\documentclass[../notes.tex]{subfiles}

\pagestyle{main}
\renewcommand{\chaptermark}[1]{\markboth{\chaptername\ \thechapter\ (#1)}{}}

\begin{document}




\chapter{The Boltzmann Factor and Partition Functions}
\section{Overview of Major Results}
\begin{itemize}
    \item \marginnote{1/10:}In this course, we will review thermochemistry from intro chem, but go deeper with statistical mechanics.
    \item TA: Haozhi.
    \begin{itemize}
        \item Did his undergrad at Oxford.
        \item Has already taught this class in the PME.
    \end{itemize}
    \item \textbf{Boltzmann constant}: The following constant. \emph{Denoted by} $\bm{k_\textbf{B}^{}}$. \emph{Given by}
    \begin{equation*}
        \kB = \SI[per-mode=symbol]{1.381e-23}{\joule\per\kelvin}
    \end{equation*}
    \begin{itemize}
        \item Equal to the quotient of the ideal gas constant and Avogadro's constant.
    \end{itemize}
    \item \textbf{Ideal gas law}: The following relationship between the pressure $P$, volume $V$, number of moles $n$, and temperature $T$ of an ideal gas, and the ideal gas constant $R$.
    \begin{equation*}
        PV = nRT
    \end{equation*}
    \begin{itemize}
        \item Multiplying by the quotient of Avogadro's constant with itself yields
        \begin{align*}
            PV &= n\NA\frac{R}{\NA}T\\
            PV &= N\kB T
        \end{align*}
        where $N$ is the number of molecules in the system.
        \item The unit for $PV$ is Joules.
        \item Thus, the above form states that $PV$ is equal to the number of particles times a tiny unit of energy.
    \end{itemize}
    \item Relating $PV$ to the kinetic energy of gas molecules/atoms\footnote{This derivation differs from that on \textcite[3-4]{bib:APChemNotes} and \textcite[18-19]{bib:PHYS13300Notes}, in that its approach is from a flux perspective.}.
    \begin{itemize}
        \item Pressure originates microscopically from the collisions of particles with the walls of their container.
        \item As such, we first seek to derive an expression for the number of collisions per second per area.
        \begin{itemize}
            \item Consider the number $N(v_x)$ particles with speed $v_x$ in the $x$-direction.
            \item The quotient $N(v_x)/V$ is the density in the container of particles with speed $v_x$.
            \item Thus, the flux of particles with velocity $v_x$ "through"/to/at the wall is this density, times the area of the wall, times the $x$-velocity of the particles.
            \item Keep in mind that the units of flux are particles (colliding with the wall with velocity $v_x$) per second.
        \end{itemize}
        \item Assume an elastic collision of each particle with the wall. Thus, when each particle of mass $m$ collides with the wall, it transfers $2mv_x$ of momentum.
        \item Another mathematical expression to use: Recall from physics that $F=\dv*{p}{t}$.
        \begin{itemize}
            \item We do not at present have an expression for the change in momentum at the wall per second. However, we do have an expression for the change in momentum at the wall per particle ($2mv_x$) and an expression for the number of particles colliding with the wall per second. Naturally the product of these two expressions is $\dv*{p}{t}$. Mathematically,
            \begin{equation*}
                F = \dv{p}{t}
                = \frac{\Delta p}{\text{particle}}\cdot\frac{\text{flux particles}}{\text{second}}
                = (2mv_x)\cdot\left( \frac{N(v_x)}{V}\cdot v_x\cdot\text{Area} \right)
            \end{equation*}
            \item Dividing both sides by area yields an expression for pressure since $P=F/\text{Area}$. In particular, $P$ denotes the pressure exerted on the wall by all particles moving with velocity $v_x$.
            \begin{equation*}
                P = \frac{2m}{V}\cdot v_x^2N(v_x)
            \end{equation*}
        \end{itemize}
        \item But, of course, we must sum over all possible $v_x$, so the total force exerted on the wall by the particles in the box is
        \begin{equation*}
            P = \int_{v_x>0}\frac{2m}{V}\cdot v_x^2N(v_x)\dd{v_x}
        \end{equation*}
        \begin{itemize}
            \item Note that we sum over only the positive values of $v_x$ because all particles moving away from the wall or not moving relative to it naturally will not collide with it.
        \end{itemize}
        \item Factor out constants and multiply by $1=N/N$.
        \begin{equation*}
            P = \frac{2mN}{V}\int_{v_x>0}v_x^2\frac{N(v_x)}{N}\dd{v_x}
        \end{equation*}
        \item Note that $N(v_x)/N=p(v_x)$, that is, the number of particles with velocity $v_x$ divided by the total number of particles equals the probability that a given particle has velocity $v_x$.
        \begin{equation*}
            P = \frac{2mN}{V}\int_{v_x>0}v_x^2p(v_x)\dd{v_x}
        \end{equation*}
        \item We know from MathChapter B in \textcite{bib:McQuarrieSimon} that
        \begin{equation*}
            \prb{v_x^2} = \int_{-\infty}^\infty v_x^2p(v_x)\dd{v_x}
            = 2\int_0^\infty v_x^2p(v_x)\dd{v_x}
        \end{equation*}
        for a symmetric probability distribution $p$ (note that the assumed \textbf{isotropy} of the gas in the box guarantees this symmetry).
        \begin{itemize}
            \item Thus,
            \begin{equation*}
                \int_{v_x>0}v_x^2p(v_x)\dd{v_x} = \frac{1}{2}\prb{v_x^2}
            \end{equation*}
        \end{itemize}
        \item It follows that
        \begin{align*}
            P &= \frac{2mN}{V}\cdot\frac{1}{2}\prb{v_x^2}\\
            &= \frac{mN}{V}\prb{v_x^2}\\
            PV &= Nm\prb{v_x^2}
        \end{align*}
        \item Another additional result: Since $v_x^2=v_y^2=v_z^2$ by the isotropy of the gas, we have that
        \begin{equation*}
            \prb{v_x^2} = \prb{v_y^2} = \prb{v_z^2}
        \end{equation*}
        \begin{itemize}
            \item Furthermore, since $v^2=v_x^2+v_y^2+v_z^2$ by the generalized Pythagorean theorem, this and the above result imply that
            \begin{align*}
                v^2 &= v_x^2+v_y^2+v_z^2\\
                &= v_x^2+v_x^2+v_x^2\\
                &= 3v_x^2\\
                \frac{1}{3}v^2 &= v_x^2
            \end{align*}
            \item Thus,
            \begin{equation*}
                \prb{v_x^2} = \prb{v_y^2} = \prb{v_z^2} = \frac{1}{3}\prb{v^2}
            \end{equation*}
        \end{itemize}
        \item It follows that
        \begin{equation*}
            PV = Nm\cdot\frac{1}{3}\prb{v^2}
        \end{equation*}
        \item Thus,
        \begin{align*}
            PV &= Nm\cdot\frac{1}{3}\prb{v^2}\\
            &= \frac{2}{3}N\cdot\frac{1}{2}m\prb{v^2}\\
            &= \frac{2}{3}N\cdot\prb{E_{KE}}\\
            \prb{E_{KE}} &= \frac{3}{2}\frac{PV}{N}\\
            \prb{E_{KE}} &= \frac{3}{2}\kB T
        \end{align*}
        \item Note that this applies to all sorts of regimes --- we used no properties of the particles (e.g., atom vs. molecule) to derive this relationship. This generalizability is why we chose a flux perspective here.
    \end{itemize}
    \item Getting the distribution of the gas energies or speed is the next logical step.
    \item First, though, we consider alternate occurrences of $\kB T$.
    \begin{itemize}
        \item The activation energy of \textcite{bib:ArrheniusEqn}: "To collide is to react" is inaccurate; it must collide with sufficient energy. The molecule must be "activated."
        \begin{equation*}
            k = A\e[-E_a/RT] = A\e[-E_a/\kB T]
        \end{equation*}
        \begin{itemize}
            \item The first $E_a$ is the molar energy of activation; the second is the molecular energy of activation.
            \item Yields the probability distribution of a molecule reacting.
        \end{itemize}
        \item Nernst equation:
        \begin{equation*}
            E_\text{cell} = E_\text{cell}^0-\frac{RT}{nF}\ln Q
        \end{equation*}
        \begin{itemize}
            \item $\ln Q$ is the ratio inside vs. outside the membrane.
            \item $F=\NA e$ where $e$ is the charge of an electron.
            \item Thus,
            \begin{equation*}
                \Delta E = \frac{RT}{nF} = \frac{\kB T}{ne}
            \end{equation*}
            \item If the potential across the membrane is approximately $\kB T$, then $\ln Q\approx 1$, so $Q\approx\e$.
            \item Thus, at body temperature ($T=\SI{310}{\kelvin}$), $\kB T/\e=\SI{26}{\milli\volt}$.
        \end{itemize}
        \item The speed of sound: Certainly sound cannot travel faster than the molecules. Therefore, we can derive the following approximation for the speed of sound.
        \begin{align*}
            \frac{1}{2}m\prb{v^2} &= \frac{3}{2}\kB T\\
            \sqrt{\prb{v^2}} &= \sqrt{\frac{3\kB T}{m}}\\
            v_\text{rms} &= \sqrt{\frac{3\kB T}{m}}
        \end{align*}
        \begin{itemize}
            \item This estimate is within $\SIrange{20}{30}{\percent}$ --- take $m$ to be the average mass of air.
        \end{itemize}
        \item de Broglie wavelength: A molecule has a kinetic energy approximately equal to $\kB T$. Additionally, the quantum mechanical kinetic energy of a molecule (see derivation directly below) aligns with this, as $\hbar^2k^2/2m\approx \kB T$. Furthermore, the particle-wave duality relates the momentum to wavelength by $p=\hbar k=h/\lambda$. Therefore,
        \begin{equation*}
            \lambda \approx \sqrt{\frac{h^2}{2m\kB T}}
        \end{equation*}
        \begin{itemize}
            \item Thus, a gas at STP has a very small de Broglie wavelength and behaves classically.
            \item Only at very low temperatures with very light gases do quantum considerations come into play.
            \item A \ce{H2} molecule at $\SI{300}{\kelvin}$ has de Broglie wavelength $\lambda=\SI{1.78}{\angstrom}$.
            \item Note that the quantum mechanical kinetic energy of a free particle is derived as follows.
            \begin{align*}
                \hat{H}\psi &= E\psi\\
                -\frac{\hbar^2}{2m}\pdv[2]{x}(\e[ikx]) &= E\e[ikx]\\
                \frac{\hbar^2k^2}{2m}\e[ikx] &= E\e[ikx]\\
                E &= \frac{\hbar^2k^2}{2m}
            \end{align*}
        \end{itemize}
    \end{itemize}
    \item \textbf{Boltzmann factor}: Gives the relative probability $p_2/p_1$ of two states $E_1,E_2$, provided their respective energies $E_1,E_2$. \emph{Given by}
    \begin{equation*}
        \frac{p_2}{p_1} = \e[-(E_2-E_1)/\kB T]
    \end{equation*}
    \begin{itemize}
        \item Consider states $E_1,E_2,E_3,\dots$, denoted by their energies.
        \item Consistency check: Given
        \begin{align*}
            \frac{p_2}{p_1} &= \e[\frac{-(E_2-E_1)}{\kB T}]&
            \frac{p_3}{p_2} &= \e[\frac{-(E_3-E_2)}{\kB T}]
        \end{align*}
        we do indeed have
        \begin{equation*}
            \frac{p_3}{p_1} = \frac{p_3}{p_2}\cdot\frac{p_2}{p_1}
            = \e[\frac{-(E_3-E_2)}{\kB T}+\frac{-(E_2-E_1)}{\kB T}]
            = \e[\frac{-(E_3-E_1)}{\kB T}]
        \end{equation*}
        \item We'll take this as God-given for now. Boltzmann derived it with a very good knowledge of the thermodynamics of freshman chemistry.
    \end{itemize}
    \item We're starting with the above exciting result, and then going back and building up to it over the next three weeks.
    \item We write the Boltzmann factor for degenerate states as follows.
    \begin{itemize}
        \item Consider four states at $E_2$ and one state at $E_1$.
        \item The Boltzmann factor still tells us that $p_2/p_1=\e[-(E_2-E_1)/\kB T]$, but we have to make the following adjustment. Indeed, the total probability of being in one of the four states at energy $E_2$ is $p(E_2)=4p_2$, while the total probability of being in the one state at energy $E_1$ is still just $p(E_1)=1p_1$.
        \item In each state $E_2$,
        \begin{equation*}
            \frac{p(E_2)}{p(E_1)} = \frac{N_2}{N_1}\e[-(E_2-E_1)/\kB T]
        \end{equation*}
    \end{itemize}
    \item The weekly quiz.
    \begin{itemize}
        \item The first quiz will be next week.
        \item A Canvas quiz -- we'll have 24 hours to start, but only 1 hour to take it.
    \end{itemize}
\end{itemize}



\section{Boltzmann Factor Examples / Partition Function}
\begin{itemize}
    \item \marginnote{1/12:}We will apply the Boltzmann factor to electronic, magnetic, translational, rotational, and vibrational molecular states.
    \item Example: Sodium lamp -- two lines at $\SI{589.6}{\nano\meter}$ and $\SI{589.0}{\nano\meter}$ with intensity ratio 1:2.
    \begin{figure}[h!]
        \centering
        \begin{tikzpicture}[
            every node/.style={black}
        ]
            \footnotesize
            \draw [yex,thick,-stealth,decorate,decoration={snake,amplitude=1pt,segment length=6pt,post length=1mm}] (2.85,2.5) -- ++(0,-2.5);
            \draw [yex,thick,-stealth,decorate,decoration={snake,amplitude=1pt,segment length=6pt,post length=1mm}] (3.45,3.5) -- ++(0,-3.5);
    
            \draw [grx,ultra thick] (0,3) node[left]{$3p$} -- ++(1,0);
            \draw [grx,ultra thick] (0,0) node[left]{$3s$} -- ++(1,0);
    
            \draw [grx,ultra thick]
                (2,3.5)   -- node[below]{$-\frac{3}{2}$} ++(0.5,0)
                ++(0.1,0) -- node[below]{$-\frac{1}{2}$} ++(0.5,0)
                ++(0.1,0) -- node[below]{$+\frac{1}{2}$}  ++(0.5,0)
                ++(0.1,0) -- node[below]{$+\frac{3}{2}$}  ++(0.5,0) node[right]{$p_{3/2}$}
            ;
            \draw [grx,ultra thick]
                (2.6,2.5) -- node[below]{$-\frac{1}{2}$} ++(0.5,0)
                ++(0.1,0) -- node[below]{$+\frac{1}{2}$} ++(0.5,0) node[right=6mm]{$p_{1/2}$}
            ;
            \draw [grx,ultra thick]
                (2.6,0) -- node[below]{$-\frac{1}{2}$} ++(0.5,0)
                ++(0.1,0) -- node[below]{$+\frac{1}{2}$} ++(0.5,0) node[right=6mm]{$s_{1/2}$}
            ;
    
            \draw [gry,very thick,densely dashed]
                (1,3) -- (2,3.5)
                (1,3) -- (2.6,2.5)
                (1,0) -- (2.6,0)
            ;
    
            \draw [|-|] (6,2.5) -- node[right]{$\Delta E$} ++(0,1);
        \end{tikzpicture}
        \caption{Sodium lamp energy levels.}
        \label{fig:sodiumLamp}
    \end{figure}
    \begin{itemize}
        \item Street lamps use this (very efficient).
        \item Also used in astronomy.
        \item In the sodium atom, there are two energy levels ($3s$ and $3p$).
        \item The states have a spin-orbit coupling effect.
        \begin{itemize}
            \item $3s$ (with $S=1/2$) splits into two degenerate states $s_{\pm 1/2}$ based on spin.
            \item $3p$ (with $L=1$ and $S=1/2$) splits into two nondegenerate states ($l=\pm 1$ [called $p_{3/2}$] and $l=0$ [called $p_{1/2}$]), which further subdivide into four (resp. two) degenerate states ($-3/2,-1/2,1/2,3/2$ and $-1/2,1/2$).
        \end{itemize}
        \item Let $\Delta E$ be the difference in energy between the $p_{3/2}$ and $p_{1/2}$. Then
        \begin{equation*}
            \frac{\Delta E}{\kB} = \frac{1}{\kB}\left( \frac{hc}{\lambda_1}-\frac{hc}{\lambda_2} \right)
            = \SI{25}{\kelvin}
        \end{equation*}
        where $\lambda_1=\SI{589.6}{\nano\meter}$ and $\lambda_2=\SI{589.0}{\nano\meter}$.
        \item Thus, $\e[-\Delta E/\kB T]\approx 1$ for $T=\SI{300}{\kelvin}$ (the temperature in the sodium vapor lamp).
        \item Therefore,
        \begin{align*}
            \frac{p(E_2)}{p(E_1)} &= \frac{4}{2}\cdot 1\\
            p(E_2) &= 2p(E_1)
        \end{align*}
    \end{itemize}
    \item Example: MRI.
    \begin{itemize}
        \item The magnetic field polarizes the spins of the hydrogen protons in our body with $\Delta E=\mu_BB$.
        \item If we also take $B=\SI{6}{\tesla}$ and $T=\SI{310}{\kelvin}$ (body temperature), then
        \begin{equation*}
            \frac{\mu_BB}{\kB T} = \num{2e-5}
        \end{equation*}
        \item Thus, very few protons actually flip, but with modern technology we can still measure this.
    \end{itemize}
    \item \textbf{Proton magnetic moment}: The magnetic moment of a proton. \emph{Denoted by} $\bm{\mu_B}$. \emph{Given by}
    \begin{equation*}
        \mu_B = \SI[per-mode=symbol]{1.4e-26}{\joule\per\tesla}
    \end{equation*}
    \item Example: Rotational.
    \begin{itemize}
        \item The rotational energy $E_J$ of a molecule depends on the angular momentum quantum number $J$ and the moment of inertia of the molecule $I=\mu R^2$ via the following relation.
        \begin{equation*}
            E_J = \frac{\hbar^2}{2I}J(J+1)
        \end{equation*}
        \item Microwave spectroscopy can be used to find molecules out in the universe.
        \item At $\SI{300}{\kelvin}$,
        \begin{equation*}
            \frac{p(J=1)}{p(J=0)} = \frac{3}{1}\e[\frac{-(E_1-E_0)}{\kB T}] = 2.95
        \end{equation*}
        \begin{itemize}
            \item As before $J=1$ corresponds to states $j=-1,0,1$.
        \end{itemize}
        \item There is a range of angular momenta due to the temperature that for $T=\SI{300}{\kelvin}$ peaks around $J=5$.
        \item See Figure \ref{fig:rotationalPopulations}.
    \end{itemize}
    \item Example: Vibrational.
    \begin{itemize}
        \item Here, $\Delta E=E_n-E_{n-1}=h\nu$ for every energy level since $E_n=h\nu(n+1/2)$.
        \item It follows that
        \begin{equation*}
            \frac{h\nu}{\kB} = \SI{2800}{\kelvin}
        \end{equation*}
        for \ce{CO}, meaning that at $\SI{300}{\kelvin}$, \ce{CO} will be largely in its ground state.
    \end{itemize}
    \item The partition function tells us everything we would want to know about a system.
    \begin{equation*}
        Q = \sum_i\e[-E_i/\kB T]
    \end{equation*}
    \begin{itemize}
        \item All we need to know is the energy of every state in the system.
        \item This is impossible for an infinite system, but the Schr\"{o}dinger equation gives us the energy of a system, so its a great place to start.
    \end{itemize}
    \item Calculating the average energy of a system from the partition function.
    \begin{itemize}
        \item Take the energy $E_1$ of the ground state to be our zero of energy.
        \item Under this assumption, we may rewrite the partition function as follows.
        \begin{equation*}
            Q = \sum_i\e[-(E_i-E_1)/\kB T]
            = \frac{p_1}{p_1}+\frac{p_2}{p_1}+\frac{p_3}{p_1}+\cdots
            = 1+\e[\frac{-(E_2-E_1)}{\kB T}]+\e[\frac{-(E_3-E_1)}{\kB T}]+\cdots
        \end{equation*}
        \item From MathChapter B, the expected energy is equal to
        \begin{equation*}
            \prb{E} = E_1p_1+E_2p_2+E_3p_3+\cdots
        \end{equation*}
        \item Since $E_1=0$, the above can be rewritten as
        \begin{equation*}
            \prb{E} = p_1\left[ E_2\frac{p_2}{p_1}+E_3\frac{p_3}{p_1}+\cdots \right]
        \end{equation*}
        \item With this result established, we now take a moment to note that
        \begin{equation*}
            \pdv{T}(\e[-E_i/\kB T]) = \frac{E_i}{\kB T^2}\e[-E_i/\kB T]
            = \frac{1}{\kB T^2}\left( E_i\frac{p_i}{p_1} \right)
        \end{equation*}
        and
        \begin{align*}
            p_1 &= 1-(p_2+p_3+\cdots)\\
            &= 1-p_1\left( \frac{p_2}{p_1}+\frac{p_3}{p_1}+\cdots \right)\\
            &= 1-p_1(Q-1)\\
            p_1 &= \frac{1}{Q}
        \end{align*}
        \item Therefore,
        \begin{align*}
            \prb{E} &= p_1\kB T^2\pdv{T}(\frac{p_1}{p_1}+\frac{p_2}{p_1}+\cdots)\\
            &= \frac{1}{Q}\kB T^2\pdv{Q}{T}\\
            \prb{E} &= \kB T^2\pdv{\ln Q}{T}
        \end{align*}
        \item The above is an important result.
    \end{itemize}
    \item Changing the origin of energy.
    \begin{itemize}
        \item We know that
        \begin{align*}
            Q(E_0) &= Q(E_0')\e[-(E_0'-E_0)/\kB T]\\
            \ln Q(E_0) &= \ln Q(E_0')-\frac{E_0'-E_0}{\kB T}
        \end{align*}
        \item Thus,
        \begin{align*}
            \prb{E}_{E_0} &= \kB T^2\pdv{T}(\ln Q(E_0))\\
            &= \kB T^2\left( \pdv{T}(\ln Q(E_0'))-\pdv{T}(\frac{E_0'-E_0}{\kB T}) \right)\\
            &= \prb{E}_{E_0'}+(E_0'-E_0)\\
            \prb{E}_{E_0}+E_0 &= \prb{E}_{E_0'}+E_0'
        \end{align*}
        \item So the change of the energy origin does indeed change the total energy by the same amount.
    \end{itemize}
\end{itemize}



\section{Calculating Average Energies}
\begin{itemize}
    \item \marginnote{1/14:}We derived that for an ideal gas, $\prb{E}=3\kB T/2$. But this may change at higher pressures.
    \item Calculating the average kinetic energy at higher temperatures.
    \begin{itemize}
        \item Use the main result from last time, which gives us the energy in terms of the partition function.
        \item We have different degrees of freedom since KE and PE are on different coordinates (KE is on speed and PE is on position).
        \item When we write the Boltzmann factor, we'll have an exponential with the sum of the kinetic and potential energy.
        \begin{equation*}
            Q = \sum_{ij}\e[-(E_{KE_i}+E_{PE_j})/\kB T]
            = \sum_{ij}\e[-E_{KE_i}/\kB T]\e[-E_{PE_j}/\kB T]
            = Q_{KE}Q_{PE}
        \end{equation*}
        \begin{itemize}
            \item The second equality holds because KE depends on the velocity coordinates and PE depends on position coordinates; thus, they are independent.
        \end{itemize}
        \item Kinetic energy partition function.
        \begin{equation*}
            E_{KE} = \frac{1}{2}mv_x^2
        \end{equation*}
        \begin{itemize}
            \item Thus,
            \begin{equation*}
                Q_{KE_{v_x}} = \int_{-\infty}^\infty\e[-\frac{1}{2}mv_x^2/\kB T]\dd{v_x} = \sqrt{\frac{2\pi \kB T}{m}}
            \end{equation*}
            \item This function doesn't depend on anything of significant import.
        \end{itemize}
        \item It follows that
        \begin{equation*}
            \prb{KE_x} = \kB T^2\pdv{T}(\ln Q_{KE_{v_x}})
            = \kB T^2\pdv{T}(\ln\sqrt{\frac{2\pi\kB}{m}}+\frac{1}{2}\ln T)
            = \frac{\kB T}{2}
        \end{equation*}
        and
        \begin{equation*}
            \prb{KE} = \prb{KE_x}+\prb{KE_y}+\prb{KE_z}
            = \frac{3}{2}\kB T
        \end{equation*}
        \item Therefore, this result holds beyond the specific case of an ideal gas!\footnote{This is why this derivation is preferable over the one presented in the book. Note that the "gas" here is assumed to be made of hard spheres.}
    \end{itemize}
    \item Now for the potential energy of a harmonic oscillator.
    \begin{itemize}
        \item $PE=\frac{1}{2}kx^2$; calculate the partition function for the coordinate $x$.
        \begin{equation*}
            Q_x = \int_{-\infty}^\infty\e[-\frac{1}{2}kx^2/\kB T]\dd{x}
            = \sqrt{\frac{2\pi \kB T}{k}}
        \end{equation*}
        \item Thus,
        \begin{equation*}
            \prb{PE_x} = \frac{\kB T}{2}
        \end{equation*}
        \item For a 3D harmonic oscillator,
        \begin{equation*}
            \prb{PE} = \frac{3}{2}\kB T
        \end{equation*}
    \end{itemize}
    \item Average potential energy of a gravitational potential.
    \begin{itemize}
        \item Apply the virial theorem (relates the average kinetic energy of a system in a conservative potential to the potential energy).
        \item Since we've shown that for any system, the average kinetic energy in one dimension is $\kB T/2$, the potential in any system will be related (i.e., have a factor of $\kB T$).
    \end{itemize}
    \item What it means to cool something down, if KE always follows the same formula.
    \begin{itemize}
        \item Although the formula does not change, $\prb{KE}\propto T$, so decreasing the temperature decreases the kinetic energy.
        \item Similarly, as things change phase, more and more potentials take hold (e.g., in the gas phase, there is no potential energy, but there is significant potential energy in the solid and liquid phases).
    \end{itemize}
    \item Rotational kinetic energy.
    \begin{itemize}
        \item Consider \ce{N2}, with its two rotational degrees of freedom.
        \item Classically,
        \begin{equation*}
            E_\text{rot} = \frac{1}{2}I\omega^2
        \end{equation*}
        \item Thus, once again,
        \begin{equation*}
            Q_\omega = \int_{-\infty}^\infty\e[-\frac{1}{2}I\omega^2/\kB T]\dd{\omega}
            = \sqrt{\frac{2\pi \kB T}{I}}
        \end{equation*}
        making
        \begin{equation*}
            \prb{E_\text{rot}} = \frac{\kB T}{2}
        \end{equation*}
        for one degree of freedom.
    \end{itemize}
    \item \textbf{Law of Dulong and Petit}: The heat capacity of elemental solids is about $3nR$.
    \begin{itemize}
        \item Observed in 1819.
        \item A major result in an era where atomic structure was just emerging.
        \item Imagine an atom bound in a three-dimensional (octahedral) potential. It's energy is thus
        \begin{equation*}
            \frac{1}{2}mv^2+\frac{1}{2}kr^2
        \end{equation*}
        \item Thus,
        \begin{align*}
            \prb{E_\text{atom}} &= \frac{3}{2}\kB T+\frac{3}{2}\kB T = 3\kB T\\
            \prb{E_\text{solid}} &= 3N\kB T = 3n\NA\kB T = 3nRT
        \end{align*}
        \item Some heat capacities are lower than $3nR$ (solids of rare gases that are heavier and need more heat to behave ideally), and some are higher (the potential is not a harmonic potential).
    \end{itemize}
    \item As experiments got better, people realized that heat capacity, as a function of temperature, decreases as $T\to\SI{0}{\kelvin}$, and was only asymptotic at $3nR$ at temperatures sufficiently close to room temperature.
    \begin{itemize}
        \item Quantum mechanics, especially the work of Einstein, solved this mystery.
        \item Atomic motion is quantized in units of energy.
        \begin{itemize}
            \item If the temperature is much higher than the quantized energies, the system behaves classically.
            \item If the temperature drops below the quantization energies of the vibration, we will not have equal population of energy levels (most will be in the ground state, making the energy 0; thus, there is no derivative of it and no heat capacity).
        \end{itemize}
    \end{itemize}
    \item Partition function of a quantum harmonic oscillator and the energy of the oscillator.
    \begin{itemize}
        \item Recall that the energies are given by $(n+1/2)h\nu$.
        \item Taking our zero of energy to be $\SI{0.5}{\hartree}$, the partition function of the vibration of the quantum harmonic oscillator is
        \begin{align*}
            Q &= 1+\e[-h\nu/\kB T]+\e[-2h\nu/\kB T]+\cdots\\
            Q &= (\e[-h\nu/\kB T])^0+(\e[-h\nu/\kB T])^1+(\e[-h\nu/\kB T])^2+\cdots\\
            Q-Q\e[-h\nu/\kB T] &= 1\\
            Q &= \frac{1}{1-\e[-h\nu/\kB T]}
        \end{align*}
        when we take the zero point energy as our zero of energy.
        \item It follows that
        \begin{align*}
            \prb{E} &= \kB T^2\pdv{T}\left[ \ln\left( \frac{1}{1-\e[-h\nu/\kB T]} \right) \right]\\
            &= \frac{h\nu}{\e[h\nu/\kB T]-1}
        \end{align*}
        \item As $T\to\infty$, $h\nu/\kB T$ gets very small. But since $\e[x]\approx 1+x$ at small $x$, as $T\to\infty$, we have that
        \begin{equation*}
            \prb{E} \approx \frac{h\nu}{(1+h\nu/\kB T)-1} = \kB T
        \end{equation*}
        \item Therefore, as $T\to\infty$, we recover the energy of a classical harmonic oscillator.
        \item On the other hand, as $T\to 0$, $E\to 0$.
    \end{itemize}
    \item Note that heat capacity $C=\pdv*{E}{T}$.
\end{itemize}



\section{Chapter 17: The Boltzmann Factor and Partition Functions}
\emph{From \textcite{bib:McQuarrieSimon}.}
\begin{itemize}
    \item \marginnote{1/23:}Up to this point, we have established that all physical systems' energy states are quantized. Now, we address questions such as "what fraction of the molecules are to be found in the ground vibrational state, the first excited vibrational state, and so on" \parencite[693]{bib:McQuarrieSimon}.
    \begin{itemize}
        \item We will see how notions such as `higher temperature systems should have more populated excited states' translate into precise mathematics.
        \item Our two most important tools to address such questions are the \textbf{Boltzmann factor} and the \textbf{partition function}.
    \end{itemize}
    \item \textbf{Boltzmann factor}: The relation between the probability that a system will be in a given state to the energy of that state. \emph{Given by}
    \begin{equation*}
        p_j \propto \e[-E_j/\kB T]
    \end{equation*}
    \item \textbf{Partition function}: A function in terms of which we can express all of the macroscopic properties of a given system, such as energy, heat capacity, and pressure. \emph{Denoted by} $\bm{Q}$, $\bm{Q(N,V,\beta)}$, $\bm{Q(N,V,T)}$. \emph{Given by}
    \begin{equation*}
        Q = \sum_j\e[-E_j/\kB T]
    \end{equation*}
    \item Determining on what quantities the energies of a macroscopic system (such as some volume of fluid, gas, or solid) depend.
    \begin{itemize}
        \item Consider an ideal gas confined to a cubic box of side length $a$ (we will not generalize this result explicitly even though it can be).
        \item In such a system, the constituent particles do not interact, so the energy of the system will be a simple sum
        \begin{equation*}
            E_j(N,V) = \epsilon_1+\cdots+\epsilon_N
        \end{equation*}
        of the energies $\epsilon_1,\dots,\epsilon_N$ of the $N$ particles with no higher-degree interaction terms necessary.
        \item Additionally, the confinement means that quantum mechanically, every particle exists in a potential of zero within the cubic box and is subject to infinite potential outside the box.
        \item Thus, if we consider only the translational energies of each particle, we may apply the particle in a 3D cubic box model from Chapter 3 of \textcite{bib:McQuarrieSimon} to learn that each
        \begin{equation*}
            \epsilon_i = \frac{h^2}{8ma^2}(n_x^2+n_y^2+n_z^2)
        \end{equation*}
        \item Notice that $E_j$ depends on $N$ and $V$ in this system via the $N$ terms in the summation and the dependence of each $\epsilon_i$ on $a=\sqrt[3]{V}$.
        \item These are the most important (and general) factors on which $E_j$ depends, and hence we often denote the energy of the $j^\text{th}$ state of the system by $E_j(N,V)$
    \end{itemize}
    \item \textbf{Heat reservoir}: An essentially infinite heat bath.
    \item \textbf{Ensemble}: A huge collection of systems with identical values of $N$, $V$, and $T$ in thermal contact with a heat reservoir at a temperature $T$.
    \begin{itemize}
        \item For a given ensemble, we denote the number of systems in state $j$ by $a_j$ and the total number of systems by $\mathcal{A}$.
    \end{itemize}
    \item Finding the relative number $a_m/a_n$ of systems in the ensemble in states $a_n$ and $a_m$.
    \begin{itemize}
        \item $a_m/a_n$ will depend on the energies $E_n$ and $E_m$ via some function $f$, i.e.,
        \begin{equation*}
            \frac{a_m}{a_n} = f(E_n,E_m)
        \end{equation*}
        \item Energies are given relative to some zero, but $a_m/a_n$ will not depend on the arbitrary choice of this zero. Thus, the math must make said zero cancel, so we take
        \begin{equation*}
            \frac{a_m}{a_n} = f(E_n-E_m)
        \end{equation*}
        \item The above equation must hold for any two energy states, so we can also write $a_l/a_m=f(E_m-E_l)$ and $a_l/a_n=f(E_n-E_l)$ for instance. But this implies that
        \begin{align*}
            \frac{a_l}{a_n} &= \frac{a_m}{a_n}\cdot\frac{a_l}{a_m}\\
            f(E_n-E_l) &= f(E_n-E_m)f(E_m-E_l)
        \end{align*}
        \item The above equation uniquely describes an exponential function, so we take $f(E)=\e[\beta E]$\footnote{Note that the base need not be $\e$, but we can take it to be $\e$ WLOG since $\e[\beta E]=(\e[\beta])^E$ and we may take $\beta$ such that $\e[\beta]$ equals any positive real number.}.
        \begin{itemize}
            \item To check that our definition of $f$ satisfies the above equation, note that
            \begin{equation*}
                \e[\beta(E_n-E_l)] = \e[\beta(E_n-E_m)]\e[\beta(E_m-E_l)]
            \end{equation*}
        \end{itemize}
    \end{itemize}
    \item Deriving an expression for $a_m$.
    \begin{itemize}
        \item From the above result, we have that
        \begin{align*}
            \frac{a_m}{a_n} &= \e[\beta(E_n-E_m)]\\
            a_m\e[\beta E_m] &= a_n\e[\beta E_n]
        \end{align*}
        i.e., that the value of $a_n\e[\beta E_n]$ is the same (hence constant) for any $n$ since $m$ is arbitrary. Thus, let $C=a_n\e[\beta E_n]$.
        \item It follows that
        \begin{align*}
            \frac{a_m}{a_n} &= \e[\beta(E_n-E_m)]\\
            a_m &= a_n\e[\beta E_n]\e[-\beta E_m]\\
            a_m &= C\e[-\beta E_m]
        \end{align*}
    \end{itemize}
    \item Determining $C$.
    \begin{itemize}
        \item We have that
        \begin{align*}
            C\sum_j\e[-\beta E_j] &= \sum_ja_j = \mathcal{A}\\
            C &= \frac{\mathcal{A}}{\sum_j\e[-\beta E_j]}
        \end{align*}
        \item Thus,
        \begin{equation*}
            \frac{a_j}{\mathcal{A}} = \frac{1}{\sum_j\e[-\beta E_j]}\e[-\beta E_j(N,V)]
        \end{equation*}
        \item Taking the limit as the number of systems in the ensemble goes to infinity makes $a_j/\mathcal{A}\to p_j$, where $p_j$ is the \emph{probability} that a system will be in state $j$ (see MathChapter B).
        \item Recognizing that the denominator above is the partition function (and a function of $N$, $V$, and $\beta$), we have that
        \begin{equation*}
            p_j(N,V,\beta) = \frac{1}{Q(N,V,\beta)}\e[\beta E_j(N,V)]
        \end{equation*}
    \end{itemize}
    \item We will later show that $\beta=1/\kB T$.
    \begin{itemize}
        \item Note, however, that from a theoretical point of view, $\beta$ can be just as useful as $T$.
    \end{itemize}
    \item Expressions for the average energy $\prb{E}$ of a system.
    \begin{itemize}
        \item From the definition of $\prb{E}$ in MathChapter B, we have that
        \begin{equation*}
            \prb{E} = \sum_j p_jE_j = \sum_j\frac{E_j\e[-\beta E_j]}{Q}
        \end{equation*}
        \item We can also express $\prb{E}$ entirely in terms of $Q$ since
        \begin{align*}
            \pdv{\ln Q}{\beta} &= -\sum_j\frac{E_j\e[-\beta E_j]}{Q} = -\prb{E}\\
            \prb{E} &= -\pdv{\ln Q}{\beta}
        \end{align*}
        \item Substituting $\beta=1/\kB T$ and applying the chain rule to $\pdv*{\ln Q}{T}$ yields
        \begin{align*}
            \pdv{f}{T} &= \pdv{f}{\beta}\cdot\pdv{\beta}{T}
            = \pdv{f}{\beta}\cdot\frac{1}{\kB}\pdv{T}(\frac{1}{T})
            = \pdv{f}{\beta}\cdot-\frac{1}{\kB T^2}\\
            \pdv{f}{\beta} &= -\kB T^2\pdv{f}{T}
        \end{align*}
        so
        \begin{equation*}
            \prb{E} = \kB T^2\pdv{\ln Q}{T}
        \end{equation*}
    \end{itemize}
    \item Using results from Chapter 14, \textcite{bib:McQuarrieSimon} calculates the average energy of a bare proton in a magnetic field, concluding that at $T=0$ (i.e., zero thermal energy), the proton orients itself in the direction of the magnetic field with certainty while as $T\to\infty$, the thermal energy is such that the proton becomes equally likely to be in either state.
    \item Calculating the average energy of a monatomic ideal gas.
    \begin{itemize}
        \item From Chapter 18,
        \begin{align*}
            Q(N,V,\beta) &= \frac{[q(V,\beta)]^N}{N!}&
            q(V,\beta) &= \left( \frac{2\pi m}{h^2\beta} \right)^{3/2}V
        \end{align*}
        \item We have that
        \begin{align*}
            \ln Q &= N\ln q-\ln N!\\
            &= -\frac{3N}{2}\ln\beta+\frac{3N}{2}\ln\left( \frac{2\pi m}{h^2} \right)+N\ln V-\ln N!\\
            &= -\frac{3N}{2}\ln\beta+\text{terms not involving }\beta
        \end{align*}
        \item Therefore,
        \begin{equation*}
            \prb{E} = -\pdv{\ln Q}{\beta} = \frac{3N}{2}\dv{\ln\beta}{\beta} = \frac{3N}{2\beta} = \frac{3}{2}N\kB T = \frac{3}{2}nRT
        \end{equation*}
    \end{itemize}
    \item The above result leads us to a fundamental postulate of physical chemistry: "The ensemble average of any quantity, as calculated using the probability distribution [$p_j=\frac{1}{Q}\e[-\beta E_j]$], is the same as the experimentally observed value of that quantity" \parencite[700]{bib:McQuarrieSimon}.
    \item The experimentally observed energy of a system is denoted by $U$.
    \item A molar quantity is denoted by an overbar (e.g., $\overline{U}$ is the experimentally observed energy of one mole of a system).
    \item Calculating the average energy of a diatomic ideal gas.
    \begin{itemize}
        \item From Chapter 18,
        \begin{align*}
            Q(N,V,\beta) &= \frac{[q(V,\beta)]^N}{N!}&
            q(V,\beta) &= \left( \frac{2\pi m}{h^2\beta} \right)^{3/2}V\cdot\frac{8\pi^2I}{h^2\beta}\cdot\frac{\e[-\beta h\nu/2]}{1-\e[-\beta h\nu]}
        \end{align*}
        for the rigid rotator-harmonic oscillator model of an ideal diatomic gas.
        \begin{itemize}
            \item The first term in the expression for $q(V,\beta)$ is translational (and identical to that of a monatomic ideal gas), the second term is rotational, and the third term is vibrational.
        \end{itemize}
        \item Using the same procedure as before, we can calculate that for one mole of a diatomic ideal gas,
        \begin{equation*}
            \overline{U} = \frac{3}{2}RT+RT+\frac{\NA h\nu}{2}+\frac{\NA h\nu\e[-\beta h\nu]}{1-\e[-\beta h\nu]}
        \end{equation*}
        \begin{itemize}
            \item "The first term represents the average translational energy, the second term represents the average rotational energy, the third term represents the zero-point vibrational energy, and the fourth term represents the average vibrational energy" \parencite[701]{bib:McQuarrieSimon}.
            \item Note that the fourth term becomes significant only at higher temperatures. 
        \end{itemize}
    \end{itemize}
    \item \textbf{Constant-volume heat capacity}: A measure of how the energy of a system changes with temperature at constant amount and volume. \emph{Denoted by} $\bm{C_V}$. \emph{Given by}
    \begin{equation*}
        C_V = \pdv{\prb{E}}{T} = \pdv{U}{T}
    \end{equation*}
    \begin{itemize}
        \item We can express $C_V$ in terms of $Q$ via our above expression for $\prb{E}$ as a function of $Q$.
    \end{itemize}
    \item For an ideal monatomic gas,
    \begin{equation*}
        \overline{C}_V = \frac{3}{2}R
    \end{equation*}
    \item For an ideal diatomic gas,
    \begin{equation*}
        \overline{C}_V = \frac{5}{2}R+R\left( \frac{h\nu}{\kB T} \right)^2\frac{\e[-h\nu/\kB T]}{(1-\e[-h\nu/\kB T])^2}
    \end{equation*}
    \item Molar heat capacity of a crystal, as per the \textbf{Einstein model of atomic crystals}.
    \begin{itemize}
        \item Because each lattice site is identical, assume further that all atoms vibrate with the same frequency.
        \item The associated partition function is thus
        \begin{equation*}
            Q = \e[-\beta U_0]\left( \frac{\e[-\beta h\nu/2]}{1-\e[\beta h\nu]} \right)^{3N}
        \end{equation*}
        where $\nu$ is characteristic of the particular crystal and $U_0$ is the \textbf{sublimation energy} (at $\SI{0}{\kelvin}$).
        \item It follows as before that
        \begin{equation*}
            \overline{C}_V = 3R\left( \frac{h\nu}{\kB T} \right)^2\frac{\e[-h\nu/\kB T]}{(1-\e[-h\nu/\kB T])^2}
        \end{equation*}
    \end{itemize}
    \item \textbf{Einstein model of atomic crystals}: A model of a crystal as $N$ atoms situated at lattice sites, with each atom vibrating as a three-dimensional harmonic oscillator.
    \item \textbf{Sublimation energy} (at $T$): The energy needed to separate all the atoms from one another at $T$.
    \item One important consequence of this result is that by experimentally measuring the heat capacity of a crystal at different temperatures, we can determine its fundamental frequency $\nu$.
    \item Another is the \textbf{law of Dulong and Petit}.
    \item \textbf{Law of Dulong and Petit}: The molar heat capacities of atomic crystals should level off at a value of $3R=\SI{24.9}{\joule\per\mole\per\kelvin}$ at high temperatures.
    \item Expressions for the average pressure $\prb{P}$ of a system.
    \begin{itemize}
        \item From Chapter 19, the pressure of a macroscopic system in state $J$ is
        \begin{equation*}
            P_j(N,V) = -\pdv{E_j}{V}
        \end{equation*}
        \item From the definition of $\prb{P}$ in MathChapter B, we have that
        \begin{equation*}
            \prb{P} = \sum_j-\pdv{E_j}{V}\frac{\e[-\beta E_j]}{Q}
        \end{equation*}
        \item We can make the above more compact since
        \begin{equation*}
            \pdv{Q}{V} = -\beta\sum_j\pdv{E_j}{V}\e[-\beta E_j] = Q\beta\prb{P}
        \end{equation*}
        so
        \begin{align*}
            \prb{P} &= \frac{1}{\beta}\pdv{\ln Q}{V}&
            \prb{P} &= \kB T\pdv{\ln Q}{V}
        \end{align*}
    \end{itemize}
    \item Just like we did with energy, we equate the ensemble average pressure with the observed pressure via $P=\prb{P}$.
    \item Deriving the ideal-gas equation of state.
    \begin{itemize}
        \item We restrict ourselves at first to the special case of a monatomic ideal gas.
        \item As before, we have that
        \begin{align*}
            \ln Q &= \frac{3N}{2}\ln\left( \frac{2\pi m}{h^2\beta} \right)+N\ln V-\ln N!\\
            &= N\ln V+\text{terms not involving }V
        \end{align*}
        \item Therefore,
        \begin{align*}
            \prb{P} &= \kB T\pdv{\ln Q}{V} = \frac{N\kB T}{V}\\
            PV &= N\kB T
        \end{align*}
        \item Since only the terms not involving $V$ change for diatomic and polyatomic gases, the above equation of state holds for all ideal gases.
    \end{itemize}
    \item The partition function associated with the van der Waals equation is
    \begin{equation*}
        Q(N,V,\beta) = \frac{1}{N!}\left( \frac{2\pi m}{h^2\beta} \right)^{3N/2}(V-Nb)^N\e[\beta aN^2/V]
    \end{equation*}
    \begin{itemize}
        \item Indeed, going through the same process as above with this equation yields
        \begin{equation*}
            \left( P+\frac{aN^2}{V^2} \right)(V-Nb) = N\kB T
        \end{equation*}
    \end{itemize}
    \item Since we lack the computational power to calculate the set $\{E_j\}$ of eigenvalues of the $N$-body Schr\"{o}dinger equation, we often approximate each $E_j$ as the sum of the energies of the constituent particles of a system.
    \item Consider a system of independent, \emph{distinguishable} particles.
    \begin{itemize}
        \item A good example of one is the Einstein model of atomic crystals, since each atom is assumed to vibrate independently of the others and each atom is distinguishable by its position in the crystal lattice.
        \item Applying the summation approximation, we get
        \begin{equation*}
            E_l(N,V) = \underbrace{\varepsilon_i^a(V)+\varepsilon_j^b(V)+\varepsilon_k^c(V)+\cdots}_{N\text{ times}}
        \end{equation*}
        where each $\varepsilon_i^a$ denotes the energy of an individual particle ($i$ being the energy state, and $a$ being the index of the particle [they are distinguishable]).
        \item Under this approximation, the partition function of the system becomes
        \begin{equation*}
            Q(N,V,T) = \sum_l\e[-\beta E_l] = \sum_{i,j,k,\dots}\e[-\beta(\varepsilon_i^a(V)+\varepsilon_j^b(V)+\varepsilon_k^c(V)+\cdots)]
        \end{equation*}
        \item Since we can sum over the indices separately (i.e., one after another), the above summation can mathematically be rewritten
        \begin{align*}
            Q(N,V,T) &= \sum_i\e[-\beta\varepsilon_i^a]\sum_j\e[-\beta\varepsilon_j^b]\sum_k\e[-\beta\varepsilon_k^c]\cdots\\
            &= q_a(V,T)q_b(V,T)q_c(V,T)\cdots
        \end{align*}
        where each $q(V,T)$ is a \textbf{molecular partition function}.
    \end{itemize}
    \item \textbf{Molecular partition function}: A partition function pertaining to a particular molecule within a system. \emph{Denoted by} $q(V,T)$. \emph{Given by}
    \begin{equation*}
        q(V,T) = \sum_j\e[-\varepsilon_j/\kB T]
    \end{equation*}
    \begin{itemize}
        \item Often able to be evaluated since they only depend on the allowed energies of individual atoms or molecules.
    \end{itemize}
    \item \textbf{Boson}: A particle whose wave function must be symmetric under the interchange of two identical particles.
    \begin{itemize}
        \item Particles of integer spin (such as photons [spin 1] and deuterons [spin 0]) are bosons.
        \item Two identical fermions \emph{can} occupy the same single-particle energy state.
    \end{itemize}
    \item \textbf{Fermions}: A particle whose wave function must be antisymmetric under the interchange of two identical particles.
    \begin{itemize}
        \item Particles of half-integer spin (such as electrons, protons, and neutrons [all with spin $1/2$]) are fermions.
        \item Two identical fermions \emph{cannot} occupy the same single-particle energy state.
    \end{itemize}
    \item Consider a system of independent, \emph{indistinguishable} particles.
    \begin{itemize}
        \item As before, we have that $E_{ijk\dots}=\varepsilon_i+\varepsilon_j+\varepsilon_k+\cdots$ for $N$ terms, but we since we cannot distinguish between particles, we cannot sum over the indices separately.
        \item Thus, our partition function is set at
        \begin{equation*}
            Q(N,V,T) = \sum_{i,j,k,\dots}\e[-\beta(\varepsilon_i+\varepsilon_j+\varepsilon_k+\cdots)]
        \end{equation*}
        \item If the particles in question are fermions, the indices are not independent of each other.
        \begin{itemize}
            \item In particular, we cannot have a $\varepsilon_1+\varepsilon_1+\varepsilon_2+\varepsilon_3+\cdots$ term in the summation because no two fermions can occupy the same single-particle energy state.
            \item This restriction omits all terms with more than one particle in the same energy state from the summation.
        \end{itemize}
        \item If the particles in question are bosons, then we must avoid summing identical terms.
        \begin{itemize}
            \item In particular, terms such as $\varepsilon_1+\varepsilon_2+\varepsilon_2+\varepsilon_2+\cdots$ and $\varepsilon_2+\varepsilon_1+\varepsilon_2+\varepsilon_2+\cdots$ represent the same state, and thus should be included only once in the summation. However, an unrestricted summation would include $N$ such terms.
            \item On the other end of the spectrum, there are $N!$ terms that include $\epsilon_1+\epsilon_2+\epsilon_3+\cdots+\epsilon_N$ in some order.
        \end{itemize}
        \item In either case, the terms causing the problems are those with two or more identical indices.
        \item If it were not for such terms, we could carry out the summation in an unrestricted manner, obtaining $[q(V,T)]^N$ as with distinguishable particles (forcing each particle to have a different state is mathematically equivalent to making them distinguishable), and then divide by $N!$, obtaining $[q(V,T)]^N/N!$, to account for the over-counting.
        \item But such terms do exist in the summation. However, there are times when their presence is negligible.
        \begin{itemize}
            \item If, for example, the number of quantum states available to any particle is significantly greater than the number of particles, it is very unlikely that two particles will ever be in the same state.
            \item Indeed, most quantum-mechanical systems have an infinite number of energy states. However, at any given temperature, many of these states will be energetically inaccessible.
            \item Thus, we can only invoke the $[q(V,T)]^N/N!$ approximation if the number of quantum states with energies less than $\kB T$ (which is roughly the average energy of a molecule) is much larger than the number of particles.
        \end{itemize}
        \item In particular, if
        \begin{equation*}
            \frac{N}{V}\left( \frac{h^2}{8m\kB T} \right)^{3/2} \ll  1
        \end{equation*}
        then we may approximate
        \begin{equation*}
            Q(N,V,T) = \frac{[q(V,T)]^N}{N!}
        \end{equation*}
        \begin{itemize}
            \item This criterion favors large particle mass, high temperature, and low density.
            \item It is easily satisfied in most conventional cases.
            \item Quantum systems that do not satisfy this criterion must be treated by special methods beyond the scope of \textcite{bib:McQuarrieSimon}.
        \end{itemize}
    \end{itemize}
    \item \textbf{Boltzmann statistics}: Statistics in which the number of available molecular states is much greater than the number of particles.
    \begin{itemize}
        \item Favored by high temperatures.
    \end{itemize}
    \item Relating the average energy $\prb{E}$ of a system to the average energy $\prb{\varepsilon}$ of a constituent molecule.
    \begin{align*}
        \prb{E} &= \kB T^2\pdv{\ln Q}{T}\\
        &= \kB T^2\pdv{T}(\ln\left( \frac{q^N}{N!} \right))\\
        &= N\kB T^2\pdv{\ln q}{T}\\
        \prb{E} &= N\prb{\varepsilon}
    \end{align*}
    \item The probability that a molecule is in its $j^\text{th}$ molecular energy state is denoted by $\pi_j$.
    \item Assuming $\varepsilon=\varepsilon_i^\text{trans}+\varepsilon_i^\text{rot}+\varepsilon_i^\text{vib}+\varepsilon_i^\text{elec}$ allows us to write $q=q_\text{trans}q_\text{rot}q_\text{vib}q_\text{elec}$ since the various energy terms are distinguishable here (hence we can sum over the indices separately).
    \item The probability that a molecule is in its $i^\text{th}$ translational, $j^\text{th}$ rotational, $k^\text{th}$ vibrational, and $l^\text{th}$ electronic state is given by
    \begin{equation*}
        \pi_{ijkl} = \frac{\e[-\varepsilon_i^\text{trans}/\kB T]\e[-\varepsilon_j^\text{rot}/\kB T]\e[-\varepsilon_k^\text{vib}/\kB T]\e[-\varepsilon_l^\text{elec}/\kB T]}{q_\text{trans}q_\text{rot}q_\text{vib}q_\text{elec}}
    \end{equation*}
    \item Since the total probability $\pi_k^\text{vib}$ that a molecule is in its $k^\text{th}$ vibrational state (for example) encompasses all probabilities of it being in any translational, rotational, or vibrational state, we have by summation that
    \begin{equation*}
        \pi_k^\text{vib} = \sum_{i,j,l}\pi_{ijkl}
        = \frac{\sum_i\left( \e[-\varepsilon_i^\text{trans}/\kB T] \right)\sum_j\left( \e[-\varepsilon_j^\text{rot}/\kB T] \right)\sum_l\left( \e[-\varepsilon_l^\text{elec}/\kB T] \right)\e[-\varepsilon_k^\text{vib}/\kB T]}{q_\text{trans}q_\text{rot}q_\text{vib}q_\text{elec}}
        = \frac{\e[-\varepsilon_k^\text{vib}/\kB T]}{q_\text{vib}}
    \end{equation*}
    \item It follows that
    \begin{equation*}
        \prb{\varepsilon^\text{vib}} = \sum_k\varepsilon_k^\text{vib}\frac{\e[-\varepsilon_k^\text{vib}/\kB T]}{q_\text{vib}}
        = -\pdv{\ln q_\text{vib}}{\beta}
        = \kB T^2\pdv{\ln q_\text{vib}}{T}
    \end{equation*}
    \begin{itemize}
        \item Analogous results hold for $\prb{\varepsilon^\text{trans}}$, $\prb{\varepsilon^\text{rot}}$, and $\prb{\varepsilon^\text{elec}}$.
    \end{itemize}
    \item Although we have written partition functions as sums over energy \emph{states} up to this point, we can also sum over energy \emph{levels} by including the degeneracy $g_J$ of the level.
    \begin{itemize}
        \item For example, since the energy and degeneracy of a rigid rotator are, respectively,
        \begin{align*}
            \varepsilon_j &= \frac{\hbar^2}{2I}J(J+1)&
            g_J &= 2J+1
        \end{align*}
        we can write
        \begin{equation*}
            q_\text{rot}(T) = \sum_{J=0}^\infty(2J+1)\e[-\hbar^2J(J+1)/2I\kB T]
        \end{equation*}
    \end{itemize}
\end{itemize}




\end{document}