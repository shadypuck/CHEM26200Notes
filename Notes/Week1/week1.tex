\documentclass[../notes.tex]{subfiles}

\pagestyle{main}
\renewcommand{\chaptermark}[1]{\markboth{\chaptername\ \thechapter\ (#1)}{}}

\begin{document}




\chapter{The Boltzmann Factor and Partition Functions}
\section{Overview of Major Results}
\begin{itemize}
    \item \marginnote{1/10:}In this course, we will review thermochemistry from intro chem, but go deeper with statistical mechanics.
    \item TA: Haozhi.
    \begin{itemize}
        \item Did his undergrad at Oxford.
        \item Has already taught this class in the PME.
    \end{itemize}
    \item \textbf{Boltzmann constant}: The following constant. \emph{Denoted by} $\bm{k_B}$. \emph{Given by}
    \begin{equation*}
        k_B = \SI[per-mode=symbol]{1.381e-23}{\joule\per\kelvin}
    \end{equation*}
    \begin{itemize}
        \item Equal to the quotient of the ideal gas constant and Avogadro's constant.
    \end{itemize}
    \item \textbf{Ideal gas law}: The following relationship between the pressure $P$, volume $V$, number of moles $n$, and temperature $T$ of an ideal gas, and the ideal gas constant $R$.
    \begin{equation*}
        PV = nRT
    \end{equation*}
    \begin{itemize}
        \item Multiplying by the quotient of Avogadro's constant with itself yields
        \begin{align*}
            PV &= nN_A\frac{R}{N_A}T\\
            PV &= Nk_BT
        \end{align*}
        where $N$ is the number of molecules in the system.
        \item The unit for $PV$ is Joules.
        \item Thus, the above form states that $PV$ is equal to the number of particles times a tiny unit of energy.
    \end{itemize}
    \item Relating $PV$ to the kinetic energy of gas molecules/atoms\footnote{This derivation differs from that on \textcite[3-4]{bib:APChemNotes} and \textcite[18-19]{bib:PHYS13300Notes}, in that its approach is from a flux perspective.}.
    \begin{itemize}
        \item Pressure originates microscopically from the collisions of particles with the walls of their container.
        \item As such, we first seek to derive an expression for the number of collisions per second per area.
        \begin{itemize}
            \item Consider the number $N(v_x)$ particles with speed $v_x$ in the $x$-direction.
            \item The quotient $N(v_x)/V$ is the density in the container of particles with speed $v_x$.
            \item Thus, the flux "through"/to/at the wall is this density, times the area of the wall, times the $x$-velocity of the particles.
        \end{itemize}
        \item Assume an elastic collision of each particle with the wall. Thus, when each particle of mass $m$ collides with the wall, it transfers $2mv_x$ of momentum.
        \item Therefore, since $F=\dv*{p}{t}$, the overall force exerted on the wall by the gas particles moving with speed $v_x$ is $2mv_x$, $N(v_x)/V\cdot v_x\cdot\text{Area}$ times per second.
        \item But, of course, we must sum over all possible $v_x$, so the total force
        \begin{equation*}
            F = \int_{v_x>0} 2mv_x\cdot\frac{N(v_x)}{V}\cdot v_x\cdot\text{Area}\dd{v_x}
        \end{equation*}
        \item It follows that
        \begin{align*}
            P &= \frac{F}{\text{Area}}\\
            &= \int_{v_x>0} 2mv_x^2\cdot\frac{N(v_x)}{V}\dd{v_x}
            \intertext{The factor of $1/2$ in the following line comes from the fact that we are only integrating over half of the possible $v_x^2$s (i.e., the positive ones).}
            &= 2m\cdot\frac{N}{V}\cdot\frac{1}{2}\prb{v_x^2}\\
            &= \frac{N}{V}m\prb{v_x^2}\\
            PV &= Nm\cdot\prb{v_x^2}
            \intertext{Assuming that the gas is not moving in any one direction means that $\prb{v_x^2}=\prb{v_y^2}=\prb{v_z^2}=\frac{1}{3}\prb{v^2}$. Therefore,}
            &= Nm\cdot\frac{1}{3}\prb{v^2}\\
            &= \frac{2}{3}N\cdot\frac{1}{2}m\prb{v^2}\\
            &= \frac{2}{3}N\cdot\prb{E_{KE}}\\
            \prb{E_{KE}} &= \frac{3}{2}\frac{PV}{N}\\
            \prb{E_{KE}} &= \frac{3}{2}k_BT
        \end{align*}
        \item Note that this applies to all sorts of regimes --- we used no properties of the particles (e.g., atom vs. molecule) to derive this relationship.
    \end{itemize}
    \item Getting the distribution of the gas energies or speed is the next logical step.
    \item First, though, we consider alternate occurrences of $k_BT$.
    \begin{itemize}
        \item The activation energy of \textcite{bib:ArrheniusEqn}: "To collide is to react" is inaccurate; it must collide with sufficient energy. The molecule must be "activated."
        \begin{equation*}
            k = A\e[-E_a/RT] = A\e[-E_a/k_BT]
        \end{equation*}
        \begin{itemize}
            \item The first $E_a$ is the molar energy of activation; the second is the molecular energy of activation.
            \item Yields the probability distribution of a molecule reacting.
        \end{itemize}
        \item Nernst equation:
        \begin{equation*}
            E_\text{cell} = E_\text{cell}^0-\frac{RT}{nF}\ln Q
        \end{equation*}
        \begin{itemize}
            \item $\ln Q$ is the ratio inside vs. outside the membrane.
            \item $F=N_Ae$ where $e$ is the charge of an electron.
            \item Thus,
            \begin{equation*}
                \Delta E = \frac{RT}{nF} = \frac{k_BT}{ne}
            \end{equation*}
            \item If the potential across the membrane is approximately $k_BT$, then $\ln Q\approx 1$, so $Q\approx\e$.
            \item Thus, at body temperature ($T=\SI{310}{\kelvin}$), $k_BT/\e=\SI{26}{\milli\volt}$.
        \end{itemize}
        \item The speed of sound: Certainly sound cannot travel faster than the molecules. Therefore, we can derive the following approximation for the speed of sound.
        \begin{align*}
            \frac{1}{2}m\prb{v^2} &= \frac{3}{2}k_BT\\
            \sqrt{\prb{v^2}} &= \sqrt{\frac{3k_BT}{m}}\\
            v_\text{rms} &= \sqrt{\frac{3k_BT}{m}}
        \end{align*}
        \begin{itemize}
            \item This estimate is within $\SIrange{20}{30}{\percent}$ --- take $m$ to be the average mass of air.
        \end{itemize}
        \item de Broglie wavelength: A molecule has a kinetic energy approximately equal to $k_BT$. Additionally, the quantum mechanical kinetic energy of a molecule aligns with this, as $\hbar^2k^2/2m\approx k_BT$. Furthermore, the particle-wave duality relates the momentum to wavelength by $p=\hbar k=h/\lambda$. Therefore,
        \begin{equation*}
            \lambda \approx \sqrt{\frac{h^2}{2mk_BT}}
        \end{equation*}
        \begin{itemize}
            \item Thus, a gas at STP has a very small de Broglie wavelength and behaves classically.
            \item Only at very low temperatures with very light gasses do quantum considerations come into play.
            \item A \ce{H2} molecule at $\SI{300}{\kelvin}$ has de Broglie wavelength $\lambda=\SI{1.78}{\angstrom}$.
            \item Note that the quantum mechanical kinetic energy of a free particle is derived as follows.
            \begin{align*}
                \hat{H}\psi &= E\psi\\
                -\frac{\hbar^2}{2m}\pdv[2]{x}(\e[ikx]) &= E\e[ikx]\\
                \frac{\hbar^2k^2}{2m}\e[ikx] &= E\e[ikx]\\
                E &= \frac{\hbar^2k^2}{2m}
            \end{align*}
        \end{itemize}
    \end{itemize}
    \item \textbf{Boltzmann factor}: Gives the relative probability $p_2/p_1$ of two states $E_1,E_2$, provided their respective energies $E_1,E_2$. \emph{Given by}
    \begin{equation*}
        \frac{p_2}{p_1} = \e[-(E_2-E_1)/k_BT]
    \end{equation*}
    \begin{itemize}
        \item Consider states $E_1,E_2,E_3,\dots$, denoted by their energies.
        \item Consistency check: Given
        \begin{align*}
            \frac{p_2}{p_1} &= \e[\frac{-(E_2-E_1)}{k_BT}]&
            \frac{p_3}{p_2} &= \e[\frac{-(E_3-E_2)}{k_BT}]
        \end{align*}
        we do indeed have
        \begin{equation*}
            \frac{p_3}{p_1} = \frac{p_3}{p_2}\cdot\frac{p_2}{p_1}
            = \e[\frac{-(E_3-E_2)}{k_BT}+\frac{-(E_2-E_1)}{k_BT}]
            = \e[\frac{-(E_3-E_1)}{k_BT}]
        \end{equation*}
        \item We'll take this as God-given for now. Boltzmann derived it with a very good knowledge of the thermodynamics of freshman chemistry.
    \end{itemize}
    \item We're starting with the above exciting result, and then going back and building up to it over the next three weeks.
    \item We write the Boltzmann factor for degenerate states as follows.
    \begin{itemize}
        \item Consider four states at $E_2$ and one state at $E_1$.
        \item The Boltzmann factor still tells us that $p_2/p_1=\e[-(E_2-E_1)/k_BT]$, but we have to make the following adjustment. Indeed, the total probability of being in one of the four states at energy $E_2$ is $p(E_2)=4p_2$, while the total probability of being in the one state at energy $E_1$ is still just $p(E_1)=1p_1$.
        \item In each state $E_2$,
        \begin{equation*}
            \frac{p(E_2)}{p(E_1)} = \frac{N_2}{N_1}\e[-(E_2-E_1)/k_BT]
        \end{equation*}
    \end{itemize}
    \item The weekly quiz.
    \begin{itemize}
        \item The first quiz will be next week.
        \item A Canvas quiz -- we'll have 24 hours to take it, but only 1 hour to take it.
    \end{itemize}
\end{itemize}



\section{Boltzmann Factor Examples / Partition Function}
\begin{itemize}
    \item \marginnote{1/12:}We will apply the Boltzmann factor to electronic, magnetic, translational, rotational, and vibrational molecular states.
    \item Example: Sodium lamp -- two lines at $\SI{589.6}{\nano\meter}$ and $\SI{589.0}{\nano\meter}$ with intensity ratio 1:2.
    \begin{figure}[h!]
        \centering
        \begin{tikzpicture}[
            every node/.style={black}
        ]
            \footnotesize
            \draw [yex,thick,-stealth,decorate,decoration={snake,amplitude=1pt,segment length=6pt,post length=1mm}] (2.85,2.5) -- ++(0,-2.5);
            \draw [yex,thick,-stealth,decorate,decoration={snake,amplitude=1pt,segment length=6pt,post length=1mm}] (3.45,3.5) -- ++(0,-3.5);
    
            \draw [grx,ultra thick] (0,3) node[left]{$3p$} -- ++(1,0);
            \draw [grx,ultra thick] (0,0) node[left]{$3s$} -- ++(1,0);
    
            \draw [grx,ultra thick]
                (2,3.5)   -- node[below]{$-\frac{3}{2}$} ++(0.5,0)
                ++(0.1,0) -- node[below]{$-\frac{1}{2}$} ++(0.5,0)
                ++(0.1,0) -- node[below]{$+\frac{1}{2}$}  ++(0.5,0)
                ++(0.1,0) -- node[below]{$+\frac{3}{2}$}  ++(0.5,0) node[right]{$p_{3/2}$}
            ;
            \draw [grx,ultra thick]
                (2.6,2.5) -- node[below]{$-\frac{1}{2}$} ++(0.5,0)
                ++(0.1,0) -- node[below]{$+\frac{1}{2}$} ++(0.5,0) node[right=6mm]{$p_{1/2}$}
            ;
            \draw [grx,ultra thick]
                (2.6,0) -- node[below]{$-\frac{1}{2}$} ++(0.5,0)
                ++(0.1,0) -- node[below]{$+\frac{1}{2}$} ++(0.5,0) node[right=6mm]{$s_{1/2}$}
            ;
    
            \draw [gry,very thick,densely dashed]
                (1,3) -- (2,3.5)
                (1,3) -- (2.6,2.5)
                (1,0) -- (2.6,0)
            ;
    
            \draw [|-|] (6,2.5) -- node[right]{$\Delta E$} ++(0,1);
        \end{tikzpicture}
        \caption{Sodium lamp energy levels.}
        \label{fig:sodiumLamp}
    \end{figure}
    \begin{itemize}
        \item Street lamps use this (very efficient).
        \item Also used in astronomy.
        \item In the sodium atom, there are two energy levels ($3s$ and $3p$).
        \item The states have a spin-orbit coupling effect.
        \begin{itemize}
            \item $3s$ (with $S=1/2$) splits into two degenerate states $s_{\pm 1/2}$ based on spin.
            \item $3p$ (with $L=1$ and $S=1/2$) splits into two nondegenerate states ($l=\pm 1$ [called $p_{3/2}$] and $l=0$ [called $p_{1/2}$]), which further subdivide into four (resp. two) degenerate states ($-3/2,-1/2,1/2,3/2$ and $-1/2,1/2$).
        \end{itemize}
        \item Let $\Delta E$ be the difference in energy between the $p_{3/2}$ and $p_{1/2}$. Then
        \begin{equation*}
            \frac{\Delta E}{k_B} = \frac{1}{k_B}\left( \frac{hc}{\lambda_1}-\frac{hc}{\lambda_2} \right)
            = \SI{25}{\kelvin}
        \end{equation*}
        where $\lambda_1=\SI{589.6}{\nano\meter}$ and $\lambda_2=\SI{589.0}{\nano\meter}$.
        \item Thus, $\e[-\Delta E/k_BT]\approx 1$ for $T=\SI{300}{\kelvin}$ (the temperature in the sodium vapor lamp).
        \item Therefore,
        \begin{align*}
            \frac{p(E_2)}{p(E_1)} &= \frac{4}{2}\cdot 1\\
            p(E_2) &= 2p(E_1)
        \end{align*}
    \end{itemize}
    \item Example: MRI.
    \begin{itemize}
        \item The magnetic field polarizes the spins of the hydrogen protons in our body with $\Delta E=\mu_BB$.
        \item If we also take $B=\SI{6}{\tesla}$ and $T=\SI{310}{\kelvin}$ (body temperature), then
        \begin{equation*}
            \frac{\mu_BB}{k_BT} = \num{2e-5}
        \end{equation*}
        \item Thus, very few protons actually flip, but with modern technology we can still measure this.
    \end{itemize}
    \item \textbf{Proton magnetic moment}: The magnetic moment of a proton. \emph{Denoted by} $\bm{\mu_B}$. \emph{Given by}
    \begin{equation*}
        \mu_B = \SI[per-mode=symbol]{1.4e-26}{\joule\per\tesla}
    \end{equation*}
    \item Example: Rotational.
    \begin{itemize}
        \item The rotational energy $E_J$ of a molecule depends on the angular momentum quantum number $J$ and the moment of inertia of the molecule $I=\mu R^2$ via the following relation.
        \begin{equation*}
            E_J = \frac{\hbar^2}{2I}J(J+1)
        \end{equation*}
        \item Microwave spectroscopy can be used to find molecules out in the universe.
        \item At $\SI{300}{\kelvin}$,
        \begin{equation*}
            \frac{p(J=1)}{p(J=0)} = \frac{3}{1}\e[\frac{-(E_1-E_0)}{k_BT}] = 2.95
        \end{equation*}
        \begin{itemize}
            \item As before $J=1$ corresponds to states $j=-1,0,1$.
        \end{itemize}
        \item See Figure 18.5 in the textbook.
        \item There is a range of angular momenta due to the temperature that for $T=\SI{300}{\kelvin}$ peaks around $J=5$.
    \end{itemize}
    \item Example: Vibrational.
    \begin{itemize}
        \item Here, $\Delta E=E_n-E_{n-1}=h\nu$ for every energy level since $E_n=h\nu(n+1/2)$.
        \item It follows that
        \begin{equation*}
            \frac{h\nu}{k_B} = \SI{2800}{\kelvin}
        \end{equation*}
        for \ce{CO}, meaning that at $\SI{300}{\kelvin}$, \ce{CO} will be largely in its ground state.
    \end{itemize}
    \item The partition function tells us everything we wanna know about a system.
    \begin{equation*}
        Q = \sum_i\e[-E_i/k_BT]
    \end{equation*}
    \begin{itemize}
        \item All we need to know is the energy of every state in the system.
        \item This is impossible for an infinite system, but the Schr\"{o}dinger equation gives us the energy of a system, so its a great place to start.
    \end{itemize}
    \item Calculating the total energy from the partition function.
    \begin{itemize}
        \item To construct it, start with
        \begin{equation*}
            Q = \frac{p_1}{p_1}+\frac{p_2}{p_1}+\frac{p_3}{p_1}+\cdots
            = 1+\e[\frac{-(E_2-E_1)}{k_BT}]+\e[\frac{-(E_3-E_1)}{k_BT}]+\cdots
        \end{equation*}
        \item The total energy is equal to
        \begin{equation*}
            \prb{E} = E_1p_1+E_2p_2+E_3p_3+\cdots
        \end{equation*}
        \item Taking $E_1=0$ gives
        \begin{equation*}
            \prb{E} = p_1\left[ E_2\frac{p_2}{p_1}+E_3\frac{p_3}{p_1}+\cdots \right]
        \end{equation*}
        \item Note that
        \begin{equation*}
            \pdv{T}(\e[-E_2/k_BT]) = \frac{E_2}{k_BT^2}\e[-E_2/k_BT]
            = \frac{1}{k_BT^2}\left( E_2\frac{p_2}{p_1} \right)
        \end{equation*}
        \item Additionally,
        \begin{align*}
            p_1 &= 1-(p_2+p_3+\cdots)\\
            &= 1-p_1\left( \frac{p_2}{p_1}+\frac{p_3}{p_1}+\cdots \right)\\
            &= 1-p_1(Q-1)\\
            p_1 &= \frac{1}{Q}
        \end{align*}
        \item Therefore,
        \begin{align*}
            \prb{E} &= p_1k_BT^2\pdv{T}(\frac{p_1}{p_1}+\frac{p_2}{p_1}+\cdots)\\
            &= p_1k_BT^2\pdv{Q}{T}\\
            &= \frac{1}{Q}k_BT^2\pdv{Q}{T}\\
            \prb{E} &= k_BT^2\pdv{T}(\ln Q)
        \end{align*}
        \item The above is an important result.
    \end{itemize}
    \item Changing the origin of energy.
    \begin{itemize}
        \item We know that
        \begin{align*}
            Q(E_0) &= Q(E_0')\e[-(E_0'-E_0)/k_BT]\\
            \ln Q(E_0) &= \ln Q(E_0')-\frac{E_0'-E_0}{k_BT}
        \end{align*}
        \item Thus,
        \begin{align*}
            \prb{E}_{E_0} &= k_BT^2\pdv{T}(\ln Q(E_0))\\
            &= k_BT^2\left( \pdv{T}(\ln Q(E_0'))-\pdv{T}(\frac{E_0'-E_0}{k_BT}) \right)\\
            &= \prb{E}_{E_0'}+(E_0'-E_0)\\
            \prb{E}_{E_0}+E_0 &= \prb{E}_{E_0'}+E_0'
        \end{align*}
        \item So the change of the energy origin does indeed change the total energy by the same amount.
    \end{itemize}
\end{itemize}



\section{Calculating Average Energies}
\begin{itemize}
    \item \marginnote{1/14:}We derived that for an ideal gas, $\prb{E}=3k_BT/2$. But this may change at higher pressures.
    \item Calculating the average kinetic energy at higher temperatures.
    \begin{itemize}
        \item Use the main result from last time, which gives us the energy in terms of the partition function.
        \item We have different degrees of freedom since KE and PE are on different coordinates (KE is on speed and PE is on position).
        \item When we write the Boltzmann factor, we'll have an exponential with the sum of the kinetic and potential energy.
        \begin{equation*}
            Q = \sum_{ij}\e[-(E_{KE_i}-E_{PE_j})/k_BT]
            = \sum_{ij}\e[-E_{KE_i}/k_BT]\e[-E_{PE_j})/k_BT]
            = Q_{KE}Q_{PE}
        \end{equation*}
        \begin{itemize}
            \item The second equality holds because KE depends on the velocity coordinates and PE depends on position coordinates; thus, they are independent.
        \end{itemize}
        \item Kinetic energy partition function.
        \begin{equation*}
            E_{KE} = \frac{1}{2}mv_x^2
        \end{equation*}
        \begin{itemize}
            \item Thus,
            \begin{equation*}
                Q_{KE_{v_x}} = \int_{-\infty}^\infty\e[-\frac{1}{2}mv_x^2/k_BT]\dd{v_x} = \sqrt{\frac{2\pi k_BT}{m}}
            \end{equation*}
            \item This function doesn't depend on anything of significant import.
        \end{itemize}
        \item It follows that
        \begin{equation*}
            \prb{KE_x} = k_BT^2\pdv{T}(\ln Q_{KE_{v_x}})
            = k_BT^2\pdv{T}(\ln\sqrt{\frac{2\pi k_B}{m}}+\frac{1}{2}\ln T)
            = \frac{k_BT}{2}
        \end{equation*}
        and
        \begin{equation*}
            \prb{KE} = \prb{KE_x}+\prb{KE_y}+\prb{KE_z}
            = \frac{3}{2}k_BT
        \end{equation*}
        \item Therefore, this result holds beyond the specific case of an ideal gas!
    \end{itemize}
    \item Now for the potential energy of a harmonic oscillator.
    \begin{itemize}
        \item $PE=\frac{1}{2}kx^2$; calculate the partition function for the coordinate $x$.
        \begin{equation*}
            Q_x = \int_{-\infty}^\infty\e[-\frac{1}{2}kx^2/k_BT]\dd{x}
            = \sqrt{\frac{2\pi k_BT}{k}}
        \end{equation*}
        \item Thus,
        \begin{equation*}
            \prb{PE_x} = \frac{k_BT}{2}
        \end{equation*}
        \item For a 3D harmonic oscillator,
        \begin{equation*}
            \prb{PE} = \frac{3}{2}k_BT
        \end{equation*}
    \end{itemize}
    \item Average potential energy of a gravitational potential.
    \begin{itemize}
        \item Apply the virial theorem (relates the average kinetic energy of a system in a conservative potential to the potential energy).
        \item Since we've shown that for any system, the average kinetic energy in one dimension is $k_BT/2$, the potential in any system will be related (i.e., have a factor of $k_BT$).
    \end{itemize}
    \item What it means to cool something down, if KE always follows the same formula.
    \begin{itemize}
        \item Although the formula does not change, $\prb{KE}\propto T$, so decreasing the temperature decreases the kinetic energy.
        \item Similarly, as things change phase, more and more potentials take hold (e.g., in the gas phase, there is no potential energy, but there is significant potential energy in the solid and liquid phases).
    \end{itemize}
    \item Rotational kinetic energy.
    \begin{itemize}
        \item Consider \ce{N2}, with its two rotational degrees of freedom.
        \item Classically,
        \begin{equation*}
            E_\text{rot} = \frac{1}{2}I\omega^2
        \end{equation*}
        \item Thus, once again,
        \begin{equation*}
            Q_\omega = \int_{-\infty}^\infty\e[-\frac{1}{2}I\omega^2/k_BT]\dd{\omega}
            = \sqrt{\frac{2\pi k_BT}{I}}
        \end{equation*}
        making
        \begin{equation*}
            \prb{E_\text{rot}} = \frac{k_BT}{2}
        \end{equation*}
        for one degree of freedom.
    \end{itemize}
    \item \textbf{Law of Dulong and Petit}: The heat capacity of elemental solids is about $3nR$.
    \begin{itemize}
        \item Observed in 1819.
        \item A major result in an era where atomic structure was just emerging.
        \item Imagine an atom bound in a three-dimensional (octahedral) potential. It's energy is thus
        \begin{equation*}
            \frac{1}{2}mv^2+\frac{1}{2}kr^2
        \end{equation*}
        \item Thus,
        \begin{align*}
            \prb{E_\text{atom}} &= \frac{3}{2}k_BT+\frac{3}{2}k_BT = 3k_BT\\
            \prb{E_\text{solid}} &= 3Nk_BT = 3nN_Ak_BT = 3nRT
        \end{align*}
        \item Some heat capacities are lower than $3nR$ (solids of rare gases that are heavier and need more heat to behave ideally), and some are higher (the potential is not a harmonic potential).
    \end{itemize}
    \item As experiments got better, people realized that heat capacity, as a function of temperature, decreases as $T\to\SI{0}{\kelvin}$, and was only asymptotic at $3nR$ at temperatures sufficiently close to room temperature.
    \begin{itemize}
        \item Quantum mechanics, especially the work of Einstein, solved this mystery.
        \item Atomic motion is quantized in units of energy.
        \begin{itemize}
            \item If the temperature is much higher than the quantized energies, the system behaves classically.
            \item If the temperature drops below the quantization energies of the vibration, we will not have equal population of energy levels (most will be in the ground state, making the energy 0; thus, there is no derivative of it and no heat capacity).
        \end{itemize}
    \end{itemize}
    \item Partition function of a quantum harmonic oscillator and the energy of the oscillator.
    \begin{itemize}
        \item Recall that the energies are given by $(n+1/2)h\nu$.
        \item The partition function of the vibration of the quantum harmonic oscillator is
        \begin{align*}
            Q &= 1+\e[-h\nu/k_BT]+\e[-2h\nu/k_BT]+\cdots\\
            Q &= (\e[-h\nu/k_BT])^0+(\e[-h\nu/k_BT])^1+(\e[-h\nu/k_BT])^2+\cdots\\
            Q-Q\e[-h\nu/k_BT] &= 1\\
            Q &= \frac{1}{1-\e[-h\nu/k_BT]}
        \end{align*}
        when we take the zero point energy as our zero of energy.
        \item It follows that
        \begin{align*}
            \prb{E} &= k_BT^2\pdv{T}\left[ \ln\left( \frac{1}{1-\e[-h\nu/k_BT]} \right) \right]\\
            &= \frac{h\nu}{\e[h\nu/k_BT]-1}
        \end{align*}
        \item As $T\to\infty$, $h\nu/k_BT$ gets very small. But since $\e[x]\approx 1+x$ at small $x$, as $T\to\infty$, we have that
        \begin{equation*}
            \prb{E} \approx \frac{h\nu}{(1+h\nu/k_BT)-1} = k_BT
        \end{equation*}
        \item Therefore, as $T\to\infty$, we recover the energy of a classical harmonic oscillator.
        \item On the other hand, as $T\to 0$, $E\to 0$.
    \end{itemize}
    \item Note that heat capacity $C=\pdv*{E}{T}$.
\end{itemize}




\end{document}