\documentclass[../notes.tex]{subfiles}

\pagestyle{main}
\renewcommand{\chaptermark}[1]{\markboth{\chaptername\ \thechapter\ (#1)}{}}

\begin{document}




\chapter{The Boltzmann Factor and Partition Functions}
\section{Overview of Major Results}
\begin{itemize}
    \item \marginnote{1/10:}In this course, we will review thermochemistry from intro chem, but go deeper with statistical mechanics.
    \item TA: Haozhi.
    \begin{itemize}
        \item Did his undergrad at Oxford.
        \item Has already taught this class in the PME.
    \end{itemize}
    \item \textbf{Boltzmann constant}: The following constant. \emph{Denoted by} $\bm{k_B}$. \emph{Given by}
    \begin{equation*}
        k_B = \SI[per-mode=symbol]{1.381e-23}{\joule\per\kelvin}
    \end{equation*}
    \begin{itemize}
        \item Equal to the quotient of the ideal gas constant and Avogadro's constant.
    \end{itemize}
    \item \textbf{Ideal gas law}: The following relationship between the pressure $P$, volume $V$, number of moles $n$, and temperature $T$ of an ideal gas, and the ideal gas constant $R$.
    \begin{equation*}
        PV = nRT
    \end{equation*}
    \begin{itemize}
        \item Multiplying by the quotient of Avogadro's constant with itself yields
        \begin{align*}
            PV &= nN_A\frac{R}{N_A}T\\
            PV &= Nk_BT
        \end{align*}
        where $N$ is the number of molecules in the system.
        \item The unit for $PV$ is Joules.
        \item Thus, the above form states that $PV$ is equal to the number of particles times a tiny unit of energy.
    \end{itemize}
    \item Relating $PV$ to the kinetic energy of gas molecules/atoms\footnote{This derivation differs from that on \textcite[3-4]{bib:APChemNotes} and \textcite[18-19]{bib:PHYS13300Notes}, in that its approach is from a flux perspective.}.
    \begin{itemize}
        \item Pressure originates microscopically from the collisions of particles with the walls of their container.
        \item As such, we first seek to derive an expression for the number of collisions per second per area.
        \begin{itemize}
            \item Consider the number $N(v_x)$ particles with speed $v_x$ in the $x$-direction.
            \item The quotient $N(v_x)/V$ is the density in the container of particles with speed $v_x$.
            \item Thus, the flux "through"/to/at the wall is this density, times the area of the wall, times the $x$-velocity of the particles.
        \end{itemize}
        \item Assume an elastic collision of each particle with the wall. Thus, when each particle of mass $m$ collides with the wall, it transfers $2mv_x$ of momentum.
        \item Therefore, since $F=\dv*{p}{t}$, the overall force exerted on the wall by the gas particles moving with speed $v_x$ is $2mv_x$, $N(v_x)/V\cdot v_x\cdot\text{Area}$ times per second.
        \item But, of course, we must sum over all possible $v_x$, so the total force
        \begin{equation*}
            F = \int_{v_x>0} 2mv_x\cdot\frac{N(v_x)}{V}\cdot v_x\cdot\text{Area}\dd{v_x}
        \end{equation*}
        \item It follows that
        \begin{align*}
            P &= \frac{F}{\text{Area}}\\
            &= \int_{v_x>0} 2mv_x^2\cdot\frac{N(v_x)}{V}\dd{v_x}
            \intertext{The factor of $1/2$ in the following line comes from the fact that we are only integrating over half of the possible $v_x^2$s (i.e., the positive ones).}
            &= 2m\cdot\frac{N}{V}\cdot\frac{1}{2}\prb{v_x^2}\\
            &= \frac{N}{V}m\prb{v_x^2}\\
            PV &= Nm\cdot\prb{v_x^2}
            \intertext{Assuming that the gas is not moving in any one direction means that $\prb{v_x^2}=\prb{v_y^2}=\prb{v_z^2}=\frac{1}{3}\prb{v^2}$. Therefore,}
            &= Nm\cdot\frac{1}{3}\prb{v^2}\\
            &= \frac{2}{3}N\cdot\frac{1}{2}m\prb{v^2}\\
            &= \frac{2}{3}N\cdot\prb{E_{KE}}\\
            \prb{E_{KE}} &= \frac{3}{2}\frac{PV}{N}\\
            \prb{E_{KE}} &= \frac{3}{2}k_BT
        \end{align*}
        \item Note that this applies to all sorts of regimes --- we used no properties of the particles (e.g., atom vs. molecule) to derive this relationship.
    \end{itemize}
    \item Getting the distribution of the gas energies or speed is the next logical step.
    \item First, though, we consider alternate occurrences of $k_BT$.
    \begin{itemize}
        \item The activation energy of \textcite{bib:ArrheniusEqn}: "To collide is to react" is inaccurate; it must collide with sufficient energy. The molecule must be "activated."
        \begin{equation*}
            k = A\e[-E_a/RT] = A\e[-E_a/k_BT]
        \end{equation*}
        \begin{itemize}
            \item The first $E_a$ is the molar energy of activation; the second is the molecular energy of activation.
            \item Yields the probability distribution of a molecule reacting.
        \end{itemize}
        \item Nernst equation:
        \begin{equation*}
            E_\text{cell} = E_\text{cell}^0-\frac{RT}{nF}\ln Q
        \end{equation*}
        \begin{itemize}
            \item $\ln Q$ is the ratio inside vs. outside the membrane.
            \item $F=N_Ae$ where $e$ is the charge of an electron.
            \item Thus,
            \begin{equation*}
                \Delta E = \frac{RT}{nF} = \frac{k_BT}{ne}
            \end{equation*}
            \item If the potential across the membrane is approximately $k_BT$, then $\ln Q\approx 1$, so $Q\approx\e$.
            \item Thus, at body temperature ($T=\SI{310}{\kelvin}$), $k_BT/\e=\SI{26}{\milli\volt}$.
        \end{itemize}
        \item The speed of sound: Certainly sound cannot travel faster than the molecules. Therefore, we can derive the following approximation for the speed of sound.
        \begin{align*}
            \frac{1}{2}m\prb{v^2} &= \frac{3}{2}k_BT\\
            \sqrt{\prb{v^2}} &= \sqrt{\frac{3k_BT}{m}}\\
            v_\text{rms} &= \sqrt{\frac{3k_BT}{m}}
        \end{align*}
        \begin{itemize}
            \item This estimate is within $\SIrange{20}{30}{\percent}$ --- take $m$ to be the average mass of air.
        \end{itemize}
        \item de Broglie wavelength: A molecule has a kinetic energy approximately equal to $k_BT$. Additionally, the quantum mechanical kinetic energy of a molecule aligns with this, as $\hbar^2k^2/2m\approx k_BT$. Furthermore, the particle-wave duality relates the momentum to wavelength by $p=\hbar k=h/\lambda$. Therefore,
        \begin{equation*}
            \lambda \approx \sqrt{\frac{h^2}{2mk_BT}}
        \end{equation*}
        \begin{itemize}
            \item Thus, a gas at STP has a very small de Broglie wavelength and behaves classically.
            \item Only at very low temperatures with very light gasses do quantum considerations come into play.
            \item A \ce{H2} molecule at $\SI{300}{\kelvin}$ has de Broglie wavelength $\lambda=\SI{1.78}{\angstrom}$.
            \item Note that the quantum mechanical kinetic energy of a free particle is derived as follows.
            \begin{align*}
                \hat{H}\psi &= E\psi\\
                -\frac{\hbar^2}{2m}\pdv[2]{x}(\e[ikx]) &= E\e[ikx]\\
                \frac{\hbar^2k^2}{2m}\e[ikx] &= E\e[ikx]\\
                E &= \frac{\hbar^2k^2}{2m}
            \end{align*}
        \end{itemize}
    \end{itemize}
    \item \textbf{Boltzmann factor}: Gives the relative probability $p_2/p_1$ of two states $E_1,E_2$, provided their respective energies $E_1,E_2$. \emph{Given by}
    \begin{equation*}
        \frac{p_2}{p_1} = \e[-(E_2-E_1)/k_BT]
    \end{equation*}
    \begin{itemize}
        \item Consider states $E_1,E_2,E_3,\dots$, denoted by their energies.
        \item Consistency check: Given
        \begin{align*}
            \frac{p_2}{p_1} &= \e[\frac{-(E_2-E_1)}{k_BT}]&
            \frac{p_3}{p_2} &= \e[\frac{-(E_3-E_2)}{k_BT}]
        \end{align*}
        we do indeed have
        \begin{equation*}
            \frac{p_3}{p_1} = \frac{p_3}{p_2}\cdot\frac{p_2}{p_1}
            = \e[\frac{-(E_3-E_2)}{k_BT}+\frac{-(E_2-E_1)}{k_BT}]
            = \e[\frac{-(E_3-E_1)}{k_BT}]
        \end{equation*}
        \item We'll take this as God-given for now. Boltzmann derived it with a very good knowledge of the thermodynamics of freshman chemistry.
    \end{itemize}
    \item We're starting with the above exciting result, and then going back and building up to it over the next three weeks.
    \item We write the Boltzmann factor for degenerate states as follows.
    \begin{itemize}
        \item Consider four states at $E_2$ and one state at $E_1$.
        \item The Boltzmann factor still tells us that $p_2/p_1=\e[-(E_2-E_1)/k_BT]$, but we have to make the following adjustment. Indeed, the total probability of being in one of the four states at energy $E_2$ is $p(E_2)=4p_2$, while the total probability of being in the one state at energy $E_1$ is still just $p(E_1)=1p_1$.
        \item In each state $E_2$,
        \begin{equation*}
            p(E_2) = \frac{N_2}{N_1}\e[-(E_2-E_1)/k_BT]
        \end{equation*}
        \begin{itemize}
            \item Where did $p_1$ go in the above equation?
        \end{itemize}
    \end{itemize}
    \item The weekly quiz.
    \begin{itemize}
        \item The first quiz will be next week.
        \item A Canvas quiz -- we'll have 24 hours to take it, but only 1 hour to take it.
    \end{itemize}
\end{itemize}




\end{document}