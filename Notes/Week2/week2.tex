\documentclass[../notes.tex]{subfiles}

\pagestyle{main}
\renewcommand{\chaptermark}[1]{\markboth{\chaptername\ \thechapter\ (#1)}{}}
\stepcounter{chapter}

\begin{document}




\chapter{Partition Functions and Ideal Gases}
\section{System Partition Functions}
\begin{itemize}
    \item \marginnote{1/19:}Decomposing the partition function of a molecule into the product of separate sums as partitioned by degrees of freedom (e.g., translation, rotation, vibration, and electronic).
    \item The partition functions of \textbf{independent}, distinguishable/indistinguishable molecules.
    \begin{itemize}
        \item We should not double count the same states.
        \item The $N!$ in $Q=q^N/N!$ is not important when calculating energy (because of the properties of the $\ln$ function), but it is very important when calculating quantities such as entropy.
    \end{itemize}
    \item \textbf{Independent} (particles): A set of particles that do not interact with one another.
    \item Discusses bosons and fermions.
    \begin{itemize}
        \item We can have a two fermions in the state $|1,1\rangle$ because it is a symmetric state.
    \end{itemize}
    \item Recall the Fermi level, the boundary between the filled and unfilled electronic states in a solid.
    \begin{itemize}
        \item If $T$ is small, this level is a hard boundary.
        \item If $T$ is large, electrons can easily be excited and the Fermi level is a soft boundary.
    \end{itemize}
    \item Does the 3D particle in a box derivation for the translation molecular partition function.
    \begin{itemize}
        \item Note that since the de Broglie wavelength $\lambda_\text{DB}=\sqrt{h^2/2mk_BT}$, we may write
        \begin{equation*}
            q_x = \sum_{n_x}\e[-h^2/8mk_BTL_x^2]
            = \sum_{n_x}\e[-\lambda_\text{DB}^2n_x^2/4L_x^2]
        \end{equation*}
    \end{itemize}
    \item The number of states are occupied/have energy within $k_BT$ of the ground state.
    \begin{itemize}
        \item $-\lambda_\text{DB}^2n_x^2/4L_x^2$ is on the order of 1, implying that $n_x$ is on the order of $2L/\lambda_\text{DB}$.
        \item It follows if $L$ is on a macroscopic scale (e.g., $L\approx\SI{1}{\meter}$) and $\lambda_\text{DB}$ is on a sub-angstrom scale that $n_x$ is on the order of $10^{10}$. When $n_x$ is at such a scale, $\e[-\lambda_\text{DB}^2n_x^2/4L_x^2]\approx 1/\e$.
        \item It follows that in a $\SI{1}{\cubic\meter}$ box, we will have about $10^{30}$ states, so we really are in a regime where the number of states is larger than the number of molecules.
    \end{itemize}
    \item More precisely, we want
    \begin{equation*}
        N \ll n_xn_yn_z = \left( \frac{8mk_BT}{h^2} \right)^{3/2}L_xL_yL_z
    \end{equation*}
    where the middle term approximates the number of states so that
    \begin{equation*}
        \frac{N}{V} \ll \left( \frac{8mk_BT}{h^2} \right)^{3/2}
    \end{equation*}
    \item Approximating the translational energy with an integral.
    \begin{itemize}
        \item Concludes with the translational partition function.
        \item Since we can approach this problem from a classical perspective (as we did last Friday) or quantum mechanically (as we did today) to achieve the same result, this system again demonstrates the relation between quantum and classical mechanics.
    \end{itemize}
\end{itemize}



\section{Molecular Partition Functions}
\begin{itemize}
    \item \marginnote{1/21:}We approximate the total molecular energy as
    \begin{equation*}
        q = q_\text{elec}q_\text{trans}q_\text{vib}q_\text{rot}
    \end{equation*}
    \item The heat capacity in the very high temperature limit where translations, rotations, and vibrations are classical.
    \begin{itemize}
        \item Translational: $\frac{3}{2}k_B$.
        \item Vibrational: Each degree of freedom ($3N-5$ for a linear molecule and $3N-6$ for a nonlinear molecule) contributes $k_B$.
        \item Rotational: Each degree of freedom ($2$ for a linear molecule and $3$ for a nonlinear molecule) contributes $\frac{1}{2}k_B$.
    \end{itemize}
    \item We can use the above to calculate the heat capacity of various molecules at very high temperatures (note, however, that at such temperatures, molecules would likely dissociate; we're simply theoretically considering the classical limit here).
    \begin{itemize}
        \item \ce{Ne}: $\frac{3}{2}k_B$.
        \item \ce{H2O}: $\frac{3}{2}k_B+3\cdot k_B+3\cdot\frac{1}{2}k_B=6k_B$.
        \item \ce{O2}: $\frac{3}{2}+1\cdot k_B+2\cdot\frac{1}{2}k_B=\frac{7}{2}k_B$.
        \item \ce{CO2}: $\frac{3}{2}k_B+4\cdot k_B+2\cdot\frac{1}{2}k_B=\frac{13}{2}k_B$.
        \item \ce{CHCl3}: $\frac{3}{2}k_B+9\cdot k_B+3\cdot\frac{1}{2}k_B=12 k_B$.
    \end{itemize}
    \item Electronic partition function.
    \begin{itemize}
        \item Consider the bottom $D_e$ of the potential well of a diatomic.
        \item $D_0$ is the ionization energy from the bottom state ($D_e\neq D_0$, but relations can be obtained via spectroscopy).
        \item It follows that
        \begin{equation*}
            q_\text{elec} = g_1\e[-(-D_e/k_BT)]+g_2\e[-E_2/k_BT]
        \end{equation*}
        \item If $\dd{T}\ll(E_2+D_e)$, then $q_\text{elec}=g_1\e[D_e/k_BT]$.
    \end{itemize}
    \item Vibrational partition function.
    \begin{itemize}
        \item As before with the law of Dulong and Petit.
        \item It's a special point where $T=h\nu/k_B$.
    \end{itemize}
    \item Rotational partition function.
    \begin{itemize}
        \item Almost always classical.
        \item The rotational energy of a polyatomic molecule will almost always be $\frac{3}{2}k_B$.
        \item Let's look at a heteronuclear diatomic, such as \ce{CO}. Derives
        \begin{equation*}
            q_\text{rot} = \sum_{J=0}^\infty(2J+1)\e[-\hbar^2J(J+1)/2Ik_BT]
        \end{equation*}
        \item The \textbf{rotational temperature} leads to
        \begin{equation*}
            q_\text{rot} = \sum_{J=0}^\infty(2J+1)\e[\Theta_\text{rot}/T] = \frac{T}{\Theta_\text{rot}}
        \end{equation*}
        \item Thus, at the temperature at which we exist, rotation is equivalent classically to quantum mechanically.
    \end{itemize}
    \item \textbf{Rotational temperature}: The following quantity. \emph{Denoted by} $\bm{\Theta_\textbf{rot}}$. \emph{Given by}
    \begin{equation*}
        \Theta_\text{rot} = \hbar^2/2Ik_B
    \end{equation*}
    \item PGS will not specify whether we need a quantum vs. classical model.
    \item Homonuclear diatomic (e.g., \ce{H2}).
    \begin{itemize}
        \item The vibrational differences in energy become visible with spectroscopy.
    \end{itemize}
    \item $q_\text{rot}=T/2\Theta_\text{rot}$.
    \item Partition functions:
    \begin{itemize}
        \item If the molecule is linear, it's of the form $T/\Theta_\text{rot}$.
        \item If the molecule is nonlinear, it's of the form $T/2\Theta_\text{rot}$.
        \item Spherical top (e.g., \ce{CH4}):
        \begin{equation*}
            \frac{\sqrt{\pi}}{\sigma}\left( \frac{T}{\Theta_\text{rot}} \right)^{3/2}
        \end{equation*}
        \item Symmetric top (e.g., \ce{NH3}):
        \begin{equation*}
            \frac{\sqrt{\pi}}{\sigma}\sqrt{\frac{T^3}{\Theta_{\text{rot},a}^2\Theta_{\text{rot},b}}}
        \end{equation*}
        \begin{itemize}
            \item $a$ and $b$ are the two different symmetry axes.
        \end{itemize}
        \item Asymmetric top (e.g., \ce{H2O}):
        \begin{equation*}
            \frac{\sqrt{\pi}}{\sigma}\sqrt{\frac{T^3}{\Theta_{\text{rot},a}\Theta_{\text{rot},b}\Theta_{\text{rot},c}}}
        \end{equation*}
    \end{itemize}
    \item Application to total energy and heat capacity of a molecule.
    \begin{itemize}
        \item We have that
        \begin{equation*}
            q = \left( \frac{2\pi mk_BT}{h^2} \right)^{3/2}\cdot V\cdot\sqrt{\frac{T^2}{\sigma\Theta_{\text{rot},a}\Theta_{\text{rot},b}\Theta_{\text{rot},c}}}\cdot\sum_1^{3N-6}\frac{\e[-\Theta_\text{vib}/2T]}{1-\e[\Theta_\text{vib}/T]}\cdot g_1\e[D_e/k_BT]
        \end{equation*}
        \item Thus,
        \begin{equation*}
            \prb{E} = k_BT^2\pdv{\ln q}{T}
            = k_BT^2\pdv{T}(\frac{3}{2}\ln T+\text{constant}+\frac{3}{2}\ln T+\text{vibration}-D_e)
        \end{equation*}
        \item The energy of the vibration is $E=k_B\Theta_\text{vib}/(\e[\Theta_\text{vib}/k_BT]-1)+k_B\Theta_\text{vib}/2$. It follows that
        \begin{equation*}
            C = \pdv{E}{T}
            = k_B\frac{\Theta_\text{vib}^2}{T^2}\frac{\e[-\Theta_\text{vib}/T]}{(1-\e[-\Theta_\text{vib}/T])}
        \end{equation*}
    \end{itemize}
\end{itemize}




\end{document}