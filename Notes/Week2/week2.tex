\documentclass[../notes.tex]{subfiles}

\pagestyle{main}
\renewcommand{\chaptermark}[1]{\markboth{\chaptername\ \thechapter\ (#1)}{}}
\stepcounter{chapter}

\begin{document}




\chapter{Partition Functions and Ideal Gases}
\section{System Partition Functions}
\begin{itemize}
    \item \marginnote{1/19:}Decomposing the partition function of a molecule into the product of separate sums as partitioned by degrees of freedom (e.g., translation, rotation, vibration, and electronic).
    \item The partition functions of \textbf{independent}, distinguishable/indistinguishable molecules.
    \begin{itemize}
        \item We should not double count the same states.
        \item The $N!$ in $Q=q^N/N!$ is not important when calculating energy (because of the properties of the $\ln$ function), but it is very important when calculating quantities such as entropy.
    \end{itemize}
    \item \textbf{Independent} (particles): A set of particles that do not interact with one another.
    \item Discusses bosons and fermions.
    \begin{itemize}
        \item We can have a two fermions in the state $|1,1\rangle$ because it is a symmetric state.
    \end{itemize}
    \item Recall the Fermi level, the boundary between the filled and unfilled electronic states in a solid.
    \begin{itemize}
        \item If $T$ is small, this level is a hard boundary.
        \item If $T$ is large, electrons can easily be excited and the Fermi level is a soft boundary.
    \end{itemize}
    \item Does the 3D particle in a box derivation for the translation molecular partition function.
    \begin{itemize}
        \item Note that since the de Broglie wavelength $\lambda_\text{DB}=\sqrt{h^2/2mk_BT}$, we may write
        \begin{equation*}
            q_x = \sum_{n_x}\e[-h^2/8mk_BTL_x^2]
            = \sum_{n_x}\e[-\lambda_\text{DB}^2n_x^2/4L_x^2]
        \end{equation*}
    \end{itemize}
    \item The number of states are occupied/have energy within $k_BT$ of the ground state.
    \begin{itemize}
        \item $-\lambda_\text{DB}^2n_x^2/4L_x^2$ is on the order of 1, implying that $n_x$ is on the order of $2L/\lambda_\text{DB}$.
        \item It follows if $L$ is on a macroscopic scale (e.g., $L\approx\SI{1}{\meter}$) and $\lambda_\text{DB}$ is on a sub-angstrom scale that $n_x$ is on the order of $10^{10}$. When $n_x$ is at such a scale, $\e[-\lambda_\text{DB}^2n_x^2/4L_x^2]\approx 1/\e$.
        \item It follows that in a $\SI{1}{\cubic\meter}$ box, we will have about $10^{30}$ states, so we really are in a regime where the number of states is larger than the number of molecules.
    \end{itemize}
    \item More precisely, we want
    \begin{equation*}
        N \ll n_xn_yn_z = \left( \frac{8mk_BT}{h^2} \right)^{3/2}L_xL_yL_z
    \end{equation*}
    where the middle term approximates the number of states so that
    \begin{equation*}
        \frac{N}{V} \ll \left( \frac{8mk_BT}{h^2} \right)^{3/2}
    \end{equation*}
    \item Approximating the translational energy with an integral.
    \begin{itemize}
        \item Concludes with the translational partition function.
        \item Since we can approach this problem from a classical perspective (as we did last Friday) or quantum mechanically (as we did today) to achieve the same result, this system again demonstrates the relation between quantum and classical mechanics.
    \end{itemize}
\end{itemize}




\end{document}