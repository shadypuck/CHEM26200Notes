\documentclass[../notes.tex]{subfiles}

\pagestyle{main}
\renewcommand{\chaptermark}[1]{\markboth{\chaptername\ \thechapter\ (#1)}{}}
\stepcounter{chapter}

\begin{document}




\chapter{Partition Functions and Ideal Gases}
\section{System Partition Functions}
\begin{itemize}
    \item \marginnote{1/19:}Decomposing the partition function of a molecule into the product of separate sums as partitioned by degrees of freedom (e.g., translation, rotation, vibration, and electronic).
    \item The partition functions of \textbf{independent}, distinguishable/indistinguishable molecules.
    \begin{itemize}
        \item We should not double count the same states.
        \item The $N!$ in $Q=q^N/N!$ is not important when calculating energy (because of the properties of the natural log), but it is very important when calculating quantities such as entropy.
    \end{itemize}
    \item \textbf{Independent} (particles): A set of particles that do not interact with one another.
    \item Discusses bosons and fermions.
    \begin{itemize}
        \item We can have a two fermions in the state $|1,1\rangle$ because it is a symmetric state.
    \end{itemize}
    \item Recall the Fermi level, the boundary between the filled and unfilled electronic states in a solid.
    \begin{itemize}
        \item If $T$ is small, this level is a hard boundary.
        \item If $T$ is large, electrons can easily be excited and the Fermi level is a soft boundary.
    \end{itemize}
    \item Does the 3D particle in a box derivation for the translation molecular partition function.
    \begin{itemize}
        \item Note that since the de Broglie wavelength $\lambda_\text{DB}=\sqrt{h^2/2mk_BT}$, we may write
        \begin{equation*}
            q_x = \sum_{n_x}\e[-h^2/8mk_BTL_x^2]
            = \sum_{n_x}\e[-\lambda_\text{DB}^2n_x^2/4L_x^2]
        \end{equation*}
    \end{itemize}
    \item The number of states are occupied/have energy within $k_BT$ of the ground state.
    \begin{itemize}
        \item $-\lambda_\text{DB}^2n_x^2/4L_x^2$ is on the order of 1, implying that $n_x$ is on the order of $2L/\lambda_\text{DB}$.
        \item It follows if $L$ is on a macroscopic scale (e.g., $L\approx\SI{1}{\meter}$) and $\lambda_\text{DB}$ is on a sub-angstrom scale that $n_x$ is on the order of $10^{10}$. When $n_x$ is at such a scale, $\e[-\lambda_\text{DB}^2n_x^2/4L_x^2]\approx 1/\e$.
        \item It follows that in a $\SI{1}{\cubic\meter}$ box, we will have about $10^{30}$ states, so we really are in a regime where the number of states is larger than the number of molecules.
    \end{itemize}
    \item More precisely, we want
    \begin{equation*}
        N \ll n_xn_yn_z = \left( \frac{8mk_BT}{h^2} \right)^{3/2}L_xL_yL_z
    \end{equation*}
    where the middle term approximates the number of states so that
    \begin{equation*}
        \frac{N}{V} \ll \left( \frac{8mk_BT}{h^2} \right)^{3/2}
    \end{equation*}
    \item Approximating the translational energy with an integral.
    \begin{itemize}
        \item Concludes with the translational partition function.
        \item Since we can approach this problem from a classical perspective (as we did last Friday) or quantum mechanically (as we did today) to achieve the same result, this system again demonstrates the relation between quantum and classical mechanics.
    \end{itemize}
\end{itemize}



\section{Molecular Partition Functions}
\begin{itemize}
    \item \marginnote{1/21:}We approximate the total molecular energy as
    \begin{equation*}
        q = q_\text{elec}q_\text{trans}q_\text{vib}q_\text{rot}
    \end{equation*}
    \item The heat capacity in the very high temperature limit where translations, rotations, and vibrations are classical.
    \begin{itemize}
        \item Translational: $\frac{3}{2}k_B$.
        \item Vibrational: Each degree of freedom ($3N-5$ for a linear molecule and $3N-6$ for a nonlinear molecule) contributes $k_B$.
        \item Rotational: Each degree of freedom ($2$ for a linear molecule and $3$ for a nonlinear molecule) contributes $\frac{1}{2}k_B$.
    \end{itemize}
    \item We can use the above to calculate the heat capacity of various molecules at very high temperatures (note, however, that at such temperatures, molecules would likely dissociate; we're simply theoretically considering the classical limit here).
    \begin{itemize}
        \item \ce{Ne}: $\frac{3}{2}k_B$.
        \item \ce{H2O}: $\frac{3}{2}k_B+3\cdot k_B+3\cdot\frac{1}{2}k_B=6k_B$.
        \item \ce{O2}: $\frac{3}{2}+1\cdot k_B+2\cdot\frac{1}{2}k_B=\frac{7}{2}k_B$.
        \item \ce{CO2}: $\frac{3}{2}k_B+4\cdot k_B+2\cdot\frac{1}{2}k_B=\frac{13}{2}k_B$.
        \item \ce{CHCl3}: $\frac{3}{2}k_B+9\cdot k_B+3\cdot\frac{1}{2}k_B=12 k_B$.
    \end{itemize}
    \item Electronic partition function.
    \begin{itemize}
        \item Consider the bottom $D_e$ of the potential well of a diatomic.
        \item $D_0$ is the ionization energy from the bottom state ($D_e\neq D_0$, but relations can be obtained via spectroscopy).
        \item It follows that
        \begin{equation*}
            q_\text{elec} = g_1\e[-(-D_e/k_BT)]+g_2\e[-E_2/k_BT]
        \end{equation*}
        \item If $\dd{T}\ll(E_2+D_e)$, then $q_\text{elec}=g_1\e[D_e/k_BT]$.
    \end{itemize}
    \item Vibrational partition function.
    \begin{itemize}
        \item As before with the law of Dulong and Petit.
        \item It's a special point where $T=h\nu/k_B$.
    \end{itemize}
    \item Rotational partition function.
    \begin{itemize}
        \item Almost always classical.
        \item The rotational energy of a polyatomic molecule will almost always be $\frac{3}{2}k_B$.
        \item Let's look at a heteronuclear diatomic, such as \ce{CO}. Derives
        \begin{equation*}
            q_\text{rot} = \sum_{J=0}^\infty(2J+1)\e[-\hbar^2J(J+1)/2Ik_BT]
        \end{equation*}
        \item The \textbf{rotational temperature} leads to
        \begin{equation*}
            q_\text{rot} = \sum_{J=0}^\infty(2J+1)\e[\Theta_\text{rot}/T] = \frac{T}{\Theta_\text{rot}}
        \end{equation*}
        \item Thus, at the temperature at which we exist, rotation is equivalent classically to quantum mechanically.
    \end{itemize}
    \item \textbf{Rotational temperature}: The following quantity. \emph{Denoted by} $\bm{\Theta_\textbf{rot}}$. \emph{Given by}
    \begin{equation*}
        \Theta_\text{rot} = \hbar^2/2Ik_B
    \end{equation*}
    \item PGS will not specify whether we need a quantum vs. classical model.
    \item Homonuclear diatomic (e.g., \ce{H2}).
    \begin{itemize}
        \item The vibrational differences in energy become visible with spectroscopy.
    \end{itemize}
    \item $q_\text{rot}=T/2\Theta_\text{rot}$.
    \item Partition functions:
    \begin{itemize}
        \item If the molecule is linear, it's of the form $T/\Theta_\text{rot}$.
        \item If the molecule is nonlinear, it's of the form $T/2\Theta_\text{rot}$.
        \item Spherical top (e.g., \ce{CH4}):
        \begin{equation*}
            \frac{\sqrt{\pi}}{\sigma}\left( \frac{T}{\Theta_\text{rot}} \right)^{3/2}
        \end{equation*}
        \item Symmetric top (e.g., \ce{NH3}):
        \begin{equation*}
            \frac{\sqrt{\pi}}{\sigma}\sqrt{\frac{T^3}{\Theta_{\text{rot},a}^2\Theta_{\text{rot},b}}}
        \end{equation*}
        \begin{itemize}
            \item $a$ and $b$ are the two different symmetry axes.
        \end{itemize}
        \item Asymmetric top (e.g., \ce{H2O}):
        \begin{equation*}
            \frac{\sqrt{\pi}}{\sigma}\sqrt{\frac{T^3}{\Theta_{\text{rot},a}\Theta_{\text{rot},b}\Theta_{\text{rot},c}}}
        \end{equation*}
    \end{itemize}
    \item Application to total energy and heat capacity of a molecule.
    \begin{itemize}
        \item We have that
        \begin{equation*}
            q = \left( \frac{2\pi mk_BT}{h^2} \right)^{3/2}\cdot V\cdot\sqrt{\frac{T^2}{\sigma\Theta_{\text{rot},a}\Theta_{\text{rot},b}\Theta_{\text{rot},c}}}\cdot\sum_1^{3N-6}\frac{\e[-\Theta_\text{vib}/2T]}{1-\e[\Theta_\text{vib}/T]}\cdot g_1\e[D_e/k_BT]
        \end{equation*}
        \item Thus,
        \begin{equation*}
            \prb{E} = k_BT^2\pdv{\ln q}{T}
            = k_BT^2\pdv{T}(\frac{3}{2}\ln T+\text{constant}+\frac{3}{2}\ln T+\text{vibration}-D_e)
        \end{equation*}
        \item The energy of the vibration is $E=k_B\Theta_\text{vib}/(\e[\Theta_\text{vib}/k_BT]-1)+k_B\Theta_\text{vib}/2$. It follows that
        \begin{equation*}
            C = \pdv{E}{T}
            = k_B\frac{\Theta_\text{vib}^2}{T^2}\frac{\e[-\Theta_\text{vib}/T]}{(1-\e[-\Theta_\text{vib}/T])}
        \end{equation*}
    \end{itemize}
\end{itemize}



\section{Chapter 18: Partition Functions and Ideal Gases}
\emph{From \textcite{bib:McQuarrieSimon}.}
\begin{itemize}
    \item \marginnote{1/23:}Herein, we will calculate the partition functions and heat capacities of ideal gases.
    \begin{itemize}
        \item We heavily rely on the expression of the partition function for a system of independent, indistinguishable particles, which ideal gases are likely to satisfy because of their low density.
    \end{itemize}
    \item Deriving the translational molecular partition function of an atom in a monatomic ideal gas.
    \begin{itemize}
        \item As mentioned in Chapter 17, if we let the container be cubic, then
        \begin{equation*}
            \varepsilon(n_x,n_y,n_z) = \frac{h^2}{8ma^2}(n_x^2+n_y^2+n_z^2)
        \end{equation*}
        \item It follows that
        \begin{align*}
            q_\text{trans} &= \sum_{n_x,n_y,n_z=1}^\infty\e[-\beta\varepsilon(n_x,n_y,n_z)]\\
            &= \sum_{n_x=1}^\infty\exp(-\frac{\beta h^2n_x^2}{8ma^2})\sum_{n_y=1}^\infty\exp(-\frac{\beta h^2n_y^2}{8ma^2})\sum_{n_z=1}^\infty\exp(-\frac{\beta h^2n_z^2}{8ma^2})\\
            &= \left[ \sum_{n=1}^\infty\exp(-\frac{\beta h^2n^2}{8ma^2}) \right]^3
        \end{align*}
        \item The above sum cannot be evaluated in closed form. However, since later terms in the summation get very small, it is an excellent approximation to replace the summation with an integral, i.e.,
        \begin{align*}
            q_\text{trans} &= \left( \int_0^\infty\e[-\beta h^2n^2/8ma^2]\dd{n} \right)^3\\
            &= \left( \sqrt{\frac{\pi}{4\beta h^2/8ma^2}} \right)^3\\
            &= \left( \sqrt{\frac{2\pi m}{\beta h^2}} \right)^3a^3\\
            &= \left( \frac{2\pi mk_BT}{h^2} \right)^{3/2}V
        \end{align*}
    \end{itemize}
    \item \marginnote{1/24:}Deriving the electronic molecular partition function of an atom in a monatomic ideal gas.
    \begin{itemize}
        \item We express the partition function here in terms of levels, i.e., by
        \begin{equation*}
            q_\text{elec} = \sum_ig_{ei}\e[-\beta\varepsilon_{ei}]
        \end{equation*}
        where $g_{ei}$ is the degeneracy and $\varepsilon_{ei}$ is the energy of the $i^\text{th}$ electronic level.
        \item Taking $\varepsilon_{e1}=0$ to be the zero of energy yields
        \begin{equation*}
            q_\text{elec} = g_{e1}+g_{e2}\e[-\beta\varepsilon_{e2}]+\cdots
        \end{equation*}
        \begin{itemize}
            \item Note that since $\varepsilon$'s are usually on the order of tens of thousands of wavenumbers, $\e[-\beta\varepsilon_{e2}]$ is around $10^{-5}$ for most atoms at ordinary temperatures, so only the first term in the summation is significantly different from zero.
            \item For some gases such as halogens, other terms may be important, but even there the sum converges very rapidly.
        \end{itemize}
        \item Using spectroscopic data, we can show that the fraction of atoms of most gases in the first excited state is very small.
        \begin{itemize}
            \item For example, the fraction of helium atoms at $\SI{300}{\kelvin}$ in the first excited state is $10^{-334}$.
            \item For fluorine, however, the fraction is on the order of $10^{-2}$, which is significant. In this case, we need to approximate $q_\text{elec}$ with more than one or two terms.
        \end{itemize}
    \end{itemize}
    \item \textcite{bib:McQuarrieSimon} recalculates the average energy, heat capacity, and pressure of a monatomic ideal gas using the above results.
    \item Diatomics.
    \begin{itemize}
        \item The translational partition function is
        \begin{equation*}
            q_\text{trans}(V,T) = \left[ \frac{2\pi(m_1+m_2)k_BT}{h^2} \right]^{3/2}V
        \end{equation*}
        \item We take the zero of rotational energy to be the $J=0$ state.
        \item We take the zero of vibrational energy to be the bottom of the internuclear potential well of the lowest electronic state (so that the energy of the ground vibrational state is $h\nu/2$).
        \item We take the zero of electronic energy to be the energy of the separated atoms at rest in their ground electronic state (so that the energy of the ground electronic state is $-D_e$\footnote{See Figure 9.7 of \textcite{bib:CHEM26100Notes}.}).
    \end{itemize}
    \item \textbf{Vibrational temperature}: The following quantity. \emph{Denoted by} $\bm{\Theta_\text{vib}}$. \emph{Given by}
    \begin{equation*}
        \Theta_\text{vib} = \frac{h\nu}{k_B}
    \end{equation*}
    \item Deriving the vibrational molecular partition function of a molecule in a diatomic ideal gas.
    \begin{align*}
        q_\text{vib}(T) &= \sum_{v=0}^\infty\e[-\beta(v+1/2)h\nu]\\
        &= \e[-\beta h\nu/2]\sum_{v=0}^\infty\e[-\beta h\nu v]\\
        &= \e[-\beta h\nu/2]\frac{1}{1-\e[-\beta h\nu]}\\
        &= \frac{\e[-\beta h\nu/2]}{1-\e[-\beta h\nu]}
    \end{align*}
    \item In terms of $\Theta_\text{vib}$,
    \begin{gather*}
        q_\text{vib}(T) = \frac{\e[-\Theta_\text{vib}/2T]}{1-\e[-\Theta_\text{vib}/T]}\\
        \prb{E_\text{vib}} = Nk_B\left( \frac{\Theta_\text{vib}}{2}+\frac{\Theta_\text{vib}}{\e[\Theta_\text{vib}/T]-1} \right)\\
        \overline{C}_\text{V,\text{vib}} = R\left( \frac{\Theta_\text{vib}}{T} \right)^2\frac{\e[-\Theta_\text{vib}/T]}{(1-\e[-\Theta_\text{vib}/T])^2}
    \end{gather*}
    \item Note that the high temperature limit of $\overline{C}_\text{V,\text{vib}}$ is $R$, and $\overline{C}_\text{V,\text{vib}}$ attains $R/2$ at $T=0.34\,\Theta_\text{vib}$.
    \item Calculating the fraction of molecules in the ground vibrational state reveals that generally, most molecules are in the ground vibrational state.
    \begin{itemize}
        \item Exceptions include \ce{Br2}, the smaller force constant and larger mass of which lead to a smaller value of $\Theta_\text{vib}$.
    \end{itemize}
    \item \textbf{Rotational temperature}: The following quantity. \emph{Denoted by} $\bm{\Theta_\text{rot}}$. \emph{Given by}
    \begin{equation*}
        \Theta_\text{rot} = \frac{\hbar^2}{2Ik_B} = \frac{hB}{k_B}
    \end{equation*}
    \begin{itemize}
        \item $B$ is the rotational constant (see Chapter 5) in the above equation.
    \end{itemize}
    \item Deriving the rotational molecular partition function of a \emph{heteronuclear} molecule in a diatomic ideal gas.
    \begin{itemize}
        \item We have
        \begin{equation*}
            q_\text{rot}(T) = \sum_{J=0}^\infty(2J+1)\e[-\Theta_\text{rot}J(J+1)/T]
        \end{equation*}
        \item As with the translational partition function, for $\Theta_\text{rot}\ll T$ (which is true for normal temperatures), we may approximate the above sum via an integral. This approximation is known as the high-temperature limit, and under it,
        \begin{align*}
            q_\text{rot}(T) &= \int_0^\infty(2J+1)\e[-\Theta_\text{rot}J(J+1)/T]\dd{J}\\
            &= \int_0^\infty\e[-\Theta_\text{rot}x/T]\dd{x}\\
            &= \frac{T}{\Theta_\text{rot}} = \frac{8\pi^2Ik_BT}{h^2}
        \end{align*}
        \item For low temperatures or molecules with large values of $\Theta_\text{rot}$ we evaluate some number of terms of the sum directly, but we will not consider these cases further.
    \end{itemize}
    \item It follows from the above that
    \begin{align*}
        \prb{E_\text{rot}} &= Nk_BT&
        \overline{C}_{V,\text{rot}} &= R
    \end{align*}
    \begin{itemize}
        \item Each of the two rotational degrees of freedom of a diatomic contributes $R/2$ to $\overline{C}_{V,\text{rot}}$.
    \end{itemize}
    \item Contrary to the other component parts of energy, higher energy rotational states are significantly occupied.
    \begin{figure}[h!]
        \centering
        \begin{tikzpicture}
            \footnotesize
            \draw (0,4) -- (0,0) -- (5,0);
            \foreach \x [evaluate=\x as \n using int(\x*4)] in {0,1,...,5} {
                \node [below] at (\x,0) {$\n$};
            }
            \draw (0,4) node[left]{$0.10$}       -- ++(0.2,0);
            \draw (0,2)  node[left]{$0.05$}       -- ++(0.2,0);
            \node [above left] {$0.00$};
    
            \draw [semithick,blx] plot[ycomb,mark=x,domain=0:5,samples=21] (\x,{40*(2*(4*\x)+1)*0.00923*e^(-0.00923*(4*\x)*((4*\x)+1))});
        \end{tikzpicture}
        \caption{The fraction of molecules in the $J^\text{th}$ rotational level for \ce{CO} at $\SI{300}{\kelvin}$.}
        \label{fig:rotationalPopulations}
    \end{figure}
    \begin{itemize}
        \item We have that the fraction $f_J$ of molecules in the $J^\text{th}$ vibrational state is
        \begin{equation*}
            f_J = \frac{(2J+1)\e[-\Theta_\text{rot}J(J+1)/T]}{q_\text{rot}} = (2J+1)(\tfrac{\Theta_\text{rot}}{T})\e[-\Theta_\text{rot}J(J+1)/T]
        \end{equation*}
        \item We can estimate the most probable value of $J$ by solving $\pdv*{f_J}{J}=0$, which gives $J=7$ in agreement with Figure \ref{fig:rotationalPopulations}.
    \end{itemize}
    \item \marginnote{1/25:}We now address the rotational molecular partition function for a \emph{homonuclear} diatomic ideal gas molecule.
    \begin{itemize}
        \item Because of the additional perpendicular $C_2$ axes of symmetry in a homonuclear diatomic compared to a heteronuclear diatomic, the diatomic's constituent atoms are `more' indistinguishable, i.e., only nuclear spin can distinguish them.
        \item "In particular, if the two nuclei have integral spins (bosons), the molecular wave function must be symmetric with respect to an interchange of the two nuclei; if the nuclei have half odd integer spin (fermions), the molecular wave function must be antisymmetric" \parencite[747]{bib:McQuarrieSimon}.
        \item This symmetry affects the population of the rotational energy levels in a way that \emph{can} be derived, but we will just state the important result, which is that for a homonuclear diatomic molecule,
        \begin{equation*}
            q_\text{rot}(T) = \frac{T}{2\Theta_\text{rot}}
        \end{equation*}
    \end{itemize}
    \item To unify the two rotational molecular partition functions, we let
    \begin{equation*}
        q_\text{rot}(T) = \frac{T}{\sigma\Theta_\text{rot}}
    \end{equation*}
    in general, where $\sigma$ is the \textbf{symmetry number}.
    \item \textbf{Symmetry number}: The number of different ways a given molecule can be rotated into a configuration indistinguishable from the original. \emph{Denoted by} $\bm{\sigma}$. \emph{Given by}
    \begin{equation*}
        \sigma =
        \begin{cases}
            1 & \text{heteronuclear}\\
            2 & \text{homonuclear}
        \end{cases}
    \end{equation*}
    \item Taking the energy of an ideal diatomic gas molecule to be a simple sum of its translational, rotational, vibrational, and electronic energies yields the molecular partition function
    \begin{equation*}
        q(V,T) = \left( \frac{2\pi Mk_BT}{h^2} \right)^{3/2}V\cdot\frac{T}{\sigma\Theta_\text{rot}}\cdot\frac{\e[-\Theta_\text{vib}/2T]}{1-\e[-\Theta_\text{vib}/T]}\cdot g_{e1}\e[D_e/k_BT]
    \end{equation*}
    where we require $\Theta_\text{rot}\ll T$, that the only populated electronic state is the ground state, that the zero of electronic energy is the separated atoms at rest in their ground electronic states, and that the zero of vibrational energy is the bottom of the internuclear potential well of the lowest electronic state.
    \item \textcite{bib:McQuarrieSimon} derives the molar energy and heat capacity of a diatomic ideal gas one more time using the above equation.
    \begin{itemize}
        \item The only difference is that the newly added electronic factor in the partition function adds a term of $-N_AD_e$ to the Chapter 17 formula for $\overline{U}$.
        \item Also note that we can greatly improve the agreement of the harmonic oscillator-rigid rotator model with even small first-order corrections, such as including centrifugal distortion and anharmonicity.
    \end{itemize}
    \item The translational and electronic molecular partition functions of an ideal polyatomic molecule are the same as those of an ideal monatomic or diatomic molecule.
    \item On the vibrational molecular partition function of an ideal polyatomic molecule.
    \begin{itemize}
        \item Recall from Chapter 13 that the vibrational motion of a polyatomic molecule can be expressed in terms of normal coordinates.
        \item Thus, the vibrational energy of a polyatomic molecule in state $v_j=0,1,2,\dots$ is
        \begin{equation*}
            \varepsilon_\text{vib} = \sum_{j=1}^{N_\text{vib}}\left( v_j+\tfrac{1}{2} \right)h\nu_j
        \end{equation*}
        where $\nu_j$ is the frequency of the $j^\text{th}$ normal mode.
        \item It follows that for a polyatomic molecule,
        \begin{gather*}
            q_\text{vib} = \prod_{j=1}^{N_\text{vib}}\frac{\e[-\Theta_{\text{vib},j}/2T]}{1-\e[-\Theta_{\text{vib},j}/T]}\\
            E_\text{vib} = Nk_B\sum_{j=1}^{N_\text{vib}}\left( \frac{\Theta_{\text{vib},j}}{2}+\frac{\Theta_{\text{vib},j}\e[-\Theta_{\text{vib},j}/T]}{1-\e[-\Theta_{\text{vib},j}/T]} \right)\\
            C_{V,\text{vib}} = Nk_B\sum_{j=1}^{N_\text{vib}}\left[ \left( \frac{\Theta_{\text{vib},j}}{T} \right)^2\frac{\e[-\Theta_{\text{vib},j}/T]}{(1-\e[-\Theta_{\text{vib},j}/T])^2} \right]
        \end{gather*}
    \end{itemize}
    \item Rotational molecular partition functions for linear molecules.
    \begin{itemize}
        \item We can still apply the rigid-rotator approximation, but with
        \begin{equation*}
            I = \sum_{j=1}^nm_jd_j^2
        \end{equation*}
        where $d_j$ is the distance of the $j^\text{th}$ nucleus from the center of mass of the molecule.
        \item Doing so yields
        \begin{equation*}
            q_\text{rot}(T) = \frac{T}{\sigma\Theta_\text{rot}}
        \end{equation*}
        where $\sigma=1$ for unsymmetrical molecules such as \ce{N2O} and \ce{COS} and $\sigma=2$ for symmetrical molecules such as \ce{CO2} and \ce{C2H2}.
    \end{itemize}
    \item Note that the symmetry number of \ce{NH3} is three.
    \item Rotational molecular partition functions for nonlinear molecules.
    \begin{itemize}
        \item Recall the discussion surrounding the principal moments of inertia in Chapter 13.
        \item We define three characteristic rotational temperatures, namely $\Theta_{\text{rot},j}=\hbar^2/2I_jk_B$ for $j=A,B,C$.
        \item Spherical top.
        \begin{itemize}
            \item In this case, $\Theta_{\text{rot},A}=\Theta_{\text{rot},B}=\Theta_{\text{rot},C}=\Theta_\text{rot}$.
            \item The quantum-mechanical spherical top can be solved exactly to give
            \begin{align*}
                \varepsilon_J &= \frac{\hbar^2}{2I}J(J+1)&
                g_J &= (2J+1)^2
            \end{align*}
            \item Now $\Theta_\text{rot}\ll T$ for almost all spherical top molecules at ordinary temperatures, and this has two important consequence. First, we can approximate the partition function with an integral. Second, we can neglect 1 in comparison with $J$ since the important values of $J$ are large. Thus, we have that
            \begingroup
            \allowdisplaybreaks
            \begin{align*}
                q_\text{rot}(T) &= \frac{1}{\sigma}\sum_{J=0}^\infty(2J+1)^2\e[-\hbar^2J(J+1)/2Ik_BT]\\
                &= \frac{1}{\sigma}\int_0^\infty(2J+1)^2\e[-\Theta_\text{rot}J(J+1)/T]\dd{J}\\
                &= \frac{1}{\sigma}\int_0^\infty 4J^2\e[-\Theta_\text{rot}J^2/T]\dd{J}\\
                &= \frac{4}{\sigma}\int_0^\infty J^2\e[-aJ^2]\dd{J}\\
                &= \frac{4}{\sigma}\cdot\frac{1}{4a}\sqrt{\frac{\pi}{a}}\\
                q_\text{rot}(T) &= \frac{\sqrt{\pi}}{\sigma}\left( \frac{T}{\Theta_\text{rot}} \right)^{3/2}
            \end{align*}
            \endgroup
        \end{itemize}
        \item Similarly, we have respectively for a symmetric top and an asymmetric top that
        \begin{align*}
            q_\text{rot}(T) &= \frac{\sqrt{\pi}}{\sigma}\left( \frac{T}{\Theta_{\text{rot},A}} \right)\sqrt{\frac{T}{\Theta_{\text{rot},C}}}&
            q_\text{rot}(T) &= \frac{\sqrt{\pi}}{\sigma}\sqrt{\frac{T}{\Theta_{\text{rot},A}\Theta_{\text{rot},B}\Theta_{\text{rot},C}}}
        \end{align*}
    \end{itemize}
    \item It follows that
    \begin{align*}
        \overline{U}_\text{rot} &= \frac{3RT}{2}&
        \overline{C}_{V,\text{rot}} &= \frac{3R}{2}
    \end{align*}
    \item Linear molecule equations.
    \begin{gather*}
        q(V,T) = \left( \frac{2\pi Mk_BT}{h^2} \right)^{3/2}V\cdot\frac{T}{\sigma\Theta_\text{rot}}\cdot\prod_{j=1}^{3n-5}\frac{\e[-\Theta_{\text{vib},j}/2T]}{1-\e[-\Theta_{\text{vib},j}/T]}\cdot g_{e1}\e[D_e/k_BT]\\
        \frac{U}{Nk_BT} = \frac{3}{2}+\frac{2}{2}+\sum_{j=1}^{3n-5}\left( \frac{\Theta_{\text{vib},j}}{2T}+\frac{\Theta_{\text{vib},j}/T}{\e[\Theta_{\text{vib},j}/T]-1} \right)-\frac{D_e}{k_BT}\\
        \frac{C_V}{Nk_B} = \frac{3}{2}+\frac{2}{2}+\sum_{j=1}^{3n-5}\left( \frac{\Theta_{\text{vib},j}}{T} \right)^2\frac{\e[-\Theta_{\text{vib},j}/T]}{(1-\e[-\Theta_{\text{vib},j}/T])^2}
    \end{gather*}
    \item Nonlinear molecule equations.
    \begin{gather*}
        q(V,T) = \left( \frac{2\pi Mk_BT}{h^2} \right)^{3/2}V\cdot\frac{\sqrt{\pi}}{\sigma}\sqrt{\frac{T^3}{\Theta_{\text{rot},A}\Theta_{\text{rot},B}\Theta_{\text{rot},C}}}\cdot\prod_{j=1}^{3n-6}\frac{\e[-\Theta_{\text{vib},j}/2T]}{1-\e[-\Theta_{\text{vib},j}/T]}\cdot g_{e1}\e[D_e/k_BT]\\
        \frac{U}{Nk_BT} = \frac{3}{2}+\frac{3}{2}+\sum_{j=1}^{3n-6}\left( \frac{\Theta_{\text{vib},j}}{2T}+\frac{\Theta_{\text{vib},j}/T}{\e[\Theta_{\text{vib},j}/T]-1} \right)-\frac{D_e}{k_BT}\\
        \frac{C_V}{Nk_B} = \frac{3}{2}+\frac{3}{2}+\sum_{j=1}^{3n-6}\left( \frac{\Theta_{\text{vib},j}}{T} \right)^2\frac{\e[-\Theta_{\text{vib},j}/T]}{(1-\e[-\Theta_{\text{vib},j}/T])^2}
    \end{gather*}
\end{itemize}




\end{document}