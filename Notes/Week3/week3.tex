\documentclass[../notes.tex]{subfiles}

\pagestyle{main}
\renewcommand{\chaptermark}[1]{\markboth{\chaptername\ \thechapter\ (#1)}{}}
\setcounter{chapter}{2}

\begin{document}




\chapter{Kinetic Theory of Gases / The First Law of Thermodynamics}
\section{Maxwell-Boltzmann Distribution}
\begin{itemize}
    \item \marginnote{1/24:}Applying the molecular partition function to the heat capacity of a water molecule.
    \begin{itemize}
        \item A water molecule has three vibrational modes, which we will denote by $\nu_1,\nu_2,\nu_3$ (corresponding to symmetric stretch, antisymmetric stretch, and bend).
        \item Main takeaway: Heat capacity can change with temperature.
        \item After a while (at several thousand kelvin), it will level off (see Figure 18.7).
    \end{itemize}
    \item Considers \ce{CO2}'s vibrational modes, too.
    \begin{itemize}
        \item The infrared absorption of the bending mode is what's associated with the Greenhouse Effect.
        \item The symmetric stretch is IR inactive due to its lack of change of dipole moment.
        \item Raman active: Change in the polarizability of the molecule.
    \end{itemize}
    \item The Maxwell-Boltzmann distribution.
    \begin{itemize}
        \item Maxwell derived it long before Boltzmann, but Boltzmann's thermodynamic derivation is much easier.
        \item We know from the Boltzmann factor that $p(E)\propto\e[-E/k_BT]$.
        \item Thus, to get the probability $p(v)$ of some speed $v$, we should have $p(v)\propto\e[-mv^2/2k_BT]$ times a constant giving the number of molecules of each speed? This yields
        \begin{equation*}
            p(v) = A4\pi v^2\e[-mv^2/2k_BT]
        \end{equation*}
        where $A$ is a normalization constant.
        \item The Maxwell-Boltzmann distribution is such that
        \begingroup
        \allowdisplaybreaks
        \begin{align*}
            1 &= \int_0^\infty p(v)\dd{v}\\
            &= A\int_0^\infty 4\pi v^2\e[-mv^2/2k_BT]\dd{v}\\
            &= A\int_0^\infty 4\pi\left( \frac{2k_BT}{m} \right)^{3/2}u^2\e[-u^2]\dd{u}\\
            &= A4\pi\left( \frac{2k_BT}{m} \right)^{3/2}\int_0^\infty u^2\e[-u^2]\dd{u}\\
            &= A4\pi\left( \frac{2k_BT}{m} \right)^{3/2}\frac{\sqrt{\pi}}{4}\\
            A &= \left( \frac{m}{2\pi k_BT} \right)^{3/2}
        \end{align*}
        \endgroup
        \item Therefore,
        \begin{equation*}
            p(v) = 4\pi\left( \frac{m}{2\pi k_BT} \right)^{3/2}v^2\e[-mv^2/2k_BT]
        \end{equation*}
        \item Any distribution that doesn't look like this isn't in thermal equilibrium.
    \end{itemize}
    \item A system with all particles having $v=0$ is at thermal equilibrium with $T=\SI{0}{\kelvin}$.
    \item A system with all particles having constant velocity in the same direction is at thermal equilibrium with $T=\SI{0}{\kelvin}$.
    \begin{itemize}
        \item Think relativity; if you're moving with them, it looks like they're not moving and thus this case is the same as the last one because you're movement doesn't affect the thermodynamics of that system.
    \end{itemize}
    \item A system with all particles having constant velocity in different directions is not at thermal equilibrium since it does not fit the bell curve but is rather a spike.
\end{itemize}



\section{The First Law of Thermodynamics}
\begin{itemize}
    \item \marginnote{1/26:}See \textcite{bib:PHYS13300Notes} for background on/content of today's lecture.
    \item Joule best quantified how we think about work and energy.
    \item \textbf{System}: Part of the world being investigated. It can contain energy, a number of particles, etc.
    \item The Newtonian way to change the energy into a system is to do work (mechanical, electrical, etc.) on the system. In chemistry, $\var{w}$ is positive if work is done on the system.
    \begin{itemize}
        \item We have that
        \begin{equation*}
            \var{w} = -P\dd{V}
        \end{equation*}
    \end{itemize}
    \item A system at thermal equilibrium has a given temperature characteristic of the system. Some property of the system indicates how hot or cold it is (e.g., volume of mercury, etc.)
    \item Measure heat transfer using a calorimeter and a thermomenter.
    \begin{itemize}
        \item Convention: Heat put into a system $\var{q}$ is positive and this heat is transferred if the temperature is lower than another system or surroundings.
        \item The heat capacity of a system times the change in temperature is equal to the heat put into the system. It is always positive as heat put into the system raises the temperature.
    \end{itemize}
    \item Example molar heat capacities.
    \begin{itemize}
        \item For water vapor at low pressure and $\SI{20}{\celsius}$, $\overline{C}_V=3R=\SI{25}{\joule\per\mole\per\kelvin}$.
        \item For liquid water, it's higher (hydrogen bonding).
        \item For ice, it's lower.
    \end{itemize}
    \item \textbf{First law of thermodynamics}: The internal energy of a system changes with heat put into the system and work done on the system.
    \begin{equation*}
        \dd{U} = \var{q}+\var{w}
    \end{equation*}
    \begin{itemize}
        \item Note that in engineering, $\dd{U}=\var{q}-\var{w}$.
    \end{itemize}
    \item \textbf{State variable}: A property that describes the system.
    \begin{itemize}
        \item For example, a system of gas molecules has a state defined by the state variables $T$, $P$, $V$, and $n$.
    \end{itemize}
    \item \textbf{State function}: A property that depends only upon the state of the system.
    \begin{itemize}
        \item For example, some equations of state for an ideal gas are $PV=nRT$ or $PV=2U/3$.
        \item The internal energy is a state function.
        \item Heat and work are not state functions because they do not depend uniquely on the values at equilibrium.
        \begin{itemize}
            \item They also depend on the way you do something.
        \end{itemize}
    \end{itemize}
    \item \textbf{Reversible process}: A process that can be represented as a path along state variables, e.g., a line on a $PV$ diagram. This implies that it is also a path where all state variables are known, and is therefore a path where the system is always in quasi-equilibrium.
    \begin{itemize}
        \item Isothermal, isochoric, isobaric, and adiabatic changes are reversible.
        \item All of these processes are analyzed exactly as in \textcite{bib:PHYS13300Notes}.
    \end{itemize}
    \item \textbf{Irreversible process}: A process that cannot be drawn on a PV diagram.
    \item Experiment to measure $\gamma$ (the ratio of specific heats):
    \begin{enumerate}
        \item Let sit at $P_0T_0$.
        \item Pump in a little gas (add $\Delta n$) and let sit, measure $P_0+\Delta P_1,T_0$.
        \item Open the valve to air quickly to $P_0$. Adiabatic expansion (cools down).
        \item Let sit to measure the new pressure $P+\Delta P_2$ when $T$ is back at $T_0$.
        \item $\gamma$ is determined from $\Delta P_1,\Delta P_2$ (this will be a homework problem).
    \end{enumerate}
    \begin{itemize}
        \item In the second step, we add some molecules into the container. We can show that $\Delta P_1/P_0=\Delta N_1/n_0$.
        \item In the third step, we let out the air, and we can show that $\Delta n_2/n_0=\gamma\Delta P_1/P_0$.
        \item In step 4, we have in the container $(n_0+\Delta n_1-\Delta n_2)RT_0=(P_0+\Delta P_2)V_0$.
        \item This implies that $\Delta P_2/\Delta P_1=1-1/\gamma$.
    \end{itemize}
\end{itemize}



\section{Enthalpy}
\begin{itemize}
    \item \marginnote{1/28:}Thermonamic derivation of the formula for $\prb{P}$ in terms of $Q$.
    \begin{itemize}
        \item We have that
        \begin{align*}
            U &= \sum p_jE_j\\
            \dd{U} &= \sum(\dd{p_j}E_j+p_j\dd{E_j})\\
            &= \underbrace{\sum\dd{p_j}E_j}_{\var{q}}+\underbrace{\sum p_j\pdv{E_j}{V}}_{-P}\dd{V}
        \end{align*}
        where the last part follows by analogy with $\dd{U}=\var{q}-P\dd{V}$.
        \item It follows that
        \begin{equation*}
            P = -\sum p_j\pdv{E_j}{V} = -\prb{\pdv{E}{V}}
        \end{equation*}
        \item Thus, we have that
        \begin{align*}
            P &= -\sum\frac{\e[-E_j/k_BT]}{Q}\pdv{E_j}{V}\\
            &= \frac{1}{Q}\sum k_BT\cdot-\frac{1}{k_BT}\e[-E_j/k_BT]\pdv{E_j}{V}\\
            &= k_BT\frac{1}{Q}\sum\pdv{E_j}(\e[-E_j/k_BT])\pdv{E_j}{V}\\
            &= k_BT\frac{1}{Q}\sum\pdv{V}(\e[-E_j/k_BT])\\
            &= k_BT\frac{1}{Q}\pdv{Q}{V}\\
            P &= k_BT\pdv{\ln Q}{V}
        \end{align*}
    \end{itemize}
    \item Applies the formula to an ideal gas of independent, indistinguishable particles to derive the ideal gas law.
    \item \textbf{Enthalpy}: A state function representign the heat put into the system at constant pressure. \emph{Denoted by} $\bm{H}$. \emph{Given by}
    \begin{equation*}
        H = U+PV
    \end{equation*}
    \item We have that
    \begin{align*}
        \dd{H} &= \dd{U}+P\dd{V}+V\dd{P}\\
        &= \var{q}-P\dd{V}+P\dd{V}+V\dd{P}\\
        &= \var{q}+V\dd{P}
    \end{align*}
    \begin{itemize}
        \item At constant pressure ($\dd{P}=0$), we have that $\dd{H}=\var{q}$.
        \item At constant volume, we have that $\dd{H}=\var{q}$ as well?
    \end{itemize}
    \item \textbf{Constant-volume heat capacity}. The following expression. \emph{Denoted by} $\bm{C_V}$. \emph{Given by}
    \begin{equation*}
        C_V = \left( \pdv{U}{T} \right)_{V,N}
    \end{equation*}
    \item \textbf{Constant-pressure heat capacity}. The following expression. \emph{Denoted by} $\bm{C_P}$. \emph{Given by}
    \begin{equation*}
        C_P = \left( \pdv{H}{T} \right)_{P,N}
    \end{equation*}
    \item For an ideal gas,
    \begin{align*}
        \dd{H} &= \dd{U}+\dd{(PV)}\\
        &= nC_V\dd{T}+nR\dd{T}\\
        &= n(C_V+R)\dd{T}
    \end{align*}
    \begin{itemize}
        \item Recall this result from \textcite{bib:PHYS13300Notes}.
    \end{itemize}
    \item Considers heat diagrams.
    \begin{itemize}
        \item Recall the enthalpy of phase changes $\Delta H_\text{fus}$, $\Delta H_\text{vap}$, and $\Delta H_\text{sub}$.
        \item It follows that
        \begin{equation*}
            H(T)-H(T_0) = \int_{T^0}^TC_p\dd{T}+\sum\Delta H_\text{phase changes}
        \end{equation*}
    \end{itemize}
    \item \textbf{Hess's Law}: $\Delta H=0$ around a closed loop.
    \begin{itemize}
        \item This is because $H$ is a state function.
    \end{itemize}
    \item \textbf{Standard enthalpy of formation}. \emph{Denoted by} $\bm{\Delta H_f^\circ}$. \emph{Units} $\textbf{kJ}\bm{/}\textbf{mol}$.
    \begin{itemize}
        \item Calculated from the constituent elements in their standard state, \SI{1}{\bar}, \SI{298.15}{\kelvin}.
    \end{itemize}
    \item We have, for example, that the $\Delta H_\text{vap}^\circ$ of a substance is the difference of its $\Delta H_f^\circ$ in its gaseous state and its $\Delta H_f^\circ$ in its liquid state.
    \item With the standard enthalpy of formation and the heat capacity $C_P(T)$, one gets the enthalpy of formation at nonstandard temperatures.
    \item To get the enthalpy of formation at non-standard pressures of chemical interest, most of the effect is from the gas components because solids and liquid enthalpy vary little with pressure.
    \item The direction of change is sometimes in the direction of \emph{positive} enthalpy change.
    \begin{itemize}
        \item This change is driven by the fact that in these cases, the direction of change is toward the most probable state.
    \end{itemize}
    \item In a reversible process, $\dd{U}=\var{q}_\text{rev}-P\dd{V}$. In this case
    \begin{equation*}
        \var{q}_\text{rev} = \dd{U}+P\dd{V} = nC_V\dd{T}+P\dd{V} \neq \dd{nC_VT+PV}
    \end{equation*}
    so $\var{q}_\text{rev}$ is not a state function.
    \begin{itemize}
        \item However,
        \begin{align*}
            \frac{\var{q}_\text{rev}}{T} &= nC_V\frac{\dd{T}}{T}+\frac{P\dd{V}}{T}\\
            &= nC_V\frac{\dd{T}}{T}+nR\frac{\dd{V}}{V}\\
            &= \dd{(nC_V\ln T+nR\ln V)}
        \end{align*}
        is a state function.
    \end{itemize}
\end{itemize}



\section{Chapter 25: The Kinetic Theory of Gases}
\emph{From \textcite{bib:McQuarrieSimon}.}
\begin{itemize}
    \item \marginnote{1/30:}\textbf{Kinetic theory of gases}: A simple model of gases in which the molecules (pictured as hard spheres) are assumed to be in constant, incessant motion, colliding with each other and with the walls of the container.
    \item \textcite{bib:McQuarrieSimon} does the KMT derivation of the ideal gas law from \textcite{bib:APChemNotes}. Some important notes follow.
    \begin{itemize}
        \item \textcite{bib:McQuarrieSimon} emphasizes the importance of
        \begin{equation*}
            PV = \frac{1}{3}Nm\prb{u^2}
        \end{equation*}
        as a fundamental equation of KMT, as it relates a macroscopic property $PV$ to a microscopic property $m\prb{u^2}$.
        \item In Chapter 17-18, we derived quantum mechanically, and then from the partition function, that the average translational energy $\prb{E_\text{trans}}$ for a single particle of an ideal gas is $\frac{3}{2}k_BT$. From classical mechanics, we also have that $\prb{E_\text{trans}}=\frac{1}{2}m\prb{u^2}$. \emph{This} is why we may let
        \begin{equation*}
            \frac{1}{2}m\prb{u^2} = \frac{3}{2}k_BT
        \end{equation*}
        recovering that the average translational kinetic energy of the molecules in a gas is directly proportional to the Kelvin temperature.
    \end{itemize}
    \item \textbf{Isotropic} (entity): An object or substance that has the same properties in any direction.
    \begin{itemize}
        \item For example, a homogeneous gas is isotropic, and this is what allows us to state that $\prb{u_x^2}=\prb{u_y^2}=\prb{u_z^2}$.
    \end{itemize}
    \item \textcite{bib:McQuarrieSimon} derives
    \begin{equation*}
        u_\text{rms} = \sqrt{\frac{3RT}{M}}
    \end{equation*}
    \begin{itemize}
        \item $u_\text{rms}$ is an estimate of the average speed since $\prb{u^2}\neq\prb{u}^2$ in general.
    \end{itemize}
    \item \textcite{bib:McQuarrieSimon} states without proof that the speed of sound $u_\text{sound}$ in a monatomic ideal gas is given by
    \begin{equation*}
        u_\text{sound} = \sqrt{\frac{5RT}{3M}}
    \end{equation*}
    \item Assumptions of the kinetic theory of gases.
    \begin{itemize}
        \item Particles collide elastically with the wall.
        \begin{itemize}
            \item Justified because although each collision will not be elastic (the particles in the wall are moving too), the average collision will be elastic.
        \end{itemize}
        \item Particles do not collide with each other.
        \begin{itemize}
            \item Justified because "if the gas is in equilibrium, on the average, any collision that deflects the path of a molecule\dots will be balanced by a collision that replaces the molecule" \parencite[1015]{bib:McQuarrieSimon}.
        \end{itemize}
    \end{itemize}
    \item Note that we can do the kinetic derivation at many levels of rigor, but more rigorous derivations offer results that differ only by constant factors on the order of unity.
    \item Deriving a theoretical equation for the distribution of the \emph{components} of molecular velocities.
    \begin{itemize}
        \item Let $h(u_x,u_y,u_z)\dd{u_x}\dd{u_y}\dd{u_z}$ be the fraction of molecules with velocity components between $u_j$ and $u_j+\dd{u_j}$ for $j=x,y,z$.
        \item Assume that the each component of the velocity of a molecule is independent of the values of the other two components\footnote{This can be proven.}. It follows statistically that
        \begin{equation*}
            h(u_x,u_y,u_z) = f(u_x)f(u_y)f(u_z)
        \end{equation*}
        \begin{itemize}
            \item Note that we use just one function $f$ for the probability distribution in each direction because the gas is isotropic.
        \end{itemize}
        \item We can use the isotropic condition to an even greater degree. Indeed, it implies that any information conveyed by $u_x$ is necessarily and sufficiently conveyed by $u_y$, $u_z$, and $u$. Thus, we may take
        \begin{equation*}
            h(u) = h(u_x,u_y,u_z) = f(u_x)f(u_y)f(u_z)
        \end{equation*}
        \item It follows that
        \begin{equation*}
            \pdv{\ln h(u)}{u_x} = \pdv{u_x}(\ln f(u_x)+\text{terms not involving }u_x)
            = \dv{\ln f(u_x)}{u_x}
        \end{equation*}
        \item Since
        \begin{align*}
            u^2 &= u_x^2+u_y^2+u_z^2\\
            \pdv{u_x}(u^2) &= \pdv{u_x}(u_x^2+u_y^2+u_z^2)\\
            2u\pdv{u}{u_x} &= 2u_x\\
            \pdv{u}{u_x} &= \frac{u_x}{u}
        \end{align*}
        we have that
        \begin{align*}
            \pdv{\ln h}{u_x} &= \dv{\ln h}{u}\pdv{u}{u_x} = \frac{u_x}{u}\dv{\ln h}{u}\\
            \frac{\dd{\ln h(u)}}{u\dd{u}} &= \frac{\dd{\ln f(u_x)}}{u_x\dd{u_x}}
        \end{align*}
        which generalizes to
        \begin{equation*}
            \frac{\dd{\ln h(u)}}{u\dd{u}} = \frac{\dd{\ln f(u_x)}}{u_x\dd{u_x}}
            = \frac{\dd{\ln f(u_y)}}{u_y\dd{u_y}}
            = \frac{\dd{\ln f(u_z)}}{u_z\dd{u_z}}
        \end{equation*}
        \item Since $u_x,u_y,u_z$ are independent, we know that the above equation is equal to a constant, which we may call $-\gamma$. It follows that for any $j=x,y,z$, we have that
        \begin{align*}
            \frac{\dd{\ln f(u_j)}}{u_j\dd{u_j}} &= -\gamma\\
            \frac{1}{f}\dv{f}{u_j} &= -\gamma u_j\\
            \int\frac{\dd{f}}{f} &= \int-\gamma u_j\dd{u_j}\\
            \ln f &= -\frac{\gamma}{2}u_j^2+C\\
            f(u_j) &= A\e[-\gamma u_j^2]
        \end{align*}
        where we have incorporated the $1/2$ into $\gamma$.
        \item To determine $A$ and $\gamma$, we let arbitrarily let $j=x$. Since $f$ is a continuous probability distribution, we may apply the normalization requirement.
        \begin{align*}
            1 &= \int_{-\infty}^\infty f(u_x)\dd{u_x}\\
            &= 2A\int_0^\infty\e[-\gamma u_x^2]\dd{u_x}\\
            &= 2A\sqrt{\frac{\pi}{4\gamma}}\\
            A &= \sqrt{\frac{\gamma}{\pi}}
        \end{align*}
        \item Additionally, since we have that $\prb{u_x^2}=\frac{1}{3}\prb{u^2}$ and $\prb{u^2}=3RT/M$, we know that $\prb{u_x^2}=RT/M$. This combined with the definition of $\prb{u_x^2}$ as a continuous probability distribution yields
        \begin{align*}
            \frac{RT}{M} &= \prb{u_x^2}\\
            &= \int_{-\infty}^\infty u_x^2f(u_x)\dd{u_x}\\
            &= 2\sqrt{\frac{\gamma}{\pi}}\int_0^\infty u_x^2\e[-\gamma u_x^2]\dd{u_x}\\
            &= 2\sqrt{\frac{\gamma}{\pi}}\cdot\frac{1}{4\gamma}\sqrt{\frac{\pi}{\gamma}}\\
            &= \frac{1}{2\gamma}\\
            \gamma &= \frac{M}{2RT}
        \end{align*}
        \item Therefore,
        \begin{equation*}
            f(u_x) = \sqrt{\frac{M}{2\pi RT}}\e[-Mu_x^2/2RT]
        \end{equation*}
        \item It is common to rewrite the above in terms of molecular quantities $m$ and $k_B$.
    \end{itemize}
    \item It follows that as temperature increases, more molecules are likely to be found with higher component velocity values.
    \item We can use the above result to show that
    \begin{equation*}
        \prb{u_x} = \int_{-\infty}^\infty u_xf(u_x)\dd{u_x} = 0
    \end{equation*}
    \item We can also calculate that $\prb{u_x^2}=RT/M$ and $m\prb{u_x}^2/2=k_BT/2$ from the above result.
    \begin{itemize}
        \item An important consequence is that the total kinetic energy is divided equally into the $x$-, $y$-, and $z$-components.
    \end{itemize}
    \item \textcite{bib:McQuarrieSimon} discusses how the Doppler effect applied to moving molecules emitting radiation makes spectral peaks wider than we'd normally predict in a phenomenon known as \textbf{Doppler broadening}.
    \item Deriving \textbf{Maxwell-Boltzmann distribution}.
    \begin{itemize}
        \item Let the probability that a molecule has speed between $u$ and $u+\dd{u}$ be defined by a continuous probability distribution $F(u)\dd{u}$. In particular, we have from the above isotropic condition that
        \begin{align*}
            F(u)\dd{u} &= f(u_x)\dd{u_x}f(u_y)\dd{u_y}f(u_z)\dd{u_z}\\
            &= \left( \frac{m}{2\pi k_BT} \right)^{3/2}\e[-m(u_x^2+u_y^2+u_z^2)/2k_BT]\dd{u_x}\dd{u_y}\dd{u_z}
        \end{align*}
        \item Considering $F$ over a \textbf{velocity space}, we realize that we may express the probability distribution $F$ as a function of $u$ via $u^2=u_x^2+u_y^2+u_z^2$ and the differential volume element in every direction over the sphere of equal velocities (a sphere by the isotropic condition) by $4\pi u^2\dd{u}=\dd{u_x}\dd{u_y}\dd{u_z}$.
        \item Thus, the Maxwell-Boltzmann distribution in terms of speed is
        \begin{equation*}
            F(u)\dd{u} = 4\pi\left( \frac{m}{2\pi k_BT} \right)^{3/2}u^2\e[-mu^2/2k_BT]\dd{u}
        \end{equation*}
    \end{itemize}
    \item \textbf{Maxwell-Boltzmann distribution}: The distribution of molecular speeds.
    \begin{figure}[H]
        \centering
        \begin{tikzpicture}
            \draw [stealth-stealth] (0,2.5) -- (0,0) -- (4,0);
            \draw [rex,thick] plot[domain=0:3.9,smooth] (\x,{2*\x*\x*e^(-\x*\x/2)});
        \end{tikzpicture}
        \caption{The Maxwell-Boltzmann distribution.}
        \label{fig:MBdist}
    \end{figure}
    \item \textbf{Velocity space}: A rectangular coordinate system in which the distances along the axes are $u_x,u_y,u_z$.
    \item We may use the above result to calculate that
    \begin{equation*}
        \prb{u} = \sqrt{\frac{8RT}{\pi m}}
    \end{equation*}
    which only differs from $u_\text{rms}$ by a factor of 0.92.
    \item \textbf{Most probable speed}: The most probable speed of a gas molecule in a sample that obeys the Maxwell-Boltzmann distribution. \emph{Denoted by} $\bm{u_\text{mp}}$. \emph{Given by}
    \begin{equation*}
        u_\text{mp} = \sqrt{\frac{2RT}{M}}
    \end{equation*}
    \begin{itemize}
        \item Derived by setting $\dv*{F}{u}=0$.
    \end{itemize}
    \item We may also express the Maxwell-Boltzmann distribution in terms of energy via $u=\sqrt{2\varepsilon/m}$ and $\dd{u}=\dd{\varepsilon}/\sqrt{2m\varepsilon}$ to give
    \begin{equation*}
        F(\varepsilon)\dd{\varepsilon} = \frac{2\pi}{(\pi k_BT)^{3/2}}\sqrt{\varepsilon}\e[-\varepsilon/k_BT]\dd{\varepsilon}
    \end{equation*}
    \item We can also confirm our previously calculated values for $\prb{u^2}$ and $\prb{\varepsilon}$.
    \item \textcite{bib:McQuarrieSimon} does a higher-level derivation of the ideal gas law that is rather analogous to the one done in class (i.e., via its flux perspective).
    \item \textcite{bib:McQuarrieSimon} discusses a simple and Nobel-prize winning experiment that verified the Maxwell-Boltzmann distribution.
\end{itemize}




\end{document}