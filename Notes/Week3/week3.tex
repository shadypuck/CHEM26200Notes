\documentclass[../notes.tex]{subfiles}

\pagestyle{main}
\renewcommand{\chaptermark}[1]{\markboth{\chaptername\ \thechapter\ (#1)}{}}
\setcounter{chapter}{2}

\begin{document}




\chapter{Kinetic Theory of Gases / The First Law of Thermodynamics}
\section{Maxwell-Boltzmann Distribution}
\begin{itemize}
    \item \marginnote{1/24:}Applying the molecular partition function to the heat capacity of a water molecule.
    \begin{itemize}
        \item A water molecule has three vibrational modes, which we will denote by $\nu_1,\nu_2,\nu_3$ (corresponding to symmetric stretch, antisymmetric stretch, and bend).
        \item Main takeaway: Heat capacity can change with temperature.
        \item After a while (at several thousand kelvin), it will level off (see Figure 18.7).
    \end{itemize}
    \item Considers \ce{CO2}'s vibrational modes, too.
    \begin{itemize}
        \item The infrared absorption of the bending mode is what's associated with the Greenhouse Effect.
        \item The symmetric stretch is IR inactive due to its lack of change of dipole moment.
        \item Raman active: Change in the polarizability of the molecule.
    \end{itemize}
    \item The Maxwell-Boltzmann distribution.
    \begin{itemize}
        \item Maxwell derived it long before Boltzmann, but Boltzmann's thermodynamic derivation is much easier.
        \item We know from the Boltzmann factor that $p(E)\propto\e[-E/k_BT]$.
        \item Thus, to get the probability $p(v)$ of some speed $v$, we should have $p(v)\propto\e[-mv^2/2k_BT]$ times a constant giving the number of molecules of each speed? This yields
        \begin{equation*}
            p(v) = A4\pi v^2\e[-mv^2/2k_BT]
        \end{equation*}
        where $A$ is a normalization constant.
        \item The Maxwell-Boltzmann distribution is such that
        \begingroup
        \allowdisplaybreaks
        \begin{align*}
            1 &= \int_0^\infty p(v)\dd{v}\\
            &= A\int_0^\infty 4\pi v^2\e[-mv^2/2k_BT]\dd{v}\\
            &= A\int_0^\infty 4\pi\left( \frac{2k_BT}{m} \right)^{3/2}u^2\e[-u^2]\dd{u}\\
            &= A4\pi\left( \frac{2k_BT}{m} \right)^{3/2}\int_0^\infty u^2\e[-u^2]\dd{u}\\
            &= A4\pi\left( \frac{2k_BT}{m} \right)^{3/2}\frac{\sqrt{\pi}}{4}\\
            A &= \left( \frac{m}{2\pi k_BT} \right)^{3/2}
        \end{align*}
        \endgroup
        \item Therefore,
        \begin{equation*}
            p(v) = 4\pi\left( \frac{m}{2\pi k_BT} \right)^{3/2}v^2\e[-mv^2/2k_BT]
        \end{equation*}
        \item Any distribution that doesn't look like this isn't in thermal equilibrium.
    \end{itemize}
    \item A system with all particles having $v=0$ is at thermal equilibrium with $T=\SI{0}{\kelvin}$.
    \item A system with all particles having constant velocity in the same direction is at thermal equilibrium with $T=\SI{0}{\kelvin}$.
    \begin{itemize}
        \item Think relativity; if you're moving with them, it looks like they're not moving and thus this case is the same as the last one because you're movement doesn't affect the thermodynamics of that system.
    \end{itemize}
    \item A system with all particles having constant velocity in different directions is not at thermal equilibrium since it does not fit the bell curve but is rather a spike.
\end{itemize}



\section{The First Law of Thermodynamics}
\begin{itemize}
    \item \marginnote{1/26:}See \textcite{bib:PHYS13300Notes} for background on/content of today's lecture.
    \item Joule best quantified how we think about work and energy.
    \item \textbf{System}: Part of the world being investigated. It can contain energy, a number of particles, etc.
    \item The Newtonian way to change the energy into a system is to do work (mechanical, electrical, etc.) on the system. In chemistry, $\var{w}$ is positive if work is done on the system.
    \begin{itemize}
        \item We have that
        \begin{equation*}
            \var{w} = -P\dd{V}
        \end{equation*}
    \end{itemize}
    \item A system at thermal equilibrium has a given temperature characteristic of the system. Some property of the system indicates how hot or cold it is (e.g., volume of mercury, etc.)
    \item Measure heat transfer using a calorimeter and a thermomenter.
    \begin{itemize}
        \item Convention: Heat put into a system $\var{q}$ is positive and this heat is transferred if the temperature is lower than another system or surroundings.
        \item The heat capacity of a system times the change in temperature is equal to the heat put into the system. It is always positive as heat put into the system raises the temperature.
    \end{itemize}
    \item Example molar heat capacities.
    \begin{itemize}
        \item For water vapor at low pressure and $\SI{20}{\celsius}$, $\overline{C}_V=3R=\SI{25}{\joule\per\mole\per\kelvin}$.
        \item For liquid water, it's higher (hydrogen bonding).
        \item For ice, it's lower.
    \end{itemize}
    \item \textbf{First law of thermodynamics}: The internal energy of a system changes with heat put into the system and work done on the system.
    \begin{equation*}
        \dd{U} = \var{q}+\var{w}
    \end{equation*}
    \begin{itemize}
        \item Note that in engineering, $\dd{U}=\var{q}-\var{w}$.
    \end{itemize}
    \item \textbf{State variable}: A property that describes the system.
    \begin{itemize}
        \item For example, a system of gas molecules has a state defined by the state variables $T$, $P$, $V$, and $n$.
    \end{itemize}
    \item \textbf{State function}: A property that depends only upon the state of the system.
    \begin{itemize}
        \item For example, some equations of state for an ideal gas are $PV=nRT$ or $PV=2U/3$.
        \item The internal energy is a state function.
        \item Heat and work are not state functions because they do not depend uniquely on the values at equilibrium.
        \begin{itemize}
            \item They also depend on the way you do something.
        \end{itemize}
    \end{itemize}
    \item \textbf{Reversible process}: A process that can be represented as a path along state variables, e.g., a line on a $PV$ diagram. This implies that it is also a path where all state variables are known, and is therefore a path where the system is always in quasi-equilibrium.
    \begin{itemize}
        \item Isothermal, isochoric, isobaric, and adiabatic changes are reversible.
        \item All of these processes are analyzed exactly as in \textcite{bib:PHYS13300Notes}.
    \end{itemize}
    \item \textbf{Irreversible process}: A process that cannot be drawn on a PV diagram.
    \item Experiment to measure $\gamma$ (the ratio of specific heats):
    \begin{enumerate}
        \item Let sit at $P_0T_0$.
        \item Pump in a little gas (add $\Delta n$) and let sit, measure $P_0+\Delta P_1,T_0$.
        \item Open the valve to air quickly to $P_0$. Adiabatic expansion (cools down).
        \item Let sit to measure the new pressure $P+\Delta P_2$ when $T$ is back at $T_0$.
        \item $\gamma$ is determined from $\Delta P_1,\Delta P_2$ (this will be a homework problem).
    \end{enumerate}
    \begin{itemize}
        \item In the second step, we add some molecules into the container. We can show that $\Delta P_1/P_0=\Delta N_1/n_0$.
        \item In the third step, we let out the air, and we can show that $\Delta n_2/n_0=\gamma\Delta P_1/P_0$.
        \item In step 4, we have in the container $(n_0+\Delta n_1-\Delta n_2)RT_0=(P_0+\Delta P_2)V_0$.
        \item This implies that $\Delta P_2/\Delta P_1=1-1/\gamma$.
    \end{itemize}
\end{itemize}



\section{Enthalpy}
\begin{itemize}
    \item \marginnote{1/28:}Thermonamic derivation of the formula for $\prb{P}$ in terms of $Q$.
    \begin{itemize}
        \item We have that
        \begin{align*}
            U &= \sum p_jE_j\\
            \dd{U} &= \sum(\dd{p_j}E_j+p_j\dd{E_j})\\
            &= \underbrace{\sum\dd{p_j}E_j}_{\var{q}}+\underbrace{\sum p_j\pdv{E_j}{V}}_{-P}\dd{V}
        \end{align*}
        where the last part follows by analogy with $\dd{U}=\var{q}-P\dd{V}$.
        \item It follows that
        \begin{equation*}
            P = -\sum p_j\pdv{E_j}{V} = -\prb{\pdv{E}{V}}
        \end{equation*}
        \item Thus, we have that
        \begin{align*}
            P &= -\sum\frac{\e[-E_j/k_BT]}{Q}\pdv{E_j}{V}\\
            &= \frac{1}{Q}\sum k_BT\cdot-\frac{1}{k_BT}\e[-E_j/k_BT]\pdv{E_j}{V}\\
            &= k_BT\frac{1}{Q}\sum\pdv{E_j}(\e[-E_j/k_BT])\pdv{E_j}{V}\\
            &= k_BT\frac{1}{Q}\sum\pdv{V}(\e[-E_j/k_BT])\\
            &= k_BT\frac{1}{Q}\pdv{Q}{V}\\
            P &= k_BT\pdv{\ln Q}{V}
        \end{align*}
    \end{itemize}
    \item Applies the formula to an ideal gas of independent, indistinguishable particles to derive the ideal gas law.
    \item \textbf{Enthalpy}: A state function representign the heat put into the system at constant pressure. \emph{Denoted by} $\bm{H}$. \emph{Given by}
    \begin{equation*}
        H = U+PV
    \end{equation*}
    \item We have that
    \begin{align*}
        \dd{H} &= \dd{U}+P\dd{V}+V\dd{P}\\
        &= \var{q}-P\dd{V}+P\dd{V}+V\dd{P}\\
        &= \var{q}+V\dd{P}
    \end{align*}
    \begin{itemize}
        \item At constant pressure ($\dd{P}=0$), we have that $\dd{H}=\var{q}$.
        \item At constant volume, we have that $\dd{H}=\var{q}$ as well?
    \end{itemize}
    \item \textbf{Constant-volume heat capacity}. The following expression. \emph{Denoted by} $\bm{C_V}$. \emph{Given by}
    \begin{equation*}
        C_V = \left( \pdv{U}{T} \right)_{V,N}
    \end{equation*}
    \item \textbf{Constant-pressure heat capacity}. The following expression. \emph{Denoted by} $\bm{C_P}$. \emph{Given by}
    \begin{equation*}
        C_P = \left( \pdv{H}{T} \right)_{P,N}
    \end{equation*}
    \item For an ideal gas,
    \begin{align*}
        \dd{H} &= \dd{U}+\dd{(PV)}\\
        &= nC_V\dd{T}+nR\dd{T}\\
        &= n(C_V+R)\dd{T}
    \end{align*}
    \begin{itemize}
        \item Recall this result from \textcite{bib:PHYS13300Notes}.
    \end{itemize}
    \item Considers heat diagrams.
    \begin{itemize}
        \item Recall the enthalpy of phase changes $\Delta H_\text{fus}$, $\Delta H_\text{vap}$, and $\Delta H_\text{sub}$.
        \item It follows that
        \begin{equation*}
            H(T)-H(T_0) = \int_{T^0}^TC_p\dd{T}+\sum\Delta H_\text{phase changes}
        \end{equation*}
    \end{itemize}
    \item \textbf{Hess's Law}: $\Delta H=0$ around a closed loop.
    \begin{itemize}
        \item This is because $H$ is a state function.
    \end{itemize}
    \item \textbf{Standard enthalpy of formation}. \emph{Denoted by} $\bm{\Delta H_f^\circ}$. \emph{Units} $\textbf{kJ}\bm{/}\textbf{mol}$.
    \begin{itemize}
        \item Calculated from the constituent elements in their standard state, \SI{1}{\bar}, \SI{298.15}{\kelvin}.
    \end{itemize}
    \item We have, for example, that the $\Delta H_\text{vap}^\circ$ of a substance is the difference of its $\Delta H_f^\circ$ in its gaseous state and its $\Delta H_f^\circ$ in its liquid state.
    \item With the standard enthalpy of formation and the heat capacity $C_P(T)$, one gets the enthalpy of formation at nonstandard temperatures.
    \item To get the enthalpy of formation at non-standard pressures of chemical interest, most of the effect is from the gas components because solids and liquid enthalpy vary little with pressure.
    \item The direction of change is sometimes in the direction of \emph{positive} enthalpy change.
    \begin{itemize}
        \item This change is driven by the fact that in these cases, the direction of change is toward the most probable state.
    \end{itemize}
    \item In a reversible process, $\dd{U}=\var{q}_\text{rev}-P\dd{V}$. In this case
    \begin{equation*}
        \var{q}_\text{rev} = \dd{U}+P\dd{V} = nC_V\dd{T}+P\dd{V} \neq \dd{nC_VT+PV}
    \end{equation*}
    so $\var{q}_\text{rev}$ is not a state function.
    \begin{itemize}
        \item However,
        \begin{align*}
            \frac{\var{q}_\text{rev}}{T} &= nC_V\frac{\dd{T}}{T}+\frac{P\dd{V}}{T}\\
            &= nC_V\frac{\dd{T}}{T}+nR\frac{\dd{V}}{V}\\
            &= \dd{(nC_V\ln T+nR\ln V)}
        \end{align*}
        is a state function.
    \end{itemize}
\end{itemize}



\section{Chapter 25: The Kinetic Theory of Gases}
\emph{From \textcite{bib:McQuarrieSimon}.}
\begin{itemize}
    \item \marginnote{1/30:}\textbf{Kinetic theory of gases}: A simple model of gases in which the molecules (pictured as hard spheres) are assumed to be in constant, incessant motion, colliding with each other and with the walls of the container.
    \item \textcite{bib:McQuarrieSimon} does the KMT derivation of the ideal gas law from \textcite{bib:APChemNotes}. Some important notes follow.
    \begin{itemize}
        \item \textcite{bib:McQuarrieSimon} emphasizes the importance of
        \begin{equation*}
            PV = \frac{1}{3}Nm\prb{u^2}
        \end{equation*}
        as a fundamental equation of KMT, as it relates a macroscopic property $PV$ to a microscopic property $m\prb{u^2}$.
        \item In Chapter 17-18, we derived quantum mechanically, and then from the partition function, that the average translational energy $\prb{E_\text{trans}}$ for a single particle of an ideal gas is $\frac{3}{2}k_BT$. From classical mechanics, we also have that $\prb{E_\text{trans}}=\frac{1}{2}m\prb{u^2}$. \emph{This} is why we may let
        \begin{equation*}
            \frac{1}{2}m\prb{u^2} = \frac{3}{2}k_BT
        \end{equation*}
        recovering that the average translational kinetic energy of the molecules in a gas is directly proportional to the Kelvin temperature.
    \end{itemize}
    \item \textbf{Isotropic} (entity): An object or substance that has the same properties in any direction.
    \begin{itemize}
        \item For example, a homogeneous gas is isotropic, and this is what allows us to state that $\prb{u_x^2}=\prb{u_y^2}=\prb{u_z^2}$.
    \end{itemize}
    \item \textcite{bib:McQuarrieSimon} derives
    \begin{equation*}
        u_\text{rms} = \sqrt{\frac{3RT}{M}}
    \end{equation*}
    \begin{itemize}
        \item $u_\text{rms}$ is an estimate of the average speed since $\prb{u^2}\neq\prb{u}^2$ in general.
    \end{itemize}
    \item \textcite{bib:McQuarrieSimon} states without proof that the speed of sound $u_\text{sound}$ in a monatomic ideal gas is given by
    \begin{equation*}
        u_\text{sound} = \sqrt{\frac{5RT}{3M}}
    \end{equation*}
    \item Assumptions of the kinetic theory of gases.
    \begin{itemize}
        \item Particles collide elastically with the wall.
        \begin{itemize}
            \item Justified because although each collision will not be elastic (the particles in the wall are moving too), the average collision will be elastic.
        \end{itemize}
        \item Particles do not collide with each other.
        \begin{itemize}
            \item Justified because "if the gas is in equilibrium, on the average, any collision that deflects the path of a molecule\dots will be balanced by a collision that replaces the molecule" \parencite[1015]{bib:McQuarrieSimon}.
        \end{itemize}
    \end{itemize}
    \item Note that we can do the kinetic derivation at many levels of rigor, but more rigorous derivations offer results that differ only by constant factors on the order of unity.
    \item Deriving a theoretical equation for the distribution of the \emph{components} of molecular velocities.
    \begin{itemize}
        \item Let $h(u_x,u_y,u_z)\dd{u_x}\dd{u_y}\dd{u_z}$ be the fraction of molecules with velocity components between $u_j$ and $u_j+\dd{u_j}$ for $j=x,y,z$.
        \item Assume that the each component of the velocity of a molecule is independent of the values of the other two components\footnote{This can be proven.}. It follows statistically that
        \begin{equation*}
            h(u_x,u_y,u_z) = f(u_x)f(u_y)f(u_z)
        \end{equation*}
        \begin{itemize}
            \item Note that we use just one function $f$ for the probability distribution in each direction because the gas is isotropic.
        \end{itemize}
        \item We can use the isotropic condition to an even greater degree. Indeed, it implies that any information conveyed by $u_x$ is necessarily and sufficiently conveyed by $u_y$, $u_z$, and $u$. Thus, we may take
        \begin{equation*}
            h(u) = h(u_x,u_y,u_z) = f(u_x)f(u_y)f(u_z)
        \end{equation*}
        \item It follows that
        \begin{equation*}
            \pdv{\ln h(u)}{u_x} = \pdv{u_x}(\ln f(u_x)+\text{terms not involving }u_x)
            = \dv{\ln f(u_x)}{u_x}
        \end{equation*}
        \item Since
        \begin{align*}
            u^2 &= u_x^2+u_y^2+u_z^2\\
            \pdv{u_x}(u^2) &= \pdv{u_x}(u_x^2+u_y^2+u_z^2)\\
            2u\pdv{u}{u_x} &= 2u_x\\
            \pdv{u}{u_x} &= \frac{u_x}{u}
        \end{align*}
        we have that
        \begin{align*}
            \pdv{\ln h}{u_x} &= \dv{\ln h}{u}\pdv{u}{u_x} = \frac{u_x}{u}\dv{\ln h}{u}\\
            \frac{\dd{\ln h(u)}}{u\dd{u}} &= \frac{\dd{\ln f(u_x)}}{u_x\dd{u_x}}
        \end{align*}
        which generalizes to
        \begin{equation*}
            \frac{\dd{\ln h(u)}}{u\dd{u}} = \frac{\dd{\ln f(u_x)}}{u_x\dd{u_x}}
            = \frac{\dd{\ln f(u_y)}}{u_y\dd{u_y}}
            = \frac{\dd{\ln f(u_z)}}{u_z\dd{u_z}}
        \end{equation*}
        \item Since $u_x,u_y,u_z$ are independent, we know that the above equation is equal to a constant, which we may call $-\gamma$. It follows that for any $j=x,y,z$, we have that
        \begin{align*}
            \frac{\dd{\ln f(u_j)}}{u_j\dd{u_j}} &= -\gamma\\
            \frac{1}{f}\dv{f}{u_j} &= -\gamma u_j\\
            \int\frac{\dd{f}}{f} &= \int-\gamma u_j\dd{u_j}\\
            \ln f &= -\frac{\gamma}{2}u_j^2+C\\
            f(u_j) &= A\e[-\gamma u_j^2]
        \end{align*}
        where we have incorporated the $1/2$ into $\gamma$.
        \item To determine $A$ and $\gamma$, we let arbitrarily let $j=x$. Since $f$ is a continuous probability distribution, we may apply the normalization requirement.
        \begin{align*}
            1 &= \int_{-\infty}^\infty f(u_x)\dd{u_x}\\
            &= 2A\int_0^\infty\e[-\gamma u_x^2]\dd{u_x}\\
            &= 2A\sqrt{\frac{\pi}{4\gamma}}\\
            A &= \sqrt{\frac{\gamma}{\pi}}
        \end{align*}
        \item Additionally, since we have that $\prb{u_x^2}=\frac{1}{3}\prb{u^2}$ and $\prb{u^2}=3RT/M$, we know that $\prb{u_x^2}=RT/M$. This combined with the definition of $\prb{u_x^2}$ as a continuous probability distribution yields
        \begin{align*}
            \frac{RT}{M} &= \prb{u_x^2}\\
            &= \int_{-\infty}^\infty u_x^2f(u_x)\dd{u_x}\\
            &= 2\sqrt{\frac{\gamma}{\pi}}\int_0^\infty u_x^2\e[-\gamma u_x^2]\dd{u_x}\\
            &= 2\sqrt{\frac{\gamma}{\pi}}\cdot\frac{1}{4\gamma}\sqrt{\frac{\pi}{\gamma}}\\
            &= \frac{1}{2\gamma}\\
            \gamma &= \frac{M}{2RT}
        \end{align*}
        \item Therefore,
        \begin{equation*}
            f(u_x) = \sqrt{\frac{M}{2\pi RT}}\e[-Mu_x^2/2RT]
        \end{equation*}
        \item It is common to rewrite the above in terms of molecular quantities $m$ and $k_B$.
    \end{itemize}
    \item It follows that as temperature increases, more molecules are likely to be found with higher component velocity values.
    \item We can use the above result to show that
    \begin{equation*}
        \prb{u_x} = \int_{-\infty}^\infty u_xf(u_x)\dd{u_x} = 0
    \end{equation*}
    \item We can also calculate that $\prb{u_x^2}=RT/M$ and $m\prb{u_x}^2/2=k_BT/2$ from the above result\footnote{See the equipartition of energy theorem from \textcite{bib:PHYS13300Notes}.}.
    \begin{itemize}
        \item An important consequence is that the total kinetic energy is divided equally into the $x$-, $y$-, and $z$-components.
    \end{itemize}
    \item \textcite{bib:McQuarrieSimon} discusses how the Doppler effect applied to moving molecules emitting radiation makes spectral peaks wider than we'd normally predict in a phenomenon known as \textbf{Doppler broadening}.
    \item Deriving \textbf{Maxwell-Boltzmann distribution}.
    \begin{itemize}
        \item Let the probability that a molecule has speed between $u$ and $u+\dd{u}$ be defined by a continuous probability distribution $F(u)\dd{u}$. In particular, we have from the above isotropic condition that
        \begin{align*}
            F(u)\dd{u} &= f(u_x)\dd{u_x}f(u_y)\dd{u_y}f(u_z)\dd{u_z}\\
            &= \left( \frac{m}{2\pi k_BT} \right)^{3/2}\e[-m(u_x^2+u_y^2+u_z^2)/2k_BT]\dd{u_x}\dd{u_y}\dd{u_z}
        \end{align*}
        \item Considering $F$ over a \textbf{velocity space}, we realize that we may express the probability distribution $F$ as a function of $u$ via $u^2=u_x^2+u_y^2+u_z^2$ and the differential volume element in every direction over the sphere of equal velocities (a sphere by the isotropic condition) by $4\pi u^2\dd{u}=\dd{u_x}\dd{u_y}\dd{u_z}$.
        \item Thus, the Maxwell-Boltzmann distribution in terms of speed is
        \begin{equation*}
            F(u)\dd{u} = 4\pi\left( \frac{m}{2\pi k_BT} \right)^{3/2}u^2\e[-mu^2/2k_BT]\dd{u}
        \end{equation*}
    \end{itemize}
    \item \textbf{Maxwell-Boltzmann distribution}: The distribution of molecular speeds.
    \begin{figure}[H]
        \centering
        \begin{tikzpicture}
            \draw [stealth-stealth] (0,2.5) -- (0,0) -- (4,0);
            \draw [rex,thick] plot[domain=0:3.9,smooth] (\x,{2*\x*\x*e^(-\x*\x/2)});
        \end{tikzpicture}
        \caption{The Maxwell-Boltzmann distribution.}
        \label{fig:MBdist}
    \end{figure}
    \item \textbf{Velocity space}: A rectangular coordinate system in which the distances along the axes are $u_x,u_y,u_z$.
    \item We may use the above result to calculate that
    \begin{equation*}
        \prb{u} = \sqrt{\frac{8RT}{\pi m}}
    \end{equation*}
    which only differs from $u_\text{rms}$ by a factor of 0.92.
    \item \textbf{Most probable speed}: The most probable speed of a gas molecule in a sample that obeys the Maxwell-Boltzmann distribution. \emph{Denoted by} $\bm{u_\text{mp}}$. \emph{Given by}
    \begin{equation*}
        u_\text{mp} = \sqrt{\frac{2RT}{M}}
    \end{equation*}
    \begin{itemize}
        \item Derived by setting $\dv*{F}{u}=0$.
    \end{itemize}
    \item We may also express the Maxwell-Boltzmann distribution in terms of energy via $u=\sqrt{2\varepsilon/m}$ and $\dd{u}=\dd{\varepsilon}/\sqrt{2m\varepsilon}$ to give
    \begin{equation*}
        F(\varepsilon)\dd{\varepsilon} = \frac{2\pi}{(\pi k_BT)^{3/2}}\sqrt{\varepsilon}\e[-\varepsilon/k_BT]\dd{\varepsilon}
    \end{equation*}
    \item We can also confirm our previously calculated values for $\prb{u^2}$ and $\prb{\varepsilon}$.
    \item \textcite{bib:McQuarrieSimon} does a higher-level derivation of the ideal gas law that is rather analogous to the one done in class (i.e., via its flux perspective).
    \item \textcite{bib:McQuarrieSimon} discusses a simple and Nobel-prize winning experiment that verified the Maxwell-Boltzmann distribution.
\end{itemize}



\section{Chapter 19: The First Law of Thermodynamics}
\emph{From \textcite{bib:McQuarrieSimon}.}
\begin{itemize}
    \item \marginnote{1/31:}\textbf{Thermodynamics}: The study of various properties and, particularly, the relations between the various properties of systems in equilibrium.
    \begin{itemize}
        \item Primarily an experimental science that is still of great practical value to the fields of today.
        \item "All the results of thermodynamics are based on three fundamental laws. These laws summarize an enormous body of experimental data, and there are absolutely no known exceptions" \parencite[765]{bib:McQuarrieSimon}.
    \end{itemize}
    \item \textbf{Classical thermodynamics}: The development of thermodynamics before the atomic theory of matter.
    \begin{itemize}
        \item Since thermodynamics was not developed in concert with the atomic theory, we can rest assured that its results will not need to be modified. However, it provides limited insight into what is going on at the molecular level.
    \end{itemize}
    \item \textbf{Statistical thermodynamics}: The molecular interpretation of thermodynamics developed since the atomic theory of matter became generally accepted.
    \begin{itemize}
        \item Chapters 17-18 are an elementary treatment of statistical thermodynamics.
        \item Since atomic structure is still being determined, these results are not on as solid of a footing as classical thermodynamics.
    \end{itemize}
    \item \textbf{First Law of Thermodynamics}: The law of conservation of energy applied to a macroscopic system.
    \item \textbf{System}: The part of the world we are investigating.
    \item \textbf{Surroundings}: Everything else.
    \item \textbf{Heat}: The manner of energy transfer that results from a temperature difference between the system and its surroundings. \emph{Denoted by} $\bm{q}$.
    \begin{itemize}
        \item Sign convention: Heat input into a system is positive; heat evolved by a system is negative.
    \end{itemize}
    \item \textbf{Work}: The transfer of energy between the system of interest and its surroundings as a result of the existence of unbalanced forces between the two. \emph{Denoted by} $\bm{w}$.
    \begin{itemize}
        \item Sign convention: Work done \emph{on} the system (i.e., that increases the energy of the system) is positive; work done \emph{by} the system (i.e., that increases the energy of the surroundings) is negative.
    \end{itemize}
    \item Work can be related to raising a mass.
    \begin{itemize}
        \item If a pressurized gas is capped by a piston with a mass $m$ on top and then it pushes the piston upwards a distance $h$, it does $w=-mgh$ of work.
        \item Knowing that the external pressure $P_\text{ext}=F/A=mg/A$ and $Ah=\Delta V$, we recover
        \begin{equation*}
            w = -P_\text{ext}\Delta V
        \end{equation*}
        \item If $P_\text{ext}$ is not constant,
        \begin{equation*}
            w = -\int_{V_i}^{V_f}P_\text{ext}\dd{V}
        \end{equation*}
    \end{itemize}
    \item \textbf{Definite state}: A state of a system in which all of the variables needed to describe the system completely are defined.
    \begin{itemize}
        \item For example, the state of one mole of an ideal gas can be described completely via $P$, $V$, and $T$. In fact, we need only specify two of these since $PV=RT$ for one mole of gas (in particular, specifying any two specifies the third).
    \end{itemize}
    \item \textbf{State function}: A property that depends only upon the state of the system, not upon how the system was brought to that state.
    \begin{itemize}
        \item State functions can be integrated in a normal way, i.e., $\Delta U=U_2-U_1=\int_1^2\dd{U}$. In particular, we need not worry about the \emph{path} from state 1 to state 2, only that we got from $U_1$ to $U_2$.
        \item Energy is a state function, but work and heat are not state functions.
    \end{itemize}
    \item \textbf{Reversible process}: An expansion or compression in which $P_\text{ext}$ and $P$ differ only infinitesimally.
    \begin{itemize}
        \item Technically, a reversible process would take infinite time, but it serves as a useful idealized limit regardless.
    \end{itemize}
    \item To calculate $w_\text{rev}$ (reversible work) for the compression of an ideal gas isothermally, we may replace $P_\text{ext}$ by the pressure of the gas $P$ to obtain
    \begin{equation*}
        w_\text{rev} = -\int_{V_1}^{V_2}P\dd{V}
        = -\int_{V_1}^{V_2}\frac{nRT}{V}\dd{V}
        = -nRT\ln\frac{V_2}{V_1}
    \end{equation*}
    \item Isothermal compression/expansion of a gas.
    \begin{figure}[h!]
        \centering
        \begin{subfigure}[b]{0.3\linewidth}
            \centering
            \begin{tikzpicture}
                \footnotesize
                \draw [stealth-stealth] (0,2) node[above]{$P$} -- (0,0) -- (3,0) node[right]{$V$};
                \draw
                    (1,0.1) -- ++(0,-0.2) node[below]{$V_1$}
                    (2,0.1) -- ++(0,-0.2) node[below]{$V_2$}
                    (0.1,1) -- ++(-0.2,0) node[left]{$P_\text{ext}$}
                    (0.1,0.5) -- ++(-0.2,0) node[left]{$P_i$}
                ;
    
                \fill [rex,opacity=0.3] (1,0) rectangle (2,1);
                \draw [orx,thick] plot[domain=0.6:2.5,smooth] (\x,{1/\x}) node[right,black]{$T$};
            \end{tikzpicture}
            \caption{Isothermal compression.}
            \label{fig:isothermalManipulationa}
        \end{subfigure}
        \begin{subfigure}[b]{0.3\linewidth}
            \centering
            \begin{tikzpicture}
                \footnotesize
                \draw [stealth-stealth] (0,2) node[above]{$P$} -- (0,0) -- (3,0) node[right]{$V$};
                \draw
                    (1,0.1)   -- ++(0,-0.2) node[below]{$V_1$}
                    (2,0.1)   -- ++(0,-0.2) node[below]{$V_2$}
                    (0.1,0.5) -- ++(-0.2,0) node[left]{$P_\text{ext}$}
                    (0.1,1) -- ++(-0.2,0) node[left]{$P_i$}
                ;
    
                \fill [rex,opacity=0.3] (1,0) rectangle (2,0.5);
                \draw [orx,thick] plot[domain=0.6:2.5,smooth] (\x,{1/\x}) node[right,black]{$T$};
            \end{tikzpicture}
            \caption{Isothermal expansion.}
            \label{fig:isothermalManipulationb}
        \end{subfigure}
        \begin{subfigure}[b]{0.3\linewidth}
            \centering
            \begin{tikzpicture}
                \footnotesize
                \draw [stealth-stealth] (0,2) node[above]{$P$} -- (0,0) -- (3,0) node[right]{$V$};
                \draw
                    (1,0.1)   -- ++(0,-0.2) node[below]{$V_1$}
                    (2,0.1)   -- ++(0,-0.2) node[below]{$V_2$}
                ;
    
                \fill [rex,opacity=0.3] (1,0) -- plot[domain=1:2,smooth] (\x,1/\x) -- (2,0);
                \draw [orx,thick] plot[domain=0.6:2.5,smooth] (\x,{1/\x}) node[right,black]{$T$};
            \end{tikzpicture}
            \caption{Reversible process.}
            \label{fig:isothermalManipulationc}
        \end{subfigure}
        \caption{Isothermal manipulation of an ideal gas.}
        \label{fig:isothermalManipulation}
    \end{figure}
    \begin{itemize}
        \item Imagine a gas at pressure $P$, volume $V$, and temperature $T$ in a container with a moveable piston at the top.
        \item Isothermal compression (Figure \ref{fig:isothermalManipulationa}).
        \begin{itemize}
            \item Let the external pressure be held constant at $P_\text{ext}$. If the volume of the gas is initially at $V_2$ and temperature $T$, its pressure will be $P_i$. Thus, to equilibriate with the external pressure, it will compress to volume $V_1$ and final pressure $P_f=P_\text{ext}$ along the isotherm. As the force doing this work is the constant external pressure, the work done will be encapsulated by the red box in Figure \ref{fig:isothermalManipulationa}. Note that the gas necessarily releases an amount of heat equivalent to the work in the red box during the course of the compression to maintain isothermal conditions.
        \end{itemize}
        \item Isothermal expansion (Figure \ref{fig:isothermalManipulationa}).
        \begin{itemize}
            \item Let the external pressure be held constant at $P_\text{ext}$. If the volume of the gas is initially at $V_1$ and temperature $T$, its pressure will be $P_i$. Thus, to equilibriate with the external pressure, it will expand to volume $V_1$ and final pressure $P_f=P_\text{ext}$ along the isotherm. As the force doing this work is the constant external pressure, the work done will be encapsulated by the red box in Figure \ref{fig:isothermalManipulationa}. Note that the gas necessarily absorbs an amount of heat equivalent to the work in the red box during the course of the compression to maintain isothermal conditions.
        \end{itemize}
        \item Reversible compression/expansion (Figure \ref{fig:isothermalManipulationc}).
        \begin{itemize}
            \item In a very slow manner, incrementally increase (resp. decrease) $P_\text{ext}$ so as to allow the gas to reversibly compress (resp. expand).
            \item Compressing a gas reversibly and isothermally does the minimum amount of work on the gas. Expanding a gas reversibly and isothermally requires the gas to do the maximum amount of work.
        \end{itemize}
        \item This work done on the gas raises the internal energy of the system, right? So shouldn't that raise the temperature, making the process not isothermal?
        \item Where does the extra energy above and below the isotherm in the irreversible processes go? Is it converted to heat?
        \item Does this mean that if you used a \SI{1000}{\kilo\gram} weight to compress a gas to half its original volume vs. using a \SI{10}{\kilo\gram} weight to compress a gas to half its original volume, the gas would get 100 times hotter in the former case?
    \end{itemize}
    \item \textbf{Path function}: A function whose value depends on the path from state 1 to state 2, not just the initial and final states.
    \begin{itemize}
        \item State functions cannot be integrated in the normal way. Mathematically, they have \textbf{inexact differentials}, i.e., we write $\int_1^2\var{w}=w$. This is because it makes no sense to write $w_1$, $w_2$, $w_2-w_1$, or $\Delta w$, for example.
        \item Work and heat are path functions.
    \end{itemize}
    \item The First Law of Thermodynamics says that $\dd{U}=\var{q}+\var{w}$ (in differential form) and $\Delta U=q+w$ (in integrated form).
    \begin{itemize}
        \item An important consequence is that even though $\var{q}$ and $\var{w}$ are separately path functions/inexact differentials, their sum is a state function/exact differential.
    \end{itemize}
    \item \textbf{Adiabatic process}: A process in which no heat is transferred between the system and its surroundings.
    \item Work during an adiabatic process.
    \begin{itemize}
        \item For an adiabatic process, $\var{q}=0$.
        \item Thus, $w=\Delta U$.
        \item But since $\Delta U$ is entirely dependent on temperature, we have that
        \begin{equation*}
            w = \Delta U
            = \int_{T_1}^{T_2}\left( \pdv{U}{T} \right)_V\dd{T}
            = \int_{T_1}^{T_2}C_V(T)\dd{T}
        \end{equation*}
        \item Why isn't this an integral of $P$ with respect to $V$?
    \end{itemize}
    \item Temperature during an adiabatic process.
    \begin{itemize}
        \item We have that
        \begin{align*}
            \dd{U} &= \dd{w}\\
            C_V(T)\dd{T} &= -P\dd{V}\\
            n\overline{C}_V(T)\dd{T} &= -\frac{nRT}{V}\dd{V}\\
            \int_{T_1}^{T_2}\frac{\overline{C}_V(T)}{T}\dd{T} &= -R\int_{V_1}^{V_2}\frac{\dd{V}}{V}
        \end{align*}
        \item In the specific case of a monatomic ideal gas, $C_V(T)=3/2$. Thus, continuing, we have
        \begin{align*}
            \frac{3}{2}\ln\frac{T_2}{T_1} &= \ln\frac{V_1}{V_2}\\
            \left( \frac{T_2}{T_1} \right)^{3/2} &= \frac{V_1}{V_2}
        \end{align*}
        \item We can also express the above in terms of pressure
        \begin{align*}
            \left( \frac{P_2V_2/nR}{P_1V_2/nR} \right)^{3/2} &= \frac{V_1}{V_2}\\
            P_1V_1^{5/3} &= P_2V_2^{5/3}
        \end{align*}
        \item For a diatomic gas, we end up with
        \begin{equation*}
            P_1V_1^{7/5} = P_2V_2^{7/5}
        \end{equation*}
        \item Note that for an isothermal expansion, Boyle's law applies: $P_1V_1=P_2V_2$.
    \end{itemize}
    \item Relating work and heat to molecular properties.
    \begin{itemize}
        \item By comparing recently derived equations with previously derived equations, we have that
        \begin{align*}
            U &= \sum_jp_jE_j\\
            \dd{U} &= \sum_jp_j\dd{E_j}+\sum_jE_j\dd{p_j}\\
            &= \sum_jp_j\left( \pdv{E_j}{V} \right)_N\dd{V}+\sum_jE_j\dd{p_j}
        \end{align*}
        \item The above equation suggests that we can interpret the first term as the average change in energy of a system caused by a small change in its volume, i.e., the average work.
        \item It follows by the First Law of Thermodynamics that we can interpret the second term as the average heat.
        \item This expresses the important but subtle notion that work results from "an infinitesimal change in the allowed energies of a system, without changing the probability distribution of its states" while heat results from "a change in the probability distribution of the states of a system, without changing the allowed energies" \parencite[780]{bib:McQuarrieSimon}.
        \item In particular, if we take the process under study to be reversible, we have
        \begin{equation*}
            \dd{U} = \underbrace{\sum_jp_j\left( \pdv{E_j}{V} \right)_N\dd{V}}_{\var{w_\text{rev}}}+\underbrace{\sum_jE_j\dd{p_j}}_{\var{q_\text{rev}}}
            = \underbrace{\sum_jp_j\left( \pdv{E_j}{V} \right)_N}_{-P}\dd{V}+\sum_jE_j\dd{p_j}
        \end{equation*}
        \item The second equality above expresses the fact that
        \begin{equation*}
            P = -\prb{\pdv{E}{V}}
        \end{equation*}
        which we previously used in Chapter 17.
    \end{itemize}
    \item For a constant-volume process, $w=0$, so we know that the heat evolved in the process $q_V=\Delta U$.
    \item Defining a state function analogous to $U$ for constant-pressure processes.
    \begin{itemize}
        \item We have from the First Law that
        \begin{equation*}
            \Delta U = q+w = q-\int_{V_1}^{V_2}P\dd{V}
        \end{equation*}
        \item Thus, at constant pressure,
        \begin{equation*}
            q_P = \Delta U+P_\text{ext}\int_{V_1}^{V_2}\dd{V}
            = \Delta U+P\Delta V
        \end{equation*}
        \item The above equation suggests how to define our new state function.
    \end{itemize}
    \item \textbf{Enthalpy}: The state function describing the heat put into a system at constant pressure. \emph{Denoted by} $\bm{H}$. \emph{Given by}
    \begin{equation*}
        H = U+PV
    \end{equation*}
    \begin{itemize}
        \item $\Delta H$ can be determined experimentally as the heat associated with a constant-pressure process.
    \end{itemize}
    \item Examples.
    \begin{itemize}
        \item For the melting of ice, $\Delta V$ is small, so $\Delta U\approx\Delta H$.
        \item For the vaporization of water, $\Delta V$ is large, so $\Delta U<\Delta H$.
        \begin{itemize}
            \item We interpret the excess by the fact that most of the energy goes into raising the internal energy of the water (i.e., breaking the hydrogen bonding), but some of it must go into increasing the volume of the system against the atmospheric pressure.
        \end{itemize}
    \end{itemize}
    \item For reactions or processes that involve ideal gases,
    \begin{equation*}
        \Delta H = \Delta U+RT\Delta n_\text{gas}
    \end{equation*}
    where $\Delta n_\text{gas}$ is the difference in the number of moles of gaseous products vs. reactants.
    \item \textbf{Extensive quantity}: A quantity that depends on the amount of substance.
    \begin{itemize}
        \item Heat capacity is an extensive quantity.
    \end{itemize}
    \item Heat capacity is a path function, as it depends on whether we heat the substance at constant volume or constant pressure.
    \item We have
    \begin{align*}
        C_V &= \left( \pdv{U}{T} \right)_V&
        C_P &= \left( \pdv{H}{T} \right)_P
    \end{align*}
    \begin{itemize}
        \item We expect $C_P>C_V$ since we also have to work against atmospheric pressure.
        \item In fact, for a monatomic ideal gas,
        \begin{align*}
            H &= U+PV\\
            &= U+nRT\\
            \dv{H}{T} &= \dv{U}{T}+nR\\
            C_P-C_V &= nR\\
            C_P &= \frac{3}{2}R+nR\\
            &= \frac{5}{2}R
        \end{align*}
        \item It follows that the difference between $C_P$ and $C_V$ is significant for gases, but not for solids and liquids.
        \item Note that we can also prove a general expression for $C_P-C_V$ (see Chapter 22).
    \end{itemize}
    \item Relative enthalpies can be determined from heat capacity data and heats of transition.
    \begin{itemize}
        \item Integrate $C_P(T)$ from $T_1$ to $T_2$, adding in $\Delta_\text{fus}H$ and $\Delta_\text{vap}H$ as necessary.
    \end{itemize}
    \item \textbf{Thermochemistry}: The branch of thermodynamics which concerns the measurement of the evolution or absorption of energy as heat associated with chemical reactions.
    \item \textcite{bib:McQuarrieSimon} reviews exothermic/endothermic reactions, and $\Delta H=H_\text{prod}-H_\text{react}$.
    \item \textbf{Hess's Law}: The additivity property of $\Delta_rH$ values.
    \item \textbf{Standard reaction enthalpy}: The enthalpy change associated with one mole of a specified reagent when all reactants and products are in their standard states. \emph{Denoted by} $\bm{\Delta_rH^\circ}$.
    \begin{itemize}
        \item An intensive quantity.
    \end{itemize}
    \item \textbf{Standard molar enthalpy of formation}: The standard reaction enthalpy for the formation of one mole of a molecule from its constituent elements. \emph{Denoted by} $\bm{\Delta_fH^\circ}$.
    \begin{itemize}
        \item We can obtain such values even if a compound cannot be formed directly from its elements via several related reactions and Hess's Law.
    \end{itemize}
\end{itemize}




\end{document}