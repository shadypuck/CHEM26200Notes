\documentclass[../notes.tex]{subfiles}

\pagestyle{main}
\renewcommand{\chaptermark}[1]{\markboth{\chaptername\ \thechapter\ (#1)}{}}
\setcounter{chapter}{2}

\begin{document}




\chapter{Kinetic Theory of Gases}
\section{Maxwell-Boltzmann Distribution}
\begin{itemize}
    \item \marginnote{1/24:}Applying the molecular partition function to the heat capacity of a water molecule.
    \begin{itemize}
        \item A water molecule has three vibrational modes, which we will denote by $\nu_1,\nu_2,\nu_3$ (corresponding to symmetric stretch, antisymmetric stretch, and bend).
        \item Main takeaway: Heat capacity can change with temperature.
        \item After a while (at several thousand kelvin), it will level off (see Figure 18.7).
    \end{itemize}
    \item Considers \ce{CO2}'s vibrational modes, too.
    \begin{itemize}
        \item The infrared absorption of the bending mode is what's associated with the Greenhouse Effect.
        \item The symmetric stretch is IR inactive due to its lack of change of dipole moment.
        \item Raman active: Change in the polarizability of the molecule.
    \end{itemize}
    \item The Maxwell-Boltzmann distribution.
    \begin{itemize}
        \item Maxwell derived it long before Boltzmann, but Boltzmann's thermodynamic derivation is much easier.
        \item We know from the boltzmann factor that $p(E)\propto\e[-E/k_BT]$.
        \item Thus, to get the probability $p(v)$ of some speed $v$, we should have $p(v)\propto\e[-mv^2/2k_BT]$ times a constant giving the number of molecules of each speed? This yields
        \begin{equation*}
            p(v) = A4\pi v^2\e[-mv^2/2k_BT]
        \end{equation*}
        where $A$ is a normalization constant.
        \item The Maxwell-Boltzmann distribution is such that
        \begingroup
        \allowdisplaybreaks
        \begin{align*}
            1 &= \int_0^\infty p(v)\dd{v}\\
            &= A\int_0^\infty 4\pi v^2\e[-mv^2/2k_BT]\dd{v}\\
            &= A\int_0^\infty 4\pi\left( \frac{2k_BT}{m} \right)^{3/2}u^2\e[-u^2]\dd{u}\\
            &= A4\pi\left( \frac{2k_BT}{m} \right)^{3/2}\int_0^\infty u^2\e[-u^2]\dd{u}\\
            &= A4\pi\left( \frac{2k_BT}{m} \right)^{3/2}\frac{\sqrt{\pi}}{4}\\
            A &= \left( \frac{m}{2\pi k_BT} \right)^{3/2}
        \end{align*}
        \endgroup
        \item Therefore,
        \begin{equation*}
            p(v) = 4\pi\left( \frac{m}{2\pi k_BT} \right)^{3/2}v^2\e[-mv^2/2k_BT]
        \end{equation*}
        \item Any distribution that doesn't look like this isn't in thermal equilibrium.
    \end{itemize}
    \item A system with all particles having $v=0$ is at thermal equilibrium with $T=\SI{0}{\kelvin}$.
    \item A system with all particles having constant velocity in the same direction is at thermal equilibrium with $T=\SI{0}{\kelvin}$.
    \begin{itemize}
        \item Think relativity; if you're moving with them, it looks like they're not moving and thus this case is the same as the last one because you're movement doesn't affect the thermodynamics of that system.
    \end{itemize}
    \item A system with all particles having constant velocity in different directions is not at thermal equilibrium since it does not fit the bell curve but is rather a spike.
\end{itemize}




\end{document}