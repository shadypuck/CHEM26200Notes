\documentclass[../notes.tex]{subfiles}

\pagestyle{main}
\renewcommand{\chaptermark}[1]{\markboth{\chaptername\ \thechapter\ (#1)}{}}
\setcounter{chapter}{5}

\begin{document}




\chapter{Chemical Equilibrium}
\section{Thermodynamics of the Rubber Band}
\begin{itemize}
    \item \marginnote{2/14:}Midterm:
    \begin{itemize}
        \item Take for two hours, any two hours tomorrow.
        \item Upload a single PDF file for your answer.
        \item Do not discuss the questions with anybody.
    \end{itemize}
    \item \textbf{Standard Gibbs free energy} (at $T$): The following energy, where $H^\circ(T)$ is the standard enthalpy at $T$ and $S^\circ(T)$ is the standard entropy at $T$. \emph{Denoted by} $\bm{G^\circ(T)}$. \emph{Given by}
    \begin{equation*}
        G^\circ(T) = H^\circ(T)-TS^\circ(T)
    \end{equation*}
    \begin{itemize}
        \item It follows that since the enthalpy is taken at constant pressure and the entropy $S(T,P)$ at nonstandard pressure is given by $S(T,P)=S^\circ(T)+R\ln P/P_0$ that the Gibbs free energy $S(T,P)$ at nonstandard pressure is
        \begin{align*}
            G(T,P) &= H^\circ(T)-TS(T,P)\\
            &= H^\circ(T)-T\left( S^\circ(T)+R\ln\frac{P}{P_0} \right)\\
            &= G^\circ(T)-RT\ln\frac{P}{P_0}
        \end{align*}
    \end{itemize}
    \item Gibbs free energy and equilibrium at constant $T,P$.
    \begin{itemize}
        \item If
        \begin{equation*}
            \ce{$a$A + $b$B -> $c$C + $d$D}
        \end{equation*}
        is in equilibrium where $A,B,C,D$ are ideal gases, then $\Delta G=0$.
        \item This implies that
        \begin{equation*}
            a\Delta G_a+b\Delta G_b = c\Delta G_c+d\Delta G_d
        \end{equation*}
        which is the \textbf{law of mass action}.
    \end{itemize}
    \item Example of phase equilibrium ($G(T)$ for solid/liquid/gas phases).
    \begin{itemize}
        \item We have from the total differential of $G$ that
        \begin{equation*}
            \left( \pdv{G}{T} \right)_P = -S
        \end{equation*}
        \item Since $S\geq 0$ always, $G$ is monotonically decreasing.
        \item During the liquid phase, there is a relatively constant slight negative slope in the $G(T)$ graph.
        \item During the gas phase, $S$ is much bigger, so there is a larger negative slope in the $G(T)$ graph.
        \item Additionally, at the heats of vaporization and fusion, the system is in equilibrium (hence the energies are the same), so the graph is continuous.
    \end{itemize}
    \item Rubber band temperature analysis.
    \begin{itemize}
        \item A rubber band heats up when stretched:
        \begin{itemize}
            \item We have that $\dd{U}=T\dd{S}+f\dd{L}$.
            \item We want to show that $(\pdv*{U}{L})_T=T(\pdv*{S}{L})_T+f$.
            \begin{align*}
                \dd{U} &= T\left( \pdv{S}{T} \right)_L\dd{T}+\left( \pdv{S}{L} \right)_T\dd{L}+f\dd{L}\\
                &= T\left( \pdv{S}{T} \right)_L\dd{T}+\left[ \left( \pdv{S}{L} \right)_T+f \right]\dd{S}
            \end{align*}
            \item All that's left is to show that $-(\pdv*{S}{L})_T=(\pdv*{f}{T})_L$, which we can do using Maxwell relations.
            \begin{equation*}
                \dd{A} = -S\dd{T}+f\dd{L}
            \end{equation*}
            \begin{equation*}
                \left( \pdv{U}{L} \right)_T = -T\left( \pdv{f}{T} \right)_L+f
            \end{equation*}
        \end{itemize}
        \item Stating the equation of state for the "ideal" rubber band.
        \begin{equation*}
            f = T\phi(L)
        \end{equation*}
        \begin{itemize}
            \item It follows that $(\pdv*{f}{T})_L=f/T$, and $(\pdv*{U}{L})_T=-T\cdot f/T+f=0$.
        \end{itemize}
        \item We are now ready to answer the question of does it cool down or heat up when stretched adiabatically.
        \begin{align*}
            \dd{U} = \left( \pdv{U}{L} \right)_T\dd{L}+\left( \pdv{U}{T} \right)_L\dd{T} &= \var{q}+f\dd{L}\\
            \left( \pdv{U}{T} \right)_L\dd{T} &= f\dd{L}\\
            C_L\dd{T} &= f\dd{L}\\
            \dd{T} &= \frac{f}{C_L}\dd{L} > 0
        \end{align*}
        so since $\dd{T}>0$, the rubber band heats up as it stretches.
        \item For intuitive motivation, PGS discusses Figure 17.1 of \textcite{bib:APChemNotes}.
    \end{itemize}
    \item Building a statistical model of the rubber band.
    \begin{itemize}
        \item Consider the rubber band to be made up of segments (you can think of each segment as part of a polymer). These segments can be oriented up or down. In a stretched rubber band, the polymers will be straight, i.e., all the segments will point the same way.
        \item The difference in energy $\Delta E$ between a segment (of length $\ell_0$) pointing up or down is $2f$.
        \item Thus, the partition function for each segment is
        \begin{equation*}
            q = \e[-f\ell_0/k_BT]+\e[f\ell_0/k_BT]
        \end{equation*}
        \begin{itemize}
            \item Note that this is the same as the partition function for a paramagnet (which can also either be up or down) except that $f\ell_0$ becomes $\mu_BB_z$.
        \end{itemize}
        \item When we sum the energies, we multiply the component partition functions. Thus,
        \begin{equation*}
            Q = q^N
        \end{equation*}
        \item We know that
        \begin{equation*}
            L = N(p+p_0+p-(-p_0))
            = Np_0(p+(-p))
            = N\ell_0\tanh\left( \frac{f\ell_0}{k_BT} \right)
        \end{equation*}
    \end{itemize}
\end{itemize}



\section{Van der Waals Phase Transitions}
\begin{itemize}
    \item \marginnote{2/16:}Wrapping up the rubber band analysis.
    \item Note that we can approximate $\tanh x\approx x$ for small $x$. Thus,
    \begin{equation*}
        L = N\ell_0\cdot\frac{f\ell_0}{k_BT}
    \end{equation*}
    for small $f$.
    \item Therefore, statistical mechanics predicts the rubber band "ideal" equation of state, $f=T\phi(L)$, where $f$ is the stretching force. Note that this is not at all like a mechanical spring constant; it is purely an entropy effect.
    \begin{itemize}
        \item $\dd{S}=\dd{S_\text{orientation}}+\dd{S_\text{thermal}}=\dd{S_\text{orientation}}+C/T\dd{T}$.
        \item $\dd{U}=C\dd{T}+T\dd{S}=0$, implying that $\dd{T}=-T/C\dd{S}$.
    \end{itemize}
    \item Adiabatic stretching decreases orientation entropy and increases the temperature.
    \item The same model predicts the Curie law of paramagnetism, $M=CB/T$.
    \begin{itemize}
        \item We have
        \begin{align*}
            M &= N\mu\tanh\frac{\mu B}{k_BT}\\
            M &= \frac{N\mu^2}{k_B}\frac{B}{T}
        \end{align*}
        where the second equation only holds for small $B$.
    \end{itemize}
    \item Adiabatic demagnetization allows you to go from \SI{4}{\kelvin} to \SI{1}{\kelvin} using \ce{He4} to \ce{He3} liquefaction. Adiabatic demagnetization increases spin entropy and reduces the temperature.
    \begin{itemize}
        \item To achieve \SI{4}{\kelvin}, we let \ce{He4} adiabatically expand, which makes it very cold. However, at temperatures below \SI{4}{\kelvin}, \ce{He4} is a liquid and it can no longer adiabatically expand like a gas.
        \item To achieve temperatures lower than \SI{4}{\kelvin}, you stick copper salt in a cryostat, subject it to a magnetic field, and cool it to \SI{4}{\kelvin} with the above method. The magnet aligns the spins. When you take the salt out of the magnetic field, the spins will randomize entropically, but this takes energy. You use the lattice energy to raise the spin energy.
    \end{itemize}
    \item The challenge of protein folding and structure determination.
    \begin{itemize}
        \item We have $\Delta G=\Delta(H-TS)$.
        \item Protein folding necessitates $\Delta S<0$ (you are creating order). Thus, we must have $\Delta H<\Delta(TS)=T\Delta S$, i.e., the protein must adopt a very, very stable conformation.
        \item Estimating $\Delta S$:
        \begin{equation*}
            \Delta S = k_B\ln W
            = k_B\ln 3^{100}
            = \SI{914}{\joule\per\kelvin\per\mole}
        \end{equation*}
        \begin{itemize}
            \item Using \textbf{Levinthal's paradox}, we can estimate the number of possible configurations of a protein.
            \item Each segment (essentially a \ce{C-C} bond) has about three dihedral angle possibilities for the segments at either end.
            \item Thus, for a protein that is 100 segments long, $W\approx 3^{100}$.
        \end{itemize}
        \item At \SI{300}{\kelvin}, $T\Delta S=\SI[per-mode=symbol]{274}{\kilo\joule\per\mole}$.
        \begin{itemize}
            \item It doesn't take much of a temperature difference to alter or prevent protein folding.
        \end{itemize}
        \item About 15 hydrogen bonds or a single disulfide bond is about \SI[per-mode=symbol]{274}{\kilo\joule\per\mole} of energy, so we can find ways to stabilize proteins over the entropic barriers.
        \item Just recently, a UChicago grad student (who had been working with a UChicago professor who's been studying the protein folding problem for a long time) headed a team at Google that has an AI that looks like it will be able to solve the protein folding problem.
    \end{itemize}
    \item \textbf{Levinthal's paradox}: The observation that finding the native folded state of a protein by a random search among all possible configurations can take an enormously long time. Yet proteins can fold in seconds or less.
    \item Real gases deviate from ideality due to molecular interactions.
    \item \textbf{Compressibility factor}: \emph{Denoted by} $\bm{z}$. \emph{Given by}
    \begin{equation*}
        z = \frac{P\overline{V}}{RT}
    \end{equation*}
    \item The van der Waals equation of state (where $\overline{V}$ is the molar volume):
    \begin{equation*}
        \left( P+\frac{a}{\overline{V}^2} \right)(\overline{V}-b) = RT
    \end{equation*}
    \begin{itemize}
        \item We can also rewrite this as
        \begin{equation*}
            P = \frac{RT}{\overline{V}-b}-\frac{a}{\overline{V}^2}
        \end{equation*}
        \item Note that $a,b\geq 0$.
        \item The compressibility is
        \begin{equation*}
            z = \frac{\overline{V}}{\overline{V}-b}-\frac{a}{RT\overline{V}}
        \end{equation*}
    \end{itemize}
    \item The van der Waals equation of state is cubic in $\overline{V}$ and can predict two different molar volumes (gas and liquid) for one pressure.
    \begin{figure}[h!]
        \centering
        \begin{tikzpicture}[
            xscale=10,yscale=0.03,
            every node/.style={black}
        ]
            \small
            \draw [stealth-stealth] (0,120) -- node[left]{$P$} (0,0) -- node[below]{$\overline{V}$} (0.6,0);
    
            \draw [rex,thick] plot[domain=0.0672:0.6,smooth,samples=500,/pgf/fpu,/pgf/fpu/output format=fixed] (\x,{(-0.0821*273*\x*\x+3.59*\x-3.59*0.0427)/(0.0427*\x*\x-\x*\x*\x)});
            \draw [rex,semithick] (0.0773,47.1) -- (0.3017,47.1);
    
            \footnotesize
            \node [below left] at (0.0672,119.849) {L};
            \node [below] at (0.092,30.481) {C};
            \node [left] at (0.0773,47.1) {D};
            \node [above] at (0.1959,52.755) {B};
            \node [above right] at (0.3017,47.1) {A};
            \node [below left] at (0.6,30.245) {G};
        \end{tikzpicture}
        \caption{The van der Waals isotherm of \ce{CO2} at \SI{0}{\celsius}.}
        \label{fig:vanDerWaalsIsotherm}
    \end{figure}
    \begin{itemize}
        \item We have that
        \begin{align*}
            RT &= \left( P+\frac{a}{\overline{V}^2} \right)(\overline{V}-b)\\
            &= P\overline{V}-bP+\frac{a}{\overline{V}}-\frac{ab}{\overline{V}^2}\\
            \frac{RT\overline{V}^2}{P} &= \overline{V}^3-b\overline{V}^2+\frac{a}{P}\overline{V}-\frac{ab}{P}\\
            0 &= \overline{V}^3-\left( b+\frac{RT}{P} \right)\overline{V}^2+\frac{a}{P}\overline{V}-\frac{ab}{P}
        \end{align*}
        \item Since it is cubic in $\overline{V}$, this means we can have up to three different molar volumes for one pressure.
        \item This also reflects the fact that as we compress a gas, it behaves ideally for a while, and then pressure is constant as condensation takes hold, and then we must apply massive amounts of pressure to make the volume any smaller.
        \item On $\overline{\text{AG}}$, the system is a gas. On $\overline{\text{LD}}$, it is a liquid. On $\overline{\text{AD}}$, pressure is constant as we compress more and more because condensation takes hold, so highly compressed gas molecules become liquid, reducing the pressure.
        \item Drawing the line $\overline{\text{AD}}$: The points at which gas stops and liquid starts must be at the same pressure.
        \item We must also have the area above and below the line equal (\textbf{Maxwell equal area contraction}).
        \begin{itemize}
            \item Phase equilibrium (like we have along $\overline{\text{AD}}$) means that free energy does not change. Mathematically, $\Delta G(\text{A})=\Delta G(\text{D})$.
            \item If A and D represent gas and liquids at the same temperature, they are in equilibrium and they must have the same molar Gibbs free energy.
            \item We have $\dd{G}=-S\dd{T}+V\dd{P}$.
            \item Since $T$ is constant (this is an isotherm), $\dd{G}=V\dd{P}$.
            \item Thus, we can integrate along the curve $\overline{\text{DA}}$ to find an appropriate $\overline{\text{DA}}$ such that the integral is zero.
            \begin{align*}
                0 &= \Delta G(\text{A})-\Delta G(\text{D})\\
                &= \int_\text{D}^\text{A}\dd{G}\\
                &= \int_\text{D}^\text{A}V\dd{P}
            \end{align*}
            \item See Problem 23-46 for the "Maxwell equal area construction rule."
        \end{itemize}
    \end{itemize}
\end{itemize}



\section{More van der Waals Phenomena}
\begin{itemize}
    \item \marginnote{2/18:}The cubic vdW equation of state predicts a critical point at $T_c$ at and above which only one $P$ for $(V,T)$ solution is possible.
    \begin{itemize}
        \item The behavior of isotherms around the critical temperature.
        \begin{itemize}
            \item For isotherms below the critical temperature, there will be a range of volumes where the gas and liquid phases are in equilibrium.
            \item For isotherms above the critical temperature, we only have the gas phase, so there is no region of constant pressure as we compress the system.
            \item This means that at the critical temperature $T=T_c$, there will only be a single volume $V_c$ and hence pressure $P_c$ at which the gas and liquid phases are in equilibrium. Mathematically, this means that each of the three roots of the cubic van der Waals equation exist at the same point $V_c$, i.e., that the equation is of the form $(V-\overline{V}_c)^3$.
        \end{itemize}
        \item Expanding, we have
        \begin{align*}
            0 &= \overline{V}^3-\left( b+\frac{RT}{P} \right)\overline{V}^2+\frac{a}{P}\overline{V}-\frac{ab}{P}\\
            &= \overline{V}^3-3\overline{V}_c\overline{V}^2+3\overline{V}_c^2\overline{V}-\overline{V}_c^3
        \end{align*}
        \item Thus, at $T_c$,
        \begin{align*}
            b+\frac{RT_c}{P_c} &= 3\overline{V}_c&
            \frac{a}{P_c} &= 3V_c^2&
            \frac{ab}{P_c} &= \overline{V}_c^3
        \end{align*}
        \item One immediate consequence is that
        \begin{align*}
            \overline{V}_c^3 &= \frac{a}{P_c}\cdot b\\
            \overline{V}_c^3 &= 3V_c^2b\\
            \overline{V}_c &= 3b
        \end{align*}
        \begin{itemize}
            \item Thus, the critical volume is on the order of magnitude of the molecular volume.
            \item Note that we can't manipulate the first equation into a different relation between $\overline{V}_c$ and $b$ using the ideal gas law substitution because this is a van der Waals gas.
        \end{itemize}
        \item It follows that
        \begin{align*}
            3V_c^2 &= \frac{a}{P_c}&
                b+\frac{RT_c}{P_c} &= 3\overline{V}_c\\
            P_c &= \frac{a}{3(3b)^2}&
                b+RT_c\cdot\frac{27b^2}{a} &= 3(3b)\\
            P_c &= \frac{a}{27b^2}&
                T_c &= \frac{8a}{27Rb}
        \end{align*}
        \item Thus,
        \begin{equation*}
            z_c = \frac{P_cV_c}{RT_c} = \frac{3}{8}
        \end{equation*}
        \begin{itemize}
            \item Indeed, the vdW equation predicts the compressibility at the critical point to be $3/8$. The Redlich-Kwong predicts $1/3$. The Peng-Robinson predicts $0.307$. The experimental values are around $0.3$. Water and ammonia deviate significantly: $0.23$ and $0.24$, respectively, due to their strong dipole moments/hydrogen bonding. Table 16.5 gives a lot of related data.
        \end{itemize}
        \item Also, at $T_c,V_c$, we have that $(\pdv*{P}{V})_T=0$ and $\kappa\to\infty$ (recall that $\kappa$ is the isothermal compressibility).
        \begin{itemize}
            \item It follows from the vdW that $\kappa\propto(\overline{V}-\overline{V}_c)^{-1}$.
            \item Experimental value: \emph{Every} gas satisfies $\kappa\propto(\overline{V}-\overline{V}_c)^{-1.24}$.
            \item The mystery was solved theoretically in the 1970s with renormalization group theory.
        \end{itemize}
    \end{itemize}
    \item Law of corresponding states.
    \begin{itemize}
        \item We define the relative pressure, volume, and temperature by
        \begin{align*}
            P_R &= \frac{P}{P_c}&
            V_R &= \frac{V}{V_c}&
            T_R &= \frac{T}{T_c}
        \end{align*}
        \item Note that
        \begin{equation*}
            \frac{RT}{P_cV_c} = \frac{RT_c}{P_cV_c}\frac{T}{T_c}
            = \frac{1}{z_c}\frac{T}{T_c}
            = \frac{8}{3}T_R
        \end{equation*}
        \item Thus,
        \begin{align*}
            RT &= \left( P+\frac{a}{\overline{V}^2} \right)(\overline{V}-b)\\
            &= \left( P+\frac{3\overline{V}_c^2P_c}{\overline{V}_c^2} \right)\left( \overline{V}-\frac{\overline{V}_c}{3} \right)\\
            \frac{RT}{P_cV_c} &= \left( \frac{P}{P_c}+\frac{3\overline{V}_c^2}{\overline{V}} \right)\left( \frac{\overline{V}}{\overline{V}_c}-\frac{1}{3} \right)\\
            \frac{8}{3}T_R &= \left( P_R+\frac{3}{\overline{V}_R^2} \right)\left( \overline{V}_R-\frac{1}{3} \right)
        \end{align*}
    \end{itemize}
    \item Virial expansion to experimentally determine the vdW coefficients $a,b$ from the compressibility near ideal conditions.
    \begin{itemize}
        \item We let
        \begin{equation*}
            z = 1+\frac{B_{2V}(T)}{\overline{V}}+\frac{B_{3V}(T)}{\overline{V}^2}+\cdots
        \end{equation*}
        where $B_{iV}(T)$ is the \textbf{$\bm{i^\textbf{th}}$ virial coefficient}.
        \item It follows that
        \begin{align*}
            \frac{P\overline{V}}{RT} &= \frac{\overline{V}}{V-b}-\frac{a}{RT\overline{V}}\\
            &= \frac{1}{1-b/\overline{V}}-\frac{a}{RT\overline{V}}\\
            &= 1+\left( \frac{b}{\overline{V}}-\frac{a}{RT\overline{V}} \right)+\text{terms in }\tfrac{1}{\overline{V}^2}+\cdots
        \end{align*}
        where we get from the second to the third line using the expansion
        \begin{equation*}
            \frac{1}{1-x} = 1+x+x^2+\cdots
        \end{equation*}
        \item Thus,
        \begin{equation*}
            B_{2V}(T) = b-\frac{a}{RT}
        \end{equation*}
    \end{itemize}
    \item Microscopic origin of the vdW coefficients from the interaction potential.
    \begin{itemize}
        \item Draws the interaction potential for a diatomic molecule.
        \begin{itemize}
            \item The repulsion comes from the Fermi exclusion principle.
        \end{itemize}
        \item Discusses dipole-induced dipole moments.
        \begin{itemize}
            \item $-1/2\propto E^2\propto 1/\pi^6$.
        \end{itemize}
        \item The origin of vdW (or London dispersion) interaction is quantum mechanical.
        \begin{itemize}
            \item We use perturbation theory to calculate the interaction between two neighboring quantum dipoles.
            \begin{equation*}
                \Delta E^{(1)} = \ev{H_{int}}{\psi_A^\circ\psi_B^\circ} = 0
            \end{equation*}
            \item Thus we need the second order correction:
            \begin{equation*}
                \Delta E^{(2)} = -\sum_{ij}\frac{|\ev{H_{int}}{\psi_A^\circ\psi_B^\circ}|}{E_a^A+E_j^B-(E_0^A+E_0^B)} \approx -\frac{c}{\pi^6}
            \end{equation*}
            where $c\geq 0$.
        \end{itemize}
    \end{itemize}
\end{itemize}




\end{document}