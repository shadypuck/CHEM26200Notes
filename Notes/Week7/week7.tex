\documentclass[../notes.tex]{subfiles}

\pagestyle{main}
\renewcommand{\chaptermark}[1]{\markboth{\chaptername\ \thechapter\ (#1)}{}}
\setcounter{chapter}{6}

\begin{document}




\chapter{Phase Diagrams and Critical Points}
\section{Virial Coefficients and Fugacity}
\begin{itemize}
    \item \marginnote{2/21:}Relation between the interaction potential and the first virial coefficient (Equation 16.25).
    \begin{itemize}
        \item Statistical mechanics is important because it gives us the relation
        \begin{equation*}
            B_{2V}(T) = -2\pi N_A\int_0^\infty\left( \e[-u(r)/k_BT]-1 \right)r^2\dd{r}
        \end{equation*}
        \begin{itemize}
            \item We can derive this with our knowledge of statistical mechanics, but PGS will not go through this.
        \end{itemize}
        \item Now recall the Lennard-Jones potential
        \begin{equation*}
            u(r) = 4\epsilon\left[ \left( \frac{\sigma}{\pi} \right)^{12}-\left( \frac{\sigma}{\pi} \right)^6 \right]
        \end{equation*}
        \begin{itemize}
            \item Note that the minimum is at $(2^{1/6}\sigma,-\epsilon)$.
            \item The limiting case of the Lennard-Jones potential is hard sphere repulsion (the repulsion as you approach a hard sphere, which is zero up until you're at the surface and then infinite repulsion). Thus, with no intermolecular attraction, $a=0$, so in this case,
            \begin{align*}
                B_{2V}(T) &= -2\pi N_A\int_0^\sigma(-1)r^2\dd{r}\\
                &= \frac{2\pi N_A\sigma^3}{3}\\
                &= b-\frac{0}{RT}
            \end{align*}
        \end{itemize}
        \item Now consider a potential that is van der Waals ($c/r^6$) up until a point and then hard sphere. In this case,
        \begin{align*}
            B_{2V}(T) &= \frac{2\pi N_A\sigma^3}{3}-2\pi N_A\int_\sigma^\infty\left( \e[-c/r^6k_BT]-1 \right)r^2\dd{r}\\
            &= \frac{2\pi N_A\sigma^3}{3}-2\pi N_A\int_\sigma^\infty\left( -\frac{c}{r^6k_BT} \right)r^2\dd{r}\\
            &= b+\frac{2\pi N_Ac}{k_BT}\cdot-\frac{1}{3\sigma^3}
        \end{align*}
        where we have used $\e[x]=1+x+\cdots$ to get from the first line to the second.
        \begin{itemize}
            \item Therefore,
            \begin{equation*}
                a = \frac{2\pi N_A^2}{3}\frac{c}{\sigma^2}
            \end{equation*}
        \end{itemize}
    \end{itemize}
    \item Derivation of the relation between $B_{2V}(T)$ and the interaction potential $u(r)$.
    \begin{itemize}
        \item Consider a system of independent, indistinguishable particles.
        \item The total Hamiltonian for the system has a kinetic energy part and an interaction part.
        \begin{equation*}
            \hat{H}(p_i,r_i) = \sum_i\frac{\hat{p}_i^2}{2m}+\sum_{i<j}\hat{u}(r_i,r_j)
        \end{equation*}
        \item The kinetic part (which ignores intermolecular interactions) will lead to the ideal gas partition function. The nonideal part of the partition function will come from the interaction potentials. Mathematically,
        \begin{align*}
            Q &= \frac{1}{N!}\int\e[-\beta E(p_i,r_i)]\dd[3]{p_i}\dd[3]{r_i}\\
            &= \frac{1}{N!}\int\exp\left\{ -\beta\left[ \sum_i\frac{p_i^2}{2m}+\sum_{i<j}u(r_i,r_j) \right] \right\}\dd[3]{p_i}\dd[3]{r_i}\\
            &= \frac{1}{N!}\left( \int\exp\left[ -\beta\sum_i\frac{p_i^2}{2m} \right]\dd[3]{p_i} \right)\left( \int\exp\left[ -\beta\sum_{i<j}u(r_i,r_j) \right]\dd[3]{r_i} \right)\\
            &= \underbrace{\frac{V^N}{N!}\left( \int\exp\left[ -\beta\sum_i\frac{p_i^2}{2m} \right]\dd[3]{p_i} \right)}_{Q_\text{ideal}}\cdot\underbrace{\frac{1}{V^N}\left( \int\exp\left[ -\beta\sum_{i<j}u(r_i,r_j) \right]\dd[3]{r_i} \right)}_{Q_u}
        \end{align*}
        \item Define
        \begin{equation*}
            f_{ij} = \e[-u(r_i,r_j)/k_BT]-1
        \end{equation*}
        \item Now note that the interaction between molecules is pretty small, and in fact $f_{ij}\to 0$ as $|r_i-r_j|\to\infty$.
        \item Thus,
        \begin{align*}
            Q_u &= \frac{1}{V^N}\int\exp\left[ -\beta\sum_{i<j}u(r_i,r_j) \right]\dd[3]{r_i}\\
            &= \frac{1}{V^N}\int\prod_{i<j}(f_{ij}+1)\dd[3]{r_i}
        \end{align*}
        \item We can do a \textbf{cluster expansion} on this small $f_{ij}$:
        \begin{equation*}
            \prod_{i<j}(f_{ij}+1) = 1+\sum_{i<j}f_{ij}+\sum_{i<j}\sum_{k<\ell}f_{ij}f_{k\ell}
        \end{equation*}
        \item In particular, $\sum_{i<j}$ is the sum of pairwise interactions while $f_{ij}\cdot f_{k\ell}$ are binary interactions, $f_{ij}f_{k\ell}f_{mn}$ are tertiary interactions, and so on and so forth.
        \item But at low density, the dominant term is the pairwise interaction so we have
        \begin{align*}
            Q_u &= \frac{1}{V^N}\int\left( 1+\sum_{i<j}f_{ij} \right)\dd[3]{r_i}\\
            &= \frac{1}{V^N}\left( V^N+\frac{N(N-1)}{2}V^{N-2}\int f_{12}\dd[3]{r_1}\dd[3]{r_2} \right)\\
            &= 1+\frac{N(N-1)}{2V}\int(\e[-\beta u(r)]-1)\dd[3]{r}
        \end{align*}
        \item It follows that
        \begin{equation*}
            Q = Q_\text{id}\left[ 1+\frac{N(N-1)}{2V}\int\left( \e[-\beta u(r)]-1 \right)\dd[3]{r} \right]
        \end{equation*}
        \item But now we need to extract an equation of state from the partition function. To do so, we use
        \begin{align*}
            P &= k_BT\left( \pdv{\ln Q}{V} \right)_{N,T}\\
            &= k_BT\left( \pdv{\ln Q_\text{id}}{V} \right)_{N,T}+k_BT\left( \pdv{\ln Q_u}{V} \right)_{N,T}
        \end{align*}
        \item We know that the first term above is equal to $Nk_BT/V$, but it takes a bit more work for the second one.
        \item We have that
        \begin{align*}
            \ln Q_u &= \ln\Bigg( 1+\underbrace{\frac{N(N-1)}{2V}}_{\substack{\text{Approximately}\\\text{the intermolecular}\\\text{distance }1/\rho^3}}\underbrace{\int\left( \e[-\beta u(r)]-1 \right)\dd[3]{r}}_{\substack{\text{Approximately}\\\text{the molecular}\\\text{volume }a^3}} \Bigg)\\
            &= \frac{N(N-1)}{2V}\int\left( \e[-\beta u(r)]-1 \right)\dd[3]{r}
        \end{align*}
        since the second term is a small number and the natural log of a small number plus 1 is approximately that small number.
        \item Thus,
        \begin{equation*}
            \left( \pdv{\ln Q_u}{V} \right)_{N,T} = -\frac{N(N-1)}{2V^2}\int\left( \e[-\beta u(r)]-1 \right)\dd[3]{r}
        \end{equation*}
        so
        \begin{align*}
            P &= \frac{Nk_BT}{V}-\frac{Nk_BT}{V}\frac{N-1}{2V}\int\left( \e[-\beta u(r)]-1 \right)\dd[3]{r}\\
            &= \frac{RT}{\overline{V}}-\frac{RT}{\overline{V}}\frac{N-1}{2V}\int\left( \e[-\beta u(r)]-1 \right)\dd[3]{r}
        \end{align*}
        \item Consequently,
        \begin{equation*}
            Z = \frac{P\overline{V}}{RT} = 1-\frac{N_A}{\overline{V}}\cdot\frac{1}{2}\int\left( \e[-\beta u(r)]-1 \right)\dd[3]{r}
        \end{equation*}
        \item Therefore, by comparison with the virial expansion,
        \begin{align*}
            B_{2V}(T) &= -\frac{N_A}{2}\int\left( \e[-\beta u(r)]-1 \right)\dd[3]{r}\\
            &= -\frac{N_A}{2}\int_0^\infty\left( \e[-\beta u(r)]-1 \right)4\pi r^2\dd{r}\\
            &= -2\pi N_A\int_0^\infty\left( \e[-u(r)/k_BT]-1 \right)r^2\dd{r}
        \end{align*}
    \end{itemize}
    \item Free energy as a function of $(T,P)$ for a real gas. Definition of fugacity and fugacity coefficients.
    \begin{itemize}
        \item We want to find $\Delta G(T,P)$.
        \item We have that $\dd{\overline{G}}=-\overline{S}\dd{T}+\overline{V}\dd{P}$. It follows that
        \begin{equation*}
            \left( \pdv{\overline{G}}{P} \right)_T = \overline{V}
        \end{equation*}
        \item Thus,
        \begin{equation*}
            \overline{G}(T,P) = \overline{G}(T,P_0)+\int_{P_0}^P\overline{V}\dd{P}
        \end{equation*}
        \item In the ideal case,
        \begin{align*}
            \overline{G}_\text{ideal}(T,P) &= \overline{G}_\text{ideal}(T,P_0)+\int_{P_0}^P\frac{RT}{P}\dd{P}\\
            &= \overline{G}_\text{ideal}(T,P_0)+RT\ln\frac{P}{P_0}
        \end{align*}
        \item In the nonideal case, we define a fugacity $f$ by
        \begin{equation*}
            \overline{G}_\text{ideal}(T,P) = \overline{G}_\text{ideal}(T,P_0)+RT\ln\frac{f}{P_0}
        \end{equation*}
        \begin{itemize}
            \item The second term in the above equation refers to the Gibbs free energy of an ideal gas at $P_0=\SI{1}{\bar}$ or $P_0=\SI{1}{\atmosphere}$. Note that even at $P_0=\SI{1}{\atmosphere}$, however, there is too much pressure for truly ideal behavior, so $f\neq P_0$.
        \end{itemize}
        \item Imagine that $\Delta\overline{G}_1$ takes us from a real gas at $(T,P)$ to an ideal gas at $(T,P)$. Then
        \begin{align*}
            \Delta\overline{G}_1 &= \overline{G}_\text{ideal}(T,P)-\overline{G}_\text{real}(T,P)\\
            &= \left[ \overline{G}_\text{ideal}(T,P_0)+RT\ln\frac{P}{P_0} \right]-\left[ \overline{G}_\text{ideal}(T,P_0)+RT\ln\frac{f}{P_0} \right]\\
            &= -RT\ln\frac{f}{P}
        \end{align*}
        \item Now let $\Delta\overline{G}_2$ take us from a real gas at $(T,P)$ to a real gas at $T$ and $P\to 0$, which will be the same as an ideal gas at $T$ and $P\to 0$. Then let $\Delta\overline{G}_3$ take us from this ideal gas at $T$ and $P\to 0$ to an ideal gas at $(T,P)$. It follows that
        \begin{align*}
            \Delta\overline{G}_2 &= -\int_{P\to 0}^P\overline{V}\dd{P'}&
            \Delta\overline{G}_2 &= \int_{P\to 0}^P\frac{RT}{P'}\dd{P'}
        \end{align*}
        \item Thus, since $\Delta\overline{G}_1=\Delta\overline{G}_2+\Delta\overline{G}_3$ ($G$ is a state function),
        \begin{equation*}
            -RT\ln\frac{f}{P} = \int_{P\to 0}^P\left( -\overline{V}+\frac{RT}{P'} \right)\dd{P'}
        \end{equation*}
        \item We then define $\gamma$ to be the \textbf{fugacity coefficient} by $\gamma=f/P$. It follows that
        \begin{equation*}
            \ln\gamma = \int_0^P\frac{z-1}{P'}\dd{P'}
        \end{equation*}
    \end{itemize}
    \item Fugacity coefficient expressed in terms of the compressibility deviation from unity.
    \begin{itemize}
        \item At low temperature, $z<1$, so $\gamma<1$ and hence $f<p$.
        \item At high pressure, $z>1$ (excluded volume), so $\gamma>1$ and hence $f>P$.
    \end{itemize}
    \item Introduces phase diagrams and their notable properties.
\end{itemize}




\end{document}