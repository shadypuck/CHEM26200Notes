\documentclass[../notes.tex]{subfiles}

\pagestyle{main}
\renewcommand{\chaptermark}[1]{\markboth{\chaptername\ \thechapter\ (#1)}{}}
\setcounter{chapter}{6}

\begin{document}




\chapter{Phase Equilibria and Solutions}
\section{Virial Coefficients and Fugacity}
\begin{itemize}
    \item \marginnote{2/21:}Relation between the interaction potential and the first virial coefficient (Equation 16.25).
    \begin{itemize}
        \item Statistical mechanics is important because it gives us the relation
        \begin{equation*}
            B_{2V}(T) = -2\pi N_A\int_0^\infty\left( \e[-u(r)/k_BT]-1 \right)r^2\dd{r}
        \end{equation*}
        \begin{itemize}
            \item We can derive this with our knowledge of statistical mechanics, but PGS will not go through this.
        \end{itemize}
        \item Now recall the Lennard-Jones potential
        \begin{equation*}
            u(r) = 4\epsilon\left[ \left( \frac{\sigma}{\pi} \right)^{12}-\left( \frac{\sigma}{\pi} \right)^6 \right]
        \end{equation*}
        \begin{itemize}
            \item Note that the minimum is at $(2^{1/6}\sigma,-\epsilon)$.
            \item The limiting case of the Lennard-Jones potential is hard sphere repulsion (the repulsion as you approach a hard sphere, which is zero up until you're at the surface and then infinite repulsion). Thus, with no intermolecular attraction, $a=0$, so in this case,
            \begin{align*}
                B_{2V}(T) &= -2\pi N_A\int_0^\sigma(-1)r^2\dd{r}\\
                &= \frac{2\pi N_A\sigma^3}{3}\\
                &= b-\frac{0}{RT}
            \end{align*}
        \end{itemize}
        \item Now consider a potential that is van der Waals ($c/r^6$) up until a point and then hard sphere. In this case,
        \begin{align*}
            B_{2V}(T) &= \frac{2\pi N_A\sigma^3}{3}-2\pi N_A\int_\sigma^\infty\left( \e[-c/r^6k_BT]-1 \right)r^2\dd{r}\\
            &= \frac{2\pi N_A\sigma^3}{3}-2\pi N_A\int_\sigma^\infty\left( -\frac{c}{r^6k_BT} \right)r^2\dd{r}\\
            &= b+\frac{2\pi N_Ac}{k_BT}\cdot-\frac{1}{3\sigma^3}
        \end{align*}
        where we have used $\e[x]=1+x+\cdots$ to get from the first line to the second.
        \begin{itemize}
            \item Therefore,
            \begin{equation*}
                a = \frac{2\pi N_A^2}{3}\frac{c}{\sigma^2}
            \end{equation*}
        \end{itemize}
    \end{itemize}
    \item Derivation of the relation between $B_{2V}(T)$ and the interaction potential $u(r)$.
    \begin{itemize}
        \item Consider a system of independent, indistinguishable particles.
        \item The total Hamiltonian for the system has a kinetic energy part and an interaction part.
        \begin{equation*}
            \hat{H}(p_i,r_i) = \sum_i\frac{\hat{p}_i^2}{2m}+\sum_{i<j}\hat{u}(r_i,r_j)
        \end{equation*}
        \item The kinetic part (which ignores intermolecular interactions) will lead to the ideal gas partition function. The nonideal part of the partition function will come from the interaction potentials. Mathematically,
        \begin{align*}
            Q &= \frac{1}{N!}\int\e[-\beta E(p_i,r_i)]\dd[3]{p_i}\dd[3]{r_i}\\
            &= \frac{1}{N!}\int\exp\left\{ -\beta\left[ \sum_i\frac{p_i^2}{2m}+\sum_{i<j}u(r_i,r_j) \right] \right\}\dd[3]{p_i}\dd[3]{r_i}\\
            &= \frac{1}{N!}\left( \int\exp\left[ -\beta\sum_i\frac{p_i^2}{2m} \right]\dd[3]{p_i} \right)\left( \int\exp\left[ -\beta\sum_{i<j}u(r_i,r_j) \right]\dd[3]{r_i} \right)\\
            &= \underbrace{\frac{V^N}{N!}\left( \int\exp\left[ -\beta\sum_i\frac{p_i^2}{2m} \right]\dd[3]{p_i} \right)}_{Q_\text{ideal}}\cdot\underbrace{\frac{1}{V^N}\left( \int\exp\left[ -\beta\sum_{i<j}u(r_i,r_j) \right]\dd[3]{r_i} \right)}_{Q_u}
        \end{align*}
        \item Define
        \begin{equation*}
            f_{ij} = \e[-u(r_i,r_j)/k_BT]-1
        \end{equation*}
        \item Now note that the interaction between molecules is pretty small, and in fact $f_{ij}\to 0$ as $|r_i-r_j|\to\infty$.
        \item Thus,
        \begin{align*}
            Q_u &= \frac{1}{V^N}\int\exp\left[ -\beta\sum_{i<j}u(r_i,r_j) \right]\dd[3]{r_i}\\
            &= \frac{1}{V^N}\int\prod_{i<j}(f_{ij}+1)\dd[3]{r_i}
        \end{align*}
        \item We can do a \textbf{cluster expansion} on this small $f_{ij}$:
        \begin{equation*}
            \prod_{i<j}(f_{ij}+1) = 1+\sum_{i<j}f_{ij}+\sum_{i<j}\sum_{k<\ell}f_{ij}f_{k\ell}
        \end{equation*}
        \item In particular, $\sum_{i<j}$ is the sum of pairwise interactions while $f_{ij}\cdot f_{k\ell}$ are binary interactions, $f_{ij}f_{k\ell}f_{mn}$ are tertiary interactions, and so on and so forth.
        \item But at low density, the dominant term is the pairwise interaction so we have
        \begin{align*}
            Q_u &= \frac{1}{V^N}\int\left( 1+\sum_{i<j}f_{ij} \right)\dd[3]{r_i}\\
            &= \frac{1}{V^N}\left( V^N+\frac{N(N-1)}{2}V^{N-2}\int f_{12}\dd[3]{r_1}\dd[3]{r_2} \right)\\
            &= 1+\frac{N(N-1)}{2V}\int(\e[-\beta u(r)]-1)\dd[3]{r}
        \end{align*}
        \item It follows that
        \begin{equation*}
            Q = Q_\text{id}\left[ 1+\frac{N(N-1)}{2V}\int\left( \e[-\beta u(r)]-1 \right)\dd[3]{r} \right]
        \end{equation*}
        \item But now we need to extract an equation of state from the partition function. To do so, we use
        \begin{align*}
            P &= k_BT\left( \pdv{\ln Q}{V} \right)_{N,T}\\
            &= k_BT\left( \pdv{\ln Q_\text{id}}{V} \right)_{N,T}+k_BT\left( \pdv{\ln Q_u}{V} \right)_{N,T}
        \end{align*}
        \item We know that the first term above is equal to $Nk_BT/V$, but it takes a bit more work for the second one.
        \item We have that
        \begin{align*}
            \ln Q_u &= \ln\Bigg( 1+\underbrace{\frac{N(N-1)}{2V}}_{\substack{\text{Approximately}\\\text{the intermolecular}\\\text{distance }1/\rho^3}}\underbrace{\int\left( \e[-\beta u(r)]-1 \right)\dd[3]{r}}_{\substack{\text{Approximately}\\\text{the molecular}\\\text{volume }a^3}} \Bigg)\\
            &= \frac{N(N-1)}{2V}\int\left( \e[-\beta u(r)]-1 \right)\dd[3]{r}
        \end{align*}
        since the second term is a small number and the natural log of a small number plus 1 is approximately that small number.
        \item Thus,
        \begin{equation*}
            \left( \pdv{\ln Q_u}{V} \right)_{N,T} = -\frac{N(N-1)}{2V^2}\int\left( \e[-\beta u(r)]-1 \right)\dd[3]{r}
        \end{equation*}
        so
        \begin{align*}
            P &= \frac{Nk_BT}{V}-\frac{Nk_BT}{V}\frac{N-1}{2V}\int\left( \e[-\beta u(r)]-1 \right)\dd[3]{r}\\
            &= \frac{RT}{\overline{V}}-\frac{RT}{\overline{V}}\frac{N-1}{2V}\int\left( \e[-\beta u(r)]-1 \right)\dd[3]{r}
        \end{align*}
        \item Consequently,
        \begin{equation*}
            Z = \frac{P\overline{V}}{RT} = 1-\frac{N_A}{\overline{V}}\cdot\frac{1}{2}\int\left( \e[-\beta u(r)]-1 \right)\dd[3]{r}
        \end{equation*}
        \item Therefore, by comparison with the virial expansion,
        \begin{align*}
            B_{2V}(T) &= -\frac{N_A}{2}\int\left( \e[-\beta u(r)]-1 \right)\dd[3]{r}\\
            &= -\frac{N_A}{2}\int_0^\infty\left( \e[-\beta u(r)]-1 \right)4\pi r^2\dd{r}\\
            &= -2\pi N_A\int_0^\infty\left( \e[-u(r)/k_BT]-1 \right)r^2\dd{r}
        \end{align*}
    \end{itemize}
    \item Free energy as a function of $(T,P)$ for a real gas. Definition of fugacity and fugacity coefficients.
    \begin{itemize}
        \item We want to find $\Delta G(T,P)$.
        \item We have that $\dd{\overline{G}}=-\overline{S}\dd{T}+\overline{V}\dd{P}$. It follows that
        \begin{equation*}
            \left( \pdv{\overline{G}}{P} \right)_T = \overline{V}
        \end{equation*}
        \item Thus,
        \begin{equation*}
            \overline{G}(T,P) = \overline{G}(T,P_0)+\int_{P_0}^P\overline{V}\dd{P}
        \end{equation*}
        \item In the ideal case,
        \begin{align*}
            \overline{G}_\text{ideal}(T,P) &= \overline{G}_\text{ideal}(T,P_0)+\int_{P_0}^P\frac{RT}{P}\dd{P}\\
            &= \overline{G}_\text{ideal}(T,P_0)+RT\ln\frac{P}{P_0}
        \end{align*}
        \item In the nonideal case, we define a fugacity $f$ by
        \begin{equation*}
            \overline{G}_\text{ideal}(T,P) = \overline{G}_\text{ideal}(T,P_0)+RT\ln\frac{f}{P_0}
        \end{equation*}
        \begin{itemize}
            \item The second term in the above equation refers to the Gibbs free energy of an ideal gas at $P_0=\SI{1}{\bar}$ or $P_0=\SI{1}{\atmosphere}$. Note that even at $P_0=\SI{1}{\atmosphere}$, however, there is too much pressure for truly ideal behavior, so $f\neq P_0$.
        \end{itemize}
        \item Imagine that $\Delta\overline{G}_1$ takes us from a real gas at $(T,P)$ to an ideal gas at $(T,P)$. Then
        \begin{align*}
            \Delta\overline{G}_1 &= \overline{G}_\text{ideal}(T,P)-\overline{G}_\text{real}(T,P)\\
            &= \left[ \overline{G}_\text{ideal}(T,P_0)+RT\ln\frac{P}{P_0} \right]-\left[ \overline{G}_\text{ideal}(T,P_0)+RT\ln\frac{f}{P_0} \right]\\
            &= -RT\ln\frac{f}{P}
        \end{align*}
        \item Now let $\Delta\overline{G}_2$ take us from a real gas at $(T,P)$ to a real gas at $T$ and $P\to 0$, which will be the same as an ideal gas at $T$ and $P\to 0$. Then let $\Delta\overline{G}_3$ take us from this ideal gas at $T$ and $P\to 0$ to an ideal gas at $(T,P)$. It follows that
        \begin{align*}
            \Delta\overline{G}_2 &= -\int_{P\to 0}^P\overline{V}\dd{P'}&
            \Delta\overline{G}_2 &= \int_{P\to 0}^P\frac{RT}{P'}\dd{P'}
        \end{align*}
        \item Thus, since $\Delta\overline{G}_1=\Delta\overline{G}_2+\Delta\overline{G}_3$ ($G$ is a state function),
        \begin{equation*}
            -RT\ln\frac{f}{P} = \int_{P\to 0}^P\left( -\overline{V}+\frac{RT}{P'} \right)\dd{P'}
        \end{equation*}
        \item We then define $\gamma$ to be the \textbf{fugacity coefficient} by $\gamma=f/P$. It follows that
        \begin{equation*}
            \ln\gamma = \int_0^P\frac{Z-1}{P'}\dd{P'}
        \end{equation*}
    \end{itemize}
    \item Fugacity coefficient expressed in terms of the compressibility deviation from unity.
    \begin{itemize}
        \item At low temperature, $Z<1$, so $\gamma<1$ and hence $f<p$.
        \item At high pressure, $Z>1$ (excluded volume), so $\gamma>1$ and hence $f>P$.
    \end{itemize}
    \item Introduces phase diagrams and their notable properties.
\end{itemize}



\section{Phase Equilibria}
\begin{itemize}
    \item \marginnote{2/23:}Goes over midterm answer key (and posted it to Canvas).
    \begin{itemize}
        \item 1d.
        \begin{itemize}
            \item We needed to say that $\max S=nk_B\ln 2$.
            \item We also needed to indicate that the slope is vertical at 0 and 1.
        \end{itemize}
        \item Question 3.
        \begin{itemize}
            \item Often looking for a derivation from some fundamental law of thermodynamics.
            \item For example, for 3a, since this is an isolated system, we know that $\dd{U}=0$. Moreover, since $\dd{U}=C_1\dd{T_1}+C_2\dd{T_2}$, we have that $C_1\dd{T_1}=-C_2\dd{T_2}$.
        \end{itemize}
        \item 3b.
        \begin{itemize}
            \item The appropriate derivation was
            \begin{align*}
                \dd{S} &= \dd{S_1}+\dd{S_2}\\
                &= \frac{C_1\dd{T_1}}{T_1}+\frac{C_2\dd{T_2}}{T_2}\\
                &= C_1\dd{T_1}\left( \frac{1}{T_1}-\frac{1}{T_2} \right)
            \end{align*}
            from which it follows since $T_2>T_1$ (and hence $1/T_1-1/T_2>0$), since $\dd{S}>0$, and since $C_1>0$ that $\dd{T_1}>0$.
        \end{itemize}
        \item I need a lot of help on Question 4.
    \end{itemize}
    \item In the video of a liquid becoming a supercritical fluid, the path along the phase diagram is along the liquid-gas coexistence curve to the critical point and beyond.
    \item The heat of vaporization actually isn't constant; it varies with temperature.
    \begin{itemize}
        \item At the critical temperature, it becomes zero and the line has vertical slope.
    \end{itemize}
    \item The densities of gas and liquid also converge as $T\to T_c$.
    \item The slope of coexistence curves on $PT$ phase diagrams; the Clapeyron equation.
    \begin{itemize}
        \item Since the molar free energies $\overline{G}_\alpha,\overline{G}_\beta$ of the two phases $\alpha,\beta$ are equal when said phases are in equilibrium,
        \begin{align*}
            \dd{\overline{G}_\alpha} &= \dd{\overline{G}_\beta}\\
            \left( \pdv{\overline{G}_\alpha}{T} \right)_P\dd{T}+\left( \pdv{\overline{G}_\alpha}{P} \right)_T\dd{P} &= \left( \pdv{\overline{G}_\beta}{T} \right)_P\dd{T}+\left( \pdv{\overline{G}_\beta}{P} \right)_T\dd{P}\\
            -\overline{S}_\alpha\dd{T}+\overline{V}_\alpha\dd{P} &= -\overline{S}_\beta\dd{T}+\overline{V}_\beta\dd{P}\\
            (\overline{S}_\beta-\overline{S}_\alpha)\dd{T} &= (\overline{V}_\beta-\overline{V}_\alpha)\dd{P}\\
            \dv{P}{T} &= \frac{\overline{S}_\beta-\overline{S}_\alpha}{\overline{V}_\beta-\overline{V}_\alpha} = \frac{\Delta\overline{S}_\text{trans}}{\Delta\overline{V}_\text{trans}}
        \end{align*}
        \item Clearly, the last line above gives the slope of the coexistence curves on a phase diagram.
        \item Since $T\Delta\overline{S}_\text{trans}=\Delta\overline{H}_\text{trans}$ ($\Delta G=0$), we also have
        \begin{equation*}
            \dv{P}{T} = \frac{\Delta\overline{H}_\text{trans}}{T\Delta\overline{V}_\text{trans}}
        \end{equation*}
        \begin{itemize}
            \item Since both $\Delta\overline{H}_\text{trans},\Delta\overline{V}_\text{trans}\to 0$ as $T\to T_c$, $\dv*{P}{T}$ depends on the rates at which the two quantities approach zero.
        \end{itemize}
    \end{itemize}
    \item Deriving an expression for the vapor pressure (in equilibrium with liquid).
    \begin{itemize}
        \item We know that $\Delta\overline{V}_\text{vap}=\overline{V}_g-\overline{V}_l$ where $\overline{V}_g\gg\overline{V}_l$. Therefore, we may approximate $\Delta\overline{V}_\text{vap}\approx\overline{V}_g$.
        \item Additionally, we have that $\overline{V}_g=RT/P$.
        \item It follows that
        \begin{align*}
            \dv{P}{T} &= \frac{\Delta\overline{H}_\text{vap}}{T\overline{V}_g}\\
            &= \frac{P\Delta\overline{H}_\text{vap}}{RT^2}\\
            \frac{\dd{P}}{P} &= \frac{\Delta\overline{H}_\text{vap}}{R}\frac{\dd{T}}{T^2}\\
            \ln\frac{P}{P_0} &= \frac{\Delta\overline{H}_\text{vap}}{R}\left( -\frac{1}{T}+\frac{1}{T_0} \right)
        \end{align*}
        \item It follows that if $P_0=\SI{1}{\atmosphere}$ and $T_0=T_b$ (the standard boiling temperature), then the vapor pressure $P$ at temperature $T$ is
        \begin{equation*}
            P = P_0\exp\left[ \frac{\Delta\overline{H}_\text{vap}}{R}\left( \frac{1}{T_b}-\frac{1}{T} \right) \right]
        \end{equation*}
        \begin{itemize}
            \item Note that we take $\Delta\overline{H}_\text{vap}$ to be the molar heat of vaporization at temperature $T$, i.e., we assume it's constant from there up until $T_b$ which it technically isn't as we mentioned earlier.
        \end{itemize}
        \item Note that you can also use this equation and $(P,T)$ data to calculate $\Delta H_\text{vap}$.
    \end{itemize}
    \item Relative slopes of S-G and L-G coexistence lines at the triple point.
    \begin{itemize}
        \item When drawing a phase diagram, you should exaggerate the discontinuity in the slopes of the S-G and L-G coexistence curves at the triple point.
        \item In particular, $\dv*{P_\text{SG}}{T}>\dv*{P_\text{LG}}{T}$ since $\Delta\overline{H}_\text{SG}=\Delta\overline{H}_\text{SL}+\Delta\overline{H}_\text{LG}>\Delta\overline{H}_\text{LG}$ and $\Delta\overline{V}_\text{SG}\approx\Delta\overline{V}_\text{LG}$.
        \item Quantitatively, the ratio of the slopes is
        \begin{equation*}
            \frac{\dv*{P_\text{SG}}{T}}{\dv*{P_\text{LG}}{T}} = \frac{\Delta\overline{H}_\text{SG}}{\Delta\overline{H}_\text{LG}}
            = 1+\frac{\Delta\overline{H}_\text{SL}}{\Delta\overline{H}_\text{LG}}
        \end{equation*}
    \end{itemize}
\end{itemize}



\section{Chemical Potential and Raoult's Law}
\begin{itemize}
    \item \marginnote{2/25:}Today we begin the topic of solutions.
    \begin{itemize}
        \item We'll introduce ideal solutions and then discuss nonideal solutions.
        \item Today: Ideal solutions, i.e., solutions in which we negate the interactions between the solutes. The statistical entropy gives rise to boiling point elevation, freezing point depression, Raoult's law, and osmotic pressure.
    \end{itemize}
    \item Chemical potential.
    \begin{itemize}
        \item To discuss equilibrium, we start by discussing the energy of a solution $G(T,P,n_1,n_2)$ where $n_1$ is the number of moles of solute and $n_2$ is the number of moles of solvent.
        \item The total differential is
        \begin{equation*}
            \dd{G} = \left( \dv{G}{T} \right)_{P,n_1,n_2}\dd{T}+\left( \dv{G}{P} \right)_{T,n_1,n_2}\dd{P}+\left( \dv{G}{n_1} \right)_{T,P,n_2}\dd{n_1}+\left( \dv{G}{n_2} \right)_{T,P,n_1}\dd{n_2}
        \end{equation*}
    \end{itemize}
    \item \textbf{Chemical potential}: The following partial derivative. \emph{Denoted by} $\bm{\mu_1}$. \emph{Given by}
    \begin{equation*}
        \mu_i = \left( \dv{G}{n_i} \right)_{T,P,n_j}
    \end{equation*}
    for all $j\neq i$.
    \begin{itemize}
        \item For a pure substance, chemical potential is the molar free energy for a given ratio of concentration.
        \item $\mu_i$ is a function of $T,P,n_j\neq n_i$.
    \end{itemize}
    \item Equilibrium in terms of relation between chemical potentials.
    \begin{itemize}
        \item Consider a system in equilibrium between the liquid and gas phases where both phases have two components $n_1^l,n_2^l$ and $n_1^g,n_2^g$.
        \item We know that the free energy $G$ is the sum of the free energies of the two systems $G^l,G^g$.
        \item Since we are in equilibrium, $\dd{G}=\dd{G^l}+\dd{G^g}=0$.
        \item At constant $T,P$, it follows that
        \begin{equation*}
            \mu_1^l\dd{n_1^l}+\mu_2^l\dd{n_2^l}+\mu_1^g\dd{n_1^g}+\mu_2^g\dd{n_2^g} = 0
        \end{equation*}
        \item Furthermore, matter is conserved, i.e., $\dd{n_i^l}=-\dd{n_i^g}$.
        \item Thus, we may write
        \begin{equation*}
            (\mu_1^l-\mu_1^g)\dd{n_1^l}+(\mu_2^l-\mu_2^g)\dd{n_2^l} = 0
        \end{equation*}
        \item It follows that $\mu_1^l-\mu_1^g=0$ and $\mu_2^l-\mu_2^g=0$, i.e., that the chemical potentials of species in each phase are equal at equilibrium.
    \end{itemize}
    \item \textbf{Raoult's law}: The partial pressure $P_i$ of each component in a solution is equal to its pure vapor pressure $P_i^*$ times its mole fraction $x_i$ in solution. \emph{Given by}
    \begin{equation*}
        P_i = x_iP_i^*
    \end{equation*}
    for all $i$.
    \begin{itemize}
        \item It's only the mole fraction of the solute that matters, not whether it's as big as a protein or as small as an ion.
        \item This is only in \textbf{ideal solutions}, however.
    \end{itemize}
    \item \textbf{Ideal solution}: A solution that has a small amount of solutes.
    \begin{itemize}
        \item In the same way that any gas is ideal at low pressure, any solution is ideal with few enough solutes.
    \end{itemize}
    \item Example of vapor pressure in equilibrium with an ideal solution of benzene and toluene.
    \begin{figure}[h!]
        \centering
        \begin{tikzpicture}[
            every node/.style=black
        ]
            \small
            \draw [stealth-stealth] (0,3) -- node[rotate=90,above=7mm]{$P$ (\si{\torr})} (0,0) -- node[below=5mm]{$x_{\ce{PhH}}$} (5.5,0);
            \draw [loosely dashed,very thin] (0,2.5) -- (5,2.5) -- (5,0);
    
            \footnotesize
            \node [below=1mm] {0};
            \draw (5,0.1) -- ++(0,-0.2) node[below]{1};
            \draw (0.1,1) -- ++(-0.2,0) node[left]{60};
            \draw (0.1,2.5) -- ++(-0.2,0) node[left]{150};
    
            \draw [grx,thick]
                (0,0) -- node[pos=0.6,below right]{$P_{\ce{PhH}}$} (5,2.5)
                (0,1) -- node[pos=0.6,above right]{$P_{\ce{PhMe}}$} (5,0)
                (0,1) -- node[pos=0.45,above,rotate=17]{$P_\text{tot}=P_{\ce{PhH}}+P_{\ce{PhMe}}$} (5,2.5)
            ;
        \end{tikzpicture}
        \caption{Raoult's law example.}
        \label{fig:RaoultsLaw}
    \end{figure}
    \item We'll see deviations from Figure \ref{fig:RaoultsLaw} later, such as Henry's law.
    \item Chemical potential of a component in an ideal solution.
    \begin{itemize}
        \item If the vapor is ideal, then
        \begin{equation*}
            \overline{G}^g(T,P) = {\overline{G}^\circ}^g(T)+RT\ln\frac{P}{P_0}
        \end{equation*}
        \item For a pure substance, we have
        \begin{align*}
            {G^\circ}^g(T)+RT\ln\frac{P}{P_0} &= {G^\circ}^l(T)\\
            P &= P_0\e[-\Delta G/RT]
        \end{align*}
        where $\Delta G={G^\circ}^g-{G^\circ}^l$.
        \item When we have a solution,
        \begin{align*}
            \overline{G}^l(T,x_1) &= \overline{G}^g(T)+RT\ln\frac{x_1P_1^*}{P_0}\\
            {\overline{G}^\circ}^l(T)+? &= \overline{G}^g(T)+RT\ln\frac{x_1P_1^*}{P_0}\\
            ? &= RT\ln\frac{x_1P_1^*}{P_0}-RT\ln\frac{P_0}{P_1^*}\\
            &= RT\ln x_1
        \end{align*}
        so
        \begin{equation*}
            {\overline{G}^\circ}^l(T,x_1) = {\overline{G}^\circ}^l(T)+RT\ln x_1
        \end{equation*}
        \item This result implies that the free energy of the pure substance with mole fraction $x_1$ in solution is equal to the free energy of the solution plus $RT\ln x_1$.
    \end{itemize}
    \item The chemical potential for an ideal solution is a pure entropy effect.
    \begin{itemize}
        \item For the mixing of two solutions, we have as with gases that
        \begin{align*}
            \Delta S &= k_B\ln W\\
            &= k_B\ln\frac{(N_1+N_2)!}{N_1!N_2!}\\
            &= k_B[(N_1+N_2)\ln(N_1+N_2)-(N_1+N_2)-(N_1\ln N_1-N_1)-(N_2\ln N_2-N_2)]\\
            &= k_B\left( N_1\ln\frac{N_1+N_2}{N_1}+N_2\ln\frac{N_1+N_2}{N_2} \right)\\
            &= -k_BN_A(n_1\ln x_1+n_x\ln x_2)\\
            &= -R(n_1\ln x_1+n_x\ln x_2)
        \end{align*}
        \item It follows that without enthalpic mixing (i.e., with $\Delta H=0$), we have
        \begin{align*}
            \Delta G &= \Delta H-T\Delta S\\
            &= RT(n_1\ln x_1+n_x\ln x_2)
        \end{align*}
        \item Thus, since $\Delta\overline{G}_1=RT\ln x_1$, $RT\ln x_1$ is purely from entropy!
        \item Note that also as before, the slope of $\Delta S$ vs $x_1$ is infinite at 0 and 1.
        \begin{itemize}
            \item This reflects the fact that mixing is a more purely entropic effect in the limit of very little solute (i.e., in the limit of an ideal solution).
        \end{itemize}
    \end{itemize}
\end{itemize}



\section{Office Hours (PGS)}
\begin{itemize}
    \item \marginnote{2/25:}Plotting the coexistence curve and determining the Maxwell equal area contraction line mathematically.
    \begin{itemize}
        \item You can do these things with Mathematica and numerical approximations.
        \item PGS does not know of an easy way to do this by hand. It looks like I would have to go all the way through the cubic formula and integrals.
    \end{itemize}
    \item I'm getting confused on treating the van der Waals equation as a cubic in $\overline{V}$, because it's graph doesn't ever cross the $x$-axis, and, in fact, the equation is asymptotic to both the $x$- and $y$-axes. So what are its roots, and exactly what properties of it as a cubic are preserved?
    \begin{itemize}
        \item It is cubic because it is cubic in $\overline{V}$ (and solving for $\overline{V}$ given $P,T$ requires solving a cubic), not because it looks like the plot of a cubic function.
    \end{itemize}
    \item Why do intermolecular interactions behave the same way as intramolecular bonding interactions? B/c we used the Lennard-Jones potential for bond distance initially, and now we've applied it to two molecules interacting through their polarity?
    \begin{itemize}
        \item The Lennard-Jones potential only applies to spherically symmetric distributions (e.g., not \ce{H2O} at short distances).
        \item Not molecules of strong dipole or charge transfer either.
        \item Thus perfect for \textbf{rare} (or inert) gases.
        \item $r^{-6}$ works well for any solids that are not charged.
    \end{itemize}
    \item What is the $\sigma$ in the hard sphere potential? Isn't it the radius of the hard sphere? Because \textcite{bib:McQuarrieSimon} says it's the diameter multiple times.
    \begin{itemize}
        \item If the molecule centered at the origin is a hard sphere of radius $\sigma/2$ and the molecule approaching the hard sphere centered at the origin is also a hard sphere of radius $\sigma/2$, then they won't interact until their centers are a distance $\sigma$ apart.
    \end{itemize}
    \item Why can we use the ideal gas law to relate $B_{iV}$ and $B_{iP}$?
    \item PSet 3, Question 2.
    \begin{itemize}
        \item Use the two state model from class. Use a partition function with two states for $N$ particles.
        \item Curie's law probably isn't gonna be valid in this regime.
        \item You need the heat capacity at 4 kelvin via the Debye model. You'll need the Debye temperature from the internet.
    \end{itemize}
\end{itemize}



\section{Chapter 16: The Properties of Gases}
\emph{From \textcite{bib:McQuarrieSimon}.}
\begin{itemize}
    \item Since the results of the Lennard-Jones potential can't be evaluated analytically, we often use other approximations that can be.
    \item \textbf{Hard-sphere potential}: The potential of a hard sphere of radius $\sigma$. \emph{Given by}
    \begin{equation*}
        u(r) =
        \begin{cases}
            \infty & r<\sigma\\
            0 & r>\sigma
        \end{cases}
    \end{equation*}
    \begin{itemize}
        \item This is the simplest representative potential.
        \item Despite its simplicity, it does account for the finite size of molecules, which turns out to be the dominant feature in determining the structure of liquids and solids.
        \item It does not account for intermolecular attractions, but since such attractions lessen as temperature increases, it is a good model for that condition.
    \end{itemize}
    \item Under the hard-sphere potential,
    \begin{align*}
        B_{2V}(T) &= -2\pi N_A\int_0^\infty\left( \e[-u(r)/k_BT]-1 \right)r^2\dd{r}\\
        &= -2\pi N_A\left[ \int_0^\sigma(0-1)r^2\dd{r}+\int_\sigma^\infty(1-1)r^2\dd{r} \right]\\
        &= \frac{2\pi\sigma^3N_A}{3}
    \end{align*}
    \begin{itemize}
        \item Thus, $B_{2V}(T)$ is equal to $1/2$ the volume of $N_A$ hard spheres of radius $\sigma$.
        \item Although this formula for the second virial coefficient is not temperature dependent, it is indeed a good approximation at high temperatures, just the same way the potential is.
    \end{itemize}
    \item \textbf{Square-well potential}: The potential of a hard-sphere of radius $\sigma$ that is attractive at a consistent level from its surface up until a finite distance away. \emph{Given by}
    \begin{equation*}
        u(r) =
        \begin{cases}
            \infty & r<\sigma\\
            -\varepsilon & \sigma<r<\lambda\sigma\\
            0 & r>\lambda\sigma
        \end{cases}
    \end{equation*}
    \begin{itemize}
        \item $\varepsilon$ is the depth of the well and $(\lambda-1)\sigma$ is its width.
    \end{itemize}
    \item Under the square-well potential,
    \begin{align*}
        B_{2V}(T) &= -2\pi N_A\left[ \int_0^\sigma(0-1)r^2\dd{r}+\int_\sigma^{\lambda\sigma}(\e[\varepsilon/k_BT]-1)r^2\dd{r}+\int_{\lambda\sigma}^\infty(1-1)r^2\dd{r} \right]\\
        &= -2\pi N_A\left[ -\frac{\sigma^3}{3}+\frac{\sigma^3}{3}(\lambda^3-1)(\e[\varepsilon/k_BT]-1)+0 \right]\\
        &= \frac{2\pi\sigma^3N_A}{3}[1-(\lambda^3-1)(\e[\varepsilon/k_BT]-1)]
    \end{align*}
    \item The agreement of the square-well potential with experimental data is very good, but it does have 3 adjustable parameters.
    \item Relating the second virial coefficient to the van der Waals constants.
    \begin{itemize}
        \item With the help of the expansion $1/(1-x)=1+x+x^2+\cdots$, we have that
        \begin{align*}
            P &= \frac{RT}{\overline{V}-b}-\frac{a}{\overline{V}^2}\\
            &= \frac{RT}{\overline{V}}\frac{1}{1-b/\overline{V}}-\frac{a}{\overline{V}^2}\\
            &= \frac{RT}{\overline{V}}\left[ 1+\frac{b}{\overline{V}}+\frac{b^2}{\overline{V}^2}+\cdots \right]-\frac{a}{\overline{V}^2}\\
            &= \frac{RT}{\overline{V}}+(RTB-a)\frac{1}{\overline{V}^2}+\frac{RTb^2}{\overline{V}^3}+\cdots
        \end{align*}
        \item It follows that
        \begin{equation*}
            Z = \frac{P\overline{V}}{RT} = 1+\left( b-\frac{a}{RT} \right)\frac{1}{\overline{V}}+\frac{b^2}{\overline{V}^2}+\cdots
        \end{equation*}
        \item Thus,
        \begin{equation*}
            B_{2V}(T) = b-\frac{a}{RT}
        \end{equation*}
        for the van der Waals equation.
    \end{itemize}
    \item Relating the van der Waals constants to the Lennard-Jones parameters.
    \begin{itemize}
        \item Consider the following intermolecular potential, which is a hybrid of the hard-sphere and Lennard-Jones potentials.
        \begin{equation*}
            u(r) =
            \begin{cases}
                \infty & r<\sigma\\
                -\frac{c_6}{r^6} & r>\sigma
            \end{cases}
        \end{equation*}
        \item We can now calculate $B_{2v}(T)$ in terms of $u(r)$ with the help of the approximation $\e[x]=1+x$ (applicable since the argument of the exponent function will be very small).
        \begin{align*}
            B_{2V}(T) &= -2\pi N_A\left[ \int_0^\sigma(0-1)r^2\dd{r}+\int_\sigma^\infty(\e[c_6/k_BTr^6]-1)r^2\dd{r} \right]\\
            &= \frac{2\pi\sigma^3N_A}{3}-\frac{2\pi N_Ac_6}{k_BT}\int_\sigma^\infty\frac{1}{r^6}\cdot r^2\dd{r}\\
            &= \frac{2\pi\sigma^3N_A}{3}-\frac{2\pi N_Ac_6}{k_BT}\left[ -\frac{1}{3r^3} \right]_\sigma^\infty\\
            &= \frac{2\pi\sigma^3N_A}{3}-\frac{2\pi N_Ac_6}{3k_BT\sigma^3}
        \end{align*}
        \item It follows by comparison with the result from the previous section that
        \begin{align*}
            a &= \frac{2\pi N_A^2c_6}{3\sigma^3}&
            b &= \frac{2\pi\sigma^3N_A}{3}
        \end{align*}
        \item Physical interpretations: $a\propto c_6$ and $b$ is one-half the volume of the molecules.
    \end{itemize}
    \item In a similar fashion, we can relate $B_{2V}(T)$ to the Redlich-Kwong constants and Peng-Robinson functions.
    \begin{align*}
        B_{2V}(T) &= B-\frac{A}{RT^{3/2}}&
        B_{2V}(T) &= \beta-\frac{\alpha}{RT}
    \end{align*}
\end{itemize}



\section{Chapter 22: Helmholtz and Gibbs Energies}
\emph{From \textcite{bib:McQuarrieSimon}.}
\begin{itemize}
    \item Generalizing the equation $\overline{G}=G^\circ+RT\ln Q$ to the case of a real gas.
    \begin{itemize}
        \item We begin by substituting the virial expansion in terms of pressure into the equation $(\pdv*{\overline{G}}{P})_T=\overline{V}$.
        \begin{align*}
            \left( \pdv{\overline{G}}{P} \right)_T &= \overline{V}\\
            &= \frac{RT}{P}\left[ 1+B_{2P}(T)P+B_{3P}(T)P^2+\cdots \right]\\
            \int_{P^\text{id}}^P\dd{\overline{G}} &= RT\int_{P^\text{id}}^P\frac{\dd{P'}}{P'}+RTB_{2P}(T)\int_{P^\text{id}}^P\dd{P'}+RTB_{3P}(T)\int_{P^\text{id}}^PP'\dd{P'}\\
            \overline{G}(T,P) &= \overline{G}(T,P^\text{id})+RT\ln\frac{P}{P^\text{id}}+RTB_{2P}(T)P+RTB_{3P}(T)\frac{P^2}{2}+\cdots
        \end{align*}
        \item Note that we neglect $P_\text{id}$ in every term except the first because it is so close to zero.
        \item Substituting $\overline{G}(T,P^\text{id})=G^\circ(T)+RT\ln P^\text{id}/P^\circ$ and combining the first two terms yields
        \begin{equation*}
            \overline{G}(T,P) = G^\circ(T)+RT\ln\frac{P}{P^\circ}+RTB_{2P}(T)P+RTB_{3P}(T)\frac{P^2}{2}+\cdots
        \end{equation*}
    \end{itemize}
    \item The above equation is exact, but complicated and different for each gas (depending on the virial coefficients).
    \item It will be more convenient to, especially for calculations involving chemical equilibria, to define the \textbf{fugacity}.
    \item \textbf{Fugacity}: A state function of $P$ and $T$ describing the nonideality of the energy of a system. \emph{Denoted by} $\bm{f}$. \emph{Given by}
    \begin{equation*}
        \overline{G}(T,P) = G^\circ(T)+RT\ln\frac{f(P,T)}{f^\circ}
    \end{equation*}
    \item Fugacity must have the property that $f(P,T)\to P$ as $P\to 0$, so that the above equation can reduce to $\overline{G}=G^\circ+RT\ln P/P^\circ$.
    \item By setting the above equal to the virial expansion, we learn that
    \begin{equation*}
        \frac{f(P,T)}{f^\circ} = \frac{P}{P^\circ}\exp[B_{2P}(T)P+B_{3P}(T)P^2+\cdots]
    \end{equation*}
    \item "The standard state of the real gas\dots is taken to be the corresponding ideal gas at one bar" \parencite[906]{bib:McQuarrieSimon}.
    \begin{itemize}
        \item Mathematically, $f^\circ=P^\circ$.
        \item Note that we can also derive this from the virial expansion expression for $f(P,T)/f^\circ$ since all of the virial coefficients are equal to zero in the standard state.
    \end{itemize}
    \item \textcite{bib:McQuarrieSimon} goes through the derivation of
    \begin{equation*}
        \ln\frac{f}{P} = \int_0^P\left( \frac{\overline{V}}{RT}-\frac{1}{P'} \right)\dd{P'}
    \end{equation*}
    exactly as in class.
    \begin{itemize}
        \item The above equation allows us to calculate the ratio of the fugacity to the pressure of a gas at any pressure and temperature, given either $P$-$V$-$T$ data or an equation of state.
    \end{itemize}
    \item \textbf{Fugacity coefficient}: The following ratio. \emph{Denoted by} $\bm{\gamma}$. \emph{Given by}
    \begin{equation*}
        \gamma = \frac{f}{P}
    \end{equation*}
    \item Also note the equation
    \begin{equation*}
        \ln\gamma = \int_0^P\frac{Z-1}{P'}\dd{P'}
    \end{equation*}
    and that we can use this equation with the reduced pressure.
\end{itemize}



\section{Chapter 23: Phase Equilibria}
\emph{From \textcite{bib:McQuarrieSimon}.}
\begin{itemize}
    \item \marginnote{2/27:}\textbf{Gibbs phase rule}: The number of components $C$ and the number of phases $P$ in a system are related to the number of degrees of freedom $F$ by the equation
    \begin{equation*}
        F = C-2-P
    \end{equation*}
    \item Describes phase diagrams.
    \item \textbf{Vapor pressure}: The pressure at which solid and gas or liquid and gas are in equilibrium.
    \item \textbf{Solid-gas coexistence curve}: The curve separating the solid and gas regions of a phase diagram.
    \begin{itemize}
        \item At points along this curve, the system will be in equilibrium between solid and gas.
        \item As such, this curve specifies the vapor pressure as a function of temperature.
    \end{itemize}
    \item \textbf{Triple point}: The point of intersection of the three lines in a phase diagram, at which the solid, liquid, and gaseous phases of the substance coexist at equilibrium.
    \item "Within a single-phase region, both the pressure and the temperature must be specified, and we say that there are two degrees of freedom within a single-phase region of a pure substance" \parencite[927]{bib:McQuarrieSimon}.
    \item "Along any of the coexistence curves, either the pressure or the temperature alone is sufficient to specify a point on the curve, so we say that there is one degree of freedom" \parencite[927]{bib:McQuarrieSimon}.
    \item "The triple point is a fixed point, so there are no degrees of freedom there" \parencite[927]{bib:McQuarrieSimon}.
    \item Thus, the number of degrees of freedom $f$ is related to the number of phases in equilibrium by the equation
    \begin{equation*}
        f = 3-p
    \end{equation*}
    \item \textbf{Normal melting point}: The temperature at which a substance melts under \SI{1}{\atmosphere} of pressure.
    \item \textbf{Standard melting point}: The temperature at which a substance melts under \SI{1}{\bar} of pressure.
    \item \textbf{Boiling point}: The temperature at which the vapor pressure equals the atmospheric pressure.
    \item \textbf{Normal boiling point}: The boiling point at \SI{1}{\atmosphere}.
    \item \textbf{Standard boiling point}: The boiling point at \SI{1}{\bar}.
    \item \textbf{Sublime}: To pass directly from the solid to the gas phase.
    \item \ce{H2O}, antimony, and bismuth all expand upon freezing.
    \item \textbf{Orthobaric densities}: The densities of two phases that are in equilibrium with each other (i.e., of a substance along a coexistence curve).
    \item The orthobaric densities of the liquid and gas phases approach each other as $T\to T_c$.
    \item $\Delta\overline{H}_\text{vap}$ decreases as $T\to T_c$.
    \begin{itemize}
        \item This is because $\Delta\overline{S}_\text{vap}\to 0$ as $T\to T_c$ (the phases become less distinct), so naturally $\Delta\overline{H}_\text{vap}=T\Delta\overline{S}_\text{vap}\to 0$ as $T\to T_c$.
    \end{itemize}
    \item \textbf{Critical opalescence}: The fluctuations between the liquid and vapor state of a fluid very near its critical point which scatter light very strongly, causing the substance in question to appear milky.
    \item Because of the critical point, it is possible to transform a gas into a liquid (or vice versa) without ever passing through a two-phase state. To do so, just follow a path out and around the critical point along the phase diagram.
    \item The solid-liquid coexistence curve does not end as abruptly as the liquid-gas coexistence curve since the differences between solid and liquid are ones of intrinsically different structure as opposed to degree of motion. Thus, the solid-liquid coexistence curve of a substance either continues on indefinitely or dead ends into another solid state (some substances have multiple, such as water, which can be solid even above its normal boiling point at very high pressures).
    \item Connecting the Gibbs energy of a substance to its phase diagram.
    \begin{figure}[h!]
        \centering
        \begin{tikzpicture}[
            every node/.style={black,align=center}
        ]
            \small
            \draw [stealth-stealth] (0,3) -- node[rotate=90,above]{$\overline{G}$ (\si[per-mode=symbol]{\kilo\joule\per\mole})} (0,0) -- node[below]{$T$ (\si{\kelvin})} (4,0);
    
            \footnotesize
            \draw [rex,thick] (0,2.9) -- node[below=2mm]{Solid} (1.6,1.9) -- node[below=4mm]{Liquid} (3.9,0);
            \draw [rex,semithick,dashed]
                (1.6,1.9) -- node[above=-1mm,xshift=5mm]{Superheated\\solid} (3.2,0.9)
                (1.6,1.9) -- node[above=-1mm,xshift=5mm]{Supercooled\\liquid} (0.3,2.97)
            ;
        \end{tikzpicture}
        \caption{The energetic stabilization of phase transitions.}
        \label{fig:gibbsExtrapolation}
    \end{figure}
    \begin{itemize}
        \item Recall that the plot of $\overline{G}(T)$ for a substance is a continuous curve with two discontinuities in the slope (one for each phase transition).
        \item Consider the point at one such discontinuity (say between the solid and liquid phases). If we were to extrapolate the liquid line to lower temperatures, we would be describing a \textbf{supercooled liquid} and vice versa with a \textbf{superheated solid}.
        \item However, a substance does not ordinarily exist as a liquid below its melting point because its Gibbs energy is reduced by transitioning to a solid. Similarly, a substance does not ordinarily exist as a solid above its melting point because its Gibbs energy is reduced by transitioning to a liquid.
        \item The influence of $G=H-TS$: At low temperatures, $TS$ is small, so we favor a small $H$ (and solids have the lowest enthalpy of the three phases). Likewise, at higher temperatures, a larger $H$ can be evened out by the large $TS$.
        \item Note that we can do the same kind of analysis for $\overline{G}(P)$. However, in this case, we have the chance that with some substances (such as water), we will go from gas to solid to liquid as pressure increases.
    \end{itemize}
    \item \textbf{Supercooled liquid}: A liquid with temperature below its freezing point that is not nevertheless not a solid.
    \item \textbf{Superheated solid}: A solid with temperature above its melting point.
    \item \textbf{Metastable state}: A state of a substance that has greater molar Gibbs energy than the usual state encountered under some conditions.
    \item Consider a system consisting of the gas and liquid phases of a pure substance in equilibrium with each other.
    \begin{itemize}
        \item If $G^l$ is the Gibbs energy of the liquid phase and $G^g$ is the Gibbs energy of the gaseous phase, then the total Gibbs energy $G$ of this system is $G=G^l+G^g$.
        \item Now if $\dd{n}$ moles are transferred from the liquid phase to the vapor phase at constant $T,P$, the corresponding change in Gibbs energy is
        \begin{equation*}
            \dd{G} = \left( \pdv{G^g}{n^g} \right)_{P,T}\dd{n^g}+\left( \pdv{G^l}{n^l} \right)_{P,T}\dd{n^l}
        \end{equation*}
        \item Since $\dd{n^l}=-\dd{n^g}$, it follows that
        \begin{equation*}
            \dd{G} = \left[ \left( \pdv{G^g}{n^g} \right)_{P,T}-\left( \pdv{G^l}{n^l} \right)_{P,T} \right]\dd{n^g}
        \end{equation*}
        \item Rewriting in terms of \textbf{chemical potentials}, we have
        \begin{equation*}
            \dd{G} = (\mu^g-\mu^l)\dd{n^g}
        \end{equation*}
    \end{itemize}
    \item \textbf{Chemical potential}: The change in Gibbs energy of a substance in a phase $\alpha$ with respect to the change in the number of moles of the substance present. \emph{Denoted by} $\bm{\mu^\alpha}$. \emph{Given by}
    \begin{equation*}
        \mu^\alpha = \left( \pdv{G^\alpha}{n^\alpha} \right)_{P,T}
    \end{equation*}
    \begin{itemize}
        \item "Just as electric current flows from a higher electric potential to a lower electric potential, matter `flows' from a higher chemical potential to a lower chemical potential" \parencite[937]{bib:McQuarrieSimon}.
        \item An equivalent definition is that since $G\propto n$ for any system, $\mu$ is the proportionality constant. Here's how we know the definitions are equivalent:
        \begin{equation*}
            \mu = \left( \pdv{G}{n} \right)_{P,T}
            = \left( \pdv{n\mu(T,P)}{n} \right)_{P,T}
            = \mu(T,P)
        \end{equation*}
        \begin{itemize}
            \item In other words, $\mu$ is an intensive quantity representing the same thing as Gibbs energy.
        \end{itemize}
    \end{itemize}
    \item If two phases are in equilibrium between liquid and gas, then since $\dd{G}=0$ and $\dd{n^g}\neq 0$, $\mu^g=\mu^l$.
    \item "If the two phases are not in equilibrium with each other, a spontaneous transfer of matter from one phase to the other will occur in the direction such that $\dd{G}<0$" \parencite[937]{bib:McQuarrieSimon}.
    \begin{itemize}
        \item For example, if $\mu^g>\mu^l$, we must have $\dd{n^g}<0$ for $\dd{G}$ to be negative, meaning that matter will transfer from the vapor phase to the liquid phase.
    \end{itemize}
    \item \textcite{bib:McQuarrieSimon} derives the equation for $\dv*{P}{T}$ exactly as in class.
    \item \textbf{Clapeyron equation}: The relation between the slope of the two-phase boundary line in a phase diagram and the values of $\Delta_\text{trs}\overline{H}$ and $\Delta_\text{trs}\overline{V}$ for a transition between those two phases. \emph{Given by}
    \begin{equation*}
        \dv{P}{T} = \frac{\Delta_\text{trs}\overline{H}}{T\Delta_\text{trs}\overline{V}}
    \end{equation*}
    \item Note that $\Delta_\text{fus}\overline{H}$ and $\Delta_\text{fus}\overline{V}$ do vary with pressure, but not always significantly.
    \item \textbf{Clausius-Clapeyron equation}: The relation between the vapor pressure of a liquid and its temperature. \emph{Given by}
    \begin{equation*}
        \dv{\ln P}{T} = \frac{\Delta_\text{vap}\overline{H}}{RT^2}
    \end{equation*}
    \begin{itemize}
        \item Two assumptions were made in the derivation of the Clausius-Clapeyron equation form the Clapeyron equation: $\overline{V}^g\gg\overline{V}^l$ and the vapor pressure is so low that the gas can be treated as ideal (so we can replace $\overline{V}^g$ with $RT/P$).
        \item Thus, the Clausius-Clapeyron equation is easier to use, but the Clapeyron equation is more general.
        \item Furthermore, if $\Delta_\text{vap}\overline{H}$ does not vary with temperature over the integration limits of $T$, then
        \begin{equation*}
            \ln\frac{P_2}{P_1} = -\frac{\Delta_\text{vap}\overline{H}}{R}\left( \frac{1}{T_2}-\frac{1}{T_1} \right)
        \end{equation*}
    \end{itemize}
    \item Taking the indefinite integral of the Clausius-Clapeyron equation shows that $\ln P$ and $1/T$ have a linear relation with slope $-\Delta_\text{vap}\overline{H}/R$.
    \begin{itemize}
        \item This relation can be used to experimentally measure $\Delta_\text{vap}\overline{H}$.
        \item By representing $\Delta_\text{vap}\overline{H}$ as a Taylor polynomial in $T$, we can also find more exact values for it based on $(P,T)$ data.
    \end{itemize}
    \item \textcite{bib:McQuarrieSimon} shows that the slope of the solid-gas coexistence curve is greater than the slope of the liquid-gas coexistence curve the same way PGS did in class.
    \item Deriving an expression for the chemical potential $\mu$ in terms of the partition function $Q$.
    \begin{itemize}
        \item We know that
        \begin{align*}
            U &= k_BT^2\left( \pdv{\ln Q}{T} \right)_{N,V}&
            S &= k_BT\left( \pdv{\ln Q}{T} \right)_{N,V}+k_B\ln Q
        \end{align*}
        \item It follows that
        \begin{align*}
            A &= U-TS\\
            &= -k_BT\ln Q
        \end{align*}
        \item Additionally, the total differential of $A(T,V,n)$ is
        \begin{align*}
            \dd{A} &= \left( \pdv{A}{T} \right)_{n,V}\dd{T}+\left( \pdv{A}{V} \right)_{n,T}\dd{V}+\left( \pdv{A}{n} \right)_{T,V}\dd{n}\\
            &= -S\dd{T}-P\dd{V}+\left( \pdv{A}{n} \right)_{T,V}\dd{n}
        \end{align*}
        and the total differential of $G(T,P,n)$ is
        \begin{align*}
            \dd{G} &= \left( \pdv{G}{T} \right)_{P,n}\dd{T}+\left( \pdv{G}{P} \right)_{T,n}\dd{P}+\left( \pdv{G}{n} \right)_{T,P}\dd{n}\\
            &= -S\dd{T}+V\dd{P}+\mu\dd{n}
        \end{align*}
        \item But since
        \begin{equation*}
            \dd{G} = \dd{A}+\dd{(PV)}
            = -S\dd{T}+V\dd{P}+\left( \pdv{A}{n} \right)_{T,V}\dd{n}
        \end{equation*}
        we have by direct comparison that
        \begin{equation*}
            \mu = \left( \pdv{G}{n} \right)_{T,P}
            = \left( \pdv{A}{n} \right)_{T,V}
        \end{equation*}
        \item It follows by substituting our previous expression for $A(Q)$ that
        \begin{equation*}
            \mu = -k_BT\left( \pdv{\ln Q}{n} \right)_{V,T}
            = -RT\left( \pdv{\ln Q}{N} \right)_{V,T}
        \end{equation*}
    \end{itemize}
    \item For an ideal gas where $Q=q^N/N!$, we have by Stirling's approximation that
    \begin{align*}
        \mu &= -RT\pdv{N}(\ln\frac{q^N}{N!})\\
        &= -RT\pdv{N}(N\ln q-N\ln N+N)\\
        &= -RT(\ln q-\ln N-1+1)\\
        &= -RT\ln\frac{q(V,T)}{N}
    \end{align*}
    \item Note that this expression for $\mu$ in terms of the partition function easily yields one for $G$ in terms of the partition function via $G=n\mu$.
    \item \textcite{bib:McQuarrieSimon} discusses alternate forms of $\mu$ and $\mu^\circ$.
\end{itemize}



\section{Chapter 24: Solutions I --- Liquid-Liquid Solutions}
\emph{From \textcite{bib:McQuarrieSimon}.}
\begin{itemize}
    \item \marginnote{2/28:}In this chapter, we consider in particular solutions of two volatile liquids.
    \item Having discussed the thermodynamics of one-component systems, we now move into the thermodynamics of two-component systems, though many of our results generalize to multicomponent solutions.
    \item \textcite{bib:McQuarrieSimon} analyzes $G(T,P,n_1,n_2)$ as in class.
    \item \textbf{Partial molar entropy}. The following quantity. \emph{Denoted by} $\overline{S}_j$. \emph{Given by}
    \begin{equation*}
        \overline{S}_j = \left( \pdv{S}{n_j} \right)_{T,P,n_i}
    \end{equation*}
    for all $i\neq j$.
    \item \textbf{Partial molar volume}. The following quantity. \emph{Denoted by} $\overline{V}_j$. \emph{Given by}
    \begin{equation*}
        \overline{V}_j = \left( \pdv{V}{n_j} \right)_{T,P,n_i}
    \end{equation*}
    for all $i\neq j$.
    \item In general, extensive thermodynamic properties $Y$ have associated partial molar quantities $\overline{Y}_j$ defined analogously to the above.
    \item \textbf{Binary solution}: A solution composed of two different liquids.
    \item \marginnote{3/1:}Deriving a formula for $G$ in terms of $n_1,n_2$ at constant $T,P$.
    \begin{itemize}
        \item At constant $T,P$, we have that
        \begin{equation*}
            \dd{G} = \mu_1\dd{n_1}+\mu_2\dd{n_2}
        \end{equation*}
        \item Now imagine that we increase the amount of each component linearly from 0 to $n_i$.
        \item Then the amount of each component present at a given time can be represented as $n(\lambda)=n_i\lambda$ where $\lambda$ varies from 0 to 1.
        \item It follows that $\dd{n_i}=n_i\dd{\lambda}$ and, since $G$ is extensive, $\dd{G}=G\dd{\lambda}$.
        \item Therefore, it follows from the above equation that
        \begin{align*}
            \int_0^1G\dd{\lambda} &= \int_0^1n_1\mu_1\dd{\lambda}+\int_0^1n_2\mu_2\dd{\lambda}\\
            G\int_0^1\dd{\lambda} &= n_1\mu_1\int_0^1\dd{\lambda}+n_2\mu_2\int_0^1\dd{\lambda}\\
            G(T,P,n_1,n_2) &= \mu_1n_1+\mu_2n_2
        \end{align*}
    \end{itemize}
    \item Similarly,
    \begin{equation*}
        Y = \sum_{i=1}^m\overline{Y}_in_i
    \end{equation*}
    for any quantity $Y$ (like entropy or volume) in an $m$-component system.
    \item Note that this implies that the final volume of a solution is not necessarily a straight sum of the component volumes, but rather depends on the partial molar volumes.
    \item Extending $\dd{G}=-S\dd{T}+V\dd{P}$ into multicomponent systems.
    \begin{itemize}
        \item If we consider the Gibbs energy as $G(T,P,n_1,\dots,n_m)$, then since $G=H-TS$, we have that
        \begin{equation*}
            \left( \pdv{G}{n_j} \right)_{T,P,n_i} = \left( \pdv{H}{n_j} \right)_{T,P,n_i}-T\left( \pdv{S}{n_j} \right)_{T,P,n_i}
        \end{equation*}
        \item It follows that
        \begin{equation*}
            \mu_j = \overline{G}_j = \overline{H}_j-T\overline{S}_j
        \end{equation*}
        \item Using the Maxwell relations
        \begin{align*}
            \overline{S}_j &= -\left( \pdv{\mu_j}{T} \right)_{P,n_i}&
            \overline{V}_j &= \left( \pdv{\mu_j}{P} \right)_{T,n_i}
        \end{align*}
        and the total differential of $\mu_j$, we can determine that
        \begin{equation*}
            \dd{\mu_j} = -\overline{S}_j\dd{T}+\overline{V}_j\dd{P}
        \end{equation*}
    \end{itemize}
    \item Similarly, the Gibbs-Helmholtz equation for multicomponent systems is
    \begin{equation*}
        \left( \pdv{\mu_j/T}{T} \right)_P = -\frac{\overline{H}_j}{T^2}
    \end{equation*}
    \item \textbf{Gibbs-Duhem equation}: The relation between the chemical potential of one component in a binary solution as a function of composition to the other. \emph{Given by}
    \begin{equation*}
        n_1\dd{\mu_1}+n_2\dd{\mu_2} = 0
    \end{equation*}
    \emph{or}
    \begin{equation*}
        x_1\dd{\mu_1}+x_2\dd{\mu_2} = 0
    \end{equation*}
    \begin{itemize}
        \item Derivation: The total differential of $\dd{G}=n_1\mu_1+n_2\mu_2$ is
        \begin{equation*}
            \dd{G} = \mu_1\dd{n_1}+\mu_2\dd{n_2}+n_1\dd{\mu_1}+n_2\dd{\mu_2}
        \end{equation*}
        from which we can subtract $\dd{G}=\mu_1\dd{n_1}+\mu_2\dd{n_2}$ to get the former form and divide that by $n_1+n_2$ to get the latter form.
    \end{itemize}
    \item Note that we can also derive Gibbs-Duhem type equations for other intensive properties (such as the partial molar volume).
    \item $\bm{Y^*}$: A property of a pure substance.
    \begin{itemize}
        \item For example, $\mu_1$ denotes the chemical potential of substance 1 (which may be affected by $n_2,\dots,n_m$) whereas $\mu_1^*$ denotes $\mu_1$ at $x_1=1$.
    \end{itemize}
    \item If one component of a binary solution obeys \textbf{Raoult's law} over the complete concentration range, the other component does, too.
    \begin{itemize}
        \item Suppose the chemical potential of one component in a binary solution is given by
        \begin{equation*}
            \mu_2 = \mu_2^*+RT\ln x_2
        \end{equation*}
        \item Then
        \begin{align*}
            \dd{\mu_1} &= -\frac{x_2}{x_1}\dd{\mu_2}\\
            &= -\frac{x_2}{x_1}\cdot RT\frac{\dd{x_2}}{x_2}\\
            &= -RT\frac{\dd{x_2}}{x_1}\\
            &= RT\frac{\dd{x_1}}{x_1}\\
            \int_1^{x_1}\dd{\mu_1} &= RT\int_1^{x_1}\frac{\dd{x_1'}}{x_1'}\\
            \mu_1 &= \mu_1^*+RT\ln x_1
        \end{align*}
    \end{itemize}
    \item \textcite{bib:McQuarrieSimon} derives, exactly as in class, $\mu_j^\alpha=\mu_j^\beta$ for any component $j$ and two phases $\alpha,\beta$ in equilibrium.
    \item An important consequence of this is that "the chemical potential of each component in liquid solution can be measured by the chemical potential of that component in the vapor phase" \parencite[970]{bib:McQuarrieSimon}.
    \begin{itemize}
        \item If the pressure $P_j$ of the vapor phase is sufficiently low\footnote{Technically, we should use fugacities throughout, but vapor pressures under standard conditions typically are small enough so that the difference between the pressure and the fugacity is negligible.}, we may treat it as ideal and employ
        \begin{equation*}
            \mu_j^\text{sln} = \mu_j^\text{vap}
            = \mu_j^\circ(T)+RT\ln P_j
        \end{equation*}
        \item It follows if $j$ is pure ($x_j=1$) that
        \begin{equation*}
            \mu_j^*(l) = \mu_j^\circ(T)+RT\ln P_j^*
        \end{equation*}
        \item By subtracting the above two equations, we can derive an expression for the the chemical potential of component $j$ in solution in terms of its chemical potential as a pure substance, its vapor pressure, and its vapor pressure as a pure substance.
        \begin{equation*}
            \mu_j^\text{sln} = \mu_j^*(l)+RT\ln\frac{P_j}{P_j^*}
        \end{equation*}
    \end{itemize}
    \item \textbf{Raoult's law}: The partial vapor pressure of each component of a solution is directly proportional to its mole fraction in solution. \emph{Given by}
    \begin{equation*}
        P_j = x_jP_j^*
    \end{equation*}
    \begin{itemize}
        \item Note that $x_j$ reflects the fraction of the solution \emph{surface} that is occupied by molecules of type $j$ (since it is these molecules that can escape into the vapor phase).
    \end{itemize}
    \item \textbf{Ideal solution}: A solution that obeys Raoult's law over the entire composition range.
    \item At a molecular level, ideal binary solutions look like two types of molecules distributed randomly throughout solution.
    \begin{itemize}
        \item This occurs when the molecules are of similar size and shape, and the intermolecular forces are all similar.
    \end{itemize}
    \item Between our expression for $\mu_j^\text{sln}$ in terms of the characteristics of the pure substance and Raoult's law, we have that
    \begin{equation*}
        \mu_j^\text{sln} = \mu_j^*(l)+RT\ln x_j
    \end{equation*}
    \item Deriving an expression for the total vapor pressure of an ideal binary solution.
    \begin{align*}
        P_\text{tot} &= P_1+P_2\\
        &= x_1P_1^*+x_2P_2^*\\
        &= (1-x_2)P_1^*+x_2P_2^*\\
        &= P_1^*+x_2(P_2^*-P_1^*)
    \end{align*}
    \begin{itemize}
        \item The above formula is a straight line that evaluates to $P_1^*$ at $x_2=0$ and $P_2^*$ at $x_2=1$, as we would expect (see Figure \ref{fig:RaoultsLaw}).
    \end{itemize}
    \item Deriving an expression for the total vapor pressure of a binary solution in terms of the mole fraction $y_2$ of component 2 in the vapor phase.
    \begin{itemize}
        \item By Dalton's law of partial pressures, which asserts that the partial pressure $P_2$ of the vapor of component 2 is equal to its mole fraction $y_2$ in the vapor phase, we have that
        \begin{equation*}
            y_2 = \frac{P_2}{P_\text{tot}}
            = \frac{x_2P_2^*}{P_\text{tot}}
            = \frac{x_2P_2^*}{P_1^*+x_2(P_2^*-P_1^*)}
        \end{equation*}
        \item Solving for $x_2$ yields
        \begin{equation*}
            x_2 = \frac{P_1^*y_2}{P_2^*-y_2(P_2^*-P_1^*)}
        \end{equation*}
        \item Therefore,
        \begin{align*}
            P_\text{tot} &= P_1^*+x_2(P_2^*-P_1^*)\\
            &= P_1^*+\frac{P_1^*y_2}{P_2^*-y_2(P_2^*-P_1^*)}(P_2^*-P_1^*)
        \end{align*}
    \end{itemize}
    \item \textbf{Pressure-composition diagram}: A diagram that plots the total vapor pressure as a function of the composition of both the liquid phase and the vapor phase.
    \begin{figure}[H]
        \centering
        \begin{tikzpicture}[xscale=5,yscale=0.1]
            \small
            \draw [stealth-stealth] (0,55) -- node[rotate=90,above=5mm]{$P_\text{tot}$ (\si{\torr})} (0,20) -- node[below=5mm]{${\color{orx}x_2}$ / ${\color{yex}y_2}$} (1.1,20);
    
            \footnotesize
            \node at (0.25,42) {Liquid};
            \node at (0.65,25) {Vapor};
            \node [rotate=26] at (0.55,32) {Liquid and vapor};
            \node [below right,yshift=-1mm] at (0,20) {0.0};
            \node [above left,xshift=-1mm]  at (0,20) {20};
            \foreach \x in {0.2,0.4,0.6,0.8,1.0} {
                \draw (\x,21) -- ++(0,-2) node[below]{$\x$};
            }
            \foreach \y in {30,40,50} {
                \draw (0.02,\y) -- ++(-0.04,0) node[left]{$\y$};
            }
    
            \draw [orx,thick] plot[domain=0:1] (\x,{20.9+\x*(45.2-20.9)});
            \draw [yex,thick] plot[domain=0:1] (\x,{20.9+(20.9*\x)/(45.2-\x*(45.2-20.9))*(45.2-20.9)});
        \end{tikzpicture}
        \caption{Pressure-composition diagram for a 1-propanol/2-propanol solution.}
        \label{fig:PCdiagram}
    \end{figure}
    \begin{itemize}
        \item In Figure \ref{fig:PCdiagram}, the brown line plots the total vapor pressure as a function of the composition of the liquid phase, and the yellow line plots the total vapor pressure as a function of the composition of the vapor phase (as per the equations derived above).
        \item A vertical line in Figure \ref{fig:PCdiagram} corresponds to a single solution as pressure varies. If the vertical line is not at zero or 1, we see that high pressures yield exclusively the liquid phase, low pressures yield exclusively the vapor phase, and there is a region of pressures where both liquid and vapor coexist.
    \end{itemize}
    \item \textbf{Tie line}: A horizontal line in a pressure-composition diagram that, for a given pressure, connects the concentration that will give entirely liquid to the composition that will give entirely vapor.
    \item Determining the relative amounts of liquid and vapor.
    \begin{itemize}
        \item The mole fractions of component 2 in the liquid and vapor phases are, respectively,
        \begin{align*}
            x_2 &= \frac{n_2^l}{n_1^l+n_2^l} = \frac{n_2^l}{n^l}&
            y_2 &= \frac{n_2^g}{n_1^g+n_2^g} = \frac{n_2^g}{n^g}
        \end{align*}
        \item The overall mole fraction of component 2 is
        \begin{equation*}
            x_a = \frac{n_2^l+n_2^g}{n^l+n^g}
        \end{equation*}
        \item It follows that
        \begin{align*}
            x_a(n^l+n^g) &= x_2n^l+y_2n^g
            \frac{n^l}{n^g} &= \frac{y_2-x_a}{x_a-x_2}
        \end{align*}
    \end{itemize}
    \item \textbf{Lever rule}: The balance of the number of moles in each phase and the excess fractions of liquid and vapor. \emph{Given by} the above.
    \begin{itemize}
        \item "Note that $n^l=0$ when $x_a=y_2$ (vapor curve) and that [$n^g=0$] when $x_a=x_2$ (liquid curve)" \parencite[973]{bib:McQuarrieSimon}.
    \end{itemize}
    \item \textbf{Temperature-composition diagram}: A diagram that plots the composition of the solution and vapor phases at various temperatures.
    \begin{itemize}
        \item Pressure is held constant in such a diagram, so choose \SI{760}{\torr}.
        \item It follows that the mole percent of component 1 at this pressure is
        \begin{align*}
            \SI{760}{\torr} &= P_2^*-x_1(P_2^*-P_1^*)\\
            x_1 &= \frac{P_2^*-\SI{760}{\torr}}{P_2^*-P_1^*}
        \end{align*}
        \item We can then find $P_1^*,P_2^*$ at a given temperature either from a table the Clausius-Clapeyron equation (though the latter will be an approximation).
        \item Lastly, we can find $y_1$ as before.
    \end{itemize}
    \item Fractional distillation in a temperature-composition diagram.
    \begin{figure}[h!]
        \centering
        \begin{tikzpicture}[xscale=5,yscale=0.1]
            \small
            \draw [stealth-stealth] (0,55) -- node[left]{$t$ (\si{\celsius})} (0,20) -- node[below]{${\color{orx}x_2}$ / ${\color{yex}y_2}$} (1.1,20);
    
            \footnotesize
            \node at (0.25,42) {Vapor};
            \node at (0.65,25) {Solution};
    
            \draw [orx,thick] plot[domain=0:1] (\x,{20.9+\x*(45.2-20.9)});
            \draw [yex,thick] (0,20.9) to[bend left=1.5] (1,45.2);
    
            \draw [dashed] (0.5,33) -- ++(-0.185,0) -- ++(0,-4.4) -- ++(-0.155,0) -- ++(0,-3.8) -- ++(-0.093,0) -- ++(0,-2.2);
        \end{tikzpicture}
        \caption{Temperature-composition diagram for a 1-propanol/2-propanol solution.}
        \label{fig:TCdiagram}
    \end{figure}
    \begin{itemize}
        \item During fractional distillation, a substance is repeatedly heated up, some vapor removed, and then heated up higher as now possible.
        \item In Figure \ref{fig:TCdiagram}, we can see how at any given ratio, we can only heat up the liquid so much before we begin vaporizing both liquids, but by removing some vapor (increasing the concentration of the less volatile substance), we can now heat it up more.
        \item The diagram also provides a pictoral representation of the law of diminishing returns (as we can never get all the way over to the top-right corner).
    \end{itemize}
    \item Calculating the energy of mixing $\Delta_\text{mix}G^\text{id}$ of two ideal solutions of pure components.
    \begin{align*}
        \Delta_\text{mix}G^\text{id} &= G^\text{sln}(T,P,n_1,n_2)-G_1^*(T,P,n_1)-G_2^*(T,P,n_2)\\
        &= n_1\mu_1^\text{sln}+n_2\mu_2^\text{sln}-n_1\mu_1^*-n_2\mu_2^*\\
        &= n_1(\mu_1^*+RT\ln x_1)+n_2(\mu_2^*+RT\ln x_2)-n_1\mu_1^*-n_2\mu_2^*\\
        &= RT(n_1\ln x_1+n_2\ln x_2)
    \end{align*}
    \begin{itemize}
        \item Since $x_1,x_2<1$, we know $\Delta_\text{mix}G^\text{id}<1$, i.e., ideal solutions always form spontaneously from their separate components.
    \end{itemize}
    \item The entropy of mixing of two ideal solutions of pure components.
    \begin{equation*}
        \Delta_\text{mix}S^\text{id} = -\left( \pdv{\Delta_\text{mix}G^\text{id}}{T} \right)_{P,n_1,n_2}
        = -R(n_1\ln x_1+n_2\ln x_2)
    \end{equation*}
    \begin{itemize}
        \item This result is identical to that for the mixing of two ideal gases.
        \item However, there are marked differences between solutions and gases, namely in the extent of intermolecular interactions. The similarity is accounted for by the fact that "in an ideal solution, the interactions in the mixture and those in the pure liquids are essentially identical" \parencite[977]{bib:McQuarrieSimon}.
    \end{itemize}
    \item We can also calculate that
    \begin{align*}
        \Delta_\text{mix}V^\text{id} &= \left( \pdv{\Delta_\text{mix}G^\text{id}} \right)_{T,n_1,n_2} = 0&
        \Delta_\text{mix}H^\text{id} &= \Delta_\text{mix}G^\text{id}+T\Delta_\text{mix}S^\text{id} = 0
    \end{align*}
    \begin{itemize}
        \item These results follow from the similarities in particles shape and size, and interaction energy, respectively, assumed for an ideal solution.
    \end{itemize}
\end{itemize}




\end{document}