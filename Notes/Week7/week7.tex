\documentclass[../notes.tex]{subfiles}

\pagestyle{main}
\renewcommand{\chaptermark}[1]{\markboth{\chaptername\ \thechapter\ (#1)}{}}
\setcounter{chapter}{6}

\begin{document}




\chapter{Phase Diagrams and Critical Points}
\section{Virial Coefficients and Fugacity}
\begin{itemize}
    \item \marginnote{2/21:}Relation between the interaction potential and the first virial coefficient (Equation 16.25).
    \begin{itemize}
        \item Statistical mechanics is important because it gives us the relation
        \begin{equation*}
            B_{2V}(T) = -2\pi N_A\int_0^\infty\left( \e[-u(r)/k_BT]-1 \right)r^2\dd{r}
        \end{equation*}
        \begin{itemize}
            \item We can derive this with our knowledge of statistical mechanics, but PGS will not go through this.
        \end{itemize}
        \item Now recall the Lennard-Jones potential
        \begin{equation*}
            u(r) = 4\epsilon\left[ \left( \frac{\sigma}{\pi} \right)^{12}-\left( \frac{\sigma}{\pi} \right)^6 \right]
        \end{equation*}
        \begin{itemize}
            \item Note that the minimum is at $(2^{1/6}\sigma,-\epsilon)$.
            \item The limiting case of the Lennard-Jones potential is hard sphere repulsion (the repulsion as you approach a hard sphere, which is zero up until you're at the surface and then infinite repulsion). Thus, with no intermolecular attraction, $a=0$, so in this case,
            \begin{align*}
                B_{2V}(T) &= -2\pi N_A\int_0^\sigma(-1)r^2\dd{r}\\
                &= \frac{2\pi N_A\sigma^3}{3}\\
                &= b-\frac{0}{RT}
            \end{align*}
        \end{itemize}
        \item Now consider a potential that is van der Waals ($c/r^6$) up until a point and then hard sphere. In this case,
        \begin{align*}
            B_{2V}(T) &= \frac{2\pi N_A\sigma^3}{3}-2\pi N_A\int_\sigma^\infty\left( \e[-c/r^6k_BT]-1 \right)r^2\dd{r}\\
            &= \frac{2\pi N_A\sigma^3}{3}-2\pi N_A\int_\sigma^\infty\left( -\frac{c}{r^6k_BT} \right)r^2\dd{r}\\
            &= b+\frac{2\pi N_Ac}{k_BT}\cdot-\frac{1}{3\sigma^3}
        \end{align*}
        where we have used $\e[x]=1+x+\cdots$ to get from the first line to the second.
        \begin{itemize}
            \item Therefore,
            \begin{equation*}
                a = \frac{2\pi N_A^2}{3}\frac{c}{\sigma^2}
            \end{equation*}
        \end{itemize}
    \end{itemize}
    \item Derivation of the relation between $B_{2V}(T)$ and the interaction potential $u(r)$.
    \begin{itemize}
        \item Consider a system of independent, indistinguishable particles.
        \item The total Hamiltonian for the system has a kinetic energy part and an interaction part.
        \begin{equation*}
            \hat{H}(p_i,r_i) = \sum_i\frac{\hat{p}_i^2}{2m}+\sum_{i<j}\hat{u}(r_i,r_j)
        \end{equation*}
        \item The kinetic part (which ignores intermolecular interactions) will lead to the ideal gas partition function. The nonideal part of the partition function will come from the interaction potentials. Mathematically,
        \begin{align*}
            Q &= \frac{1}{N!}\int\e[-\beta E(p_i,r_i)]\dd[3]{p_i}\dd[3]{r_i}\\
            &= \frac{1}{N!}\int\exp\left\{ -\beta\left[ \sum_i\frac{p_i^2}{2m}+\sum_{i<j}u(r_i,r_j) \right] \right\}\dd[3]{p_i}\dd[3]{r_i}\\
            &= \frac{1}{N!}\left( \int\exp\left[ -\beta\sum_i\frac{p_i^2}{2m} \right]\dd[3]{p_i} \right)\left( \int\exp\left[ -\beta\sum_{i<j}u(r_i,r_j) \right]\dd[3]{r_i} \right)\\
            &= \underbrace{\frac{V^N}{N!}\left( \int\exp\left[ -\beta\sum_i\frac{p_i^2}{2m} \right]\dd[3]{p_i} \right)}_{Q_\text{ideal}}\cdot\underbrace{\frac{1}{V^N}\left( \int\exp\left[ -\beta\sum_{i<j}u(r_i,r_j) \right]\dd[3]{r_i} \right)}_{Q_u}
        \end{align*}
        \item Define
        \begin{equation*}
            f_{ij} = \e[-u(r_i,r_j)/k_BT]-1
        \end{equation*}
        \item Now note that the interaction between molecules is pretty small, and in fact $f_{ij}\to 0$ as $|r_i-r_j|\to\infty$.
        \item Thus,
        \begin{align*}
            Q_u &= \frac{1}{V^N}\int\exp\left[ -\beta\sum_{i<j}u(r_i,r_j) \right]\dd[3]{r_i}\\
            &= \frac{1}{V^N}\int\prod_{i<j}(f_{ij}+1)\dd[3]{r_i}
        \end{align*}
        \item We can do a \textbf{cluster expansion} on this small $f_{ij}$:
        \begin{equation*}
            \prod_{i<j}(f_{ij}+1) = 1+\sum_{i<j}f_{ij}+\sum_{i<j}\sum_{k<\ell}f_{ij}f_{k\ell}
        \end{equation*}
        \item In particular, $\sum_{i<j}$ is the sum of pairwise interactions while $f_{ij}\cdot f_{k\ell}$ are binary interactions, $f_{ij}f_{k\ell}f_{mn}$ are tertiary interactions, and so on and so forth.
        \item But at low density, the dominant term is the pairwise interaction so we have
        \begin{align*}
            Q_u &= \frac{1}{V^N}\int\left( 1+\sum_{i<j}f_{ij} \right)\dd[3]{r_i}\\
            &= \frac{1}{V^N}\left( V^N+\frac{N(N-1)}{2}V^{N-2}\int f_{12}\dd[3]{r_1}\dd[3]{r_2} \right)\\
            &= 1+\frac{N(N-1)}{2V}\int(\e[-\beta u(r)]-1)\dd[3]{r}
        \end{align*}
        \item It follows that
        \begin{equation*}
            Q = Q_\text{id}\left[ 1+\frac{N(N-1)}{2V}\int\left( \e[-\beta u(r)]-1 \right)\dd[3]{r} \right]
        \end{equation*}
        \item But now we need to extract an equation of state from the partition function. To do so, we use
        \begin{align*}
            P &= k_BT\left( \pdv{\ln Q}{V} \right)_{N,T}\\
            &= k_BT\left( \pdv{\ln Q_\text{id}}{V} \right)_{N,T}+k_BT\left( \pdv{\ln Q_u}{V} \right)_{N,T}
        \end{align*}
        \item We know that the first term above is equal to $Nk_BT/V$, but it takes a bit more work for the second one.
        \item We have that
        \begin{align*}
            \ln Q_u &= \ln\Bigg( 1+\underbrace{\frac{N(N-1)}{2V}}_{\substack{\text{Approximately}\\\text{the intermolecular}\\\text{distance }1/\rho^3}}\underbrace{\int\left( \e[-\beta u(r)]-1 \right)\dd[3]{r}}_{\substack{\text{Approximately}\\\text{the molecular}\\\text{volume }a^3}} \Bigg)\\
            &= \frac{N(N-1)}{2V}\int\left( \e[-\beta u(r)]-1 \right)\dd[3]{r}
        \end{align*}
        since the second term is a small number and the natural log of a small number plus 1 is approximately that small number.
        \item Thus,
        \begin{equation*}
            \left( \pdv{\ln Q_u}{V} \right)_{N,T} = -\frac{N(N-1)}{2V^2}\int\left( \e[-\beta u(r)]-1 \right)\dd[3]{r}
        \end{equation*}
        so
        \begin{align*}
            P &= \frac{Nk_BT}{V}-\frac{Nk_BT}{V}\frac{N-1}{2V}\int\left( \e[-\beta u(r)]-1 \right)\dd[3]{r}\\
            &= \frac{RT}{\overline{V}}-\frac{RT}{\overline{V}}\frac{N-1}{2V}\int\left( \e[-\beta u(r)]-1 \right)\dd[3]{r}
        \end{align*}
        \item Consequently,
        \begin{equation*}
            Z = \frac{P\overline{V}}{RT} = 1-\frac{N_A}{\overline{V}}\cdot\frac{1}{2}\int\left( \e[-\beta u(r)]-1 \right)\dd[3]{r}
        \end{equation*}
        \item Therefore, by comparison with the virial expansion,
        \begin{align*}
            B_{2V}(T) &= -\frac{N_A}{2}\int\left( \e[-\beta u(r)]-1 \right)\dd[3]{r}\\
            &= -\frac{N_A}{2}\int_0^\infty\left( \e[-\beta u(r)]-1 \right)4\pi r^2\dd{r}\\
            &= -2\pi N_A\int_0^\infty\left( \e[-u(r)/k_BT]-1 \right)r^2\dd{r}
        \end{align*}
    \end{itemize}
    \item Free energy as a function of $(T,P)$ for a real gas. Definition of fugacity and fugacity coefficients.
    \begin{itemize}
        \item We want to find $\Delta G(T,P)$.
        \item We have that $\dd{\overline{G}}=-\overline{S}\dd{T}+\overline{V}\dd{P}$. It follows that
        \begin{equation*}
            \left( \pdv{\overline{G}}{P} \right)_T = \overline{V}
        \end{equation*}
        \item Thus,
        \begin{equation*}
            \overline{G}(T,P) = \overline{G}(T,P_0)+\int_{P_0}^P\overline{V}\dd{P}
        \end{equation*}
        \item In the ideal case,
        \begin{align*}
            \overline{G}_\text{ideal}(T,P) &= \overline{G}_\text{ideal}(T,P_0)+\int_{P_0}^P\frac{RT}{P}\dd{P}\\
            &= \overline{G}_\text{ideal}(T,P_0)+RT\ln\frac{P}{P_0}
        \end{align*}
        \item In the nonideal case, we define a fugacity $f$ by
        \begin{equation*}
            \overline{G}_\text{ideal}(T,P) = \overline{G}_\text{ideal}(T,P_0)+RT\ln\frac{f}{P_0}
        \end{equation*}
        \begin{itemize}
            \item The second term in the above equation refers to the Gibbs free energy of an ideal gas at $P_0=\SI{1}{\bar}$ or $P_0=\SI{1}{\atmosphere}$. Note that even at $P_0=\SI{1}{\atmosphere}$, however, there is too much pressure for truly ideal behavior, so $f\neq P_0$.
        \end{itemize}
        \item Imagine that $\Delta\overline{G}_1$ takes us from a real gas at $(T,P)$ to an ideal gas at $(T,P)$. Then
        \begin{align*}
            \Delta\overline{G}_1 &= \overline{G}_\text{ideal}(T,P)-\overline{G}_\text{real}(T,P)\\
            &= \left[ \overline{G}_\text{ideal}(T,P_0)+RT\ln\frac{P}{P_0} \right]-\left[ \overline{G}_\text{ideal}(T,P_0)+RT\ln\frac{f}{P_0} \right]\\
            &= -RT\ln\frac{f}{P}
        \end{align*}
        \item Now let $\Delta\overline{G}_2$ take us from a real gas at $(T,P)$ to a real gas at $T$ and $P\to 0$, which will be the same as an ideal gas at $T$ and $P\to 0$. Then let $\Delta\overline{G}_3$ take us from this ideal gas at $T$ and $P\to 0$ to an ideal gas at $(T,P)$. It follows that
        \begin{align*}
            \Delta\overline{G}_2 &= -\int_{P\to 0}^P\overline{V}\dd{P'}&
            \Delta\overline{G}_2 &= \int_{P\to 0}^P\frac{RT}{P'}\dd{P'}
        \end{align*}
        \item Thus, since $\Delta\overline{G}_1=\Delta\overline{G}_2+\Delta\overline{G}_3$ ($G$ is a state function),
        \begin{equation*}
            -RT\ln\frac{f}{P} = \int_{P\to 0}^P\left( -\overline{V}+\frac{RT}{P'} \right)\dd{P'}
        \end{equation*}
        \item We then define $\gamma$ to be the \textbf{fugacity coefficient} by $\gamma=f/P$. It follows that
        \begin{equation*}
            \ln\gamma = \int_0^P\frac{Z-1}{P'}\dd{P'}
        \end{equation*}
    \end{itemize}
    \item Fugacity coefficient expressed in terms of the compressibility deviation from unity.
    \begin{itemize}
        \item At low temperature, $Z<1$, so $\gamma<1$ and hence $f<p$.
        \item At high pressure, $Z>1$ (excluded volume), so $\gamma>1$ and hence $f>P$.
    \end{itemize}
    \item Introduces phase diagrams and their notable properties.
\end{itemize}



\section{Chapter 16: The Properties of Gases}
\emph{From \textcite{bib:McQuarrieSimon}.}
\begin{itemize}
    \item \marginnote{2/25:}Since the results of the Lennard-Jones potential can't be evaluated analytically, we often use other approximations that can be.
    \item \textbf{Hard-sphere potential}: The potential of a hard sphere of radius $\sigma$. \emph{Given by}
    \begin{equation*}
        u(r) =
        \begin{cases}
            \infty & r<\sigma\\
            0 & r>\sigma
        \end{cases}
    \end{equation*}
    \begin{itemize}
        \item This is the simplest representative potential.
        \item Despite its simplicity, it does account for the finite size of molecules, which turns out to be the dominant feature in determining the structure of liquids and solids.
        \item It does not account for intermolecular attractions, but since such attractions lessen as temperature increases, it is a good model for that condition.
    \end{itemize}
    \item Under the hard-sphere potential,
    \begin{align*}
        B_{2V}(T) &= -2\pi N_A\int_0^\infty\left( \e[-u(r)/k_BT]-1 \right)r^2\dd{r}\\
        &= -2\pi N_A\left[ \int_0^\sigma(0-1)r^2\dd{r}+\int_\sigma^\infty(1-1)r^2\dd{r} \right]\\
        &= \frac{2\pi\sigma^3N_A}{3}
    \end{align*}
    \begin{itemize}
        \item Thus, $B_{2V}(T)$ is equal to $1/2$ the volume of $N_A$ hard spheres of radius $\sigma$.
        \item Although this formula for the second virial coefficient is not temperature dependent, it is indeed a good approximation at high temperatures, just the same way the potential is.
    \end{itemize}
    \item \textbf{Square-well potential}: The potential of a hard-sphere of radius $\sigma$ that is attractive at a consistent level from its surface up until a finite distance away. \emph{Given by}
    \begin{equation*}
        u(r) =
        \begin{cases}
            \infty & r<\sigma\\
            -\varepsilon & \sigma<r<\lambda\sigma\\
            0 & r>\lambda\sigma
        \end{cases}
    \end{equation*}
    \begin{itemize}
        \item $\varepsilon$ is the depth of the well and $(\lambda-1)\sigma$ is its width.
    \end{itemize}
    \item Under the square-well potential,
    \begin{align*}
        B_{2V}(T) &= -2\pi N_A\left[ \int_0^\sigma(0-1)r^2\dd{r}+\int_\sigma^{\lambda\sigma}(\e[\varepsilon/k_BT]-1)r^2\dd{r}+\int_{\lambda\sigma}^\infty(1-1)r^2\dd{r} \right]\\
        &= -2\pi N_A\left[ -\frac{\sigma^3}{3}+\frac{\sigma^3}{3}(\lambda^3-1)(\e[\varepsilon/k_BT]-1)+0 \right]\\
        &= \frac{2\pi\sigma^3N_A}{3}[1-(\lambda^3-1)(\e[\varepsilon/k_BT]-1)]
    \end{align*}
    \item The agreement of the square-well potential with experimental data is very good, but it does have 3 adjustable parameters.
    \item Relating the second virial coefficient to the van der Waals constants.
    \begin{itemize}
        \item With the help of the expansion $1/(1-x)=1+x+x^2+\cdots$, we have that
        \begin{align*}
            P &= \frac{RT}{\overline{V}-b}-\frac{a}{\overline{V}^2}\\
            &= \frac{RT}{\overline{V}}\frac{1}{1-b/\overline{V}}-\frac{a}{\overline{V}^2}\\
            &= \frac{RT}{\overline{V}}\left[ 1+\frac{b}{\overline{V}}+\frac{b^2}{\overline{V}^2}+\cdots \right]-\frac{a}{\overline{V}^2}\\
            &= \frac{RT}{\overline{V}}+(RTB-a)\frac{1}{\overline{V}^2}+\frac{RTb^2}{\overline{V}^3}+\cdots
        \end{align*}
        \item It follows that
        \begin{equation*}
            Z = \frac{P\overline{V}}{RT} = 1+\left( b-\frac{a}{RT} \right)\frac{1}{\overline{V}}+\frac{b^2}{\overline{V}^2}+\cdots
        \end{equation*}
        \item Thus,
        \begin{equation*}
            B_{2V}(T) = b-\frac{a}{RT}
        \end{equation*}
        for the van der Waals equation.
    \end{itemize}
    \item Relating the van der Waals constants to the Lennard-Jones parameters.
    \begin{itemize}
        \item Consider the following intermolecular potential, which is a hybrid of the hard-sphere and Lennard-Jones potentials.
        \begin{equation*}
            u(r) =
            \begin{cases}
                \infty & r<\sigma\\
                -\frac{c_6}{r^6} & r>\sigma
            \end{cases}
        \end{equation*}
        \item We can now calculate $B_{2v}(T)$ in terms of $u(r)$ with the help of the approximation $\e[x]=1+x$ (applicable since the argument of the exponent function will be very small).
        \begin{align*}
            B_{2V}(T) &= -2\pi N_A\left[ \int_0^\sigma(0-1)r^2\dd{r}+\int_\sigma^\infty(\e[c_6/k_BTr^6]-1)r^2\dd{r} \right]\\
            &= \frac{2\pi\sigma^3N_A}{3}-\frac{2\pi N_Ac_6}{k_BT}\int_\sigma^\infty\frac{1}{r^6}\cdot r^2\dd{r}\\
            &= \frac{2\pi\sigma^3N_A}{3}-\frac{2\pi N_Ac_6}{k_BT}\left[ -\frac{1}{3r^3} \right]_\sigma^\infty\\
            &= \frac{2\pi\sigma^3N_A}{3}-\frac{2\pi N_Ac_6}{3k_BT\sigma^3}
        \end{align*}
        \item It follows by comparison with the result from the previous section that
        \begin{align*}
            a &= \frac{2\pi N_A^2c_6}{3\sigma^3}&
            b &= \frac{2\pi\sigma^3N_A}{3}
        \end{align*}
        \item Physical interpretations: $a\propto c_6$ and $b$ is one-half the volume of the molecules.
    \end{itemize}
    \item In a similar fashion, we can relate $B_{2V}(T)$ to the Redlich-Kwong constants and Peng-Robinson functions.
    \begin{align*}
        B_{2V}(T) &= B-\frac{A}{RT^{3/2}}&
        B_{2V}(T) &= \beta-\frac{\alpha}{RT}
    \end{align*}
\end{itemize}



\section{Chapter 22: Helmholtz and Gibbs Energies}
\emph{From \textcite{bib:McQuarrieSimon}.}
\begin{itemize}
    \item Generalizing the equation $\overline{G}=G^\circ+RT\ln Q$ to the case of a real gas.
    \begin{itemize}
        \item We begin by substituting the virial expansion in terms of pressure into the equation $(\pdv*{\overline{G}}{P})_T=\overline{V}$.
        \begin{align*}
            \left( \pdv{\overline{G}}{P} \right)_T &= \overline{V}\\
            &= \frac{RT}{P}\left[ 1+B_{2P}(T)P+B_{3P}(T)P^2+\cdots \right]\\
            \int_{P^\text{id}}^P\dd{\overline{G}} &= RT\int_{P^\text{id}}^P\frac{\dd{P'}}{P'}+RTB_{2P}(T)\int_{P^\text{id}}^P\dd{P'}+RTB_{3P}(T)\int_{P^\text{id}}^PP'\dd{P'}\\
            \overline{G}(T,P) &= \overline{G}(T,P^\text{id})+RT\ln\frac{P}{P^\text{id}}+RTB_{2P}(T)P+RTB_{3P}(T)\frac{P^2}{2}+\cdots
        \end{align*}
        \item Note that we neglect $P_\text{id}$ in every term except the first because it is so close to zero.
        \item Substituting $\overline{G}(T,P^\text{id})=G^\circ(T)+RT\ln P^\text{id}/P^\circ$ and combining the first two terms yields
        \begin{equation*}
            \overline{G}(T,P) = G^\circ(T)+RT\ln\frac{P}{P^\circ}+RTB_{2P}(T)P+RTB_{3P}(T)\frac{P^2}{2}+\cdots
        \end{equation*}
    \end{itemize}
    \item The above equation is exact, but complicated and different for each gas (depending on the virial coefficients).
    \item It will be more convenient to, especially for calculations involving chemical equilibria, to define the \textbf{fugacity}.
    \item \textbf{Fugacity}: A state function of $P$ and $T$ describing the nonideality of the energy of a system. \emph{Denoted by} $\bm{f}$. \emph{Given by}
    \begin{equation*}
        \overline{G}(T,P) = G^\circ(T)+RT\ln\frac{f(P,T)}{f^\circ}
    \end{equation*}
    \item Fugacity must have the property that $f(P,T)\to P$ as $P\to 0$, so that the above equation can reduce to $\overline{G}=G^\circ+RT\ln P/P^\circ$.
    \item By setting the above equal to the virial expansion, we learn that
    \begin{equation*}
        \frac{f(P,T)}{f^\circ} = \frac{P}{P^\circ}\exp[B_{2P}(T)P+B_{3P}(T)P^2+\cdots]
    \end{equation*}
    \item "The standard state of the real gas\dots is taken to be the corresponding ideal gas at one bar" \parencite[906]{bib:McQuarrieSimon}.
    \begin{itemize}
        \item Mathematically, $f^\circ=P^\circ$.
        \item Note that we can also derive this from the virial expansion expression for $f(P,T)/f^\circ$ since all of the virial coefficients are equal to zero in the standard state.
    \end{itemize}
    \item \textcite{bib:McQuarrieSimon} goes through the derivation of
    \begin{equation*}
        \ln\frac{f}{P} = \int_0^P\left( \frac{\overline{V}}{RT}-\frac{1}{P'} \right)\dd{P'}
    \end{equation*}
    exactly as in class.
    \begin{itemize}
        \item The above equation allows us to calculate the ratio of the fugacity to the pressure of a gas at any pressure and temperature, given either $P$-$V$-$T$ data or an equation of state.
    \end{itemize}
    \item \textbf{Fugacity coefficient}: The following ratio. \emph{Denoted by} $\bm{\gamma}$. \emph{Given by}
    \begin{equation*}
        \gamma = \frac{f}{P}
    \end{equation*}
    \item Also note the equation
    \begin{equation*}
        \ln\gamma = \int_0^P\frac{Z-1}{P'}\dd{P'}
    \end{equation*}
    and that we can use this equation with the reduced pressure.
\end{itemize}




\end{document}