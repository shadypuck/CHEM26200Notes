\documentclass[../notes.tex]{subfiles}

\pagestyle{main}
\renewcommand{\chaptermark}[1]{\markboth{\chaptername\ \thechapter\ (#1)}{}}
\setcounter{chapter}{7}

\begin{document}




\chapter{Solutions}
\section{Vapor Pressure Lowering and Consequences}
\begin{itemize}
    \item \marginnote{2/28:}Colligative properties.
    \begin{itemize}
        \item Vapor pressure lowering.
        \item Boiling point elevation for non-volatile solutes.
        \item Freezing point depression for solutes (excluding solids).
        \item Osmotic pressure.
    \end{itemize}
    \item \textbf{Vapor pressure lowering}: The process of adding less volatile solutes to lower the overall vapor pressure via Raoult's law.
    \item \textbf{Boiling point elevation}: The increase in the normal boiling point of a mixture due to vapor pressure lowering.
    \begin{figure}[h!]
        \centering
        \begin{tikzpicture}[
            every node/.style={text height=1.5ex,text depth=0.25ex}
        ]
            \small
            \draw [stealth-stealth] (0,4) node[above]{$\ln P$} -- (0,0) -- (3,0) node[right]{$T$};
    
            \draw [blx,thick] (1.1,1.4) to[bend right=10] (2.3,3.7);
            \draw [bly,thick] (1.6,1.4) to[bend right=10] (2.8,3.7);
    
            \node at (2.7,2) {G};
            \node at (1.3,2.8) {L};
    
            \footnotesize
            \draw [dashed]
                (-0.1,2.5) node[left]{\SI{1}{\atmosphere}} -- ++(1.9,0) -- ++(0,-2.5) node[below]{$T_b$}
                (1.8,2.5) -- ++(0.5,0) -- ++(0,-2.5) node[below]{$T_b'$}
            ;
        \end{tikzpicture}
        \caption{Boiling point elevation.}
        \label{fig:BPelevate}
    \end{figure}
    \begin{itemize}
        \item In Figure \ref{fig:BPelevate}, the dark blue line is the original liquid-gas coexistence curve and the light blue line is the liquid-gas coexistence curve with vapor pressure lowering --- notice how at any given temperature, the corresponding pressure represented by the light blue line is lower than that given by the dark blue line.
        \item However, we still need to achieve a vapor pressure of \SI{1}{\atmosphere} for the liquid to boil.
        \item We can get to such a pressure with a higher temperature, i.e., by heating the liquid to $T_b'$ instead of just $T_b$.
    \end{itemize}
    \item Calculating the change in the boiling point $\Delta T_b$.
    \begin{itemize}
        \item We can calculate the boiling point elevation from the Clausius-Clapeyron equation.
        \begin{align*}
            \frac{\Delta P}{\Delta T_b} &= \frac{\Delta\overline{H}}{T_b\Delta\overline{V}}\\
            \Delta T_b &= P^*(1-x)\frac{T_b\Delta\overline{V}}{\Delta\overline{H}}\\
            &= P^*(1-x)\frac{T_b\overline{V}_g}{\Delta\overline{H}}\\
            &= P^*(1-x)\frac{T_bRT_b}{P^*\Delta\overline{H}}\\
            &= \frac{RT_b^2}{\Delta H}(1-x)
        \end{align*}
        where $x$ is the mole fraction of solutes.
        \item We can also derive the above result from the chemical potential, where we get to the first to the second equation below because (standard) chemical potentials are equal at equilibrium.
        \begin{align*}
            \mu_l^\circ(T)+RT\ln x_1 &= \mu_g^\circ(T)+RT\ln\frac{P}{P_0}\\
            RT\ln x_1 &= RT\ln\frac{P}{P_0}+[\mu_g^\circ(T)-\mu_l^\circ(T)]\\
            &= RT\ln\frac{P}{P_0}+[\Delta\overline{H}_\text{vap}-T\Delta\overline{S}_\text{vap}]\\
            &= RT\ln\frac{P}{P_0}+[\Delta\overline{H}_\text{vap}-T_b\Delta\overline{S}_\text{vap}+(T_b-T)\Delta\overline{S}_\text{vap}]\\
            &= RT\ln\frac{P}{P_0}+\Delta T_b\cdot\frac{\Delta\overline{H}_\text{vap}}{T_b}\\
            \Delta T_b &= -\frac{RT_b^2}{\Delta\overline{H}_\text{vap}}\ln x_1\\
            &= -\frac{RT_b^2}{\Delta\overline{H}_\text{vap}}\ln(1-x_2)\\
            &\approx \frac{RT_b^2}{\Delta\overline{H}_\text{vap}}x_2
        \end{align*}
    \end{itemize}
    \item \textbf{Freezing point depression}: The decrease in the normal melting point of a mixture due to vapor pressure lowering.
    \begin{figure}[h!]
        \centering
        \begin{tikzpicture}[
            every node/.style={text height=1.5ex,text depth=0.25ex}
        ]
            \small
            \draw [stealth-stealth] (0,4) node[above]{$\ln P$} -- (0,0) -- (3,0) node[right]{$T$};
    
            \draw [blx,thick] (0.7,0.5) to[bend right=10] (1.4,2) to[bend right=10] (2.3,3.7);
            \draw [bly,thick] (1.02,1) to[bend right=10] (2.8,3.7);
    
            \node at (2.7,2) {G};
            \node at (1.6,3) {L};
            \node at (0.7,1.3) {S};
    
            \footnotesize
            \draw [dashed]
                (1.4,2) -- ++(0,-2) node[below]{$T_f$}
                (1.02,1) -- ++(0,-1) node[below]{$T_f'$}
            ;
        \end{tikzpicture}
        \caption{Freezing point depression.}
        \label{fig:FPdepress}
    \end{figure}
    \begin{itemize}
        \item We know that $\mu_g^\circ(T_f)=\mu_l^\circ(T_f)$. Thus,
        \begin{align*}
            \mu_g^\circ(T) &= \mu_l^\circ(T)+RT\ln x_i\\
            RT\ln x_i &= \mu_g^\circ(T)-\mu_l^\circ(T)\\
            &= \Delta\overline{H}_\text{fus}(T)-T\Delta\overline{S}_\text{fus}\\
            &= (T_f-T)\Delta\overline{S}_\text{fus}\\
            &= (T_f-T)\frac{\Delta\overline{H}_\text{fus}}{T_f}\\
            &= -\Delta T_f\cdot\frac{\Delta\overline{H}_\text{fus}}{T_f}\\
            \Delta T_f &= -\frac{RT_f^2}{\Delta\overline{H}_\text{fus}}\ln x_1\\
            &\approx \frac{RT_f^2}{\Delta\overline{H}_\text{fus}}x_2
        \end{align*}
        \item The final result should be
        \begin{equation*}
            \Delta T_f = -\frac{RT_f^2}{\Delta\overline{H}_\text{fus}}x_2
        \end{equation*}
    \end{itemize}
    \item The freezing point of water as more and more salt is added.
    \begin{figure}[h!]
        \centering
        \begin{tikzpicture}[
            every node/.style=black
        ]
            \small
            \draw [stealth-stealth] (0,3.5) node[above]{$T$} -- (0,0) -- (5.5,0) node[right]{$x_{\ce{NaCl}}$};
            \footnotesize
            \draw
                (0.1,3) -- ++(-0.2,0) node[left]{\SI{0}{\celsius}}
                (0.1,1) -- ++(-0.2,0) node[left]{\SI{-21}{\celsius}}
                (1.165,0.1) -- ++(0,-0.2) node[below]{23.3\%}
            ;
            
            \draw [blx,thick] (0,3) to[bend left=15] (1.165,1) to[bend right=5] (5,3.5);
            \draw [bly,thick] (0,1) -- node[below]{\ce{NaCl + 2H2O} solid} ++(5,0);
            \node [align=center] at (0.45,1.65) {Ice\\+\\brine};
            \node [align=center] at (3.5,1.5) {Brine +\\\ce{NaCl(H2O)2}};
            \node at (1.9,2.7) {Brine};
            \draw [bly,dashed] (0,3) -- (1.2,1.85);
            \node at (1.2,3.3) {Prediction} edge[out=-90,in=45,->] (1,2.1);
        \end{tikzpicture}
        \caption{Freezing point vs. solute concentration.}
        \label{fig:H2OFP}
    \end{figure}
    \item Osmotic pressure.
    \begin{itemize}
        \item Imagine a U-tube with a filter at the bottom that is porous to the solvent but nonporous to the solute.
        \item We know from Gen Chem that there will be excess pressure on the side with the impurities.
        \item Let $\pi=\rho gh$ be the extra pressure where $h$ is the height difference between the two sides and $\rho$ is the density of the solvent.
        \begin{align*}
            \mu_l^\circ = \mu_l &= \mu_l^\circ+RT\ln x_1+\pi\overline{V}\\
            0 &= RT\ln x_1+\pi\overline{V}\\
            \pi &= \frac{RT}{\overline{V}}x_2
        \end{align*}
    \end{itemize}
    \item Ocean salinity is about \SI{1}{\molar}, so $\pi=\SI{24}{\atmosphere}$. That means that in the tube, the right hand side will rise about as much as the Sears tower. Thus, the minimum amount of pressure/work you need is the height of the Sears tower minus \SI{24}{\atmosphere}. Still, this is far more efficient than distillation.
\end{itemize}



\section{Nonideal Solutions}
\begin{itemize}
    \item \marginnote{3/2:}When solutions are not ideal, we see deviations from Raoult's Law (Figure \ref{fig:RaoultsLaw}).
    \begin{itemize}
        \item However, all solutions follow Raoult's Law when they are nearly pure.
        \item When the actual partial pressure is higher, that means the substance would rather be in the vapor phase. When the actual partial pressure is lower, that means there is a favorable interaction between particles (they'd rather be in the dissolved state).
    \end{itemize}
    \item \textbf{Henry's Law}: Gives the tangent to the vapor pressure vs. mole percent graph at $x_1=0$. \emph{Given by}
    \begin{equation*}
        P_1 = k_Hx_1
    \end{equation*}
    as $x_1\to 0$.
    \item We know that $A(g)\to A(\text{solv})$.
    \begin{itemize}
        \item At equilibrium $\Delta\overline{G}=0=\Delta\overline{G}_\text{solv}+RT\ln x/P$.
        \item Thus,
        \begin{equation*}
            P = x\e[\Delta\overline{G}_\text{solv}/RT]
        \end{equation*}
        \item $x$ is the mixing entropy, $P$ is the gas phase entropy.
    \end{itemize}
    \item Microscopic enthalpy and entropy of solvation contribute to Henry's law, in addition to mixing entropy.
    \begin{itemize}
        \item Larger $k_H$ means less soluble.
    \end{itemize}
    \item Temperature dependence.
    \begin{equation*}
        k_{H,cp} = k_{H,cp}^\Theta\exp\left[ -C\cdot\left( \frac{1}{T}-\frac{1}{T^\Theta} \right) \right]
    \end{equation*}
    \begin{itemize}
        \item $\Theta$ indicates reference temperature.
        \item It follows that as $T$ increases, $k_H$ increases.
        \begin{itemize}
            \item For example, \ce{O2} is less soluble in water at higher temperatures.
            \item This is consistent with solvation being an exothermic process.
        \end{itemize}
    \end{itemize}
    \item The activity and activity coefficient of a solute. Using Raoult's law as the reference for fully miscible substances.
    \begin{itemize}
        \item We have for an ideal solution that $\mu_i=\mu_i^*+RT\ln x_i$ and for an ideal gas that $\mu_i=\mu_i^*+RT\ln\frac{P_i}{P_i^*}$.
        \item These two equations together imply Raoult's law.
        \item Now we want to keep the nice form of the above equations even with nonideality, so we do something similar to defining fugacity by defining the \textbf{activity}.
        \item $a_i\to x_i$ as $x_i\to 1$.
        \item As $x_i\to 0$, we have that
        \begin{align*}
            \mu_i^*+RT\ln a_i &= \mu_i^*+RT\ln\frac{P_i}{P_i^*}\\
            &= \mu_i^*+RT\ln\frac{k_H}{P_i^*}x_i
        \end{align*}
        i.e., $a_i\to k_Hx_i/P_i^*$.
        \item As a substance becomes less active, it becomes more reactive.
    \end{itemize}
    \item \textbf{Activity}: A measure of the nonideality of solutions. \emph{Denoted by} $\bm{a_i}$. \emph{Given by}
    \begin{equation*}
        \mu_i = \mu_i^*+RT\ln a_i
    \end{equation*}
    \item \textbf{Activity coefficient}: The following ratio. \emph{Denoted by} $\bm{\gamma}$. \emph{Given by}
    \begin{equation*}
        \gamma = \frac{a_i}{x_i}
    \end{equation*}
    \item Example: Carbon disulfide/dimethoxymethane.
    \begin{table}[h!]
        \centering
        \begin{tabular}{ccc}
            $x_{\ce{CS2}}$ & $P_{\ce{CS2}}$ (\si{\torr}) & $P_{\ce{CH2(OMe)2}}$ (\si{\torr})\\
            0 & 0 & 587\\
            0.1 & 109 & 529\\
            1 & 514 & 0\\
        \end{tabular}
        \caption{Pressure data for \ce{CS2} and \ce{CH2(OMe)2}.}
        \label{fig:PCS2CH2OMe2}
    \end{table}
    \begin{itemize}
        \item At $x_{\ce{CS2}}=0.1$, we have that $a_{\ce{CS2}}=109/514$.
        \item As $x\to 0$, $k_{H,\ce{CS2}}=\SI{1130}{\torr}$ so $a_{\ce{CS2}}=\frac{1130}{514}x_{\ce{CS2}}$.
        \item See \textcite{bib:McQuarrieSimon} for using Henry's law as the reference state.
    \end{itemize}
    \item Nonideality above the Raoult's law diagonal: More "active" than the ideal mixture.
    \item Nonideality below the Raoult's law diagonal: Less active than at low concentration.
\end{itemize}



\section{Quantifying Deviations from Raoult's Law}
\begin{itemize}
    \item \marginnote{3/4:}\textbf{Gibbs-Duhem relation}: A relation tying together the partial pressure of one substance to another, regardless of ideality/nonideality.
    \begin{itemize}
        \item We begin with
        \begin{equation*}
            \dd{G} = \sum\mu_i\dd{n_i}+\sum n_i\dd{\mu_i}
        \end{equation*}
        at constant temperature and pressure.
        \item We also know from an earlier class that
        \begin{equation*}
            G = \sum\mu_in_i
        \end{equation*}
        \item Differentiating the above, we get
        \begin{equation*}
            \dd{G} = \sum\mu_i\dd{n_i}+\sum n_i\dd{\mu_i}
        \end{equation*}
        \item It follows by setting the above two equations equal to each other that
        \begin{equation*}
            \sum n_i\dd{\mu_i} = 0
        \end{equation*}
    \end{itemize}
    \item We can apply the Gibbs-Duhem relation to our expression for the chemical potential in terms of the activity coefficient.
    \begin{itemize}
        \item From last time,
        \begin{align*}
            \mu_i &= \mu_i^*+RT\ln(\gamma_ix_i)\\
            \dd{\mu_i} &= RT\dd{(\ln\gamma_i+\ln x_i)}
        \end{align*}
        \item It follows that
        \begin{align*}
            0 &= n_1\dd{\mu_1}+n_2\dd{\mu_2}\\
            &= \frac{1}{n_1+n_2}(n_1\dd{\mu_1}+n_2\dd{\mu_2})\\
            &= \frac{n_1}{n_1+n_2}RT\dd{(\ln\gamma_1+\ln x_1)}+\frac{n_2}{n_1+n_2}RT\dd{(\ln\gamma_2+\ln x_2)}\\
            &= RT(x_1\dd{\ln x_1}+x_2\dd{\ln x_2}+x_1\dd{\ln\gamma_1}+x_2\dd{\ln\gamma_2})\\
            &= \dd{x_1}+\dd{x_2}+x_1\dd{\ln\gamma_1}+x_2\dd{\ln\gamma_2}\\
            &= \dd{x_1}-\dd{x_1}+x_1\dd{\ln\gamma_1}+x_2\dd{\ln\gamma_2}\\
            &= x_1\dd{\ln\gamma_1}+x_2\dd{\ln\gamma_2}
        \end{align*}
        \item Takeaway: If we know the activity coefficient for one component across the whole range, we can find the activity coefficient for the other component.
        \item In other words, knowing the chemical potential for one component gives us the chemical potential for the other.
    \end{itemize}
    \item Relating $P_1,P_2$ with the Gibbs-Duhem relation.
    \begin{itemize}
        \item We assume that we are in phase equilibrium. Thus, the chemical potential of the gas phase equals the chemical potential of the solution. In an equation,
        \begin{equation*}
            \mu_1^*+RT\ln\frac{P_1}{P_1^*} = \mu_1+RT\ln a_1
        \end{equation*}
        \item Thus, applying the Gibbs-Duhem relation,
        \begin{align*}
            0 &= x_1\dd{\ln P_1}+x_2\dd{\ln P_2}\\
            &= x_1\left( \pdv{\ln P_1}{x_1} \right)_{T,P}\dd{x_1}+x_2\left( \pdv{\ln P_2}{x_2} \right)_{T,P}\dd{x_2}\\
            &= x_1\left( \pdv{\ln P_1}{x_1} \right)_{T,P}-x_2\left( \pdv{\ln P_2}{x_2} \right)_{T,P}
        \end{align*}
        \item Takeaway: The slopes of the curves on a pressure-concentration diagram are related, regardless of ideality/nonideality.
    \end{itemize}
    \item \textbf{Margules equation}: An expansion to capture the experimental $P(x)$ deviation from Raoult's law.
    \begin{equation*}
        P_1 = x_1P_1^*\e[\alpha x_2^2+\beta x_2^3+\cdots]
    \end{equation*}
    \begin{itemize}
        \item An analytical form that gives us a good representation of experimentally-measured vapor pressures.
        \item Important features: Follows Henry's law at low concentrations of solute and Raoult's Law at high concentrations of solute.
    \end{itemize}
    \item Margules equation covers Raoult's law and Henry's law.
    \begin{itemize}
        \item We have by the product rule of derivatives that
        \begin{equation*}
            \dv{P_1}{x_1} = P_1^*\left[ \e[\alpha x_2^2+\beta x_2^3+\cdots]+x_1(-2\alpha x_2-3\beta x_2^2-\cdots)\e[\alpha x_2^2+\beta x_2^3+\cdots] \right]
        \end{equation*}
        \item As $x_1\to 1$, $\dv*{P_1}{x_1}\to P_1^*$. This can be seen because the above expression can be rewritten
        \begin{equation*}
            \dv{P_1}{x_1} = P_1^*\e[\alpha x_2^2+\beta x_2^3+\cdots]\left[ 1-2\alpha x_1x_2-3\beta x_1x_2^2-\cdots \right]
        \end{equation*}
        \begin{itemize}
            \item Implication: We recover Raoult's law as $x_1\to 1$.
            \item Additionally, this explains why there is no linear terms in the exponent: If there were, one of the terms in the sum on the right-hand side above wouldn't go to zero as $x_1\to 1$, meaning that $\dv*{P_1}{x_1}$ would not go to $P_1^*$.
        \end{itemize}
        \item Recovering Henry's law as $x_1\to 0$: As $x_1\to 0$, $P_1\to k_Hx_1=P_1^*\e[\alpha+\beta+\cdots]\cdot x_1$.
    \end{itemize}
    \item Margules equation parameters fitted for $P_1$ provide a related equation for $P_2$ based on Gibbs-Duhem (Problem 24-33).
    \begin{itemize}
        \item To two terms, the Margules equation tells us that
        \begin{equation*}
            P_1 = x_1P_1^*\e[\alpha x_2^2+\beta x_2^3]
        \end{equation*}
        \item As we showed earlier, the Gibbs-Duhem equation implies that
        \begin{equation*}
            x_1\pdv{\ln P_1}{x_1} = x_2\pdv{\ln P_2}{x_2}
        \end{equation*}
        \item Thus, since $x_2=1-x_1$,
        \begin{align*}
            x_1\pdv{\ln P_1}{x_1} &= x_1\pdv{x_1}(\ln x_1+\alpha x_2^2+\beta x_2^3+\text{terms not involving }x_1)\\
            &= x_1\left[ \frac{1}{x_1}+(-2\alpha x_2)+(-3\beta x_2^2) \right]\\
            &= 1+x_2(-2\alpha x_1-3\beta x_1x_2)\\
            &= 1+x_2(-2\alpha x_1-3\beta x_1(1-x_1))\\
            &= 1+x_2(-2\alpha x_1-3\beta(x_1-x_1^2))\\
            &= 1+x_2[-(2\alpha+3\beta)x_1+3\beta x_1^2]
        \end{align*}
        \item Now let
        \begin{equation*}
            P_2 = x_2P_2^*\e[ax_1^2+bx_1^3]
        \end{equation*}
        \item It follows as before that
        \begin{equation*}
            x_2\pdv{\ln P_2}{x_2} = 1+x_2(-2ax_1-3bx_1^2)
        \end{equation*}
        \item Direct comparison reveals that
        \begin{align*}
            -2a &= -(2\alpha+3\beta)&
                -3b &= 3\beta\\
            a &= \alpha+\frac{3}{2}\beta&
                b &= -\beta
        \end{align*}
        so
        \begin{equation*}
            P_2 = x_2P_2^*\e[(\alpha+\frac{3}{2}\beta)x_1^2-\beta x_1^3]
        \end{equation*}
    \end{itemize}
    \item \textbf{Regular solution}: A solution for which only $\alpha\neq 0$ and every other Margules parameter is zero.
    \begin{itemize}
        \item In this case,
        \begin{align*}
            P_1 &= x_1P_1^*\e[\alpha x_2^2]&
            P_2 &= x_2P_2^*\e[\alpha x_1^2]
        \end{align*}
        \item This means that the Raoult's activity of the two solutions is entirely symmetric over the range of compositions.
        \item This does not necessarily imply that the vapor pressures vary in the same way.
    \end{itemize}
    \item Mixing Gibbs free energy for a regular solution (24-5).
    \begin{itemize}
        \item We have
        \begin{align*}
            \frac{\Delta G_\text{mix}}{RT} &= \underbrace{n_1\mu_1^\text{sol}+n_2\mu_2^\text{sol}}_\text{mixture}-(\underbrace{n_1\mu_1^*+n_2\mu_2^*}_\text{pure})\\
            &= n_1\ln\frac{P_1}{P_1^*}+n_2\ln\frac{P_2}{P_2^*}\\
            &= n_1\ln x_1+n_2\alpha x_2^2+n_2\ln x_2+n_2\alpha x_1^2\\
            \frac{\Delta G_\text{mix}}{RT(n_1+n_2)} &= x_1\ln x_1+x_2\ln x_2+\alpha x_1x_2^2+\alpha x_2x_1^2\\
            &= x_1\ln x_1+x_2\ln x_2+\alpha x_1x_2(x_1+x_2)\\
            \frac{\Delta G_\text{mix}}{RT(n_1+n_2)} &= x_1\ln x_1+x_2\ln x_2+\alpha x_1x_2
        \end{align*}
        \item The left two terms above represent the entropy of mixing (which does not depend on interaction).
        \item The right term above is the enthalpy/entropy of mixing which depends \emph{only} on interaction.
        \item Thus, $\alpha$ is the microscopic entropy of mixing for a regular solution.
    \end{itemize}
    \item The upper critical solution temperature (UCST) (critical temperature above which the components of the mixture are miscible in all proportions).
    \begin{itemize}
        \item We have
        \begin{equation*}
            \Delta\overline{G} = RT(x_1\ln x_1+x_2\ln x_2)+\underbrace{RT\alpha}_\Omega x_1x_2
        \end{equation*}
        \item Graphing the above equation will help us determine what extent of mixing will be most energetically favorable.
        \begin{itemize}
            \item In particular, if we graph the left two terms, that tells us the entropy of mixing, which will be greatest at $x_1=x_2$.
            \item Graphing the right term on the other hand yields a parabola.
            \item As such, since $x_1\ln x_1+x_2\ln x_2$ has vertical slope at $x_1=0,1$ and the other term simply has some positive slope, entropy of mixing will always win at low concentrations of solutes, i.e., when you have a small enough concentration of solute in solvent, you will always observe mixing.
            \item However, the overall graph may have minima on the two sides, indicating that forming a fully mixed solution is not energetically favorable, but it is most favorable to separate into substance 1 with a bit of substance 2 and vice versa (think oil and water).
            \item At high temperatures, however, this effect goes away and mixing always becomes the most favorable state.
        \end{itemize}
        \item The temperature at which the overall graph only has one minimum for the first time is the UCST.
    \end{itemize}
\end{itemize}



\section{Chapter 24: Solutions I --- Liquid-Liquid Solutions}
\emph{From \textcite{bib:McQuarrieSimon}.}
\begin{itemize}
    \item There exist positive and negative deviations from Raoult's Law (see Figure \ref{fig:postiveDeviationRaoult}).
    \begin{figure}[h!]
        \centering
        \begin{tikzpicture}[xscale=5,yscale=0.5]
            \small
            \draw (0,6) -- node[rotate=90,above=7mm]{Vapor pressure (\si{\torr})} (0,0) -- node[below=5mm]{$x_1$} (1,0);
            \footnotesize
            \node [above left,xshift=-1mm,yshift=-1mm] {0};
            \node [below right,xshift=-1mm,yshift=-1mm] {0.0};
            \foreach \y [evaluate=\y as \n using int(\y*100)] in {2,4,6} {
                \draw (0.02,\y) -- ++(-0.04,0) node[left]{$\n$};
            }
            \foreach \x in {0.2,0.4,0.6,0.8,1.0} {
                \draw (\x,0.2) -- ++(0,-0.4) node[below]{$\x$};
            }
    
            \draw [gry,thick] plot [domain=0:1,smooth] (\x,{5*\x});
            \draw [gry,thick] plot [domain=0:1,smooth] (\x,{5.9-5.9*\x});
            \draw [grx,thick] plot [domain=0:1,smooth] (\x,{5*\x*e^((1-\x)^2)});
            \draw [grx,thick] plot [domain=0:1,smooth] (\x,{(5.9-5.9*\x)*e^(\x^2)});
    
            \node at (0.52,0.7) {Raoult's Law}
                edge [->,out=0,in=-95,looseness=0.5] (0.8,1)
                edge [->,out=180,in=-85,looseness=0.5] (0.23,1)
            ;
            \node at (0.17,2.5) {$P_1$};
            \node at (0.85,2.5) {$P_2$};
        \end{tikzpicture}
        \caption{Positive deviations from Raoult's law (\ce{CS2}/\ce{CH2(OMe)2} at \SI{25}{\celsius}).}
        \label{fig:postiveDeviationRaoult}
    \end{figure}
    \item If 1 and 2 denote two substances that are being mixed, positive deviations occur when 1-2 interactions are more repulsive than either 1-1 or 2-2 interactions, and vice versa for negative deviations.
    \item If one component of a binary solution exhibits positive deviations from Raoult's law, so must the other component.
    \begin{itemize}
        \item Follows from "Relating $P_1,P_2$ with the Gibbs-Duhem relation" as in class.
    \end{itemize}
    \item Molecular structure predicts deviations from Raoult's law.
    \begin{figure}[H]
        \centering
        \begin{tikzpicture}[xscale=5,yscale=0.5]
            \small
            \draw (0,6) -- node[rotate=90,above=2mm]{Vapor pressure} (0,0) -- node[below=5mm]{$x_\text{alcohol}$} (1,0);
            \footnotesize
            \node [below right,xshift=-1mm,yshift=-1mm] {0.0};
            \foreach \y in {2,4,6} {
                \draw (0.02,\y) -- ++(-0.04,0);
            }
            \foreach \x in {0.2,0.4,0.6,0.8,1.0} {
                \draw (\x,0.2) -- ++(0,-0.4) node[below]{$\x$};
            }
    
            \draw [grx,thick] plot [domain=0:1,smooth] (\x,{6*\x});
            \draw [blx,thick] plot [domain=0:1,smooth] (\x,{6*\x*e^(0.8*(1-\x)^2)});
            \draw [rex,thick] plot [domain=0:1,smooth] (\x,{6*\x*e^(1.4*(1-\x)^2)});
            \draw [orx,thick] plot [domain=0:1,smooth] (\x,{6*\x*e^(1.8*(1-\x)^2)});
    
            \node at (0.55,1) {Raoult's Law}
                edge [->,out=180,in=-85,looseness=0.5] (0.25,1.4)
            ;
        \end{tikzpicture}
        \caption{Vapor pressure of alcohol/water solutions.}
        \label{fig:VPalcoholWater}
    \end{figure}
    \begin{itemize}
        \item The blue line represents \ce{MeOH}, the red line represents \ce{EtOH}, and the orange line represents \ce{PrOH}.
        \item Notice how as molecular structure gets further from that of \ce{H2O} (the solvent), deviations become more exaggerated.
    \end{itemize}
    \item \textcite{bib:McQuarrieSimon} discusses the limiting behavior of $P_1$ in Figure \ref{fig:postiveDeviationRaoult}, noting the Raoult's law limit as $x_1\to 1$ and introducing \textbf{Henry's law}.
    \item \textbf{Henry's law}: The partial pressure of a component in solution approaches the linear function $k_{H,i}x_i$ as $x_i\to 0$.
    \item \textbf{Henry's law constant}: The constant $k_{H,i}$ in the above definition.
    \item For an ideal solution, $k_{H,i}=P_i^*$, but this is not true in general (as illustrated by Figure \ref{fig:postiveDeviationRaoult}).
    \item Example: Finding $P_1^*$ and $k_H$ from an equation for $P_1$.
    \begin{itemize}
        \item Let $P_1=180x_1\e[x_2^2+\frac{1}{2}x_2^3]$.
        \item As $x_1\to 1$, $x_2\to 0$, so $P_1\to 180x_1$. Since this form should be the Raoult's law limit, we can read off $P_1^*=\SI{180}{\torr}$.
        \item As $x_1\to 0$, $x_2\to 1$, so $P_1\to 180x_1\e[1+\frac{1}{2}]=180\e[3/2]x_1$. Since this form should be the Henry's law limit, we can read off $k_{H,1}=180\e[3/2]=\SI{807}{\torr}$.
    \end{itemize}
    \item \textcite{bib:McQuarrieSimon} derives
    \begin{equation*}
        x_1\left( \pdv{\ln P_1}{x_1} \right)_{T,P} = x_2\left( \pdv{\ln P_2}{x_2} \right)_{T,P}
    \end{equation*}
    as in class.
    \item In a binary solution, if component 1 obeys Raoult's law as $x_1\to 1$, then component 2 obeys Henry's law as $x_2\to 0$.
    \begin{itemize}
        \item Suppose component 1 satisfies $P_1\to x_1P_1^*$ as $x_1\to 1$. Then
        \begin{align*}
            x_2\left( \pdv{\ln P_2}{x_2} \right)_{T,P} &= x_1\left( \pdv{\ln P_1}{x_1} \right)_{T,P}\\
            &= x_1\pdv{x_1}(\ln x_1+\ln P_1^*)\\
            &= 1\\
            \ln P_2 &= \int\frac{1}{x_2}\dd{x_2}\\
            &= \ln x_2+C\\
            P_2 &= k_Hx_2
        \end{align*}
        as desired.
    \end{itemize}
    \item The vapor pressure curve of WLOG component 1 of a binary solution can often be represented empirically by
    \begin{equation*}
        P_1 = x_1P_1^*\e[\alpha x_2^2+\beta x_2^3]
    \end{equation*}
    \begin{itemize}
        \item $\alpha,\beta$ reflect the extent of the nonideality of the solution in some manner because $P_1=x_1P_1^*$ (ideal) for $\alpha=\beta=0$.
        \item Furthermore, generalizing from the previous example, $k_{H,1}=P_1^*\e[\alpha+\beta]$ and $k_{H,2}=P_2^*\e[\alpha+\beta/2]$ (the latter equation will be derived presently).
    \end{itemize}
    \item Deriving $P_2=x_2P_2^*\e[(\alpha+3\beta/2)x_1^2-\beta x_1^3]$ from $P_1=x_1P_1^*\e[\alpha x_2^2+\beta x_2^3]$ in an alternate manner to what was presented in class.
    \begin{itemize}
        \item We are given that $P_1=x_1P_1^*\e[\alpha x_2^2+\beta x_2^3]$ and that $x_2=1-x_1$.
        \item It follows that the chemical potential of component 1 is
        \begin{align*}
            \mu_1 &= \mu_1^\circ+RT\ln P_1\\
            &= \mu_1^\circ+RT\left[ \ln P_1^*+\ln x_1+\alpha(1-x_1)^2+\beta(1-x_1)^3 \right]
        \end{align*}
        \item It follows that the total differential of $\mu_1(x_1)$ is
        \begin{equation*}
            \dd{\mu_1} = RT\left[ \frac{1}{x_1}-2\alpha(1-x_1)-3\beta(1-x_1)^2 \right]\dd{x_1}
        \end{equation*}
        \item Thus, by the Gibbs-Duhem equation,
        \begin{align*}
            x_1\dd{\mu_1}+x_2\dd{\mu_2} &= 0\\
            \dd{\mu_2} &= -\frac{x_1}{x_2}\cdot RT\left[ \frac{1}{x_1}-2\alpha(1-x_1)-3\beta(1-x_1)^2 \right]\dd{x_1}\\
            &= RT\left[ -\frac{1}{x_2}+2\alpha x_1+3\beta x_1(1-x_1) \right]\dd{x_1}
        \end{align*}
        \item Substitute $x_2=1-x_1$ and integrate.
        \begin{align*}
            \dd{\mu_2} &= RT\left[ \frac{1}{x_2}-2\alpha(1-x_2)-3\beta x_2(1-x_2) \right]\dd{x_2}\\
            \int_1^{x_2}\dd{\mu_2} &= RT\int_1^{x_2}\left[ \frac{1}{x_2'}-2\alpha(1-x_2')-3\beta x_2'(1-x_2') \right]\dd{x_2'}\\
            \mu_2(x_2)-\mu_2(1) &= RT\left[ \ln x_2+\alpha(1-x_2)^2-\frac{3\beta}{2}(x_2^2-1)+\beta(x_2^3-1) \right]\\
            \mu_2-\mu_2^* &= RT\left[ \ln x_2+\alpha x_1^2+\frac{3\beta}{2}x_1^2-\beta x_1^3 \right]
        \end{align*}
        \item Therefore, since
        \begin{align*}
            \mu_2 &= \mu_2^\circ+RT\ln P_2&
            \mu_2^* &= \mu_2^\circ+RT\ln P_2^*
        \end{align*}
        we have that
        \begin{align*}
            \ln P_2 &= \ln P_2^*+\ln x_2+\alpha x_1^2+\frac{3\beta}{2}x_1^2-\beta x_1^3\\
            P_2 &= x_2P_2^*\e[(\alpha+3\beta/2)x_1^2-\beta x_1^3]
        \end{align*}
    \end{itemize}
\end{itemize}




\end{document}