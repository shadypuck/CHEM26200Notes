\documentclass[../notes.tex]{subfiles}

\pagestyle{main}
\renewcommand{\chaptermark}[1]{\markboth{\chaptername\ \thechapter\ (#1)}{}}
\setcounter{chapter}{7}

\begin{document}




\chapter{Solutions}
\section{Vapor Pressure Lowering and Consequences}
\begin{itemize}
    \item \marginnote{2/28:}Colligative properties.
    \begin{itemize}
        \item Vapor pressure lowering.
        \item Boiling point elevation for non-volatile solutes.
        \item Freezing point depression for solutes (excluding solids).
        \item Osmotic pressure.
    \end{itemize}
    \item \textbf{Vapor pressure lowering}: The process of adding less volatile solutes to lower the overall vapor pressure via Raoult's law.
    \item \textbf{Boiling point elevation}: The increase in the normal boiling point of a mixture due to vapor pressure lowering.
    \begin{figure}[h!]
        \centering
        \begin{tikzpicture}[
            every node/.style={text height=1.5ex,text depth=0.25ex}
        ]
            \small
            \draw [stealth-stealth] (0,4) node[above]{$\ln P$} -- (0,0) -- (3,0) node[right]{$T$};
    
            \draw [blx,thick] (1.1,1.4) to[bend right=10] (2.3,3.7);
            \draw [bly,thick] (1.6,1.4) to[bend right=10] (2.8,3.7);
    
            \node at (2.7,2) {G};
            \node at (1.3,2.8) {L};
    
            \footnotesize
            \draw [dashed]
                (-0.1,2.5) node[left]{\SI{1}{\atmosphere}} -- ++(1.9,0) -- ++(0,-2.5) node[below]{$T_b$}
                (1.8,2.5) -- ++(0.5,0) -- ++(0,-2.5) node[below]{$T_b'$}
            ;
        \end{tikzpicture}
        \caption{Boiling point elevation.}
        \label{fig:BPelevate}
    \end{figure}
    \begin{itemize}
        \item In Figure \ref{fig:BPelevate}, the dark blue line is the original liquid-gas coexistence curve and the light blue line is the liquid-gas coexistence curve with vapor pressure lowering --- notice how at any given temperature, the corresponding pressure represented by the light blue line is lower than that given by the dark blue line.
        \item However, we still need to achieve a vapor pressure of \SI{1}{\atmosphere} for the liquid to boil.
        \item We can get to such a pressure with a higher temperature, i.e., by heating the liquid to $T_b'$ instead of just $T_b$.
    \end{itemize}
    \item Calculating the change in the boiling point $\Delta T_b$.
    \begin{itemize}
        \item We can calculate the boiling point elevation from the Clausius-Clapeyron equation.
        \begin{align*}
            \frac{\Delta P}{\Delta T_b} &= \frac{\Delta\overline{H}}{T_b\Delta\overline{V}}\\
            \Delta T_b &= P^*(1-x)\frac{T_b\Delta\overline{V}}{\Delta\overline{H}}\\
            &= P^*(1-x)\frac{T_b\overline{V}_g}{\Delta\overline{H}}\\
            &= P^*(1-x)\frac{T_bRT_b}{P^*\Delta\overline{H}}\\
            &= \frac{RT_b^2}{\Delta H}(1-x)
        \end{align*}
        where $x$ is the mole fraction of solutes.
        \item We can also derive the above result from the chemical potential, where we get to the first to the second equation below because (standard) chemical potentials are equal at equilibrium.
        \begin{align*}
            \mu_l^\circ(T)+RT\ln x_1 &= \mu_g^\circ(T)+RT\ln\frac{P}{P_0}\\
            RT\ln x_1 &= RT\ln\frac{P}{P_0}+[\mu_g^\circ(T)-\mu_l^\circ(T)]\\
            &= RT\ln\frac{P}{P_0}+[\Delta\overline{H}_\text{vap}-T\Delta\overline{S}_\text{vap}]\\
            &= RT\ln\frac{P}{P_0}+[\Delta\overline{H}_\text{vap}-T_b\Delta\overline{S}_\text{vap}+(T_b-T)\Delta\overline{S}_\text{vap}]\\
            &= RT\ln\frac{P}{P_0}+\Delta T_b\cdot\frac{\Delta\overline{H}_\text{vap}}{T_b}\\
            \Delta T_b &= -\frac{RT_b^2}{\Delta\overline{H}_\text{vap}}\ln x_1\\
            &= -\frac{RT_b^2}{\Delta\overline{H}_\text{vap}}\ln(1-x_2)\\
            &\approx \frac{RT_b^2}{\Delta\overline{H}_\text{vap}}x_2
        \end{align*}
    \end{itemize}
    \item \textbf{Freezing point depression}: The decrease in the normal melting point of a mixture due to vapor pressure lowering.
    \begin{figure}[h!]
        \centering
        \begin{tikzpicture}[
            every node/.style={text height=1.5ex,text depth=0.25ex}
        ]
            \small
            \draw [stealth-stealth] (0,4) node[above]{$\ln P$} -- (0,0) -- (3,0) node[right]{$T$};
    
            \draw [blx,thick] (0.7,0.5) to[bend right=10] (1.4,2) to[bend right=10] (2.3,3.7);
            \draw [bly,thick] (1.02,1) to[bend right=10] (2.8,3.7);
    
            \node at (2.7,2) {G};
            \node at (1.6,3) {L};
            \node at (0.7,1.3) {S};
    
            \footnotesize
            \draw [dashed]
                (1.4,2) -- ++(0,-2) node[below]{$T_f$}
                (1.02,1) -- ++(0,-1) node[below]{$T_f'$}
            ;
        \end{tikzpicture}
        \caption{Freezing point depression.}
        \label{fig:FPdepress}
    \end{figure}
    \begin{itemize}
        \item We know that $\mu_g^\circ(T_f)=\mu_l^\circ(T_f)$. Thus,
        \begin{align*}
            \mu_g^\circ(T) &= \mu_l^\circ(T)+RT\ln x_i\\
            RT\ln x_i &= \mu_g^\circ(T)-\mu_l^\circ(T)\\
            &= \Delta\overline{H}_\text{fus}(T)-T\Delta\overline{S}_\text{fus}\\
            &= (T_f-T)\Delta\overline{S}_\text{fus}\\
            &= (T_f-T)\frac{\Delta\overline{H}_\text{fus}}{T_f}\\
            &= -\Delta T_f\cdot\frac{\Delta\overline{H}_\text{fus}}{T_f}\\
            \Delta T_f &= -\frac{RT_f^2}{\Delta\overline{H}_\text{fus}}\ln x_1\\
            &\approx \frac{RT_f^2}{\Delta\overline{H}_\text{fus}}x_2
        \end{align*}
        \item The final result should be
        \begin{equation*}
            \Delta T_f = -\frac{RT_f^2}{\Delta\overline{H}_\text{fus}}x_2
        \end{equation*}
    \end{itemize}
    \item The freezing point of water as more and more salt is added.
    \begin{figure}[h!]
        \centering
        \begin{tikzpicture}[
            every node/.style=black
        ]
            \small
            \draw [stealth-stealth] (0,3.5) node[above]{$T$} -- (0,0) -- (5.5,0) node[right]{$x_{\ce{NaCl}}$};
            \footnotesize
            \draw
                (0.1,3) -- ++(-0.2,0) node[left]{\SI{0}{\celsius}}
                (0.1,1) -- ++(-0.2,0) node[left]{\SI{-21}{\celsius}}
                (1.165,0.1) -- ++(0,-0.2) node[below]{23.3\%}
            ;
            
            \draw [blx,thick] (0,3) to[bend left=15] (1.165,1) to[bend right=5] (5,3.5);
            \draw [bly,thick] (0,1) -- node[below]{\ce{NaCl + 2H2O} solid} ++(5,0);
            \node [align=center] at (0.45,1.65) {Ice\\+\\brine};
            \node [align=center] at (3.5,1.5) {Brine +\\\ce{NaCl(H2O)2}};
            \node at (1.9,2.7) {Brine};
            \draw [bly,dashed] (0,3) -- (1.2,1.85);
            \node at (1.2,3.3) {Prediction} edge[out=-90,in=45,->] (1,2.1);
        \end{tikzpicture}
        \caption{Freezing point vs. solute concentration.}
        \label{fig:H2OFP}
    \end{figure}
    \item Osmotic pressure.
    \begin{itemize}
        \item Imagine a U-tube with a filter at the bottom that is porous to the solvent but nonporous to the solute.
        \item We know from Gen Chem that there will be excess pressure on the side with the impurities.
        \item Let $\pi=\rho gh$ be the extra pressure where $h$ is the height difference between the two sides and $\rho$ is the density of the solvent.
        \begin{align*}
            \mu_l^\circ = \mu_l &= \mu_l^\circ+RT\ln x_1+\pi\overline{V}\\
            0 &= RT\ln x_1+\pi\overline{V}\\
            \pi &= \frac{RT}{\overline{V}}x_2
        \end{align*}
    \end{itemize}
    \item Ocean salinity is about \SI{1}{\molar}, so $\pi=\SI{24}{\atmosphere}$. That means that in the tube, the right hand side will rise about as much as the Sears tower. Thus, the minimum amount of pressure/work you need is the height of the Sears tower minus \SI{24}{\atmosphere}. Still, this is far more efficient than distillation.
\end{itemize}



\section{Nonideal Solutions}
\begin{itemize}
    \item \marginnote{3/2:}When solutions are not ideal, we see deviations from Raoult's Law (Figure \ref{fig:RaoultsLaw}).
    \begin{itemize}
        \item However, all solutions follow Raoult's Law when they are nearly pure.
        \item When the actual partial pressure is higher, that means the substance would rather be in the vapor phase. When the actual partial pressure is lower, that means there is a favorable interaction between particles (they'd rather be in the dissolved state).
    \end{itemize}
    \item \textbf{Henry's Law}: Gives the tangent to the vapor pressure vs. mole percent graph at $x_1=0$. \emph{Given by}
    \begin{equation*}
        P_1 = \kH x_1
    \end{equation*}
    as $x_1\to 0$.
    \item We know that $A(g)\to A(\text{solv})$.
    \begin{itemize}
        \item At equilibrium $\Delta\overline{G}=0=\Delta\overline{G}_\text{solv}+RT\ln x/P$.
        \item Thus,
        \begin{equation*}
            P = x\e[\Delta\overline{G}_\text{solv}/RT]
        \end{equation*}
        \item $x$ is the mixing entropy, $P$ is the gas phase entropy.
    \end{itemize}
    \item Microscopic enthalpy and entropy of solvation contribute to Henry's law, in addition to mixing entropy.
    \begin{itemize}
        \item Larger $\kH$ means less soluble.
    \end{itemize}
    \item Temperature dependence.
    \begin{equation*}
        k_{H,cp} = k_{H,cp}^\Theta\exp\left[ -C\cdot\left( \frac{1}{T}-\frac{1}{T^\Theta} \right) \right]
    \end{equation*}
    \begin{itemize}
        \item $\Theta$ indicates reference temperature.
        \item It follows that as $T$ increases, $\kH$ increases.
        \begin{itemize}
            \item For example, \ce{O2} is less soluble in water at higher temperatures.
            \item This is consistent with solvation being an exothermic process.
        \end{itemize}
    \end{itemize}
    \item The activity and activity coefficient of a solute. Using Raoult's law as the reference for fully miscible substances.
    \begin{itemize}
        \item We have for an ideal solution that $\mu_i=\mu_i^*+RT\ln x_i$ and for an ideal gas that $\mu_i=\mu_i^*+RT\ln\frac{P_i}{P_i^*}$.
        \item These two equations together imply Raoult's law.
        \item Now we want to keep the nice form of the above equations even with nonideality, so we do something similar to defining fugacity by defining the \textbf{activity}.
        \item $a_i\to x_i$ as $x_i\to 1$.
        \item As $x_i\to 0$, we have that
        \begin{align*}
            \mu_i^*+RT\ln a_i &= \mu_i^*+RT\ln\frac{P_i}{P_i^*}\\
            &= \mu_i^*+RT\ln\frac{\kH}{P_i^*}x_i
        \end{align*}
        i.e., $a_i\to \kH x_i/P_i^*$.
        \item As a substance becomes less active, it becomes more reactive.
    \end{itemize}
    \item \textbf{Activity}: A measure of the nonideality of solutions. \emph{Denoted by} $\bm{a_i}$. \emph{Given by}
    \begin{equation*}
        \mu_i = \mu_i^*+RT\ln a_i
    \end{equation*}
    \item \textbf{Activity coefficient}: The following ratio. \emph{Denoted by} $\bm{\gamma}$. \emph{Given by}
    \begin{equation*}
        \gamma = \frac{a_i}{x_i}
    \end{equation*}
    \item Example: Carbon disulfide/dimethoxymethane.
    \begin{table}[h!]
        \centering
        \begin{tabular}{ccc}
            $x_{\ce{CS2}}$ & $P_{\ce{CS2}}$ (\si{\torr}) & $P_{\ce{CH2(OMe)2}}$ (\si{\torr})\\
            0 & 0 & 587\\
            0.1 & 109 & 529\\
            1 & 514 & 0\\
        \end{tabular}
        \caption{Pressure data for \ce{CS2} and \ce{CH2(OMe)2}.}
        \label{fig:PCS2CH2OMe2}
    \end{table}
    \begin{itemize}
        \item At $x_{\ce{CS2}}=0.1$, we have that $a_{\ce{CS2}}=109/514$.
        \item As $x\to 0$, $k_{H,\ce{CS2}}=\SI{1130}{\torr}$ so $a_{\ce{CS2}}=\frac{1130}{514}x_{\ce{CS2}}$.
        \item See \textcite{bib:McQuarrieSimon} for using Henry's law as the reference state.
    \end{itemize}
    \item Nonideality above the Raoult's law diagonal: More "active" than the ideal mixture.
    \item Nonideality below the Raoult's law diagonal: Less active than at low concentration.
\end{itemize}



\section{Quantifying Deviations from Raoult's Law}
\begin{itemize}
    \item \marginnote{3/4:}\textbf{Gibbs-Duhem relation}: A relation tying together the partial pressure of one substance to another, regardless of ideality/nonideality.
    \begin{itemize}
        \item We begin with
        \begin{equation*}
            \dd{G} = \sum\mu_i\dd{n_i}+\sum n_i\dd{\mu_i}
        \end{equation*}
        at constant temperature and pressure.
        \item We also know from an earlier class that
        \begin{equation*}
            G = \sum\mu_in_i
        \end{equation*}
        \item Differentiating the above, we get
        \begin{equation*}
            \dd{G} = \sum\mu_i\dd{n_i}+\sum n_i\dd{\mu_i}
        \end{equation*}
        \item It follows by setting the above two equations equal to each other that
        \begin{equation*}
            \sum n_i\dd{\mu_i} = 0
        \end{equation*}
    \end{itemize}
    \item We can apply the Gibbs-Duhem relation to our expression for the chemical potential in terms of the activity coefficient.
    \begin{itemize}
        \item From last time,
        \begin{align*}
            \mu_i &= \mu_i^*+RT\ln(\gamma_ix_i)\\
            \dd{\mu_i} &= RT\dd{(\ln\gamma_i+\ln x_i)}
        \end{align*}
        \item It follows that
        \begin{align*}
            0 &= n_1\dd{\mu_1}+n_2\dd{\mu_2}\\
            &= \frac{1}{n_1+n_2}(n_1\dd{\mu_1}+n_2\dd{\mu_2})\\
            &= \frac{n_1}{n_1+n_2}RT\dd{(\ln\gamma_1+\ln x_1)}+\frac{n_2}{n_1+n_2}RT\dd{(\ln\gamma_2+\ln x_2)}\\
            &= RT(x_1\dd{\ln x_1}+x_2\dd{\ln x_2}+x_1\dd{\ln\gamma_1}+x_2\dd{\ln\gamma_2})\\
            &= \dd{x_1}+\dd{x_2}+x_1\dd{\ln\gamma_1}+x_2\dd{\ln\gamma_2}\\
            &= \dd{x_1}-\dd{x_1}+x_1\dd{\ln\gamma_1}+x_2\dd{\ln\gamma_2}\\
            &= x_1\dd{\ln\gamma_1}+x_2\dd{\ln\gamma_2}
        \end{align*}
        \item Takeaway: If we know the activity coefficient for one component across the whole range, we can find the activity coefficient for the other component.
        \item In other words, knowing the chemical potential for one component gives us the chemical potential for the other.
    \end{itemize}
    \item Relating $P_1,P_2$ with the Gibbs-Duhem relation.
    \begin{itemize}
        \item We assume that we are in phase equilibrium. Thus, the chemical potential of the gas phase equals the chemical potential of the solution. In an equation,
        \begin{equation*}
            \mu_1^*+RT\ln\frac{P_1}{P_1^*} = \mu_1+RT\ln a_1
        \end{equation*}
        \item Thus, applying the Gibbs-Duhem relation,
        \begin{align*}
            0 &= x_1\dd{\ln P_1}+x_2\dd{\ln P_2}\\
            &= x_1\left( \pdv{\ln P_1}{x_1} \right)_{T,P}\dd{x_1}+x_2\left( \pdv{\ln P_2}{x_2} \right)_{T,P}\dd{x_2}\\
            &= x_1\left( \pdv{\ln P_1}{x_1} \right)_{T,P}-x_2\left( \pdv{\ln P_2}{x_2} \right)_{T,P}
        \end{align*}
        \item Takeaway: The slopes of the curves on a pressure-concentration diagram are related, regardless of ideality/nonideality.
    \end{itemize}
    \item \textbf{Margules equation}: An expansion to capture the experimental $P(x)$ deviation from Raoult's law.
    \begin{equation*}
        P_1 = x_1P_1^*\e[\alpha x_2^2+\beta x_2^3+\cdots]
    \end{equation*}
    \begin{itemize}
        \item An analytical form that gives us a good representation of experimentally-measured vapor pressures.
        \item Important features: Follows Henry's law at low concentrations of solute and Raoult's Law at high concentrations of solute.
    \end{itemize}
    \item Margules equation covers Raoult's law and Henry's law.
    \begin{itemize}
        \item We have by the product rule of derivatives that
        \begin{equation*}
            \dv{P_1}{x_1} = P_1^*\left[ \e[\alpha x_2^2+\beta x_2^3+\cdots]+x_1(-2\alpha x_2-3\beta x_2^2-\cdots)\e[\alpha x_2^2+\beta x_2^3+\cdots] \right]
        \end{equation*}
        \item As $x_1\to 1$, $\dv*{P_1}{x_1}\to P_1^*$. This can be seen because the above expression can be rewritten
        \begin{equation*}
            \dv{P_1}{x_1} = P_1^*\e[\alpha x_2^2+\beta x_2^3+\cdots]\left[ 1-2\alpha x_1x_2-3\beta x_1x_2^2-\cdots \right]
        \end{equation*}
        \begin{itemize}
            \item Implication: We recover Raoult's law as $x_1\to 1$.
            \item Additionally, this explains why there is no linear terms in the exponent: If there were, one of the terms in the sum on the right-hand side above wouldn't go to zero as $x_1\to 1$, meaning that $\dv*{P_1}{x_1}$ would not go to $P_1^*$.
        \end{itemize}
        \item Recovering Henry's law as $x_1\to 0$: As $x_1\to 0$, $P_1\to \kH x_1=P_1^*\e[\alpha+\beta+\cdots]\cdot x_1$.
    \end{itemize}
    \item Margules equation parameters fitted for $P_1$ provide a related equation for $P_2$ based on Gibbs-Duhem (Problem 24-33).
    \begin{itemize}
        \item To two terms, the Margules equation tells us that
        \begin{equation*}
            P_1 = x_1P_1^*\e[\alpha x_2^2+\beta x_2^3]
        \end{equation*}
        \item As we showed earlier, the Gibbs-Duhem equation implies that
        \begin{equation*}
            x_1\pdv{\ln P_1}{x_1} = x_2\pdv{\ln P_2}{x_2}
        \end{equation*}
        \item Thus, since $x_2=1-x_1$,
        \begin{align*}
            x_1\pdv{\ln P_1}{x_1} &= x_1\pdv{x_1}(\ln x_1+\alpha x_2^2+\beta x_2^3+\text{terms not involving }x_1)\\
            &= x_1\left[ \frac{1}{x_1}+(-2\alpha x_2)+(-3\beta x_2^2) \right]\\
            &= 1+x_2(-2\alpha x_1-3\beta x_1x_2)\\
            &= 1+x_2(-2\alpha x_1-3\beta x_1(1-x_1))\\
            &= 1+x_2(-2\alpha x_1-3\beta(x_1-x_1^2))\\
            &= 1+x_2[-(2\alpha+3\beta)x_1+3\beta x_1^2]
        \end{align*}
        \item Now let
        \begin{equation*}
            P_2 = x_2P_2^*\e[ax_1^2+bx_1^3]
        \end{equation*}
        \item It follows as before that
        \begin{equation*}
            x_2\pdv{\ln P_2}{x_2} = 1+x_2(-2ax_1-3bx_1^2)
        \end{equation*}
        \item Direct comparison reveals that
        \begin{align*}
            -2a &= -(2\alpha+3\beta)&
                -3b &= 3\beta\\
            a &= \alpha+\frac{3}{2}\beta&
                b &= -\beta
        \end{align*}
        so
        \begin{equation*}
            P_2 = x_2P_2^*\e[(\alpha+\frac{3}{2}\beta)x_1^2-\beta x_1^3]
        \end{equation*}
    \end{itemize}
    \item \textbf{Regular solution}: A solution for which only $\alpha\neq 0$ and every other Margules parameter is zero.
    \begin{itemize}
        \item In this case,
        \begin{align*}
            P_1 &= x_1P_1^*\e[\alpha x_2^2]&
            P_2 &= x_2P_2^*\e[\alpha x_1^2]
        \end{align*}
        \item This means that the Raoult's activity of the two solutions is entirely symmetric over the range of compositions.
        \item This does not necessarily imply that the vapor pressures vary in the same way.
    \end{itemize}
    \item Mixing Gibbs free energy for a regular solution (24-5).
    \begin{itemize}
        \item We have
        \begin{align*}
            \frac{\Delta G_\text{mix}}{RT} &= \underbrace{n_1\mu_1^\text{sol}+n_2\mu_2^\text{sol}}_\text{mixture}-(\underbrace{n_1\mu_1^*+n_2\mu_2^*}_\text{pure})\\
            &= n_1\ln\frac{P_1}{P_1^*}+n_2\ln\frac{P_2}{P_2^*}\\
            &= n_1\ln x_1+n_2\alpha x_2^2+n_2\ln x_2+n_2\alpha x_1^2\\
            \frac{\Delta G_\text{mix}}{RT(n_1+n_2)} &= x_1\ln x_1+x_2\ln x_2+\alpha x_1x_2^2+\alpha x_2x_1^2\\
            &= x_1\ln x_1+x_2\ln x_2+\alpha x_1x_2(x_1+x_2)\\
            \frac{\Delta G_\text{mix}}{RT(n_1+n_2)} &= x_1\ln x_1+x_2\ln x_2+\alpha x_1x_2
        \end{align*}
        \item The left two terms above represent the entropy of mixing (which does not depend on interaction).
        \item The right term above is the enthalpy/entropy of mixing which depends \emph{only} on interaction.
        \item Thus, $\alpha$ is the microscopic entropy of mixing for a regular solution.
    \end{itemize}
    \item The upper critical solution temperature (UCST) (critical temperature above which the components of the mixture are miscible in all proportions).
    \begin{itemize}
        \item We have
        \begin{equation*}
            \Delta\overline{G} = RT(x_1\ln x_1+x_2\ln x_2)+\underbrace{RT\alpha}_\Omega x_1x_2
        \end{equation*}
        \item Graphing the above equation will help us determine what extent of mixing will be most energetically favorable.
        \begin{itemize}
            \item In particular, if we graph the left two terms, that tells us the entropy of mixing, which will be greatest at $x_1=x_2$.
            \item Graphing the right term on the other hand yields a parabola.
            \item As such, since $x_1\ln x_1+x_2\ln x_2$ has vertical slope at $x_1=0,1$ and the other term simply has some positive slope, entropy of mixing will always win at low concentrations of solutes, i.e., when you have a small enough concentration of solute in solvent, you will always observe mixing.
            \item However, the overall graph may have minima on the two sides, indicating that forming a fully mixed solution is not energetically favorable, but it is most favorable to separate into substance 1 with a bit of substance 2 and vice versa (think oil and water).
            \item At high temperatures, however, this effect goes away and mixing always becomes the most favorable state.
        \end{itemize}
        \item The temperature at which the overall graph only has one minimum for the first time is the UCST.
    \end{itemize}
\end{itemize}



\section{Office Hours (PGS)}
\begin{itemize}
    \item Can you go over the spontaneous expansion of an ideal gas from one isolated container to two? Is this an adiabatic process or an isothermal process? How can the gas cool down during the expansion if it ends up at the same final temperature?
    \begin{itemize}
        \item An isoenthalpic expansion? Only as long as we can neglect the back pressure of the far wall. In this moment,
        \begin{align*}
            0 &= \Delta H\\
            &= \Delta U+\Delta KE+\Delta(PV)
        \end{align*}
        \item The gas shooting out is expanding and picking up a lot of kinetic energy, but we need to conserve enthalpy all together, so that kinetic energy comes from vibrational and spin states.
        \item After everything comes to equilibrium, though, the temperature is the same.
        \item It's irreversible, so that means that during the process, you do not have well-defined state variables.
        \item Note that no work is done even for the intermixing of two gases.
        \begin{itemize}
            \item Any work done by one gas is canceled by the work on the other gas and no work comes out of the system; it's a pure entropy effect.
            \item But work is being done by both gases; just equal and opposite?
        \end{itemize}
    \end{itemize}
    \item How does the Henry's law we learned relate to the one from Gen Chem (the amount of gas dissolved in solution is proportional to its partial pressure)?
    \begin{itemize}
        \item The Henry's law of intro chem says that concentration vs. pressure will always be a straight line.
        \item We know now that we see deviations at higher concentrations, and that Henry's law is just a good approximation at relatively low pressures.
    \end{itemize}
    \item Demixing of solutions at lower temperatures?
    \begin{itemize}
        \item The line connecting all of the two minima with peak at the UCMT is the \textbf{spinodal line}.
    \end{itemize}
\end{itemize}



\section{Chapter 24: Solutions I --- Liquid-Liquid Solutions}
\emph{From \textcite{bib:McQuarrieSimon}.}
\begin{itemize}
    \item There exist positive and negative deviations from Raoult's Law (see Figure \ref{fig:postiveDeviationRaoult}).
    \begin{figure}[h!]
        \centering
        \begin{tikzpicture}[xscale=5,yscale=0.5]
            \small
            \draw (0,6) -- node[rotate=90,above=7mm]{Vapor pressure (\si{\torr})} (0,0) -- node[below=5mm]{$x_1$} (1,0);
            \footnotesize
            \node [above left,xshift=-1mm,yshift=-1mm] {0};
            \node [below right,xshift=-1mm,yshift=-1mm] {0.0};
            \foreach \y [evaluate=\y as \n using int(\y*100)] in {2,4,6} {
                \draw (0.02,\y) -- ++(-0.04,0) node[left]{$\n$};
            }
            \foreach \x in {0.2,0.4,0.6,0.8,1.0} {
                \draw (\x,0.2) -- ++(0,-0.4) node[below]{$\x$};
            }
    
            \draw [gry,thick] plot [domain=0:1,smooth] (\x,{5*\x});
            \draw [gry,thick] plot [domain=0:1,smooth] (\x,{5.9-5.9*\x});
            \draw [grx,thick] plot [domain=0:1,smooth] (\x,{5*\x*e^((1-\x)^2)});
            \draw [grx,thick] plot [domain=0:1,smooth] (\x,{(5.9-5.9*\x)*e^(\x^2)});
    
            \node at (0.52,0.7) {Raoult's Law}
                edge [->,out=0,in=-95,looseness=0.5] (0.8,1)
                edge [->,out=180,in=-85,looseness=0.5] (0.23,1)
            ;
            \node at (0.17,2.5) {$P_1$};
            \node at (0.85,2.5) {$P_2$};
        \end{tikzpicture}
        \caption{Positive deviations from Raoult's law (\ce{CS2}/\ce{CH2(OMe)2} at \SI{25}{\celsius}).}
        \label{fig:postiveDeviationRaoult}
    \end{figure}
    \item If 1 and 2 denote two substances that are being mixed, positive deviations occur when 1-2 interactions are more repulsive than either 1-1 or 2-2 interactions, and vice versa for negative deviations.
    \item If one component of a binary solution exhibits positive deviations from Raoult's law, so must the other component.
    \begin{itemize}
        \item Follows from "Relating $P_1,P_2$ with the Gibbs-Duhem relation" as in class.
    \end{itemize}
    \item Molecular structure predicts deviations from Raoult's law.
    \begin{figure}[H]
        \centering
        \begin{tikzpicture}[xscale=5,yscale=0.5]
            \small
            \draw (0,6) -- node[rotate=90,above=2mm]{Vapor pressure} (0,0) -- node[below=5mm]{$x_\text{alcohol}$} (1,0);
            \footnotesize
            \node [below right,xshift=-1mm,yshift=-1mm] {0.0};
            \foreach \y in {2,4,6} {
                \draw (0.02,\y) -- ++(-0.04,0);
            }
            \foreach \x in {0.2,0.4,0.6,0.8,1.0} {
                \draw (\x,0.2) -- ++(0,-0.4) node[below]{$\x$};
            }
    
            \draw [grx,thick] plot [domain=0:1,smooth] (\x,{6*\x});
            \draw [blx,thick] plot [domain=0:1,smooth] (\x,{6*\x*e^(0.8*(1-\x)^2)});
            \draw [rex,thick] plot [domain=0:1,smooth] (\x,{6*\x*e^(1.4*(1-\x)^2)});
            \draw [orx,thick] plot [domain=0:1,smooth] (\x,{6*\x*e^(1.8*(1-\x)^2)});
    
            \node at (0.55,1) {Raoult's Law}
                edge [->,out=180,in=-85,looseness=0.5] (0.25,1.4)
            ;
        \end{tikzpicture}
        \caption{Vapor pressure of alcohol/water solutions.}
        \label{fig:VPalcoholWater}
    \end{figure}
    \begin{itemize}
        \item The blue line represents \ce{MeOH}, the red line represents \ce{EtOH}, and the orange line represents \ce{PrOH}.
        \item Notice how as molecular structure gets further from that of \ce{H2O} (the solvent), deviations become more exaggerated.
    \end{itemize}
    \item \textcite{bib:McQuarrieSimon} discusses the limiting behavior of $P_1$ in Figure \ref{fig:postiveDeviationRaoult}, noting the Raoult's law limit as $x_1\to 1$ and introducing \textbf{Henry's law}.
    \item \textbf{Henry's law}: The partial pressure of a component in solution approaches the linear function $k_{H,i}x_i$ as $x_i\to 0$.
    \item \textbf{Henry's law constant}: The constant $k_{H,i}$ in the above definition.
    \item For an ideal solution, $k_{H,i}=P_i^*$, but this is not true in general (as illustrated by Figure \ref{fig:postiveDeviationRaoult}).
    \item Example: Finding $P_1^*$ and $\kH$ from an equation for $P_1$.
    \begin{itemize}
        \item Let $P_1=180x_1\e[x_2^2+\frac{1}{2}x_2^3]$.
        \item As $x_1\to 1$, $x_2\to 0$, so $P_1\to 180x_1$. Since this form should be the Raoult's law limit, we can read off $P_1^*=\SI{180}{\torr}$.
        \item As $x_1\to 0$, $x_2\to 1$, so $P_1\to 180x_1\e[1+\frac{1}{2}]=180\e[3/2]x_1$. Since this form should be the Henry's law limit, we can read off $k_{H,1}=180\e[3/2]=\SI{807}{\torr}$.
    \end{itemize}
    \item \textcite{bib:McQuarrieSimon} derives
    \begin{equation*}
        x_1\left( \pdv{\ln P_1}{x_1} \right)_{T,P} = x_2\left( \pdv{\ln P_2}{x_2} \right)_{T,P}
    \end{equation*}
    as in class.
    \item In a binary solution, if component 1 obeys Raoult's law as $x_1\to 1$, then component 2 obeys Henry's law as $x_2\to 0$.
    \begin{itemize}
        \item Suppose component 1 satisfies $P_1\to x_1P_1^*$ as $x_1\to 1$. Then
        \begin{align*}
            x_2\left( \pdv{\ln P_2}{x_2} \right)_{T,P} &= x_1\left( \pdv{\ln P_1}{x_1} \right)_{T,P}\\
            &= x_1\pdv{x_1}(\ln x_1+\ln P_1^*)\\
            &= 1\\
            \ln P_2 &= \int\frac{1}{x_2}\dd{x_2}\\
            &= \ln x_2+C\\
            P_2 &= \kH x_2
        \end{align*}
        as desired.
    \end{itemize}
    \item The vapor pressure curve of WLOG component 1 of a binary solution can often be represented empirically by
    \begin{equation*}
        P_1 = x_1P_1^*\e[\alpha x_2^2+\beta x_2^3]
    \end{equation*}
    \begin{itemize}
        \item $\alpha,\beta$ reflect the extent of the nonideality of the solution in some manner because $P_1=x_1P_1^*$ (ideal) for $\alpha=\beta=0$.
        \item Furthermore, generalizing from the previous example, $k_{H,1}=P_1^*\e[\alpha+\beta]$ and $k_{H,2}=P_2^*\e[\alpha+\beta/2]$ (the latter equation will be derived presently).
    \end{itemize}
    \item Deriving $P_2=x_2P_2^*\e[(\alpha+3\beta/2)x_1^2-\beta x_1^3]$ from $P_1=x_1P_1^*\e[\alpha x_2^2+\beta x_2^3]$ in an alternate manner to what was presented in class.
    \begin{itemize}
        \item We are given that $P_1=x_1P_1^*\e[\alpha x_2^2+\beta x_2^3]$ and that $x_2=1-x_1$.
        \item It follows that the chemical potential of component 1 is
        \begin{align*}
            \mu_1 &= \mu_1^\circ+RT\ln P_1\\
            &= \mu_1^\circ+RT\left[ \ln P_1^*+\ln x_1+\alpha(1-x_1)^2+\beta(1-x_1)^3 \right]
        \end{align*}
        \item It follows that the total differential of $\mu_1(x_1)$ is
        \begin{equation*}
            \dd{\mu_1} = RT\left[ \frac{1}{x_1}-2\alpha(1-x_1)-3\beta(1-x_1)^2 \right]\dd{x_1}
        \end{equation*}
        \item Thus, by the Gibbs-Duhem equation,
        \begin{align*}
            x_1\dd{\mu_1}+x_2\dd{\mu_2} &= 0\\
            \dd{\mu_2} &= -\frac{x_1}{x_2}\cdot RT\left[ \frac{1}{x_1}-2\alpha(1-x_1)-3\beta(1-x_1)^2 \right]\dd{x_1}\\
            &= RT\left[ -\frac{1}{x_2}+2\alpha x_1+3\beta x_1(1-x_1) \right]\dd{x_1}
        \end{align*}
        \item Substitute $x_2=1-x_1$ and integrate.
        \begin{align*}
            \dd{\mu_2} &= RT\left[ \frac{1}{x_2}-2\alpha(1-x_2)-3\beta x_2(1-x_2) \right]\dd{x_2}\\
            \int_1^{x_2}\dd{\mu_2} &= RT\int_1^{x_2}\left[ \frac{1}{x_2'}-2\alpha(1-x_2')-3\beta x_2'(1-x_2') \right]\dd{x_2'}\\
            \mu_2(x_2)-\mu_2(1) &= RT\left[ \ln x_2+\alpha(1-x_2)^2-\frac{3\beta}{2}(x_2^2-1)+\beta(x_2^3-1) \right]\\
            \mu_2-\mu_2^* &= RT\left[ \ln x_2+\alpha x_1^2+\frac{3\beta}{2}x_1^2-\beta x_1^3 \right]
        \end{align*}
        \item Therefore, since
        \begin{align*}
            \mu_2 &= \mu_2^\circ+RT\ln P_2&
            \mu_2^* &= \mu_2^\circ+RT\ln P_2^*
        \end{align*}
        we have that
        \begin{align*}
            \ln P_2 &= \ln P_2^*+\ln x_2+\alpha x_1^2+\frac{3\beta}{2}x_1^2-\beta x_1^3\\
            P_2 &= x_2P_2^*\e[(\alpha+3\beta/2)x_1^2-\beta x_1^3]
        \end{align*}
    \end{itemize}
    \item \marginnote{3/5:}\textbf{Azeotrope}: A mixture for which there is no change in composition upon boiling.
    \item Vapor pressure behavior at various temperatures.
    \begin{figure}[h!]
        \centering
        \begin{tikzpicture}[scale=4]
            \small
            \draw [stealth-stealth] (0,1.2) -- node[rotate=90,above=5mm]{$P_2/P_2^*$} (0,0) -- node[below=5mm]{$x_2$} (1.1,0);
            \footnotesize
            \draw (0.025,1) -- ++(-0.05,0) node[left]{$1.0$};
            \draw (1,0.025) -- ++(0,-0.05) node[below]{$1.0$};
            \draw [dashed] (0.252,0.913) -- (0.252,-0.025) node[below]{$x_2'$};
            \draw [dashed] (0.748,0.913) -- (0.748,-0.025) node[below]{$x_2''$};
    
            \draw [semithick] plot[domain=0.034:0.966,smooth] (\x,{0.686*(\x*ln(\x)+(1-\x)*ln(1-\x))+1.3});
            \node at (0.5,1.15) {Coexistence curve}
                edge [out=180,in=60,->] (0.1,1.1)
            ;
            
            \draw [grx,thick] (0,0) -- (1,1);
            \draw [blx,thick] plot[domain=0:1,smooth] (\x,{\x*e^(1.4*(1-\x)^2)});
            \draw [rex,thick] plot[domain=0:1,smooth] (\x,{\x*e^(2*(1-\x)^2)});
            \draw [orx,thick] plot[domain=0:0.252,smooth] (\x,{\x*e^(2.3*(1-\x)^2)}) -- (0.748,0.913) to[out=0,in=-135,in looseness=0.7] (1,1);
            \draw [yex,thick] plot[domain=0:0.184,smooth] (\x,{\x*e^(2.5*(1-\x)^2)}) -- (0.816,0.972) to[out=0,in=-135,in looseness=0.7] (1,1);
    
            \node [below,xshift=1mm] at (0.5,0.710) {$T_3$};
            \node [below] at (0.4,0.822) {$T_c$};
            \node [below=-1pt] at (0.5,0.913) {$T_2$};
            \node [above] at (0.5,0.972) {$T_1$};
        \end{tikzpicture}
        \caption{The critical behavior of a binary solution.}
        \label{fig:criticalBinarySoln}
    \end{figure}
    \begin{itemize}
        \item In Figure \ref{fig:criticalBinarySoln}, $T_3>T_c>T_2>T_1$. Additionally, $P_2^*$ denotes the vapor pressure of the pure substance \emph{at each temperature}, and thus serves to "normalize" all of the curves.
        \item At or above the \textbf{consulate temperature} $T_c$, component 1 and component 2 will mix to form a homogenous solution regardless of their relative concentrations.
        \item On the other hand, consider the solution formed at $T_2<T_c$.
        \begin{itemize}
            \item If $x_2<x_2'$ or $x_2>x_2''$, a homogenous solution is formed.
            \item However, if $x_2$ is within this range, the mixture separates into a solution with component 2 concentration $x_2'$ and a solution with component 2 concentration $x_2''$, with the relative proportions dictated by what will give the constant vapor pressure indicated by Figure \ref{fig:criticalBinarySoln}.
        \end{itemize}
        \item "Points inside the \textbf{coexistence curve} represent two solution phases, whereas points below the coexistence curve represent one solution phase" \parencite[985]{bib:McQuarrieSimon}.
    \end{itemize}
    \item \textbf{Consulate temperature}: The temperature below which two liquids are not miscible in all proportions. \emph{Also known as} \textbf{critical temperature}. \emph{Denoted by} $\bm{T_c}$.
    \item A lever rule that gives the relative amounts of the two phases indicated by Figure \ref{fig:criticalBinarySoln}.
    \begin{itemize}
        \item Consider a solution of overall composition $x_2\in(x_2',x_2'')$. Let $n_1',n_2'$ and $n_1'',n_2''$ be the number of moles of the two components in the phases of composition $x_2'$ and $x_2''$, respectively.
        \item By definition, we have the relations
        \begin{align*}
            x_2' &= \frac{n_2'}{n_1'+n_2'}&
            x_2'' &= \frac{n_2''}{n_1''+n_2''}&
            x_2 &= \frac{n_2'+n_2''}{n_1'+n_1''+n_2'+n_2''}
        \end{align*}
        \item It follows that
        \begin{align*}
            x_2(n_1'+n_1''+n_2'+n_2'') &= x_2'(n_1'+n_2')+x_2''(n_1''+n_2'')\\
            \frac{n'}{n''} &= \frac{n_1'+n_2'}{n_1''+n_2''} = \frac{x_2''-x_2}{x_2-x_2'}
        \end{align*}
    \end{itemize}
    \item \textbf{Activity} (of component $j$): A state function describing the nonideality of the chemical potential of component $j$ of a solution. \emph{Denoted by} $\bm{a_j}$. \emph{Given by}
    \begin{equation*}
        \mu_j^\text{sln} = \mu_j^*+RT\ln a_j
    \end{equation*}
    \begin{itemize}
        \item It follows from Raoult's law that $a_j\to x_j$ as $x_j\to 1$.
        \item We can also represent the activity of component 1 in a binary solution empirically by
        \begin{equation*}
            a_1 = x_1\e[\alpha x_2^2+\beta x_2^3+\cdots]
        \end{equation*}
    \end{itemize}
    \item \textbf{Activity coefficient} (of component $j$): A measure of the deviation of a solution from ideality. \emph{Denoted by} $\bm{\gamma_j}$. \emph{Given by}
    \begin{equation*}
        \gamma_j = \frac{a_j}{x_j}
    \end{equation*}
    \begin{itemize}
        \item It should make intuitive sense that $\gamma_j\to 1$ as $x_j\to 1$.
    \end{itemize}
    \item Since activity and chemical potential are directly related, activity is really just another way of expressing chemical potential.
    \begin{itemize}
        \item It follows that since the chemical potentials of the components of a binary solution are related by the Gibbs-Duhem equation, the activities of said components are related by
        \begin{align*}
            0 &= x_1\dd{(\mu_j^*+RT\ln a_1)}+x_2\dd{(\mu_j^*+RT\ln a_2)}\\
            &= RTx_1\dd{\ln a_1}+RTx_2\dd{\ln a_2}\\
            &= x_1\dd{\ln a_1}+x_2\dd{\ln a_2}
        \end{align*}
        \item Consequently, if $a_1=x_1$ over the entire composition range, we can use this relation to show that $a_2=x_2$.
        \item Similarly, if $a_1=x_1\e[\alpha x_2^2]$ over the entire composition range, we can use this relation to show that $a_2=x_2\e[\alpha x_1^2]$.
    \end{itemize}
    \item An activity defined by $a_j=P_j/P_j^*$ is said to be based upon a solvent, or Raoult's law, standard state.
    \begin{itemize}
        \item This is most convenient for substances that are miscible in all proportions, i.e., when there is no clear "solute" and "solvent."
    \end{itemize}
    \item If one component is only sparingly soluble in the other, however, picking a standard state based on Henry's law is more convenient.
    \begin{itemize}
        \item In this case, $P_j\to x_jk_{H,j}$ as $x_j\to 0$ (we consider $x_j$ close to zero because component $j$ is sparingly soluble, so its concentration is naturally low).
        \item We thus have that
        \begin{equation*}
            \mu_j^\text{sln} = \mu_j^*+RT\ln\frac{k_{H,j}}{P_j^*}+RT\ln x_j
        \end{equation*}
        and must define
        \begin{equation*}
            \mu_j^\text{sln} = \mu_j^*+RT\ln\frac{k_{H,j}}{P_j^*}+RT\ln a_j
        \end{equation*}
        \item It follows that for these two equations to be the same as $x_j\to 0$, we must take $a_j=P_j/k_{H,j}$.
        \item Additionally, the standard state must be such that
        \begin{equation*}
            \mu_j = \mu_j^*+RT\ln\frac{k_{H,j}}{P_j^*}
        \end{equation*}
        or $P_j^*=k_{H,j}$.
        \item If such a standard state is not achievable in practice, it is called a \textbf{hypothetical standard state}.
    \end{itemize}
    \item Deriving an expression for the Gibbs energy of mixing of binary solutions in terms of the activity coefficients.
    \begin{itemize}
        \item We have that
        \begin{equation*}
            \Delta_\text{mix}G = n_1\mu_1^\text{sln}+n_2\mu_2^\text{sln}-n_1\mu_1^*-n_2\mu_2^*
        \end{equation*}
        \item Since $\mu_j^\text{sln}=\mu_j^*+RT\ln a_j$ for any solution and $\gamma_j=a_j/x_j$, it follows that
        \begin{align*}
            \frac{\Delta_\text{mix}G}{RT} &= n_1\ln x_1+n_2\ln x_2+n_1\ln\gamma_1+n_2\ln\gamma_2\\
            \frac{\Delta_\text{mix}\overline{G}}{RT} &= x_1\ln x_1+x_2\ln x_2+x_1\ln\gamma_1+x_2\ln\gamma_2
        \end{align*}
        \item Note that the left two terms above represent the Gibbs energy of mixing of an ideal solution.
    \end{itemize}
    \item We can use the above result to derive a formula for $\Delta_\text{mix}\overline{G}$ for a binary solution described by
    \begin{align*}
        P_1 &= x_1P_1^*\e[\alpha x_2^2]&
        P_2 &= x_2P_2^*\e[\alpha x_1^2]
    \end{align*}
    as in class. The final result is
    \begin{equation*}
        \frac{\Delta_\text{mix}\overline{G}}{RT} = x_1\ln x_1+x_2\ln x_2+\alpha x_1x_2
    \end{equation*}
    \item Statistical mechanics suggests that $\alpha$ has the form $w/RT$, where $w$ is a constant. Thus, we can write the above equation as
    \begin{equation*}
        \frac{\Delta_\text{mix}\overline{G}}{w} = \frac{RT}{w}(x_1\ln x_1+x_2\ln x_2)+x_1x_2
    \end{equation*}
    \item Visualizing the Gibbs energy of mixing.
    \begin{figure}[h!]
        \centering
        \begin{tikzpicture}[xscale=4,yscale=16]
            \small
            \draw (0,0) -- node[rotate=90,above=11mm]{$\Delta_\text{mix}\overline{G}/w$} (0,-0.2) -- node[below=7mm]{$x_1$} (1,-0.2);
            \footnotesize
            \node [below right,xshift=-1mm,yshift=-1mm] at (0,-0.2) {$0.00$};
            \node [above left,xshift=-1mm,yshift=-1mm] at (0,-0.2) {$-0.20$};
            \foreach \x in {0.25,0.50,0.75,1.00} {
                \draw (\x,-0.19375) -- ++(0,-0.0125) node[below]{$\x$};
            }
            \foreach \y in {0.00,-0.05,-0.10,-0.15} {
                \draw (0.025,\y) -- ++(-0.05,0) node[left]{$\y$};
            }
    
            \draw [rex,thick]
                plot[domain=0.0001:1,samples=100,smooth,/pgf/fpu,/pgf/fpu/output format=fixed] (\x,{0.4*(\x*ln(\x)+(1-\x)*ln(1-\x))+\x*(1-\x)})
                plot[domain=0.001:1,samples=100,smooth,/pgf/fpu,/pgf/fpu/output format=fixed] (\x,{0.5*(\x*ln(\x)+(1-\x)*ln(1-\x))+\x*(1-\x)})
                plot[domain=0.001:1,samples=100,smooth,/pgf/fpu,/pgf/fpu/output format=fixed] (\x,{0.6*(\x*ln(\x)+(1-\x)*ln(1-\x))+\x*(1-\x)})
            ;
        \end{tikzpicture}
        \caption{$\Delta_\text{mix}\overline{G}$ at various temperatures.}
        \label{fig:mixingTemperature}
    \end{figure}
    \begin{itemize}
        \item The curves with a maximum at $x_1=0.5$ have an interesting relation to the extraneous curves in the cubic van der Waals and Redlich-Kwong equations for $T<T_c$. In particular, the region between their two minima corresponds to unmixing of the solutions and should be flatter, as per Figure \ref{fig:criticalBinarySoln}.
        \item Note that the three curves plotted are for, from the top down, $RT/w=0.4$, $RT/w=0.5$, and $RT/w=0.6$.
    \end{itemize}
    \item By finding the points where $\pdv*{(\Delta_\text{mix}\overline{G}/w)}{x_1}=0$, we can plot the coexistence curve from Figure \ref{fig:criticalBinarySoln}.
    \begin{itemize}
        \item Note that for $RT/w\geq 0.5$, $\pdv*{(\Delta_\text{mix}\overline{G}/w)}{x_1}=0$ only has one solution ($x_1=0.5$), whereas for $RT/w<0.5$, $\pdv*{(\Delta_\text{mix}\overline{G}/w)}{x_1}=0$ has three solutions ($x_1=0.5$ and two others, corresponding to the other minima as in the plot of $RT/w=0.4$ in Figure \ref{fig:mixingTemperature}).
    \end{itemize}
    \item \textbf{Regular solution}: A solution that can be described by the equation $\Delta_\text{mix}\overline{G}/RT=x_1\ln x_1+x_2\ln x_2+\alpha x_1x_2$.
    \item \textbf{Excess Gibbs energy of mixing}: The extra Gibbs energy of mixing present in a solution beyond what would be present in an ideal solution. \emph{Denoted by} $\bm{G^E}$. \emph{Given by}
    \begin{equation*}
        G^E = \Delta_\text{mix}G-\Delta_\text{mix}G^\text{id}
    \end{equation*}
    \item It follows from the above that
    \begin{equation*}
        \frac{G^E}{RT} = n_1\ln\gamma_1+n_2\ln\gamma_2
    \end{equation*}
    \item \textbf{Molar excess Gibbs energy of mixing}: The excess Gibbs energy of mixing for one mole of a solution. \emph{Denoted by} $\bm{\overline{G}^E}$. \emph{Given by}
    \begin{equation*}
        \frac{\overline{G}^E}{RT} = x_1\ln\gamma_1+x_2\ln\gamma_2
    \end{equation*}
    \item If the solution in question is described by
    \begin{align*}
        P_1 &= x_1P_1^*\e[\alpha x_2^2]&
        P_2 &= x_2P_2^*\e[\alpha x_1^2]
    \end{align*}
    then we have that
    \begin{equation*}
        \frac{\overline{G}^E}{RT} = \alpha x_1x_2
    \end{equation*}
\end{itemize}



\section{Chapter 25: Solutions II --- Solid-Liquid Solutions}
\emph{From \textcite{bib:McQuarrieSimon}.}
\begin{itemize}
    \item \marginnote{3/6:}\textbf{Solute}: The sparingly soluble component in a binary solution.
    \begin{itemize}
        \item Related quantities are denoted by a subscript 2.
    \end{itemize}
    \item \textbf{Solvent}: The component in excess in a binary solution.
    \begin{itemize}
        \item Related quantities are denoted by a subscript 1.
    \end{itemize}
    \item $a_1$ (the activity of the solvent) is defined with respect to a Raoult's law standard state. $a_2$ (the activity of the solute) is defined with respect to a Henry's law standard state.
    \begin{itemize}
        \item Note that we still define $a_2$ with respect to its vapor pressure (even though that quantity may be exceedingly small) because it is still meaningful.
    \end{itemize}
    \item \textbf{Molality}: The number of moles of solute per \SI{1000}{\gram} of solvent. \emph{Denoted by} $\bm{m}$. \emph{Units} $\textbf{mol}\,\textbf{kg}^{\bm{-1}}$. \emph{Given by}
    \begin{equation*}
        m = \frac{n_2}{\SI{1000}{\gram\ solvent}}
    \end{equation*}
    \item Relating the mole fraction of solute $x_2$ and the molality $m$.
    \begin{equation*}
        x_2 = \frac{n_2}{n_1+n_2}
        = \frac{m}{\frac{\SI{1000}{\gram\per\kilo\gram}}{M_1}+m}
    \end{equation*}
    \begin{itemize}
        \item $M_1$ is the molar mass of the solvent.
        \item $\SI{1000}{\gram\per\kilo\gram}/M_1$ is the number of moles of solvent $n_1$ in \SI{1000}{\gram} of solvent.
        \item We convert from the middle equation to the right equation by dividing both the top and bottom of the fraction by \SI{1000}{\gram}.
    \end{itemize}
    \item For \ce{H2O},
    \begin{equation*}
        \frac{\SI{1000}{\gram\per\kilo\gram}}{M_1} = \SI[per-mode=symbol]{55.506}{\mole\per\kilo\gram}
    \end{equation*}
    \item We define solute activity in terms of molality by requiring that $a_2\to m$ as $m\to 0$.
    \item Under this definition, we can reexpress Henry's law in terms of molality with a related set of Henry's law constants. These give us equations like $P_2=\kH,m$.
    \item \textbf{Molarity}: The number of moles of solute per \SI{1000}{\milli\liter} of solution. \emph{Denoted by} $\bm{c}$. \emph{Units} $\textbf{mol}\,\textbf{L}^{\bm{-1}}$. \emph{Given by}
    \begin{equation*}
        c = \frac{n_2}{\SI{1000}{\milli\liter\ solution}}
    \end{equation*}
    \item We can also analogously define solute activity in terms of molarity, requiring $a_2\to c$ as $c\to 0$.
    \item Relating the solute mole fraction and molarity via the density.
    \begin{itemize}
        \item Consider \SI{1}{\liter} of a solution.
        \item We can find the mass of this quantity of solution via the density $\rho$. In particular,
        \begin{equation*}
            \text{mass of the solution per liter} = (\SI{1000}{\milli\liter\per\liter})\rho
        \end{equation*}
        \item We can extract from this, the molarity, and the molar mass of the solute $M_2$ the mass of solvent per liter.
        \begin{align*}
            \text{mass of the solvent per liter} &= \text{mass of the solution per liter}-\text{mass of the solute per liter}\\
            &= (\SI{1000}{\milli\liter\per\liter})\rho-cM_2
        \end{align*}
        \item Therefore, the number of moles of solvent per liter of solution is
        \begin{equation*}
            \frac{n_1}{\SI{1}{\liter}} = \frac{(\SI{1000}{\milli\liter\per\liter})\rho-cM_2}{M_1}
        \end{equation*}
        so
        \begin{align*}
            x_2 &= \frac{n_2}{n_1+n_2}\\
            &= \frac{n_2}{n_1+n_2}\cdot\frac{1/\SI{1}{\liter}}{1/\SI{1}{\liter}}\\
            &= \frac{c}{\frac{(\SI{1000}{\milli\liter\per\liter})\rho-cM_2}{M_1}+c}\\
            &= \frac{cM_1}{(\SI{1000}{\milli\liter\per\liter})\rho+c(M_1-M_2)}
        \end{align*}
    \end{itemize}
    \item For a nonvolatile solute, an equation of the form $a_2=P_2/\kH$ is not practical to use since $P_2$ is so small. Thus, we invoke the Gibbs-Duhem equation.
    \begin{itemize}
        \item Consider a solution of water with a small amount of a nonvolatile solute.
        \item Since $m\ll\SI{55.506}{\mole\per\kilo\gram}$, we have that
        \begin{equation*}
            x_2 \approx \frac{m}{\SI{55.506}{\mole\per\kilo\gram}}
        \end{equation*}
        \item Therefore, since $a_1\to x_1$ as $x_1\to 1$ (as we have at low concentrations), we have that
        \begin{equation*}
            \ln a_1 = \ln x_1
            = \ln(1-x_2)
            \approx -x_2
            \approx -\frac{m}{\SI{55.506}{\mole\per\kilo\gram}}
        \end{equation*}
        \item We can now use the form of the Gibbs-Duhem equation relating activities to find $a_2$.
        \item In particular, for \SI{1}{\kilo\gram} of solvent, $n_1=\SI{55.506}{\mole}$ and $n_2=m$, so we have
        \begin{equation*}
            (\SI{55.506}{\mole\per\kilo\gram})\dd{\ln a_1}+m\dd{\ln a_2} = 0
        \end{equation*}
        \item With the definition of the \textbf{osmotic coefficient}, we have that $(\SI{55.506}{\mole\per\kilo\gram})\dd{\ln a_1}=-\dd{(m\phi)}$. Thus,
        \begin{equation*}
            m\dd{\ln(\gamma_{2m}m)} = \dd{(m\phi)}
        \end{equation*}
        where $\gamma_{2m}$ is the molal activity coefficient.
        \item Therefore,
        \begin{align*}
            m\dd{\phi}+\phi\dd{m} &= m(\dd{\ln\gamma_{2m}}+\dd{\ln m})\\
            \dd{\ln\gamma_{2m}} &= \dd{\phi}+\frac{\phi-1}{m}\dd{m}\\
            \int_0^m\dd{\ln\gamma_{2m}} &= \int_0^m\dd{\phi}+\int_0^m\frac{\phi-1}{m'}\dd{m'}\\
            \ln\gamma_{2m}-\ln(1) &= \phi-1+\int_0^m\frac{\phi-1}{m'}\dd{m'}\\
            \ln\gamma_{2m} &= \phi-1+\int_0^m\frac{\phi-1}{m'}\dd{m'}
        \end{align*}
    \end{itemize}
    \item \textbf{Osmotic coefficient}: A quantity that accounts for the discrepancy between experimental solvent activities and ones predicted by the above equation. \emph{Denoted by} $\bm{\phi}$. \emph{Given by}
    \begin{equation*}
        \ln a_1 = -\frac{m\phi}{\SI{55.506}{\mole\per\kilo\gram}}
    \end{equation*}
    \begin{itemize}
        \item Since $\phi=1$ for an ideal dilute solution, the deviation of $\phi$ from unity is yet another measure of the nonideality of a solution.
        \item $\phi$ increases as $m$ increases (this should also make intuitive sense).
    \end{itemize}
    \item \textbf{Colligative property}: A property of a solution that depends (in dilute solutions) upon only the number, not the type, of solute particles.
    \item Deriving an equation for the activity of the solvent in a solution at the freezing point.
    \begin{itemize}
        \item At the freezing point, $\mu_1^s=\mu_1^\text{sln}$. In particular,
        \begin{align*}
            \mu_1^s &= \mu_1^l+RT\ln a_1\\
            \ln a_1 &= \frac{\mu_1^s-\mu_1^l}{RT}
        \end{align*}
        \item Differentiating the above with respect to $T$ at constant pressure and concentration, and substituting with the Gibbs-Helmholtz equation, give us
        \begin{align*}
            \left( \pdv{\ln a_1}{T} \right)_{P,x_1} &= \frac{1}{R}\left[ \pdv{T}(\frac{\mu_1^s}{T})-\pdv{T}(\frac{\mu_1^l}{T}) \right]_{P,x_1}\\
            &= \frac{1}{R}\left[ -\frac{\overline{H}_1^s}{T^2}+\frac{\overline{H}_1^l}{T^2} \right]\\
            &= \frac{\Delta_\text{fus}\overline{H}}{RT^2}
        \end{align*}
        \item Integrating the above from pure solvent $a_1=1$, $T=T_\text{fus}^*$ to a solution with arbitrary values of $a_1,T_\text{fus}$ allows us to calculate the activity $a_1$ of said solution.
    \end{itemize}
    \item \textbf{Freezing-point depression constant}: The proportionality constant between the change in freezing point $\Delta T_\text{fus}$ from the pure solvent to the given solution and the molality $m$ of the given solution. \emph{Denoted by} $\bm{K_f}$. \emph{Given by}
    \begin{equation*}
        \Delta T_\text{fus} = K_fm
    \end{equation*}
    \item Deriving $K_f$ from our freezing point activity equation.
    \begin{itemize}
        \item Consider a dilute solution. Assume $\Delta_\text{fus}\overline{H}$ is independent of temperature over the temperature range $(T_\text{fus},T_\text{fus}^*)$.
        \item It follows that
        \begin{align*}
            \ln a_1 &= \int_{T_\text{fus}^*}^{T_\text{fus}}\frac{\Delta_\text{fus}\overline{H}}{RT^2}\dd{T}\\
            \ln x_1 &\approx \frac{\Delta_\text{fus}\overline{H}}{R}\int_{T_\text{fus}^*}^{T_\text{fus}}\frac{\dd{T}}{T^2}\\
            \ln(1-x_2) &\approx \frac{\Delta_\text{fus}\overline{H}}{R}\left( \frac{1}{T_\text{fus}^*}-\frac{1}{T_\text{fus}} \right)\\
            -x_2 &\approx \frac{\Delta_\text{fus}\overline{H}}{R}\left( \frac{T_\text{fus}-T_\text{fus}^*}{T_\text{fus}T_\text{fus}^*} \right)
        \end{align*}
        \item Now for two more approximations. First, since $\Delta T_\text{fus}$ is small, we can set $T_\text{fus}\approx T_\text{fus}^*$ in the denominator of the above equation. Second, since $m$ is small, we may let
        \begin{equation*}
            x_2 = \frac{m}{\frac{\SI{1000}{\gram\per\kilo\gram}}{M_1}+m}
            \approx \frac{M_1m}{\SI{1000}{\gram\per\kilo\gram}}
        \end{equation*}
        \item Therefore, making the substitutions and rearranging the above gives us
        \begin{align*}
            -\frac{M_1m}{\SI{1000}{\gram\per\kilo\gram}} &= \frac{\Delta_\text{fus}\overline{H}}{R}\left( \frac{-\Delta T_\text{fus}}{(T_\text{fus}^*)^2} \right)\\
            \Delta T_\text{fus} = T_\text{fus}^*-T_\text{fus} &= \underbrace{\frac{M_1}{\SI{1000}{\gram\per\kilo\gram}}\frac{R(T_\text{fus}^*)^2}{\Delta_\text{fus}\overline{H}}}_{K_f}\cdot m
        \end{align*}
    \end{itemize}
    \item \textbf{Boiling-point elevation constant}: The proportionality constant between the change in boiling point $\Delta T_\text{vap}$ from the pure solvent to the given solution and the molality $m$ of the given solution. \emph{Denoted by} $\bm{K_b}$. \emph{Given by}
    \begin{equation*}
        \Delta T_\text{vap} = K_bm
    \end{equation*}
    \item We can derive an analogous equation to the freezing-point depression result in an analogous way, yielding
    \begin{equation*}
        \Delta T_\text{vap} = T_\text{vap}-T_\text{vap}^* = \underbrace{\frac{M_1}{\SI{1000}{\gram\per\kilo\gram}}\frac{R(T_\text{vap}^*)^2}{\Delta_\text{vap}\overline{H}}}_{K_b}\cdot m
    \end{equation*}
    \item \textbf{Semipermeable membrane}: A membrane containing pores that allow one substance to pass through but not another.
    \item \textbf{Osmotic pressure}: The hydrostatic pressure head that is built up as solvent molecules pass through a semipermeable membrane from pure solvent to solute-containing solution.
    \begin{itemize}
        \item Such a process happens to equalize the chemical potentials on the two sides of the semipermeable membrane.
    \end{itemize}
    \item Deriving an expression for the osmotic pressure.
    \begin{itemize}
        \item If $P$ is the pressure exerted on the system and $\Pi$ is the excess hydrostatic pressure on the solute-containing side, the system seeks $\Pi$ such that
        \begin{equation*}
            \mu_1^*(T,P) = \mu_1^\text{sln}(T,P+\Pi,a_1)
            = \mu_1^*(T,P+\Pi)+RT\ln a_1
        \end{equation*}
        \item It follows by the Maxwell relation for chemical potential that
        \begin{align*}
            0 &= \mu_1^*(T,P+\Pi)-\mu_1^*(T,P)+RT\ln a_1\\
            &= \int_P^{P+\Pi}\left( \pdv{\mu_1^*}{P'} \right)_T\dd{P'}+RT\ln a_1\\
            &= \int_P^{P+\Pi}\overline{V}^*\dd{P'}+RT\ln a_1
        \end{align*}
        \item Assuming that $\overline{V}_1^*$ is invariant under changing pressure, the above equation implies that
        \begin{align*}
            0 &= \Pi\overline{V}_1^*+RT\ln a_1\\
            \Pi\overline{V}_1^* &= RTx_2
        \end{align*}
        \item Since $n_2\ll n_1$, we may approximate $x_2=n_2/(n_1+n_2)\approx n_2/n_1$ to get
        \begin{align*}
            \Pi &= \frac{n_2RT}{n_1\overline{V}_1^*}\\
            &= \frac{n_2RT}{V}\\
            \Pi &= cRT
        \end{align*}
        where $V$ is the total volume of solvent.
    \end{itemize}
    \item \textbf{Van't Hoff equation} (for osmotic pressure): The above equation.
    \item \textbf{Reverse osmosis}: The process of applying pressure to seawater in excess of the osmotic pressure to force it through a rigid semipermeable membrane, resulting in fresh water.
    \begin{itemize}
        \item The most common such membrane is cellulose acetate.
    \end{itemize}
\end{itemize}




\end{document}