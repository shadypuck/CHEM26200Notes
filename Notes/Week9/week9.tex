\documentclass[../notes.tex]{subfiles}

\pagestyle{main}
\renewcommand{\chaptermark}[1]{\markboth{\chaptername\ \thechapter\ (#1)}{}}
\setcounter{chapter}{8}

\begin{document}




\chapter{Solid-Liquid Solutions and Electrochemistry}
\section{Solute Activity}
\begin{itemize}
    \item \marginnote{3/7:}At low concentration, solute activity is unity, with the chosen measure of concentration.
    \item Activity of ions in electrolytes initially decreases rapidly with concentration below \SI{1}{\molar}. The effect is much stronger than for neutral solutes.
    \item The Debye-H\"{u}ckel model reasonably explains why ion activity drops so quickly.
    \item Practical solution concentration measures are the \textbf{molarity} and \textbf{molality}, in addition to the mole fraction $x$.
    \item \textbf{Molality}: The number of moles of solute per kilogram solvent. \emph{Denoted by} $\bm{m}$. \emph{Given by}
    \begin{equation*}
        m = \frac{n_2}{\SI{1}{\kilo\gram\ solvent}}
    \end{equation*}
    \item \textbf{Concentration}: The number of moles of solute per liter solution. \emph{Also known as} \textbf{molarity}. \emph{Denoted by} $\bm{c}$. \emph{Given by}
    \begin{equation*}
        c = \frac{n_2}{\SI{1}{\liter\ solution}}
    \end{equation*}
    \item As the solvent gets purer, its vapor pressure approaches the Raoult's law limit.
    \begin{itemize}
        \item $a_1=P_1/P_1^*\to x_1$ as $x_1\to 1$.
    \end{itemize}
    \item As the solute's concentration diminishes, the solute activity approaches the Henry's law limit.
    \begin{itemize}
        \item $a_2=P_2/k_H\to x_2$ as $x_2\to 0$.
    \end{itemize}
    \item Using the vapor pressure of the solvent to determine the activity of the non-volatile solute.
    \begin{itemize}
        \item Use the Gibbs-Duhem relation,
        \begin{align*}
            0 &= n_1\dd{\ln a_1}+n_2\dd{\ln a_2}\\
            &= (\SI{55.506}{\mole\per\kilo\gram})\dd{\ln a_1}+m\dd{\ln a_2}\\
            \ln a_2 &= \int_0^m-\frac{55.506}{m'}\dd{(\ln a_1)}\dd{m'}
        \end{align*}
        \item But $\dd{(\ln a_1)}$ may not be very precise; thus, we define the \textbf{osmotic coefficient}.
        \item With this quantity, we have
        \begin{equation*}
            \dd{\ln a_1} = -\dd{(m\phi)}\cdot\frac{1}{55.506}
        \end{equation*}
        \item Thus, we have that
        \begin{align*}
            0 &= -\dd{(m\phi)}+m\dd{\ln a_2}\\
            \phi\dd{m}+m\dd{\phi} &= m\dd{\ln\gamma_2}+\dd{m}\\
            \frac{\dd{m}}{m}\phi+\dd{\phi} &= \dd{\ln\gamma_2}+\frac{\dd{m}}{m}\\
            \dd{\ln\gamma_2} &= \dd{\phi}+\left( \frac{\phi-1}{m} \right)\dd{m}\\
            \ln\gamma_2 &= (\phi-1)+\int_0^m\frac{\phi-1}{m'}\dd{m'}
        \end{align*}
    \end{itemize}
    \item \textbf{Osmotic coefficient}: The following quantity. \emph{Denoted by} $\bm{\phi}$. \emph{Given by}
    \begin{equation*}
        \phi = -\frac{\ln a_1}{x_2}
        = -\ln a_1\cdot\frac{\SI{55.506}{\mole\per\kilo\gram}}{m}
    \end{equation*}
    \begin{itemize}
        \item It is called the osmotic coefficient since $\Pi\overline{V}=-RT\ln a_1$ implies that as $x_2\to 0$, $\ln a_1\to -x_2$, so $\phi\to 1$.
        \item Even in the case of an ideal solution, $\phi\neq 1$ exactly, though, since we must mathematically \emph{approximate} $\ln a_1\approx -x_2$.
    \end{itemize}
    \item Ionic solutions deviate strongly from ideal solution, even at small molality.
    \item Mean ionic activity, molality, and activity coefficient.
    \begin{itemize}
        \item The salt $\ce{C}_{\nu_+}\ce{A}_{\nu_-}$ dissociates into $\nu_+$ \ce{C^{$z_+$}} ions and $\nu_-$ \ce{A^{$z_-$}} ions where
        \begin{equation*}
            \nu_+z_++\nu_-z_- = 0
        \end{equation*}
        for charge neutrality.
        \item Let $\nu=\nu_++\nu_-$.
        \item The chemical potential of the salt is
        \begin{align*}
            \mu_\text{salt} &= \nu_+\mu_++\nu_-\mu_-\\
            &= \nu_+\mu_+^\circ+\nu_-\mu_-^\circ+RT(\nu_+\ln a_++\nu_-\ln a_-)+RT\ln\big( \underbrace{a_+^{\nu_+}a_-^{\nu_-}}_{a_2} \big)
        \end{align*}
        \item Hence,
        \begin{align*}
            a_2 &= a_+^{\nu_+}a_-^{\nu_-}\\
            &= (\gamma_+^{\nu_+}\gamma_-^{\nu_-})(m_+^{\nu_+}m_-^{\nu_-})\\
            &= a_\pm^\nu\\
            &= \gamma_\pm^\nu m_\pm^\nu
        \end{align*}
        where $a_\pm$ is the \textbf{mean activity}, $\gamma_\pm$ is the \textbf{mean activity coefficient}, and $m_\pm$ is the \textbf{mean molality}.
        \item It follows that
        \begin{align*}
            \mu_2 &= \mu_2^\circ+RT\ln a_\pm^\nu\\
            &= \mu_2^\circ+\nu RT\ln a_\pm
        \end{align*}
    \end{itemize}
    \item Consider a solution of \ce{CaCl2} with molality $m$.
    \begin{itemize}
        \item It has
        \begin{align*}
            m_+ &= m&
            m_- &= 2m
        \end{align*}
        \item Thus,
        \begin{align*}
            m_\pm &= \sqrt{m_+^1m_-^2}\\
            &= \sqrt{4m^3}\\
            &= 4^{1/3}m
        \end{align*}
        \item We can do something similar for $\gamma_\pm$ and $a_\pm$.
    \end{itemize}
    \item Determining the activity of electrolytes.
    \begin{itemize}
        \item The derivation is pretty similar to before (for nonvolatile solvents), except that now $\ln a_1=-\nu m\phi/55.506$.
        \item It follows that
        \begin{align*}
            0 &= n_1\dd{\ln a_1}+n_2\dd{\ln a_2}\\
            &= -\nu\dd{(m\phi)}+m\nu\dd{\ln a_\pm}
        \end{align*}
        so $\nu$ cancels.
        \item This yields the same overall equation as before:
        \begin{equation*}
            \ln\gamma_\pm = (\phi-1)+\int_0^m\frac{\phi-1}{m'}\dd{m'}
        \end{equation*}
    \end{itemize}
    \item We can do the above, but ion activity can also be measured by electrochemistry.
\end{itemize}



\section{Debye-H\"{u}ckel Model}
\begin{itemize}
    \item \marginnote{3/9:}For ionic solutions, $\phi$ is fitted to a function of the form $\phi=\sum a_im^{i/2}$.
    \item For neutral solutes, $\phi$ is fitted to a function of the form $\phi=\sum a_im^i$.
    \item Debye-H\"{u}ckel model for small concentrations: $\ln\gamma_\pm\propto m^{1/2}$.
    \item We define $\pH=-\log a_1$, not $\pH=-\log[\ce{H3O+}]$.
    \begin{itemize}
        \item Thus, $\pH=-\log\gamma_\pm-\log[\ce{H3O+}]$.
        \item This also gives a method of determining $\gamma_\pm$.
    \end{itemize}
    \item Activity coefficients and interaction between solutes.
    \begin{itemize}
        \item If there are \emph{unfavorable} interactions between solutes (as in sucrose), $\gamma>1$.
        \item If there are \emph{favorable} interactions between solutes (as in \ce{NaCl}), $\gamma<1$.
        \item In terms of energy:
        \begin{align*}
            \mu &= \overline{\mu}+P\overline{V}-T\overline{S}\\
            &= \mu^*(x)+RT\ln x\\
            &= \mu^*_\text{ideal}+(\mu^*(x)-\mu_\text{ideal})+RT\ln x\\
            &= \mu^*_\text{ideal}+RT\ln a\\
            &= \mu^*_\text{ideal}+RT\ln\gamma+RT\ln x
        \end{align*}
        \item It follows that
        \begin{align*}
            RT\ln\gamma &= \mu^*(x)-\mu_\text{ideal}\\
            \gamma &= \exp\left( \frac{\mu^*(x)-\mu_\text{ideal}}{RT} \right)
        \end{align*}
        using only $\overline{\mu}\to\gamma=\exp\big( \frac{\mu^*(x)-\mu_\text{ideal}}{RT} \big)$.
        \begin{itemize}
            \item This mathematically shows that if we stabilize the ion energetically, then $\gamma<1$.
        \end{itemize}
    \end{itemize}
    \item Interactions between ions. The Debye length arises due to the screening of the charges. Calculated from the Boltzmann factor.
    \begin{itemize}
        \item Imagine a charge of magnitude $q$ at the center of our coordinate system.
        \item We have a charge density $\rho(\pi)$ at a distance $r$ from the nucleus.
        \item We have
        \begin{equation*}
            \rho(r) = \sum_iq_i\left( \frac{N_i}{V} \right)\cdot\e[-q_i\phi(r)/k_BT]
        \end{equation*}
        where the latter term is the Boltzmann factor and $\phi(r)$ is the electrostatic potential.
        \item We also have the Poisson equation,
        \begin{equation*}
            \nabla\phi(r) = -\frac{\rho(r)}{\varepsilon\varepsilon_0}
        \end{equation*}
        \item Since everything is depending on each other, it seems we are in a pickle.
        \item But if
        \begin{equation*}
            \frac{q_i\phi(r)}{k_BT} < 1
        \end{equation*}
        then
        \begin{equation*}
            \e[q_i\phi(r)/k_BT] = 1-\frac{q_i\phi(r)}{k_BT}
        \end{equation*}
        \item Therefore,
        \begin{equation*}
            \rho(r) = \sum_iq_i\left( \frac{N_i}{V} \right)\cdot\left( 1-\frac{q_i\phi(r)}{k_BT} \right)
            = \sum q_i\left( \frac{N_i}{V} \right)-\phi(r)\sum\frac{q_i^2\left( \frac{N_i}{V} \right)}{k_BT}
        \end{equation*}
        where the left term in the right expression above evaluates to zero.
        \item We can only have an excess of 1000 ions of one type of charge per liter of solution.
        \item It follows that
        \begin{align*}
            \nabla\phi(r) &= \left( \frac{\sum q_i^2\left( \frac{N_i}{V} \right)}{\varepsilon\varepsilon_0k_BT} \right)\cdot\phi(r)\\
            \pdv[2]{\phi}{r}+\cdots &= \kappa^2\cdot\phi
        \end{align*}
    \end{itemize}
    \item \textbf{Debye length}: $\kappa^{-1}$.
    \item In the one-dimensional case,
    \begin{align*}
        \pdv[2]{\phi}{x} &= \kappa^2\phi\\
        \phi(x) &= A\e[\kappa x]+B\e[-\kappa x]
    \end{align*}
    \begin{itemize}
        \item In other words, all interactions take place within the Debye length.
        \item The distance over which ion concentration differs from perfectly mixed ion potential around a charge.
        \item In electrochemistry, this is where everything happens.
    \end{itemize}
    \item Ions \textbf{screen} other ions.
    \item We have
    \begin{equation*}
        \kappa^2 = \frac{\sum q_i^2\left( \frac{N_i}{V} \right)}{\varepsilon\varepsilon_0k_BT}
    \end{equation*}
    \item \textbf{Ionic strength}: The following quantity. \emph{Denoted by} $\bm{I_c}$. \emph{Given by}
    \begin{equation*}
        I_c = \frac{1}{2}\sum z_i^2c_i
    \end{equation*}
    \item Thus, we have
    \begin{equation*}
        \kappa^{-1} = \frac{1}{\sqrt{I_c}}\cdot\sqrt{\frac{10^{-3}}{\e[2]N_A}\cdot\frac{\varepsilon\varepsilon_0k_BT}{2}}
    \end{equation*}
    \item Calculating the Debye length of \ce{NaCl}.
    \begin{itemize}
        \item For water, $\varepsilon\approx 80$.
        \item For \SI{1}{\molar} \ce{NaCl},
        \begin{equation*}
            I_c = \frac{1}{2}(1\cdot 1+1\cdot 1) = \SI{1}{\molar}
        \end{equation*}
        \item Thus, $\kappa^{-1}=\SI{3}{\angstrom}$.
    \end{itemize}
    \item Calculating the Debye length of \ce{CaCl2}.
    \begin{itemize}
        \item $I_c=\SI{3}{\molar}$.
    \end{itemize}
    \item Interactions lead to the Debye-H\"{u}ckel model for the mean ionic activity.
    \begin{itemize}
        \item We have
        \begin{equation*}
            \mu = -\frac{\kappa q^2}{8\pi\varepsilon\varepsilon_0}
        \end{equation*}
        which implies that
        \begin{equation*}
            \ln\gamma = -\frac{\kappa q^2}{8\pi\varepsilon\varepsilon_0k_BT}
        \end{equation*}
        \item Hence, by Problem 25.58,
        \begin{equation*}
            \ln\gamma_\pm = \frac{|z_+z_-|e^2\kappa}{8\pi\varepsilon\varepsilon_0k_BT}
        \end{equation*}
        where $e^2\kappa/8\pi\varepsilon\varepsilon_0$ comes from the Coulombic attraction.
        \item Thus, by the Debye-H\"{u}ckel model,
        \begin{equation*}
            \ln\gamma_\pm = -1.333|z_+z_-|\sqrt{I_c}
        \end{equation*}
    \end{itemize}
    \item One empirical extension of the Debye-H\"{u}ckel model is
    \begin{equation*}
        \ln\gamma_\pm = \frac{-1.173|z_+z_-|\sqrt{I_c}}{1+\sqrt{I_c}}+c_m
    \end{equation*}
\end{itemize}



\section{Chapter 25: Solutions II --- Solid-Liquid Solutions}
\emph{From \textcite{bib:McQuarrieSimon}.}
\begin{itemize}
    \item \marginnote{3/8:}Oppositely charged ions in solution interact with each other via a Coulombic potential, which varies proportionally to $1/r$.
    \begin{itemize}
        \item Note the difference between the strength of this potential and the LDF potential of $1/r^6$ attracting ordinary neutral molecules.
        \item It follows that solutions of electrolytes deviate from ideality much more quickly (i.e., at much lower concentrations).
    \end{itemize}
    \item When ions with higher charges (e.g., \ce{CaCl2} vs. \ce{NaCl}) are produced, the deviation from ideality is exacerbated.
    \item \textbf{$\bm{a}$-$\bm{b}$ electrolyte}: An electrolyte for which $\nu_+=a$ and $\nu_-=b$.
    \begin{itemize}
        \item For example, \ce{CaCl2} is a 1-2 electrolyte and \ce{Gd2(SO4)3} is a 2-3 electrolyte.
    \end{itemize}
    \item The chemical potential of a salt depends on the potentials of its constituent ions as per
    \begin{equation*}
        \mu_2 = \nu_+\mu_++\nu_-\mu_-
    \end{equation*}
    where $\mu_j=\mu_j^\circ+RT\ln a_j$ for $j=2,+,-$.
    \item It follows from the above and the fact that $\mu_2^\circ=\nu_+\mu_+^\circ+\nu_-\mu_-^\circ$, too, that
    \begin{align*}
        \mu_2^\circ+RT\ln a_2 &= \nu_+(\mu_+^\circ+RT\ln a_+)+\nu_-(\mu_-^\circ+RT\ln a_-)\\
        RT\ln a_2 &= \nu_+RT\ln a_++\nu_-RT\ln a_-\\
        \ln a_2 &= \nu_+\ln a_++\nu_-\ln a_-\\
        a_2 &= a_+^{\nu_+}a_-^{\nu_-}
    \end{align*}
    \item We define $\nu=\nu_++\nu_-$.
    \item \textbf{Mean ionic activity}: The following quantity. \emph{Denoted by} $a_\pm$. \emph{Given by}
    \begin{equation*}
        a_\pm^\nu = a_+^{\nu_+}a_-^{\nu_-}
    \end{equation*}
    \item We take the activity $a_2$ of an electrolyte to be its mean ionic activity to the $\nu^\text{th}$ power, i.e., $a_2=a_\pm^\nu$.
    \item We cannot determine the activities of single ions.
    \item We define the molalities of individual ions by
    \begin{align*}
        m_+ &= \nu_+m&
        m_- &= \nu_-m
    \end{align*}
    \item We define single-ion activity coefficients by
    \begin{align*}
        \gamma_+ &= \frac{a_+}{m_+}&
        \gamma_- &= \frac{a_-}{m_-}
    \end{align*}
    \item \textbf{Mean ionic molarity}: The following quantity. \emph{Denoted by} $m_\pm$. \emph{Given by}
    \begin{equation*}
        m_\pm^\nu = m_+^{\nu_+}m_-^{\nu_-}
    \end{equation*}
    \item \textbf{Mean ionic activity coefficient}: The following quantity. \emph{Denoted by} $\gamma_\pm$. \emph{Given by}
    \begin{equation*}
        \gamma_\pm^\nu = \gamma_+^{\nu_+}\gamma_-^{\nu_-}
    \end{equation*}
    \item It follows that
    \begin{equation*}
        a_2 = a_\pm^\nu
        = (m_+^{\nu_+}m_-^{\nu_-})(\gamma_+^{\nu_+}\gamma_-^{\nu_-})
        = m_\pm^\nu\gamma_\pm^\nu
    \end{equation*}
    \item \textcite{bib:McQuarrieSimon} derives
    \begin{equation*}
        \ln\gamma_\pm = (\phi-1)+\int_0^m\frac{\phi-1}{m'}\dd{m'}
    \end{equation*}
    \item For colligative properties, since $x_2$ is scaled up by a factor of $\nu$, we get
    \begin{align*}
        \Delta T_\text{fus} &= \nu K_fm&
        \Delta T_\text{vap} &= \nu K_bm&
        \Pi &= \nu cRT
    \end{align*}
    for solutions of electrolytes.
    \item Debye and H\"{u}ckel showed theoretically in 1925 that at low concentrations, the natural logarithms of the $j^\text{th}$ ion's activity coefficient and the mean ionic activity coefficient are given by
    \begin{align*}
        \ln\gamma_j &= -\frac{\kappa q_j^2}{8\pi\varepsilon_0\varepsilon_rk_BT}&
        \ln\gamma_\pm &= -|q_+q_-|\frac{\kappa}{8\pi\varepsilon_0\varepsilon_rk_BT}
    \end{align*}
    where $q_+=z_+e$ and $q_-=z_-e$, $\varepsilon_r$ is the (unitless) relative permittivity of the solvent, and $\kappa$ is given by
    \begin{equation*}
        \kappa^2 = \sum_{j=1}^s\frac{q_j^2}{\varepsilon_0\varepsilon_rk_BT}\left( \frac{N_j}{V} \right)
    \end{equation*}
    $s$ being the number of ionic species and $N_j/V$ being the number density of species $j$.
    \begin{itemize}
        \item Converting $N_j/V$ to molarity allows us to rewrite the above as
        \begin{equation*}
            \kappa^2 = N_A(\SI{1000}{\liter\per\cubic\meter})\sum_{j=1}^s\frac{q_j^2c_j}{\varepsilon_0\varepsilon_rk_BT}
        \end{equation*}
        \item This result is derived in Problems 25-50 through 25-58.
    \end{itemize}
    \item \textbf{Debye-H\"{u}ckel limiting law}: The above expression for $\ln\gamma_\pm$.
    \item \textbf{Ionic strength}: The following quantity. \emph{Denoted by} $\bm{I_c}$. \emph{Given by}
    \begin{equation*}
        I_c = \frac{1}{2}\sum_{j=1}^sz_j^2c_j
    \end{equation*}
    \begin{itemize}
        \item $c_j$ is the molarity of the $j^\text{th}$ ionic species.
    \end{itemize}
    \item Having defined the ionic strength, we can write express $\kappa$ in a third form as
    \begin{equation*}
        \kappa^2 = \frac{2e^2N_A(\SI{1000}{\liter\per\cubic\meter})}{\varepsilon_0\varepsilon_rk_BT}(I_c/\si{\mole\per\liter})
    \end{equation*}
    \item In the Debye-H\"{u}ckel limiting law, $\ln\gamma_\pm\propto\kappa\propto\sqrt{I_c}\propto\sqrt{c}$.
    \begin{itemize}
        \item This is why we curve fit $\phi$ vs. $m$ data to a polynomial in $m^{1/2}$.
    \end{itemize}
    \item As $m\to 0$, all curves $\ln\gamma_\pm$ vs. $m$ converge to the same straight line (hence this being a \emph{limiting} law).
    \item Physically interpreting $\kappa$.
    \begin{itemize}
        \item The net charge in a spherical shell of radius $r$ and thickness $\dd{r}$ surrounding an ion of charge $q_i$ is given by
        \begin{equation*}
            p_i(r)\dd{r} = -q_i\kappa^2r\e[-\kappa r]\dd{r}
        \end{equation*}
        \item Integrating this equation over all space yields $-q_i$.
        \item Thus, "the total charge surrounding an ion of charge $q_i$ is equal and of the opposite sign to $q_i$. In other words, it expresses the electroneutrality of the solution" \parencite[1033-34]{bib:McQuarrieSimon}.
        \item The above equation describes the \textbf{ionic atmosphere} about the central ion.
        \item Additionally, its maximum occurs at $\kappa^{-1}$, so we say that $\kappa$ is a measure of the thickness of the ionic atmosphere.
    \end{itemize}
    \item For a 1-1 electrolyte in aqueous solution at \SI{25}{\celsius},
    \begin{equation*}
        \kappa = \frac{\SI{304}{\pico\meter}}{\sqrt{c/\si{\mole\per\liter}}}
    \end{equation*}
    \item Assumptions of the Debye-H\"{u}ckel theory:
    \begin{itemize}
        \item Ions are point particles.
        \item They interact with a purely Coulombic potential.
        \item The solvent is a continuous medium with a uniform relative permittivity $\varepsilon_r$.
    \end{itemize}
    \item Clearly, these will start to break down as their concentration increases.
    \item The Debye-H\"{u}ckel model has been of important utility mainly in that all models that attempt to treat higher concentrations must reduce to it at lower temperatures.
    \item Real progress toward a more general theory wasn't made until the 1970s, though.
    \item \textbf{Primitive model}: Ions are considered hard spheres with charges at their centers and the solvent is considered a continuous medium with a uniform relative permittivity.
    \begin{itemize}
        \item Although, well, primitive, long-range Coulombic interactions and short-range repulsions, which this model encapsulates well, turn out to be major considerations, so the model is pretty good.
    \end{itemize}
    \item The mean spherical approximation (MSA) provides analytical solutions.
    \begin{itemize}
        \item A central result is that
        \begin{equation*}
            \ln\gamma_\pm = \ln\gamma_\pm^\text{el}+\ln\gamma^\text{HS}
        \end{equation*}
        where $\ln\gamma_\pm^\text{el}$ is an electrostatic (Coulombic) contribution to $\ln\gamma_\pm$ and $\ln\gamma^\text{HS}$ is a hard-sphere contribution.
        \item For a 1-1 electrolyte,
        \begin{equation*}
            \ln\gamma_\pm^\text{el} = \frac{x\sqrt{1+2x}-x-x^2}{4\pi\rho d^3}
        \end{equation*}
        where $\rho$ is the number density of charged particles, $d$ is the sum of the radius of a cation and anion, and $x=\kappa d$. Also,
        \begin{equation*}
            \ln\gamma^\text{HS} = \frac{4y-\frac{9}{4}y^2+\frac{3}{8}y^3}{\left( 1-\frac{y}{2} \right)^3}
        \end{equation*}
        where $y=\pi\rho d^3/6$.
    \end{itemize}
\end{itemize}




\end{document}