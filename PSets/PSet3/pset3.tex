\documentclass[../psets.tex]{subfiles}

\pagestyle{main}
\renewcommand{\leftmark}{Problem Set \thesection}
\setcounter{section}{2}

\begin{document}




\section{Deviations from Ideality}
\begin{enumerate}
    \item \marginnote{3/2:}An "ideal" rubber band is at \SI{300}{\kelvin} and a mass of \SI{200}{\gram} stretches its length by \SI{10}{\milli\meter}. Give the stretch length at \SI{330}{\kelvin}.
    \item A pill of \SI{0.1}{\mole} of gadolinium (III) sulfate (\ce{Gd2(SO4)3}) is used for adiabatic demagnetization.
    \begin{enumerate}
        \item Give the molar mass.
        \item Assuming all electrons unpaired in the $f$ orbital, give the magnetic moment.
        \item Calculate the heat transferred from the lattice to the spin degree of freedom as the field is varied from \SI{1}{\tesla} to \SI{0}{\tesla}.
        \item Estimate the heat capacity of the pill, and determine its lattice temperature if it starts at \SI{4}{\kelvin}.
    \end{enumerate}
    \item A Joule-Thomson expansion is an adiabatic and reversible expansion from a pressure $P_1$ to a pressure $P_2$, made closer to reversible by having a porous section that slows down the gas.
    \begin{enumerate}
        \item Show that the Joule-Thomson expansion conserves enthalpy.
        \item Show that an ideal gas's temperature is unchanged during a Joule-Thomson expansion.
        \item Argue why a real gas can cool (or heat up) during a Joule-Thomson expansion.
        \item Show that the temperature change is given by $(\pdv*{T}{P})_H$.
        \item Problem 22-48: Show that
        \begin{equation*}
            \left( \pdv{T}{P} \right)_H = \frac{V}{C_P}(\alpha T-1)
        \end{equation*}
        where $\alpha$ is the coefficient of thermal expansion.
        \item Problem 22-49: Show that a gas with only excluded volume cools upon expansion.
        \item Problem 22-51: Give the sign and estimate the magnitude of $(\pdv*{T}{P})_H$ for \ce{N2} gas around \SI{300}{\kelvin} from \SI{100}{\atmosphere} to \SI{1}{\atmosphere} based on van der Waals parameters.
    \end{enumerate}
\end{enumerate}




\end{document}